\title{In the Journals}
\date{2020-09-23 to 2020-09-30}
\maketitle

\section{APS Journals (prl, prxquantum, prresearch, pra, prb)}
\subsection*{\href{http://link.aps.org/doi/10.1103/PhysRevLett.125.140601}{Dynamical Phase Transitions in a 2D Classical Nonequilibrium Model via 2D Tensor Networks}}
\subsubsection*{Phillip Helms and Garnet Kin-Lic Chan (2020-09-29 prl)}
We demonstrate the power of 2D tensor networks for obtaining large deviation functions of dynamical observables in a classical nonequilibrium setting. Using these methods, we analyze the previously unstudied dynamical phase behavior of the fully 2D asymmetric simple exclusion process with biases in ...
\subsection*{\href{http://link.aps.org/doi/10.1103/PhysRevLett.125.140602}{Thermodynamic Uncertainty Relation for Arbitrary Initial States}}
\subsubsection*{Kangqiao Liu, \dots, and Masahito Ueda (2020-09-29 prl)}
The thermodynamic uncertainty relation (TUR) describes a trade-off relation between nonequilibrium currents and entropy production and serves as a fundamental principle of nonequilibrium thermodynamics. However, currently known TURs presuppose either specific initial states or an infinite-time avera...
\subsection*{\href{http://link.aps.org/doi/10.1103/PhysRevLett.125.143901}{Phase-Matching Controlled Orbital Angular Momentum Conversion in Periodically Poled Crystals}}
\subsubsection*{Yan Chen, \dots, and Shining Zhu (2020-09-29 prl)}
Nonlinear interactions between light waves can exchange energy, linear momentum, and angular momentum. The direction of energy flow between frequency components is usually determined by the conventional phase-matching condition related to the linear momentum. However, the transfer law of orbital ang...
\subsection*{\href{http://link.aps.org/doi/10.1103/PhysRevLett.125.147001}{Superconductivity in Uniquely Strained ${\mathrm{RuO}}_{2}$ Films}}
\subsubsection*{Masaki Uchida, \dots, and Masashi Kawasaki (2020-09-29 prl)}
Substrate-induced strain in RuO2 thin films generates the kind of phonons that promote superconductivity.
\subsection*{\href{http://link.aps.org/doi/10.1103/PhysRevLett.125.142503}{Measurement of the $^{229}\mathrm{Th}$ Isomer Energy with a Magnetic Microcalorimeter}}
\subsubsection*{Tomas Sikorsky, \dots, and Andreas Fleischmann (2020-09-28 prl)}
The high-precision measurement of a nuclear transition of a thorium isotope is a key step towards the development of a nuclear optical clock.
\subsection*{\href{http://link.aps.org/doi/10.1103/PhysRevLett.125.143001}{Ideal Unconventional Weyl Point in a Chiral Photonic Metamaterial}}
\subsubsection*{Yihao Yang, \dots, and Baile Zhang (2020-09-28 prl)}
A photonic unconventional Weyl point has been observed in a 3D topological chiral metamaterial, with two long surface arcs stretching over a large portion of the Brillouin zone, showing surface states are self-collimated and robust against disorder.
\subsection*{\href{http://link.aps.org/doi/10.1103/PhysRevLett.125.143201}{Coherent Suppression of Tensor Frequency Shifts through Magnetic Field Rotation}}
\subsubsection*{R. Lange, \dots, and E. Peik (2020-09-28 prl)}
We introduce a scheme to coherently suppress second-rank tensor frequency shifts in atomic clocks, relying on the continuous rotation of an external magnetic field during the free atomic state evolution in a Ramsey sequence. The method retrieves the unperturbed frequency within a single interrogatio...
\subsection*{\href{http://link.aps.org/doi/10.1103/PhysRevLett.125.143601}{Atom-Photon Spin-Exchange Collisions Mediated by Rydberg Dressing}}
\subsubsection*{Fan Yang, \dots, and Li You (2020-09-28 prl)}
We show that a single photon propagating through a Rydberg-dressed atomic ensemble can exchange its spin state with a single atom. Such a spin-exchange collision exhibits both dissipative and coherent features, depending on the interaction strength. For strong interaction, the collision dissipativel...
\subsection*{\href{http://link.aps.org/doi/10.1103/PhysRevLett.125.144501}{Fluid-Mediated Force on a Particle Due to an Oscillating Plate and Its Effect on Deposition Measurements by a Quartz Crystal Microbalance}}
\subsubsection*{Itzhak Fouxon, \dots, and Alexander Leshansky (2020-09-28 prl)}
Interaction of particles with boundaries is a fundamental problem in many fields of physics. In this Letter, we theoretically examine the fluid-mediated interaction between a horizontally oscillating plate and a spherical particle, revealing emergence of the novel nonlinear vertical force exerted on...
\subsection*{\href{http://link.aps.org/doi/10.1103/PhysRevLett.125.145001}{Magnetic Signatures of Radiation-Driven Double Ablation Fronts}}
\subsubsection*{P. Campbell, \dots, and L. Willingale (2020-09-28 prl)}
In experiments performed with the OMEGA EP laser system, magnetic field generation in double ablation fronts was observed. Proton radiography measured the strength, spatial profile, and temporal dynamics of self-generated magnetic fields as the target material was varied between plastic, aluminum, c...
\subsection*{\href{http://link.aps.org/doi/10.1103/PhysRevLett.125.145301}{Functional Monochalcogenides: Raman Evidence Linking Properties, Structure, and Metavalent Bonding}}
\subsubsection*{Christophe Bellin, \dots, and Abhay Shukla (2020-09-28 prl)}
Pressure- and temperature-dependent Raman scattering in GeSe, SnSe, and GeTe for pressures beyond 50 GPa and for temperatures ranging from 78 to 800 K allow us to identify structural and electronic phase transitions, similarities between GeSe and SnSe, and differences with GeTe. Calculations help to...
\subsection*{\href{http://link.aps.org/doi/10.1103/PhysRevLett.125.146001}{Transition Path Dynamics of a Dielectric Particle in a Bistable Optical Trap}}
\subsubsection*{Niels Zijlstra, \dots, and Benjamin Schuler (2020-09-28 prl)}
Many processes in chemistry, physics, and biology involve rare events in which the system escapes from a metastable state by surmounting an activation barrier. Examples range from chemical reactions, protein folding, and nucleation events to the catastrophic failure of bridges. A challenge in unders...
\subsection*{\href{http://link.aps.org/doi/10.1103/PhysRevLett.125.146101}{Universal Gas Adsorption Mechanism for Flat Nanobubble Morphologies}}
\subsubsection*{Nikolai D. Petsev, \dots, and M. Scott Shell (2020-09-28 prl)}
The adsorption of gas molecules at the substrate beneath interfacial nanobubbles modifies the energy of the solid-gas interface, and therefore affects their morphology. In this work, we describe a simple thermodynamic model that captures the influence of gas adsorption and gives flat bubble shapes w...
\subsection*{\href{http://link.aps.org/doi/10.1103/PhysRevLett.125.146401}{Higher-Order Weyl Semimetals}}
\subsubsection*{Hai-Xiao Wang, \dots, and Jian-Hua Jiang (2020-09-28 prl)}
Higher-order topology yields intriguing multidimensional topological phenomena, while Weyl semimetals have unconventional properties such as chiral anomaly. However, so far, Weyl physics remain disconnected with higher-order topology. Here, we report the theoretical discovery of higher-order Weyl se...
\subsection*{\href{http://link.aps.org/doi/10.1103/PhysRevLett.125.146801}{Magnetism of Topological Boundary States Induced by Boron Substitution in Graphene Nanoribbons}}
\subsubsection*{Niklas Friedrich, \dots, and José Ignacio Pascual (2020-09-28 prl)}
Graphene nanoribbons (GNRs), low-dimensional platforms for carbon-based electronics, show the promising perspective to also incorporate spin polarization in their conjugated electron system. However, magnetism in GNRs is generally associated with localized states around zigzag edges, difficult to fa...
\subsection*{\href{http://link.aps.org/doi/10.1103/PhysRevLett.125.149001}{Comment on “Einstein-Gauss-Bonnet Gravity in Four-Dimensional Spacetime”}}
\subsubsection*{Metin Gürses, \dots, and Bayram Tekin (2020-09-28 prl)}
Author(s): Metin Gürses, Tahsin Çağrı Şişman, and Bayram Tekin[Phys. Rev. Lett. 125, 149001] Published Mon Sep 28, 2020
\subsection*{\href{http://link.aps.org/doi/10.1103/PhysRevLett.125.149002}{Comment on “Einstein-Gauss-Bonnet Gravity in Four-Dimensional Spacetime”}}
\subsubsection*{Julio Arrechea, \dots, and Alejandro Jiménez-Cano (2020-09-28 prl)}
Author(s): Julio Arrechea, Adrià Delhom, and Alejandro Jiménez-Cano[Phys. Rev. Lett. 125, 149002] Published Mon Sep 28, 2020
\subsection*{\href{http://link.aps.org/doi/10.1103/PhysRevLett.125.130401}{Quantifying Dynamical Coherence with Dynamical Entanglement}}
\subsubsection*{Thomas Theurer, \dots, and Martin B. Plenio (2020-09-25 prl)}
Coherent superposition and entanglement are two fundamental aspects of nonclassicality. Here we provide a quantitative connection between the two on the level of operations by showing that the dynamical coherence of an operation upper bounds the dynamical entanglement that can be generated from it w...
\subsection*{\href{http://link.aps.org/doi/10.1103/PhysRevLett.125.130402}{Renormalization Group Flow of the Jaynes-Cummings Model}}
\subsubsection*{Anton Ilderton (2020-09-25 prl)}
The Jaynes-Cummings model is a cornerstone of light-matter interactions. While finite, the model provides an illustrative example of renormalization in perturbation theory. We show, however, that exact renormalization reveals a rich nonperturbative structure, and that the model provides a physical e...
\subsection*{\href{http://link.aps.org/doi/10.1103/PhysRevLett.125.131301}{New Freezeout Mechanism for Strongly Interacting Dark Matter}}
\subsubsection*{Juri Smirnov and John F. Beacom (2020-09-25 prl)}
We present a new mechanism for thermally produced dark matter, based on a semi-annihilation-like process, $χ+χ+\mathrm{SM}→χ+\mathrm{SM}$, with intriguing consequences for the properties of dark matter. First, its mass is low, $≲1\text{ }\text{ }\mathrm{GeV}$ (but $≳5\text{ }\text{ }\mathrm{keV}$ to...
\subsection*{\href{http://link.aps.org/doi/10.1103/PhysRevLett.125.131604}{All Tree-Level Correlators for M Theory on ${\mathrm{AdS}}_{7}×{S}^{4}$}}
\subsubsection*{Luis F. Alday and Xinan Zhou (周稀楠) (2020-09-25 prl)}
We present a constructive derivation of all four-point tree-level holographic correlators for eleven dimensional supergravity on ${\mathrm{AdS}}_{7}×{S}^{4}$. These correlators correspond to four-point functions of arbitrary one-half BPS operators in the six-dimensional (2,0) theory at large central...
\subsection*{\href{http://link.aps.org/doi/10.1103/PhysRevLett.125.132501}{First Observation of Multiple Transverse Wobbling Bands of Different Kinds in $^{183}\mathrm{Au}$}}
\subsubsection*{S. Nandi, \dots, and A. Goswami (2020-09-25 prl)}
We report the first observation of two wobbling bands in $^{183}\mathrm{Au}$, both of which were interpreted as the transverse wobbling (TW) band but with different behavior of their wobbling energies as a function of spin. It increases (decreases) with spin for the positive (negative) parity config...
\subsection*{\href{http://link.aps.org/doi/10.1103/PhysRevLett.125.134102}{Testing Critical Slowing Down as a Bifurcation Indicator in a Low-Dissipation Dynamical System}}
\subsubsection*{M. Marconi, \dots, and J. R. Tredicce (2020-09-25 prl)}
We study a two-dimensional low-dissipation nonautonomous dynamical system, with a control parameter that is swept linearly in time across a transcritical bifurcation. We investigate the relaxation time of a perturbation applied to a variable of the system and we show that critical slowing down may o...
\subsection*{\href{http://link.aps.org/doi/10.1103/PhysRevLett.125.135001}{Weakly Magnetized, Hall Dominated Plasma Couette Flow}}
\subsubsection*{K. Flanagan, \dots, and C. B. Forest (2020-09-25 prl)}
Electromagnetic fields rotate a plasma and produce conditions that resemble the region around a newly forming star.
\subsection*{\href{http://link.aps.org/doi/10.1103/PhysRevLett.125.137202}{Direct Observation of the Statics and Dynamics of Emergent Magnetic Monopoles in a Chiral Magnet}}
\subsubsection*{N. Kanazawa, \dots, and Y. Tokura (2020-09-25 prl)}
In the three-dimensional (3D) Heisenberg model, topological point defects known as spin hedgehogs behave as emergent magnetic monopoles, i.e., quantized sources and sinks of gauge fields that couple strongly to conduction electrons, and cause unconventional transport responses such as the gigantic H...
\subsection*{\href{http://link.aps.org/doi/10.1103/PhysRevLett.125.139901}{Erratum: Structural Origin of Enhanced Dynamics at the Surface of a Glassy Alloy [Phys. Rev. Lett. 119, 245501 (2017)]}}
\subsubsection*{Gang Sun, \dots, and Peter Harrowell (2020-09-25 prl)}
Author(s): Gang Sun, Shibu Saw, Ian Douglass, and Peter Harrowell[Phys. Rev. Lett. 125, 139901] Published Fri Sep 25, 2020
\subsection*{\href{http://link.aps.org/doi/10.1103/PhysRevLett.125.131804}{Solar Axions Cannot Explain the XENON1T Excess}}
\subsubsection*{Luca Di Luzio, \dots, and Enrico Nardi (2020-09-24 prl)}
Solar-axion explanations for the excess of electron recoil events reported by the XENON1T experiment are in severe tension with astrophysical data.
\subsection*{\href{http://link.aps.org/doi/10.1103/PhysRevLett.125.131805}{Inverse Primakoff Scattering as a Probe of Solar Axions at Liquid Xenon Direct Detection Experiments}}
\subsubsection*{James B. Dent, \dots, and Adrian Thompson (2020-09-24 prl)}
We show that XENON1T and future liquid xenon (LXe) direct detection experiments are sensitive to axions through the standard ${g}_{aγ}aF\stackrel{˜}{F}$ operators due to inverse-Primakoff scattering. This previously neglected channel significantly improves the sensitivity to the axion-photon couplin...
\subsection*{\href{http://link.aps.org/doi/10.1103/PhysRevLett.125.131806}{Reexamining the Solar Axion Explanation for the XENON1T Excess}}
\subsubsection*{Christina Gao, \dots, and Yi-Ming Zhong (2020-09-24 prl)}
The XENON1T collaboration has observed an excess in electronic recoil events below 5 keV over the known background, which could originate from beyond-the-standard-model physics. The solar axion is a well-motivated model that has been proposed to explain the excess, though it has tension with astroph...
\subsection*{\href{http://link.aps.org/doi/10.1103/PhysRevLett.125.133401}{Quasiparticle Lifetime of the Repulsive Fermi Polaron}}
\subsubsection*{Haydn S. Adlong, \dots, and Jesper Levinsen (2020-09-24 prl)}
We investigate the metastable repulsive branch of a mobile impurity coupled to a degenerate Fermi gas via short-range interactions. We show that the quasiparticle lifetime of this repulsive Fermi polaron can be experimentally probed by driving Rabi oscillations between weakly and strongly interactin...
\subsection*{\href{http://link.aps.org/doi/10.1103/PhysRevLett.125.133603}{Topological Anderson Insulator in Disordered Photonic Crystals}}
\subsubsection*{Gui-Geng Liu, \dots, and Baile Zhang (2020-09-24 prl)}
By varying the amount of disorder in a photonic crystal, researchers can control several topological features of the crystal.
\subsection*{\href{http://link.aps.org/doi/10.1103/PhysRevLett.125.137201}{Depth-Resolved Magnetization Dynamics Revealed by X-Ray Reflectometry Ferromagnetic Resonance}}
\subsubsection*{D. M. Burn, \dots, and T. Hesjedal (2020-09-24 prl)}
Researchers combine ferromagnetic resonance with x-ray reflectivity to map out the complex spin behavior of a magnetic multilayer.
\subsection*{\href{http://link.aps.org/doi/10.1103/PhysRevLett.125.138001}{Sheared Amorphous Packings Display Two Separate Particle Transport Mechanisms}}
\subsubsection*{Dong Wang, \dots, and Hu Zheng (2020-09-24 prl)}
Shearing granular materials induces nonaffine displacements. Such nonaffine displacements have been studied extensively, and are known to correlate with plasticity and other mechanical features of amorphous packings. A well known example is shear transformation zones as captured by the local deviati...
\subsection*{\href{http://link.aps.org/doi/10.1103/PhysRevLett.125.131803}{First Precision Measurement of the Parity Violating Asymmetry in Cold Neutron Capture on $^{3}\mathrm{He}$}}
\subsubsection*{M. T. Gericke et al. (nHe3 Collaboration) (2020-09-23 prl)}
A new measurement from the n3He Collaboration advances understanding of parity violation in few-nucleon systems.
\subsection*{\href{http://link.aps.org/doi/10.1103/PhysRevLett.125.133604}{Interaction-Induced Transparency for Strong-Coupling Polaritons}}
\subsubsection*{Johannes Lang, \dots, and Francesco Piazza (2020-09-23 prl)}
The propagation of light in strongly coupled atomic media takes place through the formation of polaritons—hybrid quasiparticles resulting from a superposition of an atomic and a photonic excitation. Here we consider the propagation under the condition of electromagnetically induced transparency and ...
\subsection*{\href{http://link.aps.org/doi/10.1103/PhysRevLett.125.134101}{Chaos and Quantum Scars in Bose-Josephson Junction Coupled to a Bosonic Mode}}
\subsubsection*{Sudip Sinha and S. Sinha (2020-09-23 prl)}
We consider a model describing Bose-Josephson junction (BJJ) coupled to a single bosonic mode exhibiting quantum phase transition (QPT). Onset of chaos above QPT is observed from semiclassical dynamics as well from spectral statistics. Based on entanglement entropy, we analyze the ergodic behavior o...
\subsection*{\href{http://link.aps.org/doi/10.1103/PhysRevLett.125.136402}{Bethe-Salpeter Equation at the Critical End Point of the Mott Transition}}
\subsubsection*{Erik G. C. P. van Loon, \dots, and Andrey A. Katanin (2020-09-23 prl)}
Strong repulsive interactions between electrons can lead to a Mott metal-insulator transition. The dynamical mean-field theory (DMFT) explains the critical end point and the hysteresis region usually in terms of single-particle concepts, such as the spectral function and the quasiparticle weight. In...
\subsection*{\href{http://link.aps.org/doi/10.1103/PhysRevLett.125.138002}{Crowding-Enhanced Diffusion: An Exact Theory for Highly Entangled Self-Propelled Stiff Filaments}}
\subsubsection*{Suvendu Mandal, \dots, and Hartmut Löwen (2020-09-23 prl)}
A dense network inhibits the rotation of self-propelled protein filaments but unexpectedly promotes their overall diffusivity.
\subsection*{\href{http://link.aps.org/doi/10.1103/PRXQuantum.1.010309}{Quasiprobability Distribution for Heat Fluctuations in the Quantum Regime}}
\subsubsection*{Amikam Levy and Matteo Lostaglio (2020-09-25 prxquantum)}
A new tool applied to fluctuation theorems captures quantum phenomena that have no analogs in stochastic thermodynamics.
\subsection*{\href{http://link.aps.org/doi/10.1103/PhysRevResearch.2.032073}{Fragmented monopole crystal, dimer entropy, and Coulomb interactions in ${\mathrm{Dy}}_{2}{\mathrm{Ir}}_{2}{\mathrm{O}}_{7}$}}
\subsubsection*{V. Cathelin, \dots, and E. Lhotel (2020-09-29 prresearch)}
This paper shows that the Dy2Ir2O7 pyrochlore compound stabilizes a fragmented monopole crystal phase at low temperature, and that it yields the residual entropy predicted by theory.
\subsection*{\href{http://link.aps.org/doi/10.1103/PhysRevResearch.2.033484}{Tunable quantum switcher and router of single atoms using localized artificial magnetic fields}}
\subsubsection*{Yan-Jun Zhao, \dots, and Wu-Ming Liu (2020-09-29 prresearch)}
The authors propose to generate localized artificial magnetic fields for cold atoms using two thin Raman laser beams in a two-rung two-leg ladder.
\subsection*{\href{http://link.aps.org/doi/10.1103/PhysRevResearch.2.033510}{Exciton-polariton interference controlled by electric field}}
\subsubsection*{D. K. Loginov, \dots, and Y. Masumoto (2020-09-29 prresearch)}
The authors show how the spectral oscillations in reflectance spectra caused by the interference of exciton-polaritonic waves in a wide quantum well are sensitive to an applied electric field, which gives rise to the inversion of the oscillation phase at some critical value of the field strength.
\subsection*{\href{http://link.aps.org/doi/10.1103/PhysRevResearch.2.033511}{Chiral anomalies induced transport in Weyl metals in quantizing magnetic field}}
\subsubsection*{Kamal Das, \dots, and Amit Agarwal (2020-09-29 prresearch)}
The authors explore the signatures of quantum chiral anomalies in Weyl superfluids in transport experiments at high magnetic fields.
\subsection*{\href{http://link.aps.org/doi/10.1103/PhysRevResearch.2.033512}{Quantum Zeno effect appears in stages}}
\subsubsection*{Kyrylo Snizhko, \dots, and Alessandro Romito (2020-09-29 prresearch)}
The work investigates the stochastic dynamics of a qubit under continuous partial measurement, and show that the Zeno regime is reached via a cascade of transitions, each happening at a different measurement strength.
\subsection*{\href{http://link.aps.org/doi/10.1103/PhysRevResearch.2.033514}{Parameter estimation for strong phase transitions in supranuclear matter using gravitational-wave astronomy}}
\subsubsection*{Peter T. H. Pang, \dots, and Chris Van Den Broeck (2020-09-29 prresearch)}
Using Bayesian inference techniques, this paper shows that a strong phase transition from hadronic matter to more exotic forms of matter can have a measurable imprint on the gravitational-wave signal of binary neutron-star mergers.
\subsection*{\href{http://link.aps.org/doi/10.1103/PhysRevResearch.2.033515}{Intrinsic sign problems in topological quantum field theories}}
\subsubsection*{Adam Smith, \dots, and Zohar Ringel (2020-09-29 prresearch)}
This work shows that a class of topologically ordered lattice models have intrinsic sign-problems, which cannot be removed by any local unitary transformation. This is achieved by relating the anyonic statistics of the topological excitations to the non-negativity of the ground states.
\subsection*{\href{http://link.aps.org/doi/10.1103/PhysRevResearch.2.033516}{Local chemical bonding and structural properties in ${\mathrm{Ti}}_{3}\mathrm{Al}{\mathrm{C}}_{2}$ MAX phase and ${\mathrm{Ti}}_{3}{\mathrm{C}}_{2}{T}_{x}$ MXene probed by Ti $1s$ x-ray absorption spectroscopy}}
\subsubsection*{Martin Magnuson and Lars-Åke Näslund (2020-09-29 prresearch)}
The authors investigate the MAX phase material Ti3AlC2 and the corresponding MXene material Ti3C2Tx, where the latter was examined before and after a series of heat treatments. The Ti-C bond lengths in the Ti3C2-layers are altered when the stronger interacting F and O in Ti3C2Tx replace the Al-monolayer in Ti3AlC2 and an additional heat treatment to 750 °C changes the Ti-O/F coordination.
\subsection*{\href{http://link.aps.org/doi/10.1103/PhysRevResearch.2.033517}{Experimental demonstration of cavity-free optical isolators and optical circulators}}
\subsubsection*{En-Ze Li, \dots, and Bao-Sen Shi (2020-09-29 prresearch)}
The authors present experimental results pertaining to the realization of optical isolators and circulators. This paper realizes it experimentally with a cross-Kerr nonlinearity achieved with a medium comprising a thermal vapor of N-type atoms
\subsection*{\href{http://link.aps.org/doi/10.1103/PhysRevResearch.2.033518}{Time-dependent properties of interacting active matter: Dynamical behavior of one-dimensional systems of self-propelled particles}}
\subsubsection*{Lorenzo Caprini and Umberto Marini Bettolo Marconi (2020-09-29 prresearch)}
This paper studies the dynamical properties of a one-dimensional active matter system at high density, and observe the spontaneous alignment of the particles’ velocities giving rise to ordered velocity domains.
\subsection*{\href{http://link.aps.org/doi/10.1103/PhysRevResearch.2.033519}{Geometric detection of hierarchical backbones in real networks}}
\subsubsection*{Elisenda Ortiz, \dots, and M. Ángeles Serrano (2020-09-29 prresearch)}
This work presents a geometric framework for the detection of hierarchical ordering in real complex networks and provides a filtering mechanism able to extract hierarchical backbones which favor cooperation in evolutionary dynamics modelling social dilemmas
\subsection*{\href{http://link.aps.org/doi/10.1103/PhysRevResearch.2.033520}{Apparent superballistic dynamics in one-dimensional random walks with biased detachment}}
\subsubsection*{Chapin S. Korosec, \dots, and Nancy R. Forde (2020-09-29 prresearch)}
The authors find a random walk can be tuned to exhibit dynamics from conventional diffusion to superballistic motion by adjusting detachment from its track.
\subsection*{\href{http://link.aps.org/doi/10.1103/PhysRevResearch.2.032072}{Electronic correlation and geometrical frustration in molecular solids: A systematic ab initio study of ${β}^{′}\text{−}X{[\mathrm{Pd}{(\mathrm{dmit})}_{2}]}_{2}$}}
\subsubsection*{Takahiro Misawa, \dots, and Takao Tsumuraya (2020-09-28 prresearch)}
The authors derive ab initio low-energy effective Hamiltonians for molecular solids Pd(dmit)2 salts and explore their solution in the magnetic properties of their compounds.
\subsection*{\href{http://link.aps.org/doi/10.1103/PhysRevResearch.2.033501}{Two-body mobility edge in the Anderson-Hubbard model in three dimensions: Molecular versus scattering states}}
\subsubsection*{Filippo Stellin and Giuliano Orso (2020-09-28 prresearch)}
The authors study the phase diagram of Anderson localization for a system of two particles moving in a random three-dimensional lattice and coupled by onsite, Hubbard, interactions.
\subsection*{\href{http://link.aps.org/doi/10.1103/PhysRevResearch.2.033502}{Mitigation of strong electromagnetic pulses on the LMJ-PETAL facility}}
\subsubsection*{M. Bardon, \dots, and V. T. Tikhonchuk (2020-09-28 prresearch)}
This paper shows how electromagnetic pulses mitigation is performed on the LMJ-PETAL laser facility. In particular, the authors show mitigation suppression for joint shots with appropriate time delay between the LMJ nanosecond beams and the PETAL picosecond beam.
\subsection*{\href{http://link.aps.org/doi/10.1103/PhysRevResearch.2.033503}{Collapse of the simple localized $3{d}^{1}$ orbital picture in Mott insulator}}
\subsubsection*{Shunsuke Kitou, \dots, and Hiroshi Sawa (2020-09-28 prresearch)}
This work introduces a direct determination method of an orbital state in materials, using synchrotron X-ray diffraction and an electron density analysis.
\subsection*{\href{http://link.aps.org/doi/10.1103/PhysRevResearch.2.033504}{Two-axis two-spin squeezed states}}
\subsubsection*{Jonas Kitzinger, \dots, and Tim Byrnes (2020-09-28 prresearch)}
This paper examines the properties of the states generated by the two-spin generalization of the two-axis countertwisting Hamiltonian. The spin squeezing produces correlations for arbitrary spin directions, and can violate the Bell-CHSH inequality with quadratic spin correlators
\subsection*{\href{http://link.aps.org/doi/10.1103/PhysRevResearch.2.033505}{Rényi entanglement entropy of Fermi and non-Fermi liquids: Sachdev-Ye-Kitaev model and dynamical mean field theories}}
\subsubsection*{Arijit Haldar, \dots, and Sumilan Banerjee (2020-09-28 prresearch)}
The paper presents a path-integral method for calculating entanglement entropies for interacting fermions using a representation of Renyi entropies of a subsystem in terms of fermionic displacement operators.
\subsection*{\href{http://link.aps.org/doi/10.1103/PhysRevResearch.2.033506}{Phase diagram of solitons in the polar phase of a spin-1 Bose-Einstein condensate}}
\subsubsection*{I-Kang Liu, \dots, and Hiromitsu Takeuchi (2020-09-28 prresearch)}
This work shows the soliton structures in a spin-1 Bose-Einstein condensate under the influence of quadratic Zeeman energy as well as the spin-dependent collisions between atoms.
\subsection*{\href{http://link.aps.org/doi/10.1103/PhysRevResearch.2.033507}{Probing two-level systems with electron spin inversion recovery of defects at the $\mathrm{Si}/{\mathrm{SiO}}_{2}$ interface}}
\subsubsection*{M. Belli, \dots, and R. de Sousa (2020-09-28 prresearch)}
The authors show that measurements of nonexponential time decay of dangling-bond spin magnetization produces information about amorphous two-level systems at the silicon/silicon-oxide interface.
\subsection*{\href{http://link.aps.org/doi/10.1103/PhysRevResearch.2.033508}{Work as an external quantum observable and an operational quantum work fluctuation theorem}}
\subsubsection*{Konstantin Beyer, \dots, and Walter T. Strunz (2020-09-28 prresearch)}
The authors propose an operational definition of quantum work which allows to determine free energy differences for unknown Hamiltonians.
\subsection*{\href{http://link.aps.org/doi/10.1103/PhysRevResearch.2.033509}{Competing interactions in dysprosium garnets and generalized magnetic phase diagram of $\mathrm{S}=\frac{1}{2}$ spins on a hyperkagome network}}
\subsubsection*{I. A. Kibalin, \dots, and S. Petit (2020-09-28 prresearch)}
The authors use a combination of different neutron scattering techniques to demonstrate that the rare-earth anisotropy in the hyperkagome lattice of garnet can be tuned away from the standard Ising picture.
\subsection*{\href{http://link.aps.org/doi/10.1103/PhysRevResearch.2.033491}{Quantum damping of skyrmion crystal eigenmodes due to spontaneous quasiparticle decay}}
\subsubsection*{Alexander Mook, \dots, and Daniel Loss (2020-09-25 prresearch)}
The authors reveal a magnetic-field tunable spectral magnon broadening due to spontaneous quasiparticle decay in ferromagnetic skyrmion crystals, and study their quantum damping properties.
\subsection*{\href{http://link.aps.org/doi/10.1103/PhysRevResearch.2.033492}{Quantum Zermelo problem for general energy resource bounds}}
\subsubsection*{Josep Maria Bofill, \dots, and Wolfgang Quapp (2020-09-25 prresearch)}
The authors show how the quantum Zermelo Hamiltonian, under certain energy-resource bounds, supplies a realizable time-optimal control protocol to manipulate the evolution of quantum state.
\subsection*{\href{http://link.aps.org/doi/10.1103/PhysRevResearch.2.033493}{Majorana oscillations and parity crossings in semiconductor nanowire-based transmon qubits}}
\subsubsection*{J. Ávila, \dots, and R. Aguado (2020-09-25 prresearch)}
This paper investigates the microwave response of transmon qubits based on semiconducting nanowire Josephson junctions in the topological regime.
\subsection*{\href{http://link.aps.org/doi/10.1103/PhysRevResearch.2.033494}{Floquet engineering of twisted double bilayer graphene}}
\subsubsection*{Martin Rodriguez-Vega, \dots, and Gregory A. Fiete (2020-09-25 prresearch)}
The authors consider twisted double-bilayer graphene driven by a light source subjected to free space and waveguide boundary conditions.
\subsection*{\href{http://link.aps.org/doi/10.1103/PhysRevResearch.2.033495}{Time-induced second-order topological superconductors}}
\subsubsection*{Raditya Weda Bomantara (2020-09-25 prresearch)}
The author proposes the generation of second-order topological superconductors by encoding some necessary topology in the time-domain.
\subsection*{\href{http://link.aps.org/doi/10.1103/PhysRevResearch.2.033496}{Interfacial-hybridization-modified Ir ferromagnetism and electronic structure in ${\mathrm{LaMnO}}_{3}\text{/}{\mathrm{SrIrO}}_{3}$ superlattices}}
\subsubsection*{Yujun Zhang, \dots, and Hiroki Wadati (2020-09-25 prresearch)}
This work explores the mechanism of interfacial coupling in LaMnO3/SrIrO3 superlattices by x-ray spectroscopies and first-principles calculations. The superlattice-period dependent properties of the Ir magnetic moments can be attributed to the realignment of electron spin during the formation of the interfacial molecular orbital.
\subsection*{\href{http://link.aps.org/doi/10.1103/PhysRevResearch.2.033497}{Discrimination of thermal baths by single-qubit probes}}
\subsubsection*{Ilaria Gianani, \dots, and Vittorio Giovannetti (2020-09-25 prresearch)}
The authors use indirect probing to discriminate thermal reservoirs with either bosonic or fermionic statistics and temperatures, by means of qubits
\subsection*{\href{http://link.aps.org/doi/10.1103/PhysRevResearch.2.033498}{Nonequilibrium readiness and precision of Gaussian quantum thermometers}}
\subsubsection*{Luca Mancino, \dots, and Mauro Paternostro (2020-09-25 prresearch)}
The authors study quantum thermometers using Gaussian states and show that the speed of their response is governed by their quantum properties while their precision reaches its optimal value at thermalization.
\subsection*{\href{http://link.aps.org/doi/10.1103/PhysRevResearch.2.033499}{Discovering symmetry invariants and conserved quantities by interpreting siamese neural networks}}
\subsubsection*{Sebastian J. Wetzel, \dots, and Vijay Ganesh (2020-09-25 prresearch)}
The authors train a Siamese neural network to decide whether two different descriptions describe the same physical object. The neural network learns to identify the objects by calculating the underlying invariants and conserved quantities.
\subsection*{\href{http://link.aps.org/doi/10.1103/PhysRevResearch.2.033500}{Origin of large-amplitude oscillations of dust particles in a plasma sheath}}
\subsubsection*{Joshua Méndez Harper, \dots, and Justin C. Burton (2020-09-25 prresearch)}
The authors investigate the how particles levitating in a plasma extract energy from their environment to produce stable, large-amplitude vertical oscillations
\subsection*{\href{http://link.aps.org/doi/10.1103/PhysRevResearch.2.033479}{Chiral excitonic instability of two-dimensional tilted Dirac cones}}
\subsubsection*{Daigo Ohki, \dots, and Akito Kobayashi (2020-09-24 prresearch)}
This work explores excitonic pairing instability in Zeeman-split two- dimensional Dirac cones with titled valleys, considering self-energy corrections via renormalization-group technique and a ladder-type vertex. The authors demonstrate how the pairing is affected by in-plane magnetic field and small carrier doping near charge neutrality.
\subsection*{\href{http://link.aps.org/doi/10.1103/PhysRevResearch.2.033480}{Beyond linear coupling in microwave optomechanics}}
\subsubsection*{D. Cattiaux, \dots, and E. Collin (2020-09-24 prresearch)}
This paper investigates the nonlinear effects that imprint the self-oscillating state of a nanomechanical oscillator embedded in a microwave cavity.
\subsection*{\href{http://link.aps.org/doi/10.1103/PhysRevResearch.2.033481}{Inclined convection in a layer of liquid water with poorly conducting boundaries}}
\subsubsection*{Stefano Castellini, \dots, and Alberto Vailati (2020-09-24 prresearch)}
This work reports a transition between two convective heat transfer regimes, occurring when a horizontal layer of liquid water is inclined at an angle smaller than one degree in the presence of poorly conducting boundaries.
\subsection*{\href{http://link.aps.org/doi/10.1103/PhysRevResearch.2.033482}{Tuning the orbital angular momentum of high harmonics by manipulating the collinear photon channels in two-color high-harmonic generation}}
\subsubsection*{Zhe Wang, \dots, and Qing Liao (2020-09-24 prresearch)}
This work investigates the characteristics of collinear photon channels in the two-color vortex high-order harmonic generation, and proposes a method to generate the harmonic vortices with well-defined and tunable orbital angular momentum.
\subsection*{\href{http://link.aps.org/doi/10.1103/PhysRevResearch.2.033483}{Stochasticity in radiative polarization of ultrarelativistic electrons in an ultrastrong laser pulse}}
\subsubsection*{Ren-Tong Guo, \dots, and Jian-Xing Li (2020-09-24 prresearch)}
The authors elucidate the impact of stochastic photon emissions on the electron spin dynamics and propose two methods to qualitatively observe the signatures of the stochastic effects of photon emissions with currently achievable laser facilities.
\subsection*{\href{http://link.aps.org/doi/10.1103/PhysRevResearch.2.033485}{Multi-delay complexity collapse}}
\subsubsection*{S. Kamyar Tavakoli and André Longtin (2020-09-24 prresearch)}
The authors show that increasing the number of delays in nonlinear time-delayed dynamical systems can cause a reduction of complexity, using the KS and permutation entropies in the Lang-Kobayashi semiconductor laser model as well as the Mackey-Glass equation.
\subsection*{\href{http://link.aps.org/doi/10.1103/PhysRevResearch.2.033486}{Quantum Lifshitz points and fluctuation-induced first-order phase transitions in imbalanced Fermi mixtures}}
\subsubsection*{Piotr Zdybel and Pawel Jakubczyk (2020-09-24 prresearch)}
This paper discusses the nature of the superfluid quantum phase transition in imbalanced Fermi mixtures, and shows that a quantum Lifshitz point can be obtained by fine-tuning the scattering length for experimentally relevant sets of parameters.
\subsection*{\href{http://link.aps.org/doi/10.1103/PhysRevResearch.2.033487}{Spin-helix-driven insulating phase in two-dimensional lattice}}
\subsubsection*{HaRu K. Park, \dots, and SungBin Lee (2020-09-24 prresearch)}
This paper studies the emergent SU(2) symmetry in a spin-orbit-coupled system and specifically on a two-dimensional lattice, and observe stabilization the magnetic insulator with spiral-like magnetic ordering.
\subsection*{\href{http://link.aps.org/doi/10.1103/PhysRevResearch.2.033489}{On-demand generation of higher-order Fock states in quantum-dot–cavity systems}}
\subsubsection*{M. Cosacchi, \dots, and V. M. Axt (2020-09-24 prresearch)}
The authors explore preparation protocols for higher-order photonic Fock states in solid-state quantum-dot–cavity systems
\subsection*{\href{http://link.aps.org/doi/10.1103/PhysRevResearch.2.033490}{Spatial inhomogeneity and the metal-insulator transition in ${\mathrm{Ca}}_{3}({\mathrm{Ru}}_{1−x}{\mathrm{Ti}}_{x}{)}_{2}{\mathrm{O}}_{7}$}}
\subsubsection*{Frank Lechermann, \dots, and Andrew J. Millis (2020-09-24 prresearch)}
The authors perform a large-scale many-body investigation of Ti-doped Ca3Ru2O7 and show that the metal-to-insulator transition is driven by the interplay of strong electronic correlations and the doping-induced spread of crystal-field levels on the Ru sublattice.
\subsection*{\href{http://link.aps.org/doi/10.1103/PhysRevResearch.2.033468}{Particle flows around an intruder}}
\subsubsection*{Satoshi Takada and Hisao Hayakawa (2020-09-23 prresearch)}
The authors study the the drag force acting on a stationary spherical intruder in particle flows by controlling the ratio of the injected speed of the particles to the thermal speed, using molecular dynamics simulations.
\subsection*{\href{http://link.aps.org/doi/10.1103/PhysRevResearch.2.033470}{Gate-tunable cross-plane heat dissipation in single-layer transition metal dichalcogenides}}
\subsubsection*{Zhun-Yong Ong, \dots, and Linyou Cao (2020-09-23 prresearch)}
This paper develops a theory of electronic thermal boundary conductance mediated by remote phonon scattering for the single-layer transition metal dichalcogenide semiconductors MoS2 and WS2.
\subsection*{\href{http://link.aps.org/doi/10.1103/PhysRevResearch.2.033471}{Observation of a strongly ferromagnetic spinor Bose-Einstein condensate}}
\subsubsection*{SeungJung Huh, \dots, and Jae-yoon Choi (2020-09-23 prresearch)}
The authors show strongly ferromagnetic spinor condensates of 7Li atoms, where the spin interaction energy is comparable to the spin-independent energy.
\subsection*{\href{http://link.aps.org/doi/10.1103/PhysRevResearch.2.033472}{Plunging in the Dirac sea using graphene quantum dots}}
\subsubsection*{François Fillion-Gourdeau, \dots, and Steve MacLean (2020-09-23 prresearch)}
The authors investigate the generation of electron-hole pairs in graphene by the Coulomb potential of a passing electron and show that charge carriers are generated via adiabatic pair production around avoided crossings.
\subsection*{\href{http://link.aps.org/doi/10.1103/PhysRevResearch.2.033473}{Anomalous magnetic anisotropy and magnetic nanostructure in pure Fe induced by high-pressure torsion straining}}
\subsubsection*{Y. Oba, \dots, and H. Mamiya (2020-09-23 prresearch)}
This paper shows the formation of nanosized spin misalignment in pure Fe processed via high-pressure torsion straining.
\subsection*{\href{http://link.aps.org/doi/10.1103/PhysRevResearch.2.033474}{Strong zero-field Förster resonances in K-Rb Rydberg systems}}
\subsubsection*{J. Susanne Otto, \dots, and Amita B. Deb (2020-09-23 prresearch)}
This work reveals strong interspecies Förster resonances between Rydberg-excited atomic rubidium and potassium with ultralong-range zero-field interactions. The paper shows how this may exploited in an optical transistor based on two spatially separated atomic ensembles
\subsection*{\href{http://link.aps.org/doi/10.1103/PhysRevResearch.2.033475}{Fast and robust quantum state transfer in a topological Su-Schrieffer-Heeger chain with next-to-nearest-neighbor interactions}}
\subsubsection*{Felippo M. D'Angelis, \dots, and François Impens (2020-09-23 prresearch)}
This work proposes a method of fast quantum state transfer in topological spin chains based on a dynamical control of Next-to-Nearest Neighbor interactions.
\subsection*{\href{http://link.aps.org/doi/10.1103/PhysRevResearch.2.033476}{Real-space cluster dynamical mean-field theory: Center-focused extrapolation on the one- and two particle-levels}}
\subsubsection*{Marcel Klett, \dots, and Philipp Hansmann (2020-09-23 prresearch)}
This paper uses cellular dynamical mean-field theory for the two-dimensional Hubbard model to propose a cluster center-focused-extrapolation scheme with faster convergence.
\subsection*{\href{http://link.aps.org/doi/10.1103/PhysRevResearch.2.033477}{Dynamics of transposable elements generates structure and symmetries in genetic sequences}}
\subsubsection*{Giampaolo Cristadoro, \dots, and Eduardo G. Altmann (2020-09-23 prresearch)}
The paper shows how the observations of symmetry and structure in DNA sequences can emerge from a dynamical system that models the action of transposal elements.
\subsection*{\href{http://link.aps.org/doi/10.1103/PhysRevResearch.2.033478}{Effective self-similar expansion of a Bose-Einstein condensate: Free space versus confined geometries}}
\subsubsection*{David Viedma and Michele Modugno (2020-09-23 prresearch)}
The authors numerically explore self-similar solutions of three dimensional Bose-Einstein condensates in expansion, both in free space and in waveguide settings.
\subsection*{\href{http://link.aps.org/doi/10.1103/PhysRevA.102.031304}{Formation and dynamics of quantum hydrodynamical breathing-ring solitons}}
\subsubsection*{Samuel N. Alperin and Natalia G. Berloff (2020-09-29 pra)}
We show that exciton-polariton condensates may exhibit a fundamental, self-localized nonlinear excitation in quantum hydrodynamical systems, which takes the form of a dark ring-shaped breather. We predict that these structures form spontaneously and remain stable under a combination of uniform reson...
\subsection*{\href{http://link.aps.org/doi/10.1103/PhysRevA.102.032420}{Robust data encodings for quantum classifiers}}
\subsubsection*{Ryan LaRose and Brian Coyle (2020-09-29 pra)}
Data representation is crucial for the success of machine-learning models. In the context of quantum machine learning with near-term quantum computers, equally important considerations of how to efficiently input (encode) data and effectively deal with noise arise. In this paper, we study data encod...
\subsection*{\href{http://link.aps.org/doi/10.1103/PhysRevA.102.032421}{Interplay between charge and spin thermal entanglement in Hubbard dimers}}
\subsubsection*{Fabiana Souza, \dots, and Maria S. S. Pereira (2020-09-29 pra)}
We study quantum entanglement in half-filled Hubbard dimers at finite temperatures under an external magnetic field. Due to the itinerant nature of the electrons and their fundamental indistinguishability, we employ a site-based evaluation of entanglement via the concurrence in three distinct sector...
\subsection*{\href{http://link.aps.org/doi/10.1103/PhysRevA.102.032627}{Universal nonadiabatic geometric gates protected by dynamical decoupling}}
\subsubsection*{X. Wu and P. Z. Zhao (2020-09-29 pra)}
Dynamical decoupling provides an effective method to protect nonadiabatic geometric quantum computation against environment-induced decoherence. Although some dynamical-decoupling-protected nonadiabatic geometric gates were proposed, they need to combine some other dynamical gates to perform univers...
\subsection*{\href{http://link.aps.org/doi/10.1103/PhysRevA.102.033114}{Electron-rotation coupling in diatomics under strong-field excitation}}
\subsubsection*{Yan Rong Liu, \dots, and Song Bin Zhang (2020-09-29 pra)}
The photoexcitation and photodissociation of diatomic molecules by intense pulse lasers has been the subject of extensive investigations over the past decades. However, the usually employed theoretical framework neglects the coupling between the molecular rotational angular momentum ($\mathbf{R}$) a...
\subsection*{\href{http://link.aps.org/doi/10.1103/PhysRevA.102.033339}{Robust Weyl points in a one-dimensional superlattice with transverse spin-orbit coupling}}
\subsubsection*{Xi-Wang Luo and Chuanwei Zhang (2020-09-29 pra)}
Weyl points, synthetic magnetic monopoles in the three-dimensional momentum space, are the key features of topological Weyl semimetals. The observation of Weyl points in ultracold atomic gases usually relies on the realization of high-dimensional spin-orbit coupling (SOC) for two pseudospin states (...
\subsection*{\href{http://link.aps.org/doi/10.1103/PhysRevA.102.033340}{Thermodynamics and magnetism in the two-dimensional to three-dimensional crossover of the Hubbard model}}
\subsubsection*{Eduardo Ibarra-García-Padilla, \dots, and Richard T. Scalettar (2020-09-29 pra)}
The realization of antiferromagnetic (AF) correlations in ultracold fermionic atoms on an optical lattice is a significant achievement. Experiments have been carried out in one, two, and three dimensions, and have also studied anisotropic configurations with stronger tunneling in some lattice direct...
\subsection*{\href{http://link.aps.org/doi/10.1103/PhysRevA.102.031303}{Coherent control of reactive scattering at low temperatures: Signatures of quantum interference in the differential cross sections for $\mathrm{F}+ {\mathrm{H}}_{2}$ and $\mathrm{F}+\mathrm{HD}$}}
\subsubsection*{Adrien Devolder, \dots, and Paul Brumer (2020-09-28 pra)}
Fundamental entanglement related challenges have prevented quantum-interference-based control (i.e., coherent control) of collisional cross sections from being implemented in the laboratory. Here, differential cross sections for reactive scattering at low temperatures are shown to provide a unique o...
\subsection*{\href{http://link.aps.org/doi/10.1103/PhysRevA.102.032220}{Observation of nonlocality sharing via not-so-weak measurements}}
\subsubsection*{Tianfeng Feng, \dots, and Xiaoqi Zhou (2020-09-28 pra)}
Nonlocality plays a fundamental role in quantum information science. Recently, it has been theoretically predicted and experimentally demonstrated that the nonlocality of an entangled pair may be shared among multiple observers using weak measurements with moderate strength. Here we devise an optima...
\subsection*{\href{http://link.aps.org/doi/10.1103/PhysRevA.102.032221}{Discrete Wigner functions from informationally complete quantum measurements}}
\subsubsection*{John B. DeBrota and Blake C. Stacey (2020-09-28 pra)}
Wigner functions provide a way to do quantum physics using quasiprobabilities, that is, “probability” distributions that can go negative. Informationally complete positive operator-valued measures, a much younger subject than phase-space formulations of quantum mechanics, are less familiar but provi...
\subsection*{\href{http://link.aps.org/doi/10.1103/PhysRevA.102.032626}{Simulated randomized benchmarking of a dynamically corrected cross-resonance gate}}
\subsubsection*{R. K. L. Colmenar, \dots, and J. P. Kestner (2020-09-28 pra)}
We theoretically consider a cross-resonance (CR) gate implemented by pulse sequences proposed by Calderon-Vargas and Kestner [Phys. Rev. Lett. 118, 150502 (2017)]. These sequences mitigate systematic error to first order, but their effectiveness is limited by one-qubit gate imperfections. Using addi...
\subsection*{\href{http://link.aps.org/doi/10.1103/PhysRevA.102.032820}{Interatomic Coulombic decay of a Li dimer in a coupled electron and nuclear dynamics approach}}
\subsubsection*{R. Cabrera-Trujillo, \dots, and L. S. Cederbaum (2020-09-28 pra)}
Interatomic Coulombic decay (ICD) is a fundamental process between atoms or molecules via a neighbor interaction that produces a relaxation of an electronically excited atom or molecule when embedded in an environment. Due to the physical nature of the process, the electronic and nuclear degrees of ...
\subsection*{\href{http://link.aps.org/doi/10.1103/PhysRevA.102.033336}{Linear response of a periodically driven thermal dipolar gas}}
\subsubsection*{Reuben R. W. Wang, \dots, and John L. Bohn (2020-09-28 pra)}
We study the nonequilibrium dynamics of an ultracold, nondegenerate dipolar gas of $^{164}\mathrm{Dy}$ atoms in a cylindrically symmetric harmonic trap. To do so, we investigate the normal modes and linear response of the gas when driven by means of periodic modulations to the trap axial frequency. ...
\subsection*{\href{http://link.aps.org/doi/10.1103/PhysRevA.102.033337}{Local quench spectroscopy of many-body quantum systems}}
\subsubsection*{L. Villa, \dots, and L. Sanchez-Palencia (2020-09-28 pra)}
Quench spectroscopy is a relatively new method which enables the investigation of spectral properties of many-body quantum systems by monitoring the out-of-equilibrium dynamics of real-space observables after a quench. So far the approach has been devised for global quenches or using local engineeri...
\subsection*{\href{http://link.aps.org/doi/10.1103/PhysRevA.102.033338}{Semiclassical dynamics of a disordered two-dimensional Hubbard model with long-range interactions}}
\subsubsection*{Adam S. Sajna and Anatoli Polkovnikov (2020-09-28 pra)}
Quench dynamics in a two-dimensional system of interacting fermions is analyzed within the semiclassical truncated Wigner approximation (TWA). The models with short-range and long-range interactions are considered. We show that in the latter case, the TWA is very accurate, becoming asymptotically ex...
\subsection*{\href{http://link.aps.org/doi/10.1103/PhysRevA.102.033526}{Nonreciprocal enhancement of optomechanical second-order sidebands in a spinning resonator}}
\subsubsection*{Wen-An Li, \dots, and Yuan Chen (2020-09-28 pra)}
We theoretically study optomechanically induced second-order sideband generation in a spinning resonator. Due to the splitting of resonance frequencies of the countercirculating modes via the Sagnac effect, we find that second-order sidebands can be enhanced in one direction but suppressed in the ot...
\subsection*{\href{http://link.aps.org/doi/10.1103/PhysRevA.102.033723}{Control of interference and diffraction of a three-level atom in a double-slit scheme with cavity fields}}
\subsubsection*{Mario Miranda and Miguel Orszag (2020-09-28 pra)}
A double-cavity with a quantum mechanical and a classical field is placed immediately behind a double-slit in order to analyze the wave-particle duality. Both fields have common nodes and antinodes through which a three-level atom passes after crossing the double-slit. The atom-field interaction is ...
\subsection*{\href{http://link.aps.org/doi/10.1103/PhysRevA.102.033724}{Time-resolved detection of photon–surface-plasmon coupling at the single-quanta level}}
\subsubsection*{Chun-Yuan Cheng, \dots, and Chih-Sung Chuu (2020-09-28 pra)}
The interplay of nonclassical light and surface plasmons has attracted considerable attention due to fundamental interests and potential applications. To gain more insight into the quantum nature of the photon–surface-plasmon coupling, time-resolved detection of the interaction is invaluable. Here w...
\subsection*{\href{http://link.aps.org/doi/10.1103/PhysRevA.102.033725}{Transmission spectra of bistable systems: From the ultraquantum to the classical regime}}
\subsubsection*{Evgeny V. Anikin, \dots, and Igor M. Sokolov (2020-09-28 pra)}
We present an analytical and numerical study of the fluorescence spectra of a bistable driven system by means of the Keldysh diagram technique in pseudoparticle representation. The spectra exhibit smooth transition between the ultraquantum and the quasiclassical limits and indicate the threshold val...
\subsection*{\href{http://link.aps.org/doi/10.1103/PhysRevA.102.033726}{Quantum limited source localization and pair superresolution in two dimensions under finite-emission bandwidth}}
\subsubsection*{Sudhakar Prasad (2020-09-28 pra)}
Optically localizing a single quasimonochromatic source to subdiffractive precisions entails, in the photon-counting limit, a minimum photon cost that scales as the squared ratio of the width $w$ of the optical system's point-spread function (PSF) and the sought localization precision, $d$, i.e., as...
\subsection*{\href{http://link.aps.org/doi/10.1103/PhysRevA.102.033727}{Enantiomeric-excess determination based on nonreciprocal-transition-induced spectral-line elimination}}
\subsubsection*{Xun-Wei Xu, \dots, and Ai-Xi Chen (2020-09-28 pra)}
The spontaneous emission spectrum of a multilevel atom or molecule with nonreciprocal transition is investigated. It is shown that the nonreciprocal transition can lead to the elimination of a spectral line in the spontaneous emission spectrum. As an application, we show that nonreciprocal transitio...
\subsection*{\href{http://link.aps.org/doi/10.1103/PhysRevA.102.033728}{Absorption and delayed reemission in an array of atoms strongly coupled to a waveguide}}
\subsubsection*{Mingxia Huo and Ying Li (2020-09-28 pra)}
In the system of a waveguide coupled with an array of atoms, we study the absorption and delayed reemission of light. By adjusting energy levels of atoms, we can engineer a dissipation channel of the atomic collective mode coupled to the waveguide, thereby realizing the high-fidelity absorption and ...
\subsection*{\href{http://link.aps.org/doi/10.1103/PhysRevA.102.032219}{Quantum transport on generalized scale-free networks}}
\subsubsection*{Cássio Macêdo Maciel, \dots, and Mircea Galiceanu (2020-09-25 pra)}
We consider quantum transport on generalized scale-free networks (GSFNs) in the continuous-time quantum walk (CTQW) model. The efficiency of the transport is monitored through the exact and the average return probabilities. In this model these probabilities are fully determined by the eigenvalues an...
\subsection*{\href{http://link.aps.org/doi/10.1103/PhysRevA.102.032419}{Maximal-value condition of coherence measures holds for mixed states if and only if it does for pure states}}
\subsubsection*{Xiao-Dan Cui, \dots, and D. M. Tong (2020-09-25 pra)}
While various coherence measures based on the framework for quantifying coherence [T. Baumgratz, M. Cramer, and M. B. Plenio, Phys. Rev. Lett. 113, 140401 (2014)] have been proposed, it is often asked whether a coherence measure fulfills the maximal value condition that only maximally coherent state...
\subsection*{\href{http://link.aps.org/doi/10.1103/PhysRevA.102.032624}{Driven spin chains as high-quality quantum routers}}
\subsubsection*{Darvin Wanisch and Stephan Fritzsche (2020-09-25 pra)}
We propose a setup, based on a periodically driven spin chain, that can realize a high-quality quantum router. We present two protocols, which utilize this setup, that can either generate highly entangled two-qubit states over an arbitrary distance or transfer single-qubit states with high fidelity ...
\subsection*{\href{http://link.aps.org/doi/10.1103/PhysRevA.102.032625}{Continuous-variable quantum key distribution under strong channel polarization disturbance}}
\subsubsection*{Wenyuan Liu, \dots, and Yongmin Li (2020-09-25 pra)}
In a commercial fiber-based quantum key distribution (QKD) system, the state of polarization (SOP) of the optical fields is inevitably disturbed by random birefringence of the standard single-mode fiber due to an external complex environment. We analyze theoretically the effect of SOP fluctuations o...
\subsection*{\href{http://link.aps.org/doi/10.1103/PhysRevA.102.033113}{Phase-matching analysis in high-order harmonic generation with nonzero orbital angular momentum Laguerre-Gaussian beams}}
\subsubsection*{Cheng Jin, \dots, and C. D. Lin (2020-09-25 pra)}
Through high-order harmonic generation driven by intense ultrashort vortex infrared or midinfrared lasers, a nonzero orbital angular momentum can be imprinted onto extreme ultraviolet (XUV) or soft-x-ray (SXR) light pulses. Here we simulate the generation of vortex XUV harmonics in the gas medium as...
\subsection*{\href{http://link.aps.org/doi/10.1103/PhysRevA.102.033334}{Nonlinear two-photon Rabi-Hubbard model: Superradiance, photon, and photon-pair Bose-Einstein condensates}}
\subsubsection*{Shifeng Cui, \dots, and G. G. Batrouni (2020-09-25 pra)}
We study the ground-state phase diagram of a nonlinear two-photon Rabi-Hubbard (RH) model in one dimension using quantum Monte Carlo simulations and density-matrix renormalization-group calculations. Our model includes a nonlinear photon-photon interaction term. Absent this term, the RH model has on...
\subsection*{\href{http://link.aps.org/doi/10.1103/PhysRevA.102.033335}{Towards a quantum Monte Carlo–based density functional including finite-range effects: Excitation modes of a $^{39}\mathrm{K}$ quantum droplet}}
\subsubsection*{V. Cikojević, \dots, and J. Boronat (2020-09-25 pra)}
Some discrepancies between experimental results on quantum droplets made of a mixture of $^{39}\mathrm{K}$ atoms in different hyperfine states and their analysis within extended Gross-Pitaevskii theory (which incorporates beyond mean-field corrections) have been recently solved by introducing finite...
\subsection*{\href{http://link.aps.org/doi/10.1103/PhysRevA.102.033722}{Postselection-free, hyperentangled photon pairs in a periodically poled lithium-niobate ridge waveguide}}
\subsubsection*{Ramesh Kumar, \dots, and Joyee Ghosh (2020-09-25 pra)}
In this paper, we propose the generation of hyperentangled photon pairs using type-II spontaneous parametric down-conversion in a biperiod, 5\% MgO-doped lithium-niobate ridge waveguide. The photon pairs are entangled in spatial mode and polarization degrees of freedom. A pulsed laser source at 687 n...
\subsection*{\href{http://link.aps.org/doi/10.1103/PhysRevA.102.031701}{Collective dipole-dipole interactions in planar nanocavities}}
\subsubsection*{Helge Dobbertin, \dots, and Stefan Scheel (2020-09-24 pra)}
A generalized microscopic interaction model that takes into account the noncollisional atom-atom and atom-wall interaction is presented. The model reveals density-dependent line shifts and broadenings and tells us how to control such effects with the coatings used on nanocavities.
\subsection*{\href{http://link.aps.org/doi/10.1103/PhysRevA.102.032417}{Algorithm for tailoring a quadratic lattice with a local squeezed reservoir to stabilize generic chiral states with nonlocal entanglement}}
\subsubsection*{Yariv Yanay (2020-09-24 pra)}
We demonstrate an approach to the generation of custom entangled many-body states through reservoir engineering, using the symmetry properties of bosonic lattice systems coupled to a local squeezed reservoir [Yanay and Clerk, Phys. Rev. A 98, 043615 (2018)]. We outline an algorithm where, beginning ...
\subsection*{\href{http://link.aps.org/doi/10.1103/PhysRevA.102.032418}{Classical simulation of noncontextual Pauli Hamiltonians}}
\subsubsection*{William M. Kirby and Peter J. Love (2020-09-24 pra)}
Noncontextual Pauli Hamiltonians decompose into sets of Pauli terms to which joint values may be assigned without contradiction. We construct a quasiquantized model for noncontextual Pauli Hamiltonians. Using this model, we give an algorithm to classically simulate the noncontextual variational quan...
\subsection*{\href{http://link.aps.org/doi/10.1103/PhysRevA.102.032818}{Ejected-electron-energy and angular dependence of fully differential ionization cross sections in medium-velocity proton collisions with He and ${\mathrm{H}}_{2}$}}
\subsubsection*{M. Dhital, \dots, and M. Schulz (2020-09-24 pra)}
We have measured fully momentum-analyzed recoiling target ions and scattered projectiles, produced in ionization of He and ${\mathrm{H}}_{2}$ by 75 keV proton impact, in coincidence. The momentum of the ejected electrons was deduced from momentum conservation. From the data we extracted fully differ...
\subsection*{\href{http://link.aps.org/doi/10.1103/PhysRevA.102.032819}{Effect of the orientation of Rydberg atoms on their collisional ionization cross section}}
\subsubsection*{Akilesh Venkatesh and Francis Robicheaux (2020-09-24 pra)}
Collisional ionization between two Rydberg atoms in relative motion is examined. A classical trajectory Monte Carlo method is used to determine the cross sections associated with Penning ionization. The dependence of the ionization cross section on the magnitude and the direction of orbital angular ...
\subsection*{\href{http://link.aps.org/doi/10.1103/PhysRevA.102.033333}{Cold-atom quantum simulator to explore pairing, condensation, and pseudogaps in extended Hubbard-Holstein models}}
\subsubsection*{J. P. Hague, \dots, and C. MacCormick (2020-09-24 pra)}
We describe a quantum simulator for the Hubbard-Holstein model (HHM), comprising two dressed Rydberg atom species held in a monolayer by independent painted potentials, predicting that boson-mediated preformed pairing and Berezinskii-Kosterlitz-Thouless (BKT) transition temperatures are experimental...
\subsection*{\href{http://link.aps.org/doi/10.1103/PhysRevA.102.033524}{Revealing the modal content of obstructed beams}}
\subsubsection*{Jonathan Pinnell, \dots, and Abdelhalim Bencheikh (2020-09-24 pra)}
In this paper, we propose a predictor or indicator of the self-healing ability of coherent structured light beams: the field's modal content. Specifically, the fidelity between the obstructed and unobstructed beams' modal spectrum serves as a useful measure of the degree to which the beam will likel...
\subsection*{\href{http://link.aps.org/doi/10.1103/PhysRevA.102.033525}{Orbital angular momentum dichroism caused by the interaction of electric and magnetic dipole moments and the geometrical asymmetry of chiral metal nanoparticles}}
\subsubsection*{Yangzhe Guo, \dots, and Yurui Fang (2020-09-24 pra)}
Circular dichroism (CD) caused by the response of a chiral object to circularly polarized light has been well established, and the strong CD of plasmonic metamolecules has also become of interest in recent years; however, their response if the light also has orbital angular momentum is unclear. In t...
\subsection*{\href{http://link.aps.org/doi/10.1103/PhysRevA.102.032218}{Nature of the nonequilibrium phase transition in the non-Markovian driven Dicke model}}
\subsubsection*{Rex Lundgren, \dots, and Mohammad F. Maghrebi (2020-09-23 pra)}
The Dicke model famously exhibits a phase transition to a superradiant phase with a macroscopic population of photons and is realized in multiple settings in open quantum systems. In this paper, we study a variant of the Dicke model where the cavity mode is lossy due to the coupling to a Markovian e...
\subsection*{\href{http://link.aps.org/doi/10.1103/PhysRevA.102.032416}{$α$-logarithmic negativity}}
\subsubsection*{Xin Wang and Mark M. Wilde (2020-09-23 pra)}
The logarithmic negativity of a bipartite quantum state is a widely employed entanglement measure in quantum information theory due to the fact that it is easy to compute and serves as an upper bound on distillable entanglement. More recently, the $κ$ entanglement of a bipartite state was shown to b...
\subsection*{\href{http://link.aps.org/doi/10.1103/PhysRevA.102.033332}{Stückelberg interferometry using spin-orbit-coupled cold atoms in an optical lattice}}
\subsubsection*{Shuang Liang, \dots, and Zhihao Lan (2020-09-23 pra)}
Time evolution of spin-orbit-coupled cold atoms in an optical lattice is studied, with a two-band energy spectrum having two avoided crossings. A force is applied such that the atoms experience two consecutive Landau-Zener tunnelings while transversing the avoided crossings. Stückelberg interference...
\subsection*{\href{http://link.aps.org/doi/10.1103/PhysRevA.102.033522}{General laws of the propagation of few-cycle optical pulses with orbital angular momentum in free space}}
\subsubsection*{Miguel A. Porras and Raúl García-Álvarez (2020-09-23 pra)}
We conduct a theoretical study of the propagation of few-cycle ultrafast vortices (UFVs) carrying orbital angular momentum (OAM) in free space. Our analysis reveals much more complex temporal dynamics than that of few-cycle fundamental Gaussian-like beams, particularly when approaching the single-cy...
\subsection*{\href{http://link.aps.org/doi/10.1103/PhysRevA.102.033523}{Observation and analysis of creation, decay, and regeneration of annular soliton clusters in a lossy cubic-quintic optical medium}}
\subsubsection*{Albert S. Reyna, \dots, and Cid B. de Araújo (2020-09-23 pra)}
We observe and analyze formation, decay, and subsequent regeneration of ring-shaped clusters of (2+1)-dimensional spatial solitons (filaments) in a medium with cubic-quintic (focusing-defocusing) self-interaction and strong dissipative nonlinearity. The cluster of filaments, which remains stable ove...
\subsection*{\href{http://link.aps.org/doi/10.1103/PhysRevA.102.033720}{Nonequilibrium diagrammatic approach to strongly interacting photons}}
\subsubsection*{Johannes Lang, \dots, and Francesco Piazza (2020-09-23 pra)}
We develop a nonequilibrium field-theoretical approach based on a systematic diagrammatic expansion for strongly interacting photons in optically dense atomic media. We consider the case where the characteristic photon-propagation range ${L}_{P}$ is much larger than the interatomic spacing $a$ and w...
\subsection*{\href{http://link.aps.org/doi/10.1103/PhysRevA.102.033721}{Tunable multiwindow magnomechanically induced transparency, Fano resonances, and slow-to-fast light conversion}}
\subsubsection*{Kamran Ullah, \dots, and Özgür E. Müstecaplıoğlu (2020-09-23 pra)}
We investigate the absorption and transmission properties of a weak probe field under the influence of a strong control field in a cavity magnomechanical system. The system consists of two ferromagnetic-material yttrium iron garnet (YIG) spheres coupled to a single cavity mode. In addition to two ma...
\subsection*{\href{http://link.aps.org/doi/10.1103/PhysRevB.102.094110}{Doping quantum materials: Defects and impurities in $\mathrm{Fe}{\mathrm{Ga}}_{3}$}}
\subsubsection*{J. C. Alvarez-Quiceno, \dots, and G. M. Dalpian (2020-09-29 prb)}
Defects and impurities strongly dictate the physical properties of materials by tuning conductivity, modulating magnetic response, and changing optical properties. Here we analyze how these entities behave in the prototype quantum material ${\mathrm{FeGa}}_{3}$. We have studied its intrinsic defects...
\subsection*{\href{http://link.aps.org/doi/10.1103/PhysRevB.102.094206}{Success and breakdown of the T-matrix approximation for phonon-disorder scattering}}
\subsubsection*{S. Thébaud, \dots, and T. Berlijn (2020-09-29 prb)}
We examine the validity of the widely used T-matrix approximation for treating phonon-disorder scattering by implementing an unfolding algorithm that allows simulation of disorder up to tens of millions of atoms. The T-matrix approximation breaks down for low-energy flexure phonons that play an impo...
\subsection*{\href{http://link.aps.org/doi/10.1103/PhysRevB.102.094315}{Time-dependent variational principle with ancillary Krylov subspace}}
\subsubsection*{Mingru Yang and Steven R. White (2020-09-29 prb)}
We propose an improved scheme to do the time-dependent variational principle (TDVP) in finite matrix product states (MPSs) for two-dimensional systems or one-dimensional systems with long-range interactions. We present a method to represent the time-evolving state in a MPS with its basis enriched by...
\subsection*{\href{http://link.aps.org/doi/10.1103/PhysRevB.102.100103}{Formation of nanoscale polarized clusters as precursors of electronic ferroelectricity probed by conductance noise spectroscopy}}
\subsubsection*{Jens Müller, \dots, and Takahiko Sasaki (2020-09-29 prb)}
We investigate the low-frequency charge-carrier dynamics of a molecular dimer-Mott insulator ${β}^{′}−{(\mathrm{BEDT}−\mathrm{TTF})}_{2}{\mathrm{ICl}}_{2}$, where the freezing of charge fluctuations on the dimers gives rise to electronic ferroelectricity. We show that conductance fluctuation (noise)...
\subsection*{\href{http://link.aps.org/doi/10.1103/PhysRevB.102.100508}{Electronic inhomogeneity and band structure in superstructural ${\mathrm{CuO}}_{2}$ planes of infinite-layer ${\mathrm{Sr}}_{0.94}{\mathrm{La}}_{0.06}{\mathrm{CuO}}_{2+y}$ films}}
\subsubsection*{Rui-Feng Wang, \dots, and Qi-Kun Xue (2020-09-29 prb)}
Too many Dollar signs...
\subsection*{\href{http://link.aps.org/doi/10.1103/PhysRevB.102.104116}{Detonation-induced transformation of graphite to hexagonal diamond}}
\subsubsection*{Elissaios Stavrou, \dots, and Trevor M. Willey (2020-09-29 prb)}
We explore the structural evolution of highly oriented pyrolytic graphite (HOPG) under detonation-induced shock conditions using in situ synchrotron x-ray diffraction in the ns timescale. We observe the formation of hexagonal diamond (lonsdaleite) at pressures above 50 GPa, in qualitative agreement ...
\subsection*{\href{http://link.aps.org/doi/10.1103/PhysRevB.102.104117}{Domain wall generated polarity in ferroelastics: Results from resonance piezoelectric spectroscopy, piezoelectric force microscopy, and optical second harmonic generation measurements in $\mathrm{La}\mathrm{Al}{\mathrm{O}}_{3}$ with twin and tweed microstructures}}
\subsubsection*{Hiroko Yokota, \dots, and Yoshiaki Uesu (2020-09-29 prb)}
Ferroelastic $\mathrm{La}\mathrm{Al}{\mathrm{O}}_{3}$ (space group $R\overline{3}c$) exists with two different microstructures: twins and tweed. Both microstructures contain electrical dipole moments. Polarity inside ferroelastic twin walls has been shown using two complementary experimental techniq...
\subsection*{\href{http://link.aps.org/doi/10.1103/PhysRevB.102.104514}{Spin magnetometry as a probe of stripe superconductivity in twisted bilayer graphene}}
\subsubsection*{E. J. König, \dots, and A. M. Tsvelik (2020-09-29 prb)}
The discovery of alternating superconducting and insulating ground states in magic angle graphene has suggested an intriguing analogy with cuprate high-${T}_{c}$ materials. Here we argue that the network states of small angle twisted bilayer graphene (TBG) afford a further perspective on the cuprate...
\subsection*{\href{http://link.aps.org/doi/10.1103/PhysRevB.102.115161}{Crossover between weak antilocalization and weak localization in few-layer $\mathrm{W}{\mathrm{Te}}_{2}$: Role of electron-electron interactions}}
\subsubsection*{Xurui Zhang, \dots, and Xiaoyan Shi (2020-09-29 prb)}
We report electron transport studies in an encapsulated few-layer $\mathrm{W}{\mathrm{Te}}_{2}$ at low temperatures and high magnetic fields. The magnetoconductance reveals a temperature-induced crossover between weak antilocalization and weak localization in the quantum diffusive regime. We show th...
\subsection*{\href{http://link.aps.org/doi/10.1103/PhysRevB.102.115309}{Polarity-field driven conductivity in ${\mathrm{SrTiO}}_{3}/{\mathrm{LaAlO}}_{3}$: A hybrid functional study}}
\subsubsection*{Sébastien Lemal, \dots, and Philippe Ghosez (2020-09-29 prb)}
Too many Dollar signs...
\subsection*{\href{http://link.aps.org/doi/10.1103/PhysRevB.102.121109}{Meta-GGA performance in solids at almost GGA cost}}
\subsubsection*{Daniel Mejía-Rodríguez and S. B. Trickey (2020-09-29 prb)}
A recent modification, $\mathrm{r}^{2}\mathrm{SCAN}$, of the strongly constrained and appropriately normed (SCAN) meta-generalized gradient approximation (GGA) exchange-correlation functional mostly eliminates numerical instabilities and attendant integration grid sensitivities exhibited by SCAN. He...
\subsection*{\href{http://link.aps.org/doi/10.1103/PhysRevB.102.125204}{Localized dimers drive strong anharmonicity and low lattice thermal conductivity in $\mathrm{Zn}{\mathrm{Se}}_{2}$}}
\subsubsection*{Tiantian Jia, \dots, and Georg K. H. Madsen (2020-09-29 prb)}
We calculate the lattice thermal conductivities of the pyrite-type $\mathrm{Zn}{\mathrm{Se}}_{2}$ at pressures of 0 and 10 GPa by using the linearized phonon Boltzmann transport equation. We obtain a very low value (0.69 W/mK at room temperature at 0 GPa), comparable to the best thermoelectric mater...
\subsection*{\href{http://link.aps.org/doi/10.1103/PhysRevB.102.125310}{Origin of the hysteresis of magnetoconductance in a supramolecular spin-valve based on a $\mathrm{Tb}{\mathrm{Pc}}_{2}$ single-molecule magnet}}
\subsubsection*{Kieran Hymas and Alessandro Soncini (2020-09-29 prb)}
We present a time-dependent microscopic model for Coulomb blockade transport through an experimentally realized supramolecular spin-valve device driven by an oscillating magnetic field, in which the $4f$-electron magnetic states of an array of $\mathrm{Tb}{\mathrm{Pc}}_{2}$ single-molecule magnets (...
\subsection*{\href{http://link.aps.org/doi/10.1103/PhysRevB.102.094205}{Invariance of the relation between α relaxation and β relaxation in metallic glasses to variations of pressure and temperature}}
\subsubsection*{B. Wang, \dots, and K. L. Ngai (2020-09-28 prb)}
Dielectric relaxation experiments performed at ambient and elevated pressures P in molecular, ionic, and polymeric glass formers have established that the relation of the Johari-Goldstein (JG) β-relaxation time ${τ}_{β}(T,P)$ to the α-relaxation time ${τ}_{α}(T,P)$ is invariant to changes of T and P...
\subsection*{\href{http://link.aps.org/doi/10.1103/PhysRevB.102.094312}{Effective medium theory for a photonic pseudospin-$\frac{1}{2}$ system}}
\subsubsection*{Neng Wang, \dots, and Guo Ping Wang (2020-09-28 prb)}
Photonic pseudospin-$\frac{1}{2}$ systems which exhibit Dirac cone dispersion at Brillouin zone corners in analogy to graphene, have been extensively studied in recent years. However, it is known that a linear band crossing of two bands cannot emerge at the center of a Brillouin zone in a two-dimens...
\subsection*{\href{http://link.aps.org/doi/10.1103/PhysRevB.102.094313}{Selective imaging of the terahertz electric field of the phonon-polariton in ${\mathrm{LiNbO}}_{3}$}}
\subsubsection*{Keita Matsumoto and Takuya Satoh (2020-09-28 prb)}
Coherent phonon-polaritons have attracted a considerable amount of interest owing to their relevance to nonlinear optics and terahertz (THz)-wave emissions. Therefore, it is important to analyze the THz electric fields of phonon-polaritons. However, in the majority of previous measurements, only a s...
\subsection*{\href{http://link.aps.org/doi/10.1103/PhysRevB.102.094314}{Anomalous electronic and thermoelectric transport properties in cubic ${\mathrm{Rb}}_{3}\mathrm{AuO}$ antiperovskite}}
\subsubsection*{Yinchang Zhao, \dots, and Jun Ni (2020-09-28 prb)}
We use first-principles calculations combined with self-consistent phonon (SCP) theory, electron-phonon ($e−ph$) coupling, and the Boltzmann transport equation (BTE) to investigate the electronic and thermoelectric transport properties in the cubic ${\mathrm{Rb}}_{3}\mathrm{AuO}$ antiperovskite with...
\subsection*{\href{http://link.aps.org/doi/10.1103/PhysRevB.102.094435}{Relationship between magnetic ordering and gigantic magnetocaloric effect in ${\mathrm{HoB}}_{2}$ studied by neutron diffraction experiment}}
\subsubsection*{Noriki Terada, \dots, and Hideaki Kitazawa (2020-09-28 prb)}
Too many Dollar signs...
\subsection*{\href{http://link.aps.org/doi/10.1103/PhysRevB.102.094436}{Experimental exploration of the vector nature of the dynamic order parameter near dynamic magnetic phase transitions}}
\subsubsection*{M. Quintana, \dots, and A. Berger (2020-09-28 prb)}
We have devised an experimental study to explore the vector nature of the dynamic order parameter       $Q$   and the corresponding dynamic fluctuation ${{σ}}^{Q}$ and susceptibility ${{χ}}^{Q}$ tensors in the vicinity of the dynamic phase transition (DPT). For this purpose, we have ...
\subsection*{\href{http://link.aps.org/doi/10.1103/PhysRevB.102.094437}{Microscopic theory of spin Hall magnetoresistance}}
\subsubsection*{T. Kato, \dots, and M. Matsuo (2020-09-28 prb)}
We consider a microscopic theory for the spin Hall magnetoresistance (SMR). We generally formulate a spin conductance at an interface between a normal metal and a magnetic insulator in terms of spin susceptibilities. We reveal that SMR is composed of static and dynamic parts. The static part, which ...
\subsection*{\href{http://link.aps.org/doi/10.1103/PhysRevB.102.094438}{Spin Hall effect in the $α$ and $β$ phases of $\mathrm{T}{\mathrm{a}}_{x}{\mathrm{W}}_{1\text{−}x}$ alloys}}
\subsubsection*{Lijuan Qian, \dots, and Gang Xiao (2020-09-28 prb)}
Too many Dollar signs...
\subsection*{\href{http://link.aps.org/doi/10.1103/PhysRevB.102.094519}{Charge density wave and superconductivity competition in ${\mathrm{Lu}}_{5}{\mathrm{Ir}}_{4}{\mathrm{Si}}_{10}$: A proton irradiation study}}
\subsubsection*{Maxime Leroux, \dots, and Ulrich Welp (2020-09-28 prb)}
Real-space modulated charge density waves (CDW) are a ubiquitous feature in many families of superconductors. In particular, how CDW relates to superconductivity is an active and open question that has recently gathered much interest since CDWs have been discovered in many cuprates superconductors. ...
\subsection*{\href{http://link.aps.org/doi/10.1103/PhysRevB.102.100203}{Wannier band transitions in disordered $π$-flux ladders}}
\subsubsection*{Jahan Claes and Taylor L. Hughes (2020-09-28 prb)}
Boundary obstructed topological insulators are an unusual class of higher-order topological insulators with topological characteristics determined by the so-called Wannier bands. Boundary obstructed phases can harbor hinge/corner modes, but these modes can often be destabilized by a phase transition...
\subsection*{\href{http://link.aps.org/doi/10.1103/PhysRevB.102.100505}{Spontaneous strain and magnetization in doped topological insulators with nematic and chiral superconductivity}}
\subsubsection*{R. S. Akzyanov, \dots, and A. L. Rakhmanov (2020-09-28 prb)}
We show that spontaneous strain and magnetization can arise in the doped topological insulators with a two-component superconducting vector order parameter. The details of the effects depend on the symmetry of the order parameter, whether it is nematic or chiral. The transition from the nematic stat...
\subsection*{\href{http://link.aps.org/doi/10.1103/PhysRevB.102.100506}{Spin accumulation induced by a singlet supercurrent}}
\subsubsection*{Morten Amundsen and Jacob Linder (2020-09-28 prb)}
We show that a supercurrent carried by spinless singlet Cooper pairs can induce a spin accumulation in the normal metal interlayer of a Josephson junction. This phenomenon occurs when a nonequilibrium spin-energy mode is excited in the normal metal, for instance, by an applied temperature gradient b...
\subsection*{\href{http://link.aps.org/doi/10.1103/PhysRevB.102.100507}{Odd triplet superconductivity induced by a moving condensate}}
\subsubsection*{M. A. Silaev, \dots, and A. M. Bobkov (2020-09-28 prb)}
It has been commonly accepted that a magnetic field suppresses superconductivity by inducing the ordered motion of Cooper pairs. We demonstrate that a magnetic field can instead provide a generation of superconducting correlations by inducing the motion of a superconducting condensate. This effect a...
\subsection*{\href{http://link.aps.org/doi/10.1103/PhysRevB.102.104115}{${\mathrm{Ca}}_{2}\mathrm{Mn}{\mathrm{O}}_{4}$ structural path: Following the negative thermal expansion at the local scale}}
\subsubsection*{Pedro Rocha-Rodrigues, \dots, and Armandina M. L. Lopes (2020-09-28 prb)}
The oxygen octahedral rotations in ${\mathrm{Ca}}_{2}\mathrm{Mn}{\mathrm{O}}_{4}$, the first member of the $\mathrm{Ca}\mathrm{O}{({\mathrm{CaMnO}}_{3})}_{n}$ Ruddlesden-Popper family, is probed through a set of complementary techniques, including temperature-dependent neutron and x-ray diffraction,...
\subsection*{\href{http://link.aps.org/doi/10.1103/PhysRevB.102.104308}{Role of intraband dynamics in the generation of circularly polarized high harmonics from solids}}
\subsubsection*{N. Klemke, \dots, and N. Tancogne-Dejean (2020-09-28 prb)}
Recent studies have demonstrated that the polarization states of high harmonics from solids can differ from those of the driving pulses. To gain insights on the microscopic origin of this behavior, we perform one-particle intraband-only calculations and reproduce some of the most striking observatio...
\subsection*{\href{http://link.aps.org/doi/10.1103/PhysRevB.102.104309}{Resonant-amplified and invisible Bragg scattering based on spin coalescing modes}}
\subsubsection*{K. L. Zhang and Z. Song (2020-09-28 prb)}
Unlike a real magnetic field, which separates the energy levels of a particle with opposite spin polarization, a complex field can lead to a special kind of spectral degeneracy, known as exceptional point (EP), at which two spin eigenmodes coalesce. It allows an EP impurity to be an invisible scatte...
\subsection*{\href{http://link.aps.org/doi/10.1103/PhysRevB.102.104433}{Field evolution of the spin-liquid candidate $\mathrm{Yb}\mathrm{Mg}\mathrm{Ga}{\mathrm{O}}_{4}$}}
\subsubsection*{Sebastian Bachus, \dots, and Alexander A. Tsirlin (2020-09-28 prb)}
We report magnetization, heat capacity, thermal expansion, and magnetostriction measurements down to millikelvin temperatures on the triangular antiferromagnet $\mathrm{Yb}\mathrm{Mg}\mathrm{Ga}{\mathrm{O}}_{4}$. Our data exclude the formation of the distinct $\frac{1}{3}$ plateau phase observed in ...
\subsection*{\href{http://link.aps.org/doi/10.1103/PhysRevB.102.104513}{Superconductivity of underdoped PrFeAs(O,F) investigated via point-contact spectroscopy and nuclear magnetic resonance}}
\subsubsection*{D. Daghero, \dots, and T. Shiroka (2020-09-28 prb)}
Underdoped PrFeAs(O,F), one of the lesser known members of the 1111 family of iron-based superconductors, was investigated in detail by means of transport, magnetometry, nuclear magnetic resonance (NMR) measurements, and point-contact Andreev-reflection spectroscopy (PCARS). PCARS measurements on si...
\subsection*{\href{http://link.aps.org/doi/10.1103/PhysRevB.102.115157}{Comparing the generalized Kadanoff-Baym ansatz with the full Kadanoff-Baym equations for an excitonic insulator out of equilibrium}}
\subsubsection*{Riku Tuovinen, \dots, and Michael A. Sentef (2020-09-28 prb)}
We investigate out-of-equilibrium dynamics in an excitonic insulator (EI) with a finite-momentum pairing perturbed by a laser-pulse excitation and a sudden coupling to fermionic baths. The transient dynamics of the excitonic order parameter is resolved using the full nonequilibrium Green's function ...
\subsection*{\href{http://link.aps.org/doi/10.1103/PhysRevB.102.115158}{Extremely large magnetoresistance, anisotropic Hall effect, and Fermi surface topology in single-crystalline $\mathrm{W}{\mathrm{Si}}_{2}$}}
\subsubsection*{Rajib Mondal, \dots, and Arumugam Thamizhavel (2020-09-28 prb)}
We report on the observation of a nonsaturating, extremely large magnetoresistance (XMR) and the Fermi surface topology of a high quality $\mathrm{W}{\mathrm{Si}}_{2}$ single crystal grown by the Czochralski method. The magnetoresistance at $T=2\phantom{\rule{0.16em}{0ex}}\mathrm{K}$ reaches a value...
\subsection*{\href{http://link.aps.org/doi/10.1103/PhysRevB.102.115159}{Charge density waves in Weyl semimetals}}
\subsubsection*{Dan Sehayek, \dots, and A. A. Burkov (2020-09-28 prb)}
We present a theory of charge density wave (CDW) states in Weyl semimetals and their interplay with the chiral anomaly. In particular, we demonstrate a special nature of the shortest-period CDW state, which is obtained when the separation between the Weyl nodes equals exactly half a primitive recipr...
\subsection*{\href{http://link.aps.org/doi/10.1103/PhysRevB.102.115160}{Susceptibility anisotropy and its disorder evolution in models for Kitaev materials}}
\subsubsection*{Eric C. Andrade, \dots, and Matthias Vojta (2020-09-28 prb)}
Mott insulators with strong spin-orbit coupling display a strongly anisotropic response to applied magnetic fields. This applies in particular to Kitaev materials, with $α−{\mathrm{RuCl}}_{3}$ and ${\mathrm{Na}}_{2}{\mathrm{IrO}}_{3}$ representing two important examples. Both show a magnetically ord...
\subsection*{\href{http://link.aps.org/doi/10.1103/PhysRevB.102.115308}{Generalized elastodynamic model for nanophotonics}}
\subsubsection*{J. V. Alvarez, \dots, and Dani Torrent (2020-09-28 prb)}
A self-consistent theory for the classical description of the interaction of light and matter at the nanoscale is presented, which takes into account spatial dispersion. Up to now, the Maxwell equations in nanostructured materials with spatial dispersion have been solved by the introduction of the s...
\subsection*{\href{http://link.aps.org/doi/10.1103/PhysRevB.102.115435}{Optically induced nonreciprocity by a plasmonic pump in semiconductor wires}}
\subsubsection*{Kil-Song Song, \dots, and Yong-Ha Han (2020-09-28 prb)}
In most studies on all-optical diodes, spatial asymmetry has been by necessity applied to break Lorentz reciprocity. Here we suggest a paradigm for optically induced nonreciprocity in semiconductor wires which are spatially asymmetry-free and provide a very simple and efficient platform for plasmoni...
\subsection*{\href{http://link.aps.org/doi/10.1103/PhysRevB.102.115436}{Effects of magnetic fields on the Datta-Das spin field-effect transistor}}
\subsubsection*{K. Sarkar, \dots, and R. I. Shekhter (2020-09-28 prb)}
A Datta-Das spin field-effect transistor is built of a heterostructure with a Rashba spin-orbit interaction (SOI) at the interface (or quantum well) separating two possibly magnetized reservoirs. The particle and spin currents between the two reservoirs are driven by chemical potentials that are (po...
\subsection*{\href{http://link.aps.org/doi/10.1103/PhysRevB.102.121305}{Microscopic details of stripes and bubbles in the quantum Hall regime}}
\subsubsection*{Josef Oswald and Rudolf A. Römer (2020-09-28 prb)}
We use a fully self-consistent laterally resolved Hartree-Fock approximation for numerically addressing the electron configurations at higher Landau levels in the quantum Hall regime for near-macroscopic sample sizes. At low disorder we find spatially resolved, stripe- and bubblelike charge-density ...
\subsection*{\href{http://link.aps.org/doi/10.1103/PhysRevB.102.121408}{Two-dimensional chiral stacking orders in quasi-one-dimensional charge density waves}}
\subsubsection*{Sun-Woo Kim, \dots, and Tae-Hwan Kim (2020-09-28 prb)}
Chirality manifests in various forms in nature. However, there is no evidence of the chirality in one-dimensional charge density wave (CDW) systems. Here, we have explored the chirality among quasi-one-dimensional CDW ground states with the aid of scanning tunneling microscopy, symmetry analysis, an...
\subsection*{\href{http://link.aps.org/doi/10.1103/PhysRevB.102.125149}{Variational wave functions for the spin-Peierls transition in the Su-Schrieffer-Heeger model with quantum phonons}}
\subsubsection*{Francesco Ferrari, \dots, and Federico Becca (2020-09-28 prb)}
We introduce variational wave functions to evaluate the ground-state properties of spin-phonon coupled systems described by the Su-Schrieffer-Heeger model. Quantum spins and phonons are treated on equal footing within a Monte Carlo sampling, and different regimes are investigated. We show that the p...
\subsection*{\href{http://link.aps.org/doi/10.1103/PhysRevB.102.125150}{Dynamic properties of the warm dense electron gas based on $ab initio$ path integral Monte Carlo simulations}}
\subsubsection*{Paul Hamann, \dots, and Michael Bonitz (2020-09-28 prb)}
There is growing interest in warm dense matter (WDM), an exotic state on the border between condensed matter and plasmas. Due to the simultaneous importance of quantum and correlation effects, WDM is complicated to treat theoretically. A key role has been played by ab initio path integral Monte Carl...
\subsection*{\href{http://link.aps.org/doi/10.1103/PhysRevB.102.125151}{Experimental verification of a temperature-induced topological phase transition in ${\mathrm{TlBiS}}_{2}$ and $\mathrm{Tl}\mathrm{Bi}{\mathrm{Se}}_{2}$}}
\subsubsection*{Takehito Imai, \dots, and Taichi Okuda (2020-09-28 prb)}
Temperature dependence of the band structure, as well as the spin-polarization on the topologically trivial insulator ${\mathrm{TlBiS}}_{2}$ and nontrivial insulator $\mathrm{Tl}\mathrm{Bi}{\mathrm{Se}}_{2}$, have been investigated by spin- and angle-resolved photoelectron spectroscopy. Despite the ...
\subsection*{\href{http://link.aps.org/doi/10.1103/PhysRevB.102.125432}{Strong exciton-phonon interaction assisting simultaneous enhancement of photoluminescence and Raman scattering from suspended carbon nanotubes}}
\subsubsection*{Hisashi Sumikura, \dots, and Masaya Notomi (2020-09-28 prb)}
We have observed intensity enhancement in exciton photoluminescence and $G$-band Raman scattering of suspended pristine semiconducting carbon nanotubes (CNTs) under resonant excitation while preventing surrounding dielectric Coulomb screening and emission quenching effects. Photoluminescence excitat...
\subsection*{\href{http://link.aps.org/doi/10.1103/PhysRevB.102.125433}{Optical second- and third-harmonic generation on excitons in ZnSe/BeTe quantum wells}}
\subsubsection*{Johannes Mund, \dots, and Manfred Bayer (2020-09-28 prb)}
Optical harmonic generation on excitons is found in ZnSe/BeTe quantum wells with type-II band alignment. Two experimental approaches with spectrally broad femtosecond laser pulses and spectrally narrow picosecond pulses are used for spectroscopic studies by means of second and third harmonic generat...
\subsection*{\href{http://link.aps.org/doi/10.1103/PhysRevB.102.125434}{Enhanced near-field radiation in both TE and TM waves through excitation of Mie resonance}}
\subsubsection*{Yizhi Hu, \dots, and Yue Yang (2020-09-28 prb)}
Near-field radiative heat transfer between two bodies can exceed the far-field blackbody limitation predicted by Planck's law due to the evanescent waves tunneling or coupling of additional surface modes, which typically can occur only in TM modes for nonmagnetic materials. The Mie resonance may hav...
\subsection*{\href{http://link.aps.org/doi/10.1103/PhysRevB.102.125435}{In situ grazing-incidence x-ray scattering study of pulsed-laser deposition of Pt layers}}
\subsubsection*{V. Holý, \dots, and T. Baumbach (2020-09-28 prb)}
We present a methodical study of grazing-incidence small-angle x-ray scattering performed in situ during pulsed-laser deposition of Pt on sapphire substrates. From measured two-dimensional intensity distributions in reciprocal space we calculated horizontal and vertical intensity projections and com...
\subsection*{\href{http://link.aps.org/doi/10.1103/PhysRevB.102.094109}{Evidence for a soft phonon mode driven Peierls-type distortion in ${\mathrm{Sc}}_{3}\mathrm{Co}{\mathrm{C}}_{4}$}}
\subsubsection*{Jan Langmann, \dots, and Georg Eickerling (2020-09-25 prb)}
We provide experimental and theoretical evidence for the realization of the Peierls-type structurally distorted state in the quasi-one-dimensional superconductor ${\mathrm{Sc}}_{3}\mathrm{Co}{\mathrm{C}}_{4}$ by a phonon-softening mechanism. The transition from the high- to the final low-temperature...
\subsection*{\href{http://link.aps.org/doi/10.1103/PhysRevB.102.094433}{Hidden magnetic order in the triangular-lattice magnet ${\mathrm{Li}}_{2}{\mathrm{MnTeO}}_{6}$}}
\subsubsection*{E. A. Zvereva, \dots, and M.-H. Whangbo (2020-09-25 prb)}
The manganese tellurate ${\mathrm{Li}}_{2}{\mathrm{MnTeO}}_{6}$ consists of trigonal spin lattices made up of ${\mathrm{Mn}}^{4+}$ (${d}^{3}$, $S=3/2$) ions. The magnetic properties of this compound were characterized by several experimental techniques, which include magnetic susceptibility, specifi...
\subsection*{\href{http://link.aps.org/doi/10.1103/PhysRevB.102.094434}{Higher-order ferromagnetic resonances in out-of-plane saturated Co/Au magnetic multilayers}}
\subsubsection*{L. Fallarino, \dots, and J. Lindner (2020-09-25 prb)}
Artificial ferromagnetic (FM)/nonmagnetic multilayers, with large enough FM thickness to prevent the dominance of interface anisotropies, offer a straightforward insight into the understanding and control of perpendicular standing spin wave (PSSW) modes. Here we present a study of the static and dyn...
\subsection*{\href{http://link.aps.org/doi/10.1103/PhysRevB.102.094518}{Superconducting islands with topological Josephson junctions based on semiconductor nanowires}}
\subsubsection*{J. Ávila, \dots, and R. Aguado (2020-09-25 prb)}
We theoretically study superconducting islands based on semiconductor-nanowire Josephson junctions and take into account the presence of subgap quasiparticle excitations in the spectrum of the junction. Our method extends the standard model Hamiltonian for a superconducting charge qubit and replaces...
\subsection*{\href{http://link.aps.org/doi/10.1103/PhysRevB.102.100303}{Nonreciprocal photonic topological order driven by uniform optical pumping}}
\subsubsection*{Robert Duggan, \dots, and Andrea Alù (2020-09-25 prb)}
Photonic topological insulators (PTIs) have raised significant interest in recent years, in part because of their protected edge propagation. Most PTI realizations have been based on reciprocal platforms, with protection only from symmetry-preserving defects. Nonreciprocal PTIs protect against arbit...
\subsection*{\href{http://link.aps.org/doi/10.1103/PhysRevB.102.104112}{Phonon dispersion throughout the iron spin crossover in ferropericlase}}
\subsubsection*{Michel L. Marcondes, \dots, and Renata M. Wentzcovitch (2020-09-25 prb)}
Ferropericlase (Fp), $({\mathrm{Mg}}_{1−x}{\mathrm{Fe}}_{x})\mathrm{O}$, is the second most abundant phase in the Earth's lower mantle. At relevant pressure-temperature conditions, iron in Fp undergoes a high spin (HS), S = 2, to low spin (LS), S = 0, state change. The nature of this phenomenon is q...
\subsection*{\href{http://link.aps.org/doi/10.1103/PhysRevB.102.104114}{Spin thermometry and spin relaxation of optically detected ${\mathrm{Cr}}^{3+}$ ions in ruby ${\mathrm{Al}}_{2}{\mathrm{O}}_{3}$}}
\subsubsection*{Vikas K. Sewani, \dots, and Arne Laucht (2020-09-25 prb)}
After 150 years of research, the gemstone ruby (Cr3+ in Al2O3) still reveals interesting properties. Here, the authors use its spin properties to conduct all-optical thermometry inside a dilution refrigerator, reaching lattice temperatures as low as 143 mK under continuous, low-power laser excitation. They measure the spin relaxation time down to these unprecedented low temperatures, and demonstrate optically detected magnetic resonance within the rich spin structure of the Cr3+ ions.
\subsection*{\href{http://link.aps.org/doi/10.1103/PhysRevB.102.104205}{Glass-forming ability of elemental zirconium}}
\subsubsection*{Sébastien Becker, \dots, and Noël Jakse (2020-09-25 prb)}
We report large-scale molecular dynamics simulations of the glass formation from the liquid phase and homogeneous nucleation phenomena of pure zirconium. For this purpose, we have built a modified embedded atom model potential in order to reproduce relevant structural, dynamic, and thermodynamic pro...
\subsection*{\href{http://link.aps.org/doi/10.1103/PhysRevB.102.104432}{Anomalous helimagnetic domain shrinkage due to the weakening of the Dzyaloshinskii-Moriya interaction in CrAs}}
\subsubsection*{B. Y. Pan, \dots, and D. L. Feng (2020-09-25 prb)}
CrAs is a well-known helimagnet with the double-helix structure originating from the competition between the Dzyaloshinskii-Moriya interaction (DMI) and antiferromagnetic exchange interaction $J$. By resonant soft-x-ray scattering, we observe the magnetic peak (0 0 ${q}_{m}$) that emerges at the hel...
\subsection*{\href{http://link.aps.org/doi/10.1103/PhysRevB.102.104512}{Critical role of the sign structure in the doped Mott insulator: Luther-Emery versus Fermi-liquid-like state in quasi-one-dimensional ladders}}
\subsubsection*{Hong-Chen Jiang, \dots, and Zheng-Yu Weng (2020-09-25 prb)}
The mechanism of superconductivity in a purely interacting electron system has been one of the most challenging issues in condensed matter physics. In the BCS theory, the Landau's Fermi liquid is the ground state of weakly interacting fermions dictated by the fermion sign structure, and the supercon...
\subsection*{\href{http://link.aps.org/doi/10.1103/PhysRevB.102.115151}{Unconventional topological insulators from extended topological band degeneracies}}
\subsubsection*{Zhongbo Yan (2020-09-25 prb)}
A general and beautiful picture for the realization of topological insulators is that the mass term of the Dirac model has a nodal surface wrapping one Dirac point. We show that this geometric picture based on Dirac points can be generalized to extended band degeneracies with nontrivial topological ...
\subsection*{\href{http://link.aps.org/doi/10.1103/PhysRevB.102.115155}{Isotope effects in x-ray absorption spectra of liquid water}}
\subsubsection*{Chunyi Zhang, \dots, and Xifan Wu (2020-09-25 prb)}
The isotope effects in x-ray absorption spectra of liquid water are studied by a many-body approach within electron-hole excitation theory. The molecular structures of both light and heavy water are modeled by path-integral molecular dynamics based on the advanced deep-learning technique. The neural...
\subsection*{\href{http://link.aps.org/doi/10.1103/PhysRevB.102.115156}{Melting of the critical behavior of a Tomonaga-Luttinger liquid under dephasing}}
\subsubsection*{Jean-Sébastien Bernier, \dots, and Dario Poletti (2020-09-25 prb)}
Strongly correlated quantum systems often display universal behavior as, in certain regimes, their properties are found to be independent of the microscopic details of the underlying system. An example of such a situation is the Tomonaga-Luttinger liquid description of one-dimensional strongly corre...
\subsection*{\href{http://link.aps.org/doi/10.1103/PhysRevB.102.115307}{Strain-engineered widely tunable perfect absorption angle in black phosphorus from first principles}}
\subsubsection*{Mohammad Alidoust, \dots, and Jaakko Akola (2020-09-25 prb)}
Using the density functional theory of electronic structure, we compute the anisotropic dielectric response of bulk black phosphorus subject to strain. Employing the obtained permittivity tensor, we solve Maxwell's equations and study the electromagnetic response of a layered structure comprising a ...
\subsection*{\href{http://link.aps.org/doi/10.1103/PhysRevB.102.115434}{Accessing long timescales in the relaxation dynamics of spins coupled to a conduction-electron system using absorbing boundary conditions}}
\subsubsection*{Michael Elbracht and Michael Potthoff (2020-09-25 prb)}
The relaxation time of a classical spin interacting with a large conduction-electron system is computed for a weak magnetic field, which initially drives the spin out of equilibrium. We trace the spin and the conduction-electron dynamics on a timescale which exceeds the characteristic electronic sca...
\subsection*{\href{http://link.aps.org/doi/10.1103/PhysRevB.102.125145}{Tunable and enhanced Rashba spin-orbit coupling in iridate-manganite heterostructures}}
\subsubsection*{T. S. Suraj, \dots, and M. S. Ramachandra Rao (2020-09-25 prb)}
Too many Dollar signs...
\subsection*{\href{http://link.aps.org/doi/10.1103/PhysRevB.102.125146}{Pressure-induced suppression of ferromagnetism in ${\mathrm{CePd}}_{2}{\mathrm{P}}_{2}$}}
\subsubsection*{T. A. Elmslie, \dots, and J. J. Hamlin (2020-09-25 prb)}
The correlated electron material ${\mathrm{CePd}}_{2}{\mathrm{P}}_{2}$ crystallizes in the ${\mathrm{ThCr}}_{2}{\mathrm{Si}}_{2}$ structure and orders ferromagnetically at 29 K. Prior work by Lai et al. [Phys. Rev. B 97, 224406 (2018)] found evidence for a ferromagnetic quantum critical point induce...
\subsection*{\href{http://link.aps.org/doi/10.1103/PhysRevB.102.125147}{Transport in conductors and rectifiers: Mean-field Redfield equations and nonequilibrium Green's functions}}
\subsubsection*{Zekun Zhuang, \dots, and J. B. Marston (2020-09-25 prb)}
We derive a closed equation of motion for the one particle density matrix of a quantum system coupled to multiple baths using the Redfield master equation combined with a mean-field approximation. The steady-state solution may be found analytically with perturbation theory. Application of the method...
\subsection*{\href{http://link.aps.org/doi/10.1103/PhysRevB.102.125148}{Twofold quadruple Weyl nodes in chiral cubic crystals}}
\subsubsection*{Tiantian Zhang, \dots, and Shuichi Murakami (2020-09-25 prb)}
Unlike conventional Weyl nodes, unconventional ones carry a quantized monopole charge $\mathcal{C}>1$, and their existence needs the protection of crystalline symmetries in addition to translation symmetry. There have been many studies on unconventional Weyl nodes, yet we have so far missed one, ...
\subsection*{\href{http://link.aps.org/doi/10.1103/PhysRevB.102.125309}{Finite frequency backscattering current noise at a helical edge}}
\subsubsection*{B. V. Pashinsky, \dots, and I. S. Burmistrov (2020-09-25 prb)}
Magnetic impurities with sufficient anisotropy could account for the observed strong deviation of the edge conductance of 2D topological insulators from the anticipated quantized value. In this work we consider such a helical edge coupled to dilute impurities with an arbitrary spin $S$ and a general...
\subsection*{\href{http://link.aps.org/doi/10.1103/PhysRevB.102.094311}{Phonon hydrodynamics in crystalline GeTe at low temperature}}
\subsubsection*{Kanka Ghosh, \dots, and Jean-Luc Battaglia (2020-09-24 prb)}
A first-principles density functional method along with the direct solution of linearized Boltzmann transport equations are employed to systematically analyze the low-temperature thermal transport in crystalline GeTe. The extensive thermal transport simulations, ranging from room temperature to cryo...
\subsection*{\href{http://link.aps.org/doi/10.1103/PhysRevB.102.094431}{Field-induced tricritical behavior in the Néel-type skyrmion host ${\mathrm{GaV}}_{4}{\mathrm{S}}_{8}$}}
\subsubsection*{Bingjie Liu, \dots, and Zhe Qu (2020-09-24 prb)}
Too many Dollar signs...
\subsection*{\href{http://link.aps.org/doi/10.1103/PhysRevB.102.094432}{Microscopic mechanism of high-temperature ferromagnetism in Fe, Mn, and Cr-doped InSb, InAs, and GaSb magnetic semiconductors}}
\subsubsection*{Jing-Yang You, \dots, and Gang Su (2020-09-24 prb)}
In recent experiments, high Curie temperatures ${T}_{c}$ above room temperature were reported in ferromagnetic semiconductors Fe-doped GaSb and InSb, while low ${T}_{c}$ between 20 K to 90 K were observed in some other semiconductors with the same crystal structure, including Fe-doped InAs and Mn-do...
\subsection*{\href{http://link.aps.org/doi/10.1103/PhysRevB.102.104113}{Three-dimensional higher-order topological acoustic system with multidimensional topological states}}
\subsubsection*{Shengjie Zheng, \dots, and Dejie Yu (2020-09-24 prb)}
Topologically protected gapless edge/surface states are phases of quantum matter which behave as massless Dirac fermions and are immune to disorders and continuous perturbations. Recently, a new class of topological insulators (TIs) with gapped edge states and in-gap corner states has been theoretic...
\subsection*{\href{http://link.aps.org/doi/10.1103/PhysRevB.102.104204}{From bulk descriptions to emergent interfaces: Connecting the Ginzburg-Landau and elastic-line models}}
\subsubsection*{Nirvana Caballero, \dots, and Thierry Giamarchi (2020-09-24 prb)}
Controlling interfaces is highly relevant from a technological point of view. However, their rich and complex behavior makes them very difficult to describe theoretically, and hence to predict. In this work, we establish a procedure to connect two levels of descriptions of interfaces: for a bulk des...
\subsection*{\href{http://link.aps.org/doi/10.1103/PhysRevB.102.104431}{Experimental observation of quantum many-body excitations of ${E}_{8}$ symmetry in the Ising chain ferromagnet ${\mathrm{CoNb}}_{2}{\mathrm{O}}_{6}$}}
\subsubsection*{Kirill Amelin, \dots, and Zhe Wang (2020-09-24 prb)}
Close to the quantum critical point of the transverse-field Ising spin-chain model, an exotic dynamic spectrum was predicted to emerge upon a perturbative longitudinal field. The dynamic spectrum consists of eight particles and is governed by the symmetry of the ${E}_{8}$ Lie algebra. Here we report...
\subsection*{\href{http://link.aps.org/doi/10.1103/PhysRevB.102.115150}{Magnetic and spin-liquid phases in the frustrated $t−{t}^{′}$ Hubbard model on the triangular lattice}}
\subsubsection*{Luca F. Tocchio, \dots, and Federico Becca (2020-09-24 prb)}
The Hubbard model and its strong-coupling version, the Heisenberg one, have been widely studied on the triangular lattice to capture the essential low-temperature properties of different materials. One example is given by transition metal dichalcogenides, as $1\mathrm{T}−{\mathrm{TaS}}_{2}$, where a...
\subsection*{\href{http://link.aps.org/doi/10.1103/PhysRevB.102.115152}{Charge-magnon conversion at the topological insulator/antiferromagnetic insulator interface}}
\subsubsection*{L. Y. Liao, \dots, and C. Song (2020-09-24 prb)}
Efficient generation of magnons is highly desirable for magnonics. Here we theoretically investigate the charge-magnon conversion at the interface of topological insulator/antiferromagnetic insulator (AFMI), where the Dirac electrons and the antiferromagnetic magnons are coupled by the fluctuation o...
\subsection*{\href{http://link.aps.org/doi/10.1103/PhysRevB.102.115153}{Electronic band structure of phosphorus-doped single crystal diamond: Dynamic Jahn-Teller distortion of the tetrahedral donor ground state}}
\subsubsection*{V. D. Blank, \dots, and S. A. Terentiev (2020-09-24 prb)}
Too many Dollar signs...
\subsection*{\href{http://link.aps.org/doi/10.1103/PhysRevB.102.115154}{Generalized string-net model for unitary fusion categories without tetrahedral symmetry}}
\subsubsection*{Alexander Hahn and Ramona Wolf (2020-09-24 prb)}
The Levin-Wen model of string-net condensation explains how topological phases emerge from the microscopic degrees of freedom of a physical system. However, the original construction is not applicable to all unitary fusion categories since some additional symmetries for the $F$-symbols are imposed. ...
\subsection*{\href{http://link.aps.org/doi/10.1103/PhysRevB.102.115429}{Super-Klein tunneling of Dirac fermions through electrostatic gratings in graphene}}
\subsubsection*{Alonso Contreras-Astorga, \dots, and Vít Jakubský (2020-09-24 prb)}
We use the Wick-rotated time-dependent supersymmetry to construct models of two-dimensional Dirac fermions in the presence of an electrostatic grating. We show that there appears omnidirectional perfect transmission through the grating at specific energy. Additionally to being transparent for incomi...
\subsection*{\href{http://link.aps.org/doi/10.1103/PhysRevB.102.115432}{Microwave response of interacting oxide two-dimensional electron systems}}
\subsubsection*{D. Tabrea, \dots, and J. Falson (2020-09-24 prb)}
We present an experimental study on microwave illuminated high mobility MgZnO/ZnO based two-dimensional electron systems with different electron densities and, hence, varying Coulomb interaction strength. The photoresponse of the low-temperature dc resistance in perpendicular magnetic field is exami...
\subsection*{\href{http://link.aps.org/doi/10.1103/PhysRevB.102.115433}{Continuous crossover from two-dimensional to one-dimensional electronic properties for metallic silicide nanowires}}
\subsubsection*{Stephan Appelfeller, \dots, and Mario Dähne (2020-09-24 prb)}
In a joint experimental and theoretical study on metallic $\mathrm{Tb}{\mathrm{Si}}_{2}$ nanowires, we observe a continuous crossover from a two-dimensional (2D) to a quasi-one-dimensional (1D) electronic structure by reduction of the nanowire width. The nanowires were grown by self-organization on ...
\subsection*{\href{http://link.aps.org/doi/10.1103/PhysRevB.102.119902}{Erratum: Accurate electronic band gaps of two-dimensional materials from the local modified Becke-Johnson potential [Phys. Rev. B 101, 245163 (2020)]}}
\subsubsection*{Tomáš Rauch, \dots, and Silvana Botti (2020-09-24 prb)}
Author(s): Tomáš Rauch, Miguel A. L. Marques, and Silvana Botti[Phys. Rev. B 102, 119902] Published Thu Sep 24, 2020
\subsection*{\href{http://link.aps.org/doi/10.1103/PhysRevB.102.119903}{Erratum: Dropping an impurity into a Chern insulator: A polaron view on topological matter [Phys. Rev. B 99, 081105(R) (2019)]}}
\subsubsection*{A. Camacho-Guardian, \dots, and G. M. Bruun (2020-09-24 prb)}
Author(s): A. Camacho-Guardian, N. Goldman, P. Massignan, and G. M. Bruun[Phys. Rev. B 102, 119903] Published Thu Sep 24, 2020
\subsection*{\href{http://link.aps.org/doi/10.1103/PhysRevB.102.121303}{Particle-hole Pfaffian order in a translationally and rotationally invariant system}}
\subsubsection*{Chen Sun, \dots, and D. E. Feldman (2020-09-24 prb)}
The PH-Pfaffian (particle-hole Pfaffian) topological order has been proposed as a candidate order for the $ν=5/2$ quantum Hall effect. The PH-Pfaffian liquid is known to be the ground state in several coupled wire and coupled stripe constructions. No translationally and rotationally invariant models...
\subsection*{\href{http://link.aps.org/doi/10.1103/PhysRevB.102.121407}{Direct measurement of fast ortho-para conversion of molecularly chemisorbed ${\mathrm{H}}_{2}$ on Pd(210)}}
\subsubsection*{H. Ueta, \dots, and K. Fukutani (2020-09-24 prb)}
Ortho-para conversion of molecularly chemisorbed ${\mathrm{H}}_{2}$ on a Pd(210) surface at a surface temperature of 50 K is reported. A combination of a pulsed molecular beam, photostimulated desorption, and resonance-enhanced multiphoton ionization techniques was used for probing the change in the...
\subsection*{\href{http://link.aps.org/doi/10.1103/PhysRevB.102.125141}{Competing orders at higher-order Van Hove points}}
\subsubsection*{Laura Classen, \dots, and Michael M. Scherer (2020-09-24 prb)}
Van Hove points are special points in the energy dispersion, where the density of states exhibits analytic singularities. When a Van Hove point is close to the Fermi level, tendencies towards density wave orders, Pomeranchuk orders, and superconductivity can all be enhanced, often in more than one c...
\subsection*{\href{http://link.aps.org/doi/10.1103/PhysRevB.102.125142}{Dynamically induced topology and quantum monodromies in a proximity quenched gapless wire}}
\subsubsection*{D. Dahan, \dots, and B. Seradjeh (2020-09-24 prb)}
We study the quench dynamics of a topologically trivial one-dimensional gapless wire following its sudden coupling to topological bound states. We find that as the bound states leak into and propagate through the wire, signatures of their topological nature survive and remain measurable over a long ...
\subsection*{\href{http://link.aps.org/doi/10.1103/PhysRevB.102.125143}{Combination of tensor network states and Green's function Monte Carlo}}
\subsubsection*{Mingpu Qin (2020-09-24 prb)}
We propose an approach to study the ground state of quantum many-body systems in which tensor network states, specifically projected entangled pair states (PEPSs), and Green's function Monte Carlo (GFMC) are combined. PEPSs, by design, encode the area law which governs the scaling of entanglement en...
\subsection*{\href{http://link.aps.org/doi/10.1103/PhysRevB.102.125144}{Corner states and topological transitions in two-dimensional higher-order topological sonic crystals with inversion symmetry}}
\subsubsection*{Zhan Xiong, \dots, and Jian-Hua Jiang (2020-09-24 prb)}
Macroscopic two-dimensional (2D) sonic crystals with inversion symmetry are studied to reveal higher-order topological physics in classical wave systems. By tuning a single geometry parameter, the band topology of the bulk and the edges can be controlled simultaneously. The bulk band gap forms an ac...
\subsection*{\href{http://link.aps.org/doi/10.1103/PhysRevB.102.125431}{Optically induced electron spin currents in the Kretschmann configuration}}
\subsubsection*{Daigo Oue and Mamoru Matsuo (2020-09-24 prb)}
We investigate electron spin currents induced optically via plasmonic modes in the Kretschmann configuration. By utilizing the scattering matrix formalism, we take the plasmonic mode coupled to an external laser drive into consideration and calculate induced magnetization in the metal. The spatial d...
\subsection*{\href{http://link.aps.org/doi/10.1103/PhysRevB.102.094204}{Entanglement transition in the projective transverse field Ising model}}
\subsubsection*{Nicolai Lang and Hans Peter Büchler (2020-09-23 prb)}
Random quantum circuits have emerged as useful toy models for entanglement transitions in quantum many-body systems. Here, the authors study a projective version of the transverse-field Ising model without unitary dynamics where the competition between two noncommuting measurements drives an entanglement transition. The authors identify a nonlocal classical process that captures the entanglement dynamics and relate the model to bond percolation in space-time. This leads to a conformal field theory description of the system at criticality which they verify numerically for large systems.
\subsection*{\href{http://link.aps.org/doi/10.1103/PhysRevB.102.094310}{Self-averaging in many-body quantum systems out of equilibrium: Approach to the localized phase}}
\subsubsection*{E. Jonathan Torres-Herrera, \dots, and Lea F. Santos (2020-09-23 prb)}
The self-averaging behavior of interacting many-body quantum systems has been mostly studied at equilibrium. The present paper addresses what happens out of equilibrium, as the increase of the strength of on-site disorder takes the system to the localized phase. We consider two local and two nonloca...
\subsection*{\href{http://link.aps.org/doi/10.1103/PhysRevB.102.094429}{$\mathrm{Co}{({\mathrm{NO}}_{3})}_{2}$ as an inverted umbrella-type chiral noncoplanar ferrimagnet}}
\subsubsection*{I. L. Danilovich, \dots, and A. N. Vasiliev (2020-09-23 prb)}
Low-dimensional magnetic systems tend to reveal exotic spin liquid ground states or form peculiar types of long-range order. The realization of either ordered or disordered ground states in triangular, honeycomb, or kagome lattices is dictated by the competition of exchange interactions, being sensitive also to anisotropy and to the spin value of magnetic ions. Here, the authors present the case of a chiral noncoplanar umbrella-type ferrimagnet formed in cobalt (II) dinitrate, Co(NO3)2.
\subsection*{\href{http://link.aps.org/doi/10.1103/PhysRevB.102.094430}{Bosonic representation of a Lipkin-Meshkov-Glick model with Markovian dissipation}}
\subsubsection*{Jan C. Louw, \dots, and Johannes N. Kriel (2020-09-23 prb)}
We study the dynamics of a Lipkin-Meshkov-Glick model in the presence of Markovian dissipation, with a focus on late-time dynamics and the approach to thermal equilibrium. Making use of a vectorized bosonic representation of the corresponding Lindblad master equation, we use degenerate perturbation ...
\subsection*{\href{http://link.aps.org/doi/10.1103/PhysRevB.102.094516}{Interplay between superconductivity and non-Fermi liquid behavior at a quantum critical point in a metal. III. The $γ$ model and its phase diagram across $γ=1$}}
\subsubsection*{Yi-Ming Wu, \dots, and Andrey V. Chubukov (2020-09-23 prb)}
In this paper we continue our analysis of the interplay between the pairing and the non-Fermi liquid behavior in a metal for a set of quantum-critical models with an effective dynamical electron-electron interaction $V({\mathrm{Ω}}_{m})∝1/{|{\mathrm{Ω}}_{m}|}^{γ}$ (the $γ$ model). We analyze both th...
\subsection*{\href{http://link.aps.org/doi/10.1103/PhysRevB.102.094517}{Signatures of a long-range spin-triplet component in an Andreev interferometer}}
\subsubsection*{Anatoly F. Volkov (2020-09-23 prb)}
Too many Dollar signs...
\subsection*{\href{http://link.aps.org/doi/10.1103/PhysRevB.102.104110}{Evidence for Goldstone-like and Higgs-like structural modes in the model $\mathrm{Pb}{\mathrm{Mg}}_{1/3}{\mathrm{Nb}}_{2/3}{\mathrm{O}}_{3}$ relaxor ferroelectric}}
\subsubsection*{Sergey Prosandeev, \dots, and Brahim Dkhil (2020-09-23 prb)}
Effective Hamiltonian simulations are conducted to unveil the nature of the low-frequency polar modes in the prototype relaxor ferroelectric, $\mathrm{Pb}({\mathrm{Mg}}_{1/3}{\mathrm{Nb}}_{2/3}){\mathrm{O}}_{3}$. Above the so-called ${T}^{*}$ temperature, only a single soft-mode exists, with its fre...
\subsection*{\href{http://link.aps.org/doi/10.1103/PhysRevB.102.104111}{Optical method to detect the relationship between chirality of reciprocal space chiral multifold fermions and real space chiral crystals}}
\subsubsection*{Yan Sun, \dots, and Claudia Felser (2020-09-23 prb)}
The chirality of chiral multifold fermions in reciprocal space is related to the chirality of crystal lattice structures in real space. In this study, we propose a strategy to detect and identify multifold fermions of opposite chirality in nonmagnetic systems using second-order optical transports. C...
\subsection*{\href{http://link.aps.org/doi/10.1103/PhysRevB.102.104203}{Many-body localization of bosons in an optical lattice: Dynamics in disorder-free potentials}}
\subsubsection*{Ruixiao Yao and Jakub Zakrzewski (2020-09-23 prb)}
The phenomenon of many-body Stark localization of bosons in tilted optical lattice is studied. Despite the fact that no disorder is necessary for Stark localization to occur, it is very similar to well-known many-body localization (MBL) in sufficiently strong disorder. Not only the mean gap ratio re...
\subsection*{\href{http://link.aps.org/doi/10.1103/PhysRevB.102.104305}{Ubiquity of zeros of the Loschmidt amplitude for mixed states in different physical processes and its implication}}
\subsubsection*{Xu-Yang Hou, \dots, and Chih-Chun Chien (2020-09-23 prb)}
The Loschmidt amplitude of the purified states of mixed-state density matrices is shown to have zeros when the system undergoes a quasistatic, quench, or Uhlmann process. While the Loschmidt-amplitude zero of a quench process corresponds to a dynamical quantum phase transition (DQPT) accompanied by ...
\subsection*{\href{http://link.aps.org/doi/10.1103/PhysRevB.102.104306}{Kibble-Zurek mechanism from different angles: The transverse XY model and subleading scalings}}
\subsubsection*{Björn Ladewig, \dots, and Sebastian Diehl (2020-09-23 prb)}
The Kibble-Zurek mechanism describes the saturation of critical scaling upon dynamically approaching a phase transition. This is a consequence of the breaking of adiabaticity due to the scale set by the slow drive. By driving the gap parameter, this can be used to determine the leading critical expo...
\clearpage
\section{Nature}
\subsection*{\href{https://www.nature.com/articles/d41586-020-02502-2}{A recipe to reverse the loss of nature}}
\subsubsection*{Brett A. Bryan and Carla L. Archibald (2020-09-09T00:00:00)}
(News \& Views)How can the decline in global biodiversity be reversed, given the need to supply food? Computer modelling provides a way to assess the effectiveness of combining various conservation and food-system interventions to tackle this issue.
\subsection*{\href{https://www.nature.com/articles/d41586-020-02461-8}{Genetically engineered yeast makes medicinal plant products}}
\subsubsection*{José Montaño López and José L. Avalos (2020-09-02T00:00:00)}
(News \& Views)Yeast has been engineered to convert simple sugars and amino acids into drugs that inhibit a neurotransmitter molecule. The work marks a step towards making the production of these drugs more reliable and sustainable.
\subsection*{\href{https://www.nature.com/articles/d41586-020-02656-z}{Elusive photonic crystals come a step closer}}
\subsubsection*{John C.  Crocker (2020-09-23T00:00:00)}
(News \& Views)Researchers have long sought materials in which light behaves the way electrons do in semiconductors. A workable approach for growing such materials in bulk now seems at hand, and could lead to advances in computing.
\subsection*{\href{https://www.nature.com/articles/d41586-020-02504-0}{Calcium channel in plants helps shut the door on intruders}}
\subsubsection*{Keiko Yoshioka and Wolfgang Moeder (2020-09-07T00:00:00)}
(News \& Views)Disease-causing microorganisms can invade plants through leaf pores called stomata, which close rapidly in a calcium-dependent manner on detecting such danger. The calcium channels involved have now finally been identified.
\subsection*{\href{https://www.nature.com/articles/s41586-020-2729-3}{Host–microbiota maladaptation in colorectal cancer}}
\subsubsection*{Alina Janney, and Elizabeth H. Mann (2020-09-23T00:00:00)}
(Review Article)This Review describes the interplay between host genetics, host immunity and the gut microbiome in the modulation of colorectal cancer, and discusses the role of specific bacterial species and metabolites alongside technological advances that will facilitate more in-depth investigation of the microbiome in disease.
\subsection*{\href{https://www.nature.com/articles/s41586-020-2735-5}{Third-order nanocircuit elements for neuromorphic engineering}}
\subsubsection*{Suhas Kumar, and Ziwen Wang (2020-09-23T00:00:00)}
(Article)Electrophysical processes are used to create third-order nanoscale circuit elements, and these are used to realize a transistorless network that can perform Boolean operations and find solutions to a computationally hard graph-partitioning problem.
\subsection*{\href{https://www.nature.com/articles/s41586-020-2718-6}{Colloidal diamond}}
\subsubsection*{Mingxin He, \dots, and David J. Pine (2020-09-23T00:00:00)}
(Article)Self-assembly of cubic diamond crystals is demonstrated, by using precursor clusters of particles with carefully placed ‘sticky’ patches that attract and bind adjacent clusters in specific geometries.
\subsection*{\href{https://www.nature.com/articles/s41586-020-2733-7}{Light-driven post-translational installation of reactive protein side chains}}
\subsubsection*{Brian Josephson, \dots, and Benjamin G. Davis (2020-09-23T00:00:00)}
(Article)A wide range of side chains are installed into proteins by addition of photogenerated alkyl or difluroalkyl radicals, providing access to new functionality and reactivity in proteins.
\subsection*{\href{https://www.nature.com/articles/s41586-020-2727-5}{The hysteresis of the Antarctic Ice Sheet}}
\subsubsection*{Julius Garbe, \dots, and Ricarda Winkelmann (2020-09-23T00:00:00)}
(Article)Modelling shows that the Antarctic Ice Sheet exhibits multiple temperature thresholds beyond which ice loss would become irreversible, and once melted, the ice sheet can regain its previous mass only if the climate cools well below pre-industrial temperatures.
\subsection*{\href{https://www.nature.com/articles/s41586-020-2686-x}{Mapping carbon accumulation potential from global natural forest regrowth}}
\subsubsection*{Susan C. Cook-Patton, \dots, and Bronson W. Griscom (2020-09-23T00:00:00)}
(Article)A one-kilometre-resolution map of aboveground carbon accumulation rates of forest regrowth shows 100-fold variation across the globe, with rates 32\% higher on average than IPCC estimates.
\subsection*{\href{https://www.nature.com/articles/s41586-020-2705-y}{Bending the curve of terrestrial biodiversity needs an integrated strategy}}
\subsubsection*{David Leclère, \dots, and Lucy Young (2020-09-10T00:00:00)}
(Article)To promote the recovery of the currently declining global trends in terrestrial biodiversity, increases in both the extent of land under conservation management and the sustainability of the global food system from farm to fork are required.
\subsection*{\href{https://www.nature.com/articles/s41586-020-2721-y}{Metabolic trait diversity shapes marine biogeography}}
\subsubsection*{Curtis Deutsch, and Brad Seibel (2020-09-16T00:00:00)}
(Article)A tight coupling between metabolic rate, efficacy of oxygen supply and the temperature sensitivities of marine animals predicts a variety of geographical niches that better aligns with the distributions of species than models of either temperature or oxygen alone.
\subsection*{\href{https://www.nature.com/articles/s41586-020-2720-z}{Evolution of the endothelin pathway drove neural crest cell diversification}}
\subsubsection*{Tyler A. Square, \dots, and Daniel M. Medeiros (2020-09-16T00:00:00)}
(Article)CRISPR–Cas9-mediated disruption of the endothelin-signalling pathway in the sea lamprey Petromyzon marinus and the frog Xenopus laevis were used to delineate ancient and lineage-specific roles of endothelin signalling and provide insights into vertebrate evolution.
\subsection*{\href{https://www.nature.com/articles/s41586-020-2702-1}{The calcium-permeable channel OSCA1.3 regulates plant stomatal immunity}}
\subsubsection*{Kathrin Thor, \dots, and Cyril Zipfel (2020-08-26T00:00:00)}
(Article)A study in Arabidopsis thaliana shows that the immune receptor-associated cytosolic kinase BIK1 phosphorylates OSCA1.3 and identifies OSCA1.3 as the pathogen-responsive Ca2+-permeable channel that regulates stomatal closure.
\subsection*{\href{https://www.nature.com/articles/s41586-020-2724-8}{Homeostatic mini-intestines through scaffold-guided organoid morphogenesis}}
\subsubsection*{Mikhail Nikolaev, \dots, and Matthias P. Lutolf (2020-09-16T00:00:00)}
(Article)Miniature gut tubes grown in vitro from mouse intestinal stem cells are perfusable, can be colonized with microorganisms and exhibit a similar arrangement and diversity of specialized cell types to intestines in vivo.
\subsection*{\href{https://www.nature.com/articles/s41586-020-2726-6}{Red blood cell tension protects against severe malaria in the Dantu blood group}}
\subsubsection*{Silvia N. Kariuki, \dots, and Julian C. Rayner (2020-09-16T00:00:00)}
(Article)The rare blood group Dantu is known to protect against severe malaria, and a mechanism is proposed here: Dantu red blood cells have a high membrane tension that prevents invasion by malaria parasites.
\subsection*{\href{https://www.nature.com/articles/s41586-020-2558-4}{Hydroxychloroquine use against SARS-CoV-2 infection in non-human primates}}
\subsubsection*{Pauline Maisonnasse, \dots, and Roger Le Grand (2020-07-22T00:00:00)}
(Article)Hydroxychloroquine did not confer protection against SARS-CoV-2 infection or reduce the viral load after infection in macaques; these findings do not support the use of hydroxychloroquine as an antiviral drug treatment of COVID-19 in humans.
\subsection*{\href{https://www.nature.com/articles/s41586-020-2575-3}{Chloroquine does not inhibit infection of human lung cells with SARS-CoV-2}}
\subsubsection*{Markus Hoffmann, \dots, and Stefan Pöhlmann (2020-07-22T00:00:00)}
(Article)Expression of TMPRSS2—a protease that activates SARS-CoV-2 for entry into cells—renders SARS-CoV-2 insensitive to chloroquine.
\subsection*{\href{https://www.nature.com/articles/s41586-020-2425-3}{The liver–brain–gut neural arc maintains the Treg cell niche in the gut}}
\subsubsection*{Toshiaki Teratani, \dots, and Takanori Kanai (2020-06-11T00:00:00)}
(Article)A liver–brain–gut neural circuit responds to the gut microenvironment and regulates the activity of peripheral regulatory T cells in the colon by controlling intestinal antigen-presenting cells in a muscarinic signalling-dependent manner.
\subsection*{\href{https://www.nature.com/articles/s41586-020-2444-0}{A substrate-specific mTORC1 pathway underlies Birt–Hogg–Dubé syndrome}}
\subsubsection*{Gennaro Napolitano, \dots, and Andrea Ballabio (2020-07-01T00:00:00)}
(Article)Dysregulation of an mTORC1 substrate-specific mechanism leads to constitutive activation of TFEB, and promotes kidney cystogenesis and tumorigenesis in a mouse model of Birt–Hogg–Dubé syndrome.
\subsection*{\href{https://www.nature.com/articles/s41586-020-2732-8}{Plasticity of ether lipids promotes ferroptosis susceptibility and evasion}}
\subsubsection*{Yilong Zou, \dots, and Stuart L. Schreiber (2020-09-16T00:00:00)}
(Article)The cellular organelles peroxisomes contribute to the sensitivity of cells to ferroptosis by synthesizing polyunsaturated ether phospholipids, and changes in the abundances of these lipids are associated with altered sensitivity to ferroptosis during cell-state transitions.
\subsection*{\href{https://www.nature.com/articles/s41586-020-2725-7}{Bridging of DNA breaks activates PARP2–HPF1 to modify chromatin}}
\subsubsection*{Silvija Bilokapic, \dots, and Mario Halic (2020-09-16T00:00:00)}
(Article)The PARP2–HPF1 histone-modifying complex bridges two nucleosomes to align broken DNA ends for ligation, initiating conformational changes that activate PARP2 and enable DNA damage repair.
\subsection*{\href{https://www.nature.com/articles/s41586-020-2650-9}{Biosynthesis of medicinal tropane alkaloids in yeast}}
\subsubsection*{Prashanth Srinivasan and Christina D. Smolke (2020-09-02T00:00:00)}
(Article)The alkaloid drugs hyoscyamine and scopolamine are synthesized from sugars and amino acids in yeast, using 26 genes from yeast, plants, bacteria and animals, protein engineering and a vacuole transporter to enable functional expression of a key acyltransferase.
\subsection*{\href{https://www.nature.com/articles/s41586-020-2674-1}{Zebrafish prrx1a mutants have normal hearts}}
\subsubsection*{Federico Tessadori, \dots, and Jeroen Bakkers (2020-09-23T00:00:00)}
(Matters Arising)
\subsection*{\href{https://www.nature.com/articles/s41586-020-2675-0}{Reply to: Zebrafish prrx1a mutants have normal hearts}}
\subsubsection*{Noemi Castroviejo, \dots, and M. Angela Nieto (2020-09-23T00:00:00)}
(Matters Arising)
\clearpage
\section{arXiv}
\subsection*{\href{http://arxiv.org/abs/2009.14197v1}{Revisiting the equality conditions of the data processing inequality for  the sandwiched Rényi divergence}}
\subsubsection*{Jinzhao Wang and Henrik Wilming (2020-09-29)}
We provide a transparent, simple and unified treatment of recent results on
the equality conditions for the data processing inequality (DPI) of the
sandwiched quantum R\'enyi divergence, including the statement that equality in
the data processing implies recoverability via the Petz recovery map for the
full range of $\alpha$ recently proven by Jen\v cov\'a. We also obtain a new
set of equality conditions, generalizing a previous result by Leditzky et al.

\subsection*{\href{http://arxiv.org/abs/2009.14189v1}{Duality and domain wall dynamics in a twisted Kitaev chain}}
\subsubsection*{C. M. Morris, \dots, and N. P. Armitage (2020-09-29)}
The Ising chain in transverse field is a paradigmatic model for a host of
physical phenomena, including spontaneous symmetry breaking, topological
defects, quantum criticality, and duality. Although the quasi-1D ferromagnet
CoNb$_2$O$_6$ has been put forward as the best material example of the
transverse field Ising model, it exhibits significant deviations from ideality.
Through a combination of THz spectroscopy and theory, we show that
CoNb$_2$O$_6$ in fact is well described by a different model with strong bond
dependent interactions, which we dub the {\it twisted Kitaev chain}, as these
interactions share a close resemblance to a one-dimensional version of the
intensely studied honeycomb Kitaev model. In this model the ferromagnetic
ground state of CoNb$_2$O$_6$ arises from the compromise between two distinct
alternating axes rather than a single easy axis. Due to this frustration, even
at zero applied field domain-wall excitations have quantum motion that is
described by the celebrated Su-Schriefer-Heeger model of polyacetylene. This
leads to rich behavior as a function of field. Despite the anomalous domain
wall dynamics, close to a critical transverse field the twisted Kitaev chain
enters a universal regime in the Ising universality class. This is reflected by
the observation that the excitation gap in CoNb$_2$O$_6$ in the ferromagnetic
regime closes at a rate precisely twice that of the paramagnet. This originates
in the duality between domain walls and spin-flips and the topological
conservation of domain wall parity. We measure this universal ratio `2' to high
accuracy -- the first direct evidence for the Kramers-Wannier duality in
nature.

\subsection*{\href{http://arxiv.org/abs/2009.14185v1}{CMOS-based cryogenic control of silicon quantum circuits}}
\subsubsection*{Xiao Xue, \dots, and Lieven M. K. Vandersypen (2020-09-29)}
The most promising quantum algorithms require quantum processors hosting
millions of quantum bits when targeting practical applications. A major
challenge towards large-scale quantum computation is the interconnect
complexity. In current solid-state qubit implementations, a major bottleneck
appears between the quantum chip in a dilution refrigerator and the room
temperature electronics. Advanced lithography supports the fabrication of both
CMOS control electronics and qubits in silicon. When the electronics are
designed to operate at cryogenic temperatures, it can ultimately be integrated
with the qubits on the same die or package, overcoming the wiring bottleneck.
Here we report a cryogenic CMOS control chip operating at 3K, which outputs
tailored microwave bursts to drive silicon quantum bits cooled to 20mK. We
first benchmark the control chip and find electrical performance consistent
with 99.99\% fidelity qubit operations, assuming ideal qubits. Next, we use it
to coherently control actual silicon spin qubits and find that the cryogenic
control chip achieves the same fidelity as commercial instruments. Furthermore,
we highlight the extensive capabilities of the control chip by programming a
number of benchmarking protocols as well as the Deutsch-Josza algorithm on a
two-qubit quantum processor. These results open up the path towards a fully
integrated, scalable silicon-based quantum computer.

\subsection*{\href{http://arxiv.org/abs/2009.14184v1}{Field theory of higher-order topological crystalline response,  generalized global symmetries and elasticity tetrads}}
\subsubsection*{Jaakko Nissinen (2020-09-29)}
We discuss the higher-order topological field theory and response of
topological crystalline insulators with no other symmetries. We show how the
topology and geometry of the system is organised in terms of the elasticity
tetrads which are ground state degrees of freedom labelling lattice topological
charges, higher-form conservation laws and responses on sub-dimensional
manifolds of the bulk system. In a crystalline insulator, they classify
higher-order global symmetries in a transparent fashion. This coincides with
the dimensional hierarchy of topological terms, the multipole expansion, and
anomaly inflow, related to a mixed number of elasticity tetrads and
electromagnetic gauge fields. In the continuum limit of the elasticity tetrads,
the semi-classical expansion can be used to derive the higher-order or embedded
topological responses to global U(1) symmetries, such as electromagnetic gauge
fields with explicit formulas for the higher-order quasi-topological invariants
in terms of the elasticity tetrads and Green's functions. The topological
responses and readily generalized in parameter space to allow for e.g.
multipole pumping. Our simple results further bridge the recently appreciated
connections between topological field theory, higher form symmetries and gauge
fields, fractonic excitations and topological defects with restricted mobility
elasticity in crystalline insulators.

\subsection*{\href{http://arxiv.org/abs/2009.14173v1}{Shot noise distinguishes Majorana fermions from vortices injected in the  edge mode of a chiral p-wave superconductor}}
\subsubsection*{C. W. J. Beenakker and D. O. Oriekhov (2020-09-29)}
The chiral edge modes of a topological superconductor support two types of
excitations: fermionic quasiparticles known as Majorana fermions and
$\pi$-phase domain walls known as edge vortices. Edge vortices are injected
pairwise into counter-propagating edge modes by a flux bias or voltage bias
applied to a Josephson junction. An unpaired edge mode carries zero electrical
current on average, but there are time-dependent current fluctuations. We
calculate the shot noise power produced by a sequence of edge vortices and find
that it increases logarithmically with their spacing - even if the spacing is
much larger than the core size so the vortices do not overlap. This nonlocality
produces an anomalous V log V increase of the shot noise in a voltage-biased
geometry, which serves as a distinguishing feature in comparison with the
linear-in-V Majorana fermion shot noise.

\subsection*{\href{http://arxiv.org/abs/2009.14172v1}{High-fidelity single-shot readout of single electron spin in diamond  with spin-to-charge conversion}}
\subsubsection*{Qi Zhang, \dots, and Jiangfeng Du (2020-09-29)}
High fidelity single-shot readout of qubits is a crucial component for
fault-tolerant quantum computing and scalable quantum networks. In recent
years, the nitrogen-vacancy (NV) center in diamond has risen as a leading
platform for the above applications. The current single-shot readout of the NV
electron spin relies on resonance fluorescence method at cryogenic temperature.
However, the the spin-flip process interrupts the optical cycling transition,
therefore, limits the readout fidelity. Here, we introduce a spin-to-charge
conversion method assisted by near-infrared (NIR) light to suppress the
spin-flip error. This method leverages high spin-selectivity of cryogenic
resonance excitation and high flexibility of photonionization. We achieve an
overall fidelity $>$ 95\% for the single-shot readout of an NV center electron
spin in the presence of high strain and fast spin-flip process. With further
improvements, this technique has the potential to achieve spin readout fidelity
exceeding the fault-tolerant threshold, and may also find applications on
integrated optoelectronic devices.

\subsection*{\href{http://arxiv.org/abs/2009.14166v1}{Cascaded superconducting junction refrigerators: optimization and  performance limits}}
\subsubsection*{A. Kemppinen, \dots, and M. Prunnila (2020-09-29)}
We demonstrate highly transparent vanadium-silicon and aluminium-silicon
tunnel junctions, where silicon is doped to remain conducting even in cryogenic
temperatures. We discuss using them in a cascaded electronic refrigerator with
two or more refrigeration stages, and where different superconducting gaps are
needed for different temperatures. The optimization of the whole cascade is a
multidimensional problem, but we present an approximative optimization
criterion that can be used as a figure of merit for a single stage only.

\subsection*{\href{http://arxiv.org/abs/2009.14161v1}{Tracking collective cell motion by topological data analysis}}
\subsubsection*{L. L. Bonilla, and C. Trenado (2020-09-29)}
By modifying and calibrating an active vertex model to experiments, we have
simulated numerically a confluent cellular monolayer spreading on an empty
space and the collision of two monolayers of different cells in an antagonistic
migration assay. Cells are subject to inertial forces and to active forces that
try to align their velocities with those of neighboring ones. In agreement with
experiments, spreading tests exhibit finger formation in the moving interfaces,
swirls in the velocity field, and the polar order parameter and correlation and
swirl lengths increase with time. Cells inside the tissue have smaller area
than those at the interface, as observed in recent experiments. In antagonistic
migration assays, a population of fluidlike Ras cells invades a population of
wild type solidlike cells having shape parameters above and below the geometric
critical value, respectively. Cell mixing or segregation depends on the
junction tensions between different cells. We reproduce experimentally observed
antagonistic migration assays by assuming that a fraction of cells favor
mixing, the others segregation, and that these cells are randomly distributed
in space. To characterize and compare the structure of interfaces between cell
types or of interfaces of spreading cellular monolayers in an automatic manner,
we apply topological data analysis to experimental data and to numerical
simulations. We use time series of numerical simulation data to automatically
group, track and classify advancing interfaces of cellular aggregates by means
of bottleneck or Wasserstein distances of persistent homologies. These
topological data analysis techniques are scalable and could be used in studies
involving large amounts of data. Besides applications to wound healing and
metastatic cancer, these studies are relevant for tissue engineering,
biological effects of materials, tissue and organ regeneration.

\subsection*{\href{http://arxiv.org/abs/2009.14159v1}{Quantum phase transitions in the spin-1 Kitaev-Heisenberg chain}}
\subsubsection*{Wen-Long You, \dots, and Andrzej M. Oleś (2020-09-29)}
Recently, it has been proposed that higher-spin analogues of the Kitaev
interactions $K>0$ may also occur in a number of materials with strong Hund's
and spin-orbit coupling. In this work, we use Lanczos diagonalization and
density matrix renormalization group methods to investigate numerically the
$S=1$ Kitaev-Heisenberg model. The ground-state phase diagram and quantum phase
transitions are investigated by employing local and nonlocal spin correlations.
We identified two ordered phases at negative Heisenberg coupling $J<0$:
a~ferromagnetic phase with $\langle S_i^zS_{i+1}^z\rangle>0$ and an
intermediate left-left-right-right phase with $\langle
S_i^xS_{i+1}^x\rangle\neq 0$. A~quantum spin liquid is stable near the Kitaev
limit, while a topological Haldane phase is found for $J>0$.

\subsection*{\href{http://arxiv.org/abs/2009.14147v1}{Graph Theory Based Approach to Characterize Self Interstitial Cluster  Morphologies}}
\subsubsection*{Utkarsh Bhardwaj, and Manoj Warrier (2020-09-29)}
Morphology of self interstitial atom (SIA) clusters formed after a collision
cascade is an important aspect of radiation damage. We present a method to
characterize the morphology of a cluster by precisely identifying its
constituent homogeneous components. The constituent components are identified
as parallel bundles of SIAs, rings and other configurations based on the
properties of alignment of the SIA lines and their neighborhood relationships.
We reduce the problem of decomposition of a cluster into components and
characterizing them into graph theory problems of finding connected components
and finding cycles in a graph representation of a cluster.
  The method is used to study over 1000 clusters formed in W collision cascades
for energies ranging from 50 keV to 200 keV. We show the typical cluster shapes
for each morphology type identified using the method and compare the structural
description with the results from dislocation analysis. The description is
found to be in agreement for components with big parallel bundle of SIA. We
demonstrate with examples that for other cases such as a mixed cluster, the
presented method provides a better description of the structural details. The
study gives statistical distribution of different morphologies across energies
and their properties such as component sizes and orientations.

\subsection*{\href{http://arxiv.org/abs/2009.14143v1}{Structural Phase Dependent Giant Interfacial Spin Transparency in  W/CoFeB Thin Film Heterostructure}}
\subsubsection*{Surya Narayan Panda, \dots, and Anjan Barman (2020-09-29)}
Pure spin current has transfigured the energy-efficient spintronic devices
and it has the salient characteristic of transport of the spin angular
momentum. Spin pumping is a potent method to generate pure spin current and for
its increased efficiency high effective spin-mixing conductance (Geff) and
interfacial spin transparency (T) are essential. Here, a giant T is reported in
Sub/W(t)/Co20Fe60B20(d)/SiO2(2 nm) heterostructures in \beta-tungsten (\beta-W)
phase by employing all-optical time-resolved magneto-optical Kerr effect
technique. From the variation of Gilbert damping with W and CoFeB thicknesses,
the spin diffusion length of W and spin-mixing conductances are extracted.
Subsequently, T is derived as 0.81 \pm 0.03 for the \beta-W/CoFeB interface. A
sharp variation of Geff and T with W thickness is observed in consonance with
the thickness-dependent structural phase transition and resistivity of W. The
spin memory loss and two-magnon scattering effects are found to have negligible
contributions to damping modulation as opposed to spin pumping effect which is
reconfirmed from the invariance of damping with Cu spacer layer thickness
inserted between W and CoFeB. The observation of giant interfacial spin
transparency and its strong dependence on crystal structures of W will be
important for pure spin current based spin-orbitronic devices.

\subsection*{\href{http://arxiv.org/abs/2009.14126v1}{Universal quantum computing using electro-nuclear wavefunctions of  rare-earth ions}}
\subsubsection*{Manuel Grimm, \dots, and Markus Müller (2020-09-29)}
We propose a scheme for universal quantum computing based on Kramers
rare-earth ions. Their nuclear spins in the presence of a Zeeman-split
electronic crystal field ground state act as 'passive' qubits which store
quantum information. The qubits can be activated optically by fast coherent
transitions to excited crystal field states with a magnetic moment. The dipole
interaction between these states is used to implement CNOT gates. We compare
our proposal with a similar one based on phosphorus donor atoms in silicon and
discuss the significantly improved CNOT gate time as compared to rare-earth
implementations via the slower dipole blockade.

\subsection*{\href{http://arxiv.org/abs/2009.14122v1}{The role of spin-flip assisted or orbital mixing tunneling on transport  through strongly correlated multilevel quantum dot}}
\subsubsection*{D. Krychowski and S. Lipiński (2020-09-29)}
Using the slave boson Kotliar-Ruckenstein approach (SBMFA) for N level
Anderson model, we compare fully symmetric SU(N) Kondo resonances occurring for
spin and orbital conserving tunneling with many-body resonances for the dot
with broken symmetry caused by spin, orbital or full spin-orbital mixing. As a
result of interorbital or spin flip processes new interference paths emerge,
which manifests in the occurrence of antibonding Dicke like and bonding Kondo
like resonances. The analytical expressions for linear conductances and linear
temperature thermopower coefficient for arbitrary N are found.

\subsection*{\href{http://arxiv.org/abs/2009.14118v1}{Electron Thermalization and Relaxation in Laser-Heated Nickel by  Few-Femtosecond Core-Level Transient Absorption Spectroscopy}}
\subsubsection*{Hung-Tzu Chang, \dots, and Stephen R. Leone (2020-09-29)}
Direct measurements of photoexcited carrier dynamics in nickel are made using
few-femtosecond extreme ultraviolet (XUV) transient absorption spectroscopy at
the nickel M$_{2,3}$ edge. It is observed that the core-level absorption
lineshape of photoexcited nickel can be described by a Gaussian broadening
($\sigma$) and a red shift ($\omega_{s}$) of the ground state absorption
spectrum. Theory predicts, and the experimental results verify that, after
initial rapid carrier thermalization, the electron temperature increase
($\Delta T$) is linearly proportional to the Gaussian broadening factor
$\sigma$, providing quantitative real-time tracking of the relaxation of the
electron temperature. Measurements reveal an electron cooling time for 50 nm
thick polycrystalline nickel films of 640$\pm$80 fs. With hot thermalized
carriers, the spectral red shift exhibits a power-law relationship with the
change in electron temperature of $\omega_{s}\propto\Delta T^{1.5}.$ Rapid
electron thermalization via carrier-carrier scattering accompanies and follows
the nominal 4 fs photoexcitation pulse until the carriers reach a quasi thermal
equilibrium. Entwined with a <6 fs instrument response function, a 13 fs
carrier thermalization time is estimated from the experimental data. The study
provides a prototypical example of measuring electron temperature and
thermalization in metals in real time, and it lays a foundation for further
investigation of photoinduced phase transitions and carrier transport in metals
with core-level absorption spectroscopy.

\subsection*{\href{http://arxiv.org/abs/2009.14107v1}{High-pressure induced magnetic phase transition in half-metallic  $\textbf{KBeO}_\textbf{3}$ perovskite}}
\subsubsection*{M. Hamlat, \dots, and H. Boutaleb (2020-09-29)}
In this paper, we present the study of the structural, mechanical,
magneto-electronic and thermodynamic properties of the perovskite KBeO$_3$. The
calculations were performed by the full potential augmented plane wave method,
implemented in the WIEN2k code which is based on density functional theory,
using generalized gradient approximation. The computed formation energy and
elastic constants indicate the synthesizability and mechanical stability of
KBeO$_3$. Moreover, our results showed that the latter is a half-metallic
material with half-metallic gap of 0.67 eV and an integer magnetic moment of
3$\mu_{\text{B}}$ per unit cell. In addition, KBeO$_3$ maintains the
half-metallic character under the pressure up to about 97 GPa corresponding to
the predicted magnetic-phase transition pressure from ferromagnetic to
non-magnetic state. The volume ratio $V/V_{0}$, bulk modulus, heat capacity,
thermal expansion and the Debye temperature are analyzed using the
quasi-harmonic Debye model.

\subsection*{\href{http://arxiv.org/abs/2009.14098v1}{Measuring ion oscillations at the quantum level with fluorescence light}}
\subsubsection*{G. Cerchiari, \dots, and R. Blatt (2020-09-29)}
We demonstrate an optical method for detecting the mechanical oscillations of
an atom with single-phonon sensitivity. The measurement signal results from the
interference between the light scattered by a single trapped atomic ion and
that of its mirror image. The motion of the atom modulates the interference
path length and hence the photon detection rate. We detect the oscillations of
the atom in the Doppler cooling limit and reconstruct average trajectories in
phase space. We demonstrate single-phonon sensitivity near the ground state of
motion after EIT cooling. These results could be applied for motion detection
of other light scatterers of fundamental interest, such as trapped
nanoparticles.

\subsection*{\href{http://arxiv.org/abs/2009.14095v1}{Spin transfer torque in Mn$_3$Ga-based ferrimagnetic tunnel junctions  from first principles}}
\subsubsection*{Maria Stamenova, \dots, and Stefano Sanvito (2020-09-29)}
We report on first principles calculations of spin-transfer torque (STT) in
epitaxial magnetic tunnel junctions (MTJs) based on ferrimagnetic tetragonal
Mn$_3$Ga electrodes, both as analyser in a Fe/MgO stack, and also in an
analogous stack with Mn$_3$Ga electrode (instead of Fe) as polariser. Solving
the ballistic transport problem (NEFG+DFT) for the non-equilibrium spin density
in a scattering region extended to over 7.6 nm into the Mn$_3$Ga electrode, we
find long-range spatial oscillations of the STT decaying on a length scale of a
few tens of Angstr\"oms, both in the linear response regime and for finite
bias. The oscillatory behavior of the STT in Mn$_3$Ga is robust against
variations in the stack geometry (e.g. the barrier thickness and the interface
spacing) and the applied bias voltage, which may affect the phase and the
amplitude of the spacial oscillation, but the high (carrier) frequency mode is
only responsive to variations in the longitudinal lattice constant of Mn$_3$Ga
(for fixed in-plane geometry) without being commensurate with the lattice. Our
interpretation of the long range STT oscillations is based on the bulk
electronic structure of Mn$_3$Ga, taking also into account the spin-filtering
properties of the MgO barrier. Comparison to a fully Mn$_3$Ga-based stack shows
similar STT oscillations but a significant enhancement of the TMR effect at the
Fermi level and the STT at the interface due to resonant tunneling. From the
calculated energy dependence of the spin-polarised transmissions at 0 V, we
anticipate asymmetric or symmetric TMR as a function of the applied bias
voltage for the Fe-based and the all-Mn$_3$Ga stacks, respectively, which also
both exhibit a sign change below $\pm1$ V. In the latter, symmetric, case we
expect a TMR peak at zero, which is larger for the thinner barriers because of
spin-polarised resonant tunneling.

\subsection*{\href{http://arxiv.org/abs/2009.14092v1}{Detection of odor quality and ripening stage of Mangifera indica L. by  graphdiyne nanosheet -- a DFT outlook}}
\subsubsection*{V. Nagarajan and R. Chandiramouli (2020-09-29)}
Using first-principles calculation, geometrical stability together with
electronic properties of graphdiyne nanosheet (Gdn-NS) is investigated. The
structural stability of Gdn-NS is established with the support of phonon band
structure and cohesive energy. The main objective of the present study is to
check the odor quality of Mangifera indica L. (mangoes) fruits during the
various ripening stage with the influence of Gdn-NS material. In addition, the
adsorption of various volatiles, namely ethyl butanoate, myrcene,
(E,Z,Z)-1,3,4,8-undecatetraene and $\gamma$-octalactone aromas on Gdn-NS is
explored with the significant parameters including Bader charge transfer,
energy gap, average energy gap changes and adsorption energy. The sensitivity
of volatiles emitting from various ripening stages of mango on Gdn-NS were
explored with the influence of density of states spectrum. The outcomes of the
proposed work help us to check the ripening stage and odor quality of Mangifera
indica L. by Gdn-NS material using density functional theory.

\subsection*{\href{http://arxiv.org/abs/2009.14090v1}{Classical Casimir free energy for two Drude spheres of arbitrary radii:  A plane-wave approach}}
\subsubsection*{Tanja Schoger and Gert-Ludwig Ingold (2020-09-29)}
We derive an exact analytic expression for the high-temperature limit of the
Casimir interaction between two Drude spheres of arbitrary radii. Specifically,
we determine the Casimir free energy by using the scattering approach in the
plane-wave basis. Within a round-trip expansion, we are led to consider the
combinatorics of certain partitions of the round trips. The relation between
the Casimir free energy and the capacitance matrix of two spheres is discussed.
Previously known results for the special cases of a sphere-plane geometry as
well as two spheres of equal radii are recovered. An asymptotic expansion for
small distances between the two spheres is determined and analytical
expressions for the coefficients are given.

\subsection*{\href{http://arxiv.org/abs/2009.14080v1}{Optimal covariant quantum measurements}}
\subsubsection*{Erkka Haapasalo and Juha-Pekka Pellonpää (2020-09-29)}
We discuss symmetric quantum measurements and the associated covariant
observables modelled, respectively, as instruments and positive-operator-valued
measures. The emphasis of this work are the optimality properties of the
measurements, namely, extremality, informational completeness, and the rank-1
property which contrast the complementary class of (rank-1) projection-valued
measures. The first half of this work concentrates solely on finite-outcome
measurements symmetric w.r.t. finite groups where we derive exhaustive
characterizations for the pointwise Kraus-operators of covariant instruments
and necessary and sufficient extremality conditions using these
Kraus-operators. We motivate the use of covariance methods by showing that
observables covariant with respect to symmetric groups contain a family of
representatives from both of the complementary optimality classes of
observables and show that even a slight deviation from a rank-1
projection-valued measure can yield an extreme informationally complete rank-1
observable. The latter half of this work derives similar results for continuous
measurements in (possibly) infinite dimensions. As an example we study
covariant phase space instruments, their structure, and extremality properties.

\subsection*{\href{http://arxiv.org/abs/2009.14076v1}{Noise induced effects at nano-structured thin films growth during  deposition in plasma-condensate devices}}
\subsubsection*{V. O. Kharchenko, and D. O. Kharchenko (2020-09-29)}
We perform a comprehensive study of noise-induced effects in a stochastic
model of reaction-diffusion type, describing nano-structured thin films growth
at condensation. We introduce an external flux of adsorbate between neighbour
monoatomic layers caused by the electrical field presence near substrate in
plasma-condensate devices. We take into account that the strength of the
electric field fluctuates around its mean value. We discuss a competing
influence of the regular and stochastic parts of the external flux onto the
dynamics of adsorptive system. It will be shown that the introduced
fluctuations induce first-order phase transition in a homogeneous system,
govern the pattern formation in a spatially extended system; these parts of the
flux control the dynamics of the patterning, spatial order, morphology of the
surface, growth law of the mean size of adsorbate islands, type and linear size
of surface structures. The influence of the intensity of fluctuations onto
scaling and statistical properties of the nano-structured surface is analysed
in detail. This study provides an insight into the details of noise induced
effects at pattern formation processes in anisotropic adsorptive systems.

\subsection*{\href{http://arxiv.org/abs/2009.14067v1}{Anomalous diffusion in umbrella comb}}
\subsubsection*{A. Iomin (2020-09-29)}
Anomalous transport in a circular comb is considered. The circular motion
takes place for a fixed radius, while radii are continuously distributed along
the circle. Two scenarios of the anomalous transport, related to the reflecting
and periodic angular boundary conditions, are studied. The first scenario with
the reflection boundary conditions for the circular diffusion corresponds to
the conformal mapping of a 2D comb Fokker-Planck equation on the circular comb.
This topologically constraint motion is named umbrella comb model. In this
case, the reflecting boundary conditions are imposed on the circular (rotator)
motion, while the radial motion corresponds to geometrical Brownian motion with
vanishing to zero boundary conditions on infinity. The radial diffusion is
described by the log-normal distribution, which corresponds to exponentially
fast motion with the mean squared displacement (MSD) of the order of $\exp(t)$.
The second scenario corresponds to the circular diffusion with periodic
boundary conditions and the outward radial diffusion with vanishing to zero
boundary conditions at infinity. In this case the radial motion is subdiffusion
with the MSD of the order of $t^{1/2}$. The circular motion in both scenarios
is a superposition of cosine functions that results in the stationary Bernoulli
polynomials for the probability distributions.

\subsection*{\href{http://arxiv.org/abs/2009.14055v1}{Modelling Mullins Effect Induced by Chain Delamination and Reattachment}}
\subsubsection*{Daoyuan Qian and Fanlong Meng (2020-09-29)}
We propose a continuum theory to model the Mullins effect, which is
ubiquitously observed in polymer composites. In the theory, the softening of
the materials during the stretching process is accounted for by considering the
delamination of polymer chains from nano-/micro-sized fillers, and the recovery
effect during the de-stretching process is due to the reattachment of the
polymer chains to nano-/micro-sized fillers. By incorporating the chain
entanglements, Log-Normal distribution of the mesh size in the network, etc.,
we can obtain a good agreement between our numerical calculation results and
existing experimental data. This physical theory can be easily adapted to meet
more practical needs and utilised in analysing mechanic properties of polymer
composites.

\subsection*{\href{http://arxiv.org/abs/2009.14051v1}{Dimensional Crossover Tuned by Pressure in Layered Magnetic NiPS3}}
\subsubsection*{Xiaoli Ma, \dots, and Xiaohui Yu (2020-09-29)}
Layered magnetic transition-metal thiophosphate NiPS3 has unique
two-dimensional (2D) magnetic properties and electronic behavior. The
electronic band structure and corresponding magnetic state are expected to
sensitive to the interlayer interaction, which can be tuned by external
pressure. Here, we report an insulator-metal transition accompanied with
magnetism collapse during the 2D-3D crossover in structure induced by
hydrostatic pressure. A two-stage phase transition from monoclinic (C2=m) to
trigonal (P-31m) lattice is identified by ab initio simulation and confirmed by
high-pressure XRD and Raman data, corresponding to a layer by layer slip
mechanism along the a-axis. Temperature dependence resistance measurements and
room temperature infrared spectroscopy show that the insulator-metal transition
occurs near 20 GPa as well as magnetism collapse, which is further confirmed by
low temperature Raman measurement and theoretical calculation. These results
establish a strong correlation among the structural change, electric transport,
and magnetic phase transition and expand our understandings about the layered
magnetic materials.

\subsection*{\href{http://arxiv.org/abs/2009.14049v1}{Quantum Borrmann effect for dissipation-immune photon-photon  correlations}}
\subsubsection*{Alexander V. Poshakinskiy and Alexander N. Poddubny (2020-09-29)}
We study theoretically the second-order correlation function $g^{(2)}(t)$ for
photons transmitted through a periodic Bragg-spaced array of superconducting
qubits, coupled to a waveguide. We demonstrate that photon bunching and
anti-bunching persist much longer than both radiative and non-radiative
lifetimes of a single qubit. The photon-photon correlations become immune to
non-radiative dissipation due to the Borrmann effect, that is a strongly
non-Markovian collective feature of light-qubit coupling inherent to the Bragg
regime. This persistence of quantum correlations opens new avenues for
enhancing the performance of setups of waveguide quantum electrodynamics.

\subsection*{\href{http://arxiv.org/abs/2009.14038v1}{Electron-phonon coupling of epigraphene at millikelvin temperatures}}
\subsubsection*{Bayan Karimi, \dots, and Sergey Kubatkin (2020-09-29)}
We investigate the basic charge and heat transport properties of charge
neutral epigraphene at sub-kelvin temperatures, demonstrating nearly
logarithmic dependence of electrical conductivity over more than two decades in
temperature. Using graphene's sheet conductance as in-situ thermometer, we
present a measurement of electron-phonon heat transport at mK temperatures and
show that it obeys the $T^4$ dependence characteristic for clean
two-dimensional conductor. Based on our measurement we predict the
noise-equivalent power of $\sim 10^{-22}~{\rm W}/\sqrt{{\rm Hz}}$ of
epigraphene bolometer at the low end of achievable temperatures.

\subsection*{\href{http://arxiv.org/abs/2009.14032v1}{Inertial self-propelled particles}}
\subsubsection*{Lorenzo Caprini and Umberto Marini Bettolo Marconi (2020-09-29)}
We study a self-propelled particle moving in a solvent with the active
Ornstein Uhlenbeck dynamics in the underdamped regime to evaluate the influence
of the inertia. We focus on the properties of potential-free and harmonically
confined underdamped active particles, studying how the single-particle
trajectories modify for different values of the drag coefficient. In both
cases, we solve the dynamics in terms of correlation matrices and steady-state
probability distribution functions revealing the explicit correlations between
velocity and active force. We also evaluate the influence of the inertia on the
time-dependent properties of the system, discussing the mean square
displacement and the time-correlations of particle positions and velocities.
Particular attention is devoted to the study of the Virial active pressure
unveiling the role of the inertia on this observable.

\subsection*{\href{http://arxiv.org/abs/2009.14031v1}{Selection rules of twistronic angles in 2D material flakes via  dislocation theory}}
\subsubsection*{Shuze Zhu, \dots, and Harley T. Johnson (2020-09-29)}
Interlayer rotation angle couples strongly to the electronic states of
twisted van der Waals layers. However, not every angle is energetically
favorable. Recent experiments on rotation-tunable electronics reveal the
existence of a discrete set of angles at which the rotation-tunable electronics
assume the most stable configurations. Nevertheless, a quantitative map for
locating these intrinsically preferred twist angles in twisted bilayer system
has not been available, posing challenges for the on-demand design of twisted
electronics that are intrinsically stable at desired twist angles. Here we
reveal a simple mapping between intrinsically preferred twist angles and
geometry of the twisted bilayer system, in the form of geometric scaling laws
for a wide range of intrinsically preferred twist angles as a function of only
geometric parameters of the rotating flake on a supporting layer. We reveal
these scaling laws for triangular and hexagonal flakes since they frequently
appear in chemical vapor deposition growth. We also present a general method
for handling arbitrary flake geometry. Such dimensionless scaling laws possess
universality for all kinds of two-dimensional material bilayer systems,
providing abundant opportunities for the on-demand design of intrinsic
"twistronics". For example, the set of increasing magic-sizes that
intrinsically prefers zero-approaching sequence of multiple magic-angles in
bilayer graphene system can be revealed.

\subsection*{\href{http://arxiv.org/abs/2009.14028v1}{Certification of highly entangled measurements and nonlocality via  scalable entanglement-swapping on quantum computers}}
\subsubsection*{Elisa Bäumer, and Armin Tavakoli (2020-09-29)}
Increasingly sophisticated quantum computers motivate the exploration of
their abilities in certifying genuine quantum phenomena. Here, we demonstrate
the power of state-of-the-art IBM quantum computers in correlation experiments
inspired by quantum networks. Our experiments feature up to 12 qubits and
require the implementation of paradigmatic Bell-State Measurements for scalable
entanglement-swapping. First, we demonstrate quantum communication advantages
in up to nine-qubit systems while only assuming that the quantum computer
operates on qubits. Harvesting these communication advantages, we are able to
certify 82 basis elements as entangled in a 512-outcome measurement. Then, we
relax the qubit assumption and consider quantum nonlocality in a scenario with
multiple independent entangled states arranged in a star configuration. We
report quantum violations of source-independent Bell inequalities for up to ten
qubits. Our results demonstrate the ability of quantum computers to outperform
classical limitations and certify scalable entangled measurements.

\subsection*{\href{http://arxiv.org/abs/2009.14008v1}{Mechanistic Understanding of Entanglement and Heralding in Cascade  Emitters}}
\subsubsection*{Kobra N. Avanaki and George C. Schatz (2020-09-29)}
Semiconductor quantum light sources are favorable for a wide range of quantum
photonic tasks, particularly quantum computing and quantum information
processing. Here we theoretically investigate the properties of quantum
emitters (QEs) as a source of entangled photons with practical quantum
properties including heralding of on-demand single photons. Through the
theoretical analysis, we characterize the properties of a cascade (biexciton)
emitter, including (1) studies of single-photon purity, (2) investigating the
first- and second- order correlation functions, and (3) determining the Schmidt
number of the entangled photons. The analytical expression derived for the
Schmidt number of the cascade emitters reveals a strong dependence on the ratio
of decay rates of the first and second photons. Looking into the joint spectral
density of the generated biphotons, we show how the purity and the degree of
entanglement are connected to the production of heralded single photons.
  Our model is further developed to include polarization effects, fine
structure splitting, and the emission delay between the exciton and biexciton
emission. The extended model offers more details about the underlying mechanism
of entangled photon production, and it provides additional degrees of freedom
for manipulating the system and characterizing purity of the output photon. The
theoretical investigations and the analysis provide a cornerstone for the
experimental design and engineering of on-demand single photons.

\subsection*{\href{http://arxiv.org/abs/2009.14000v1}{Pitfalls and solutions for perovskite transparent conductors}}
\subsubsection*{Liang Si, \dots, and Karsten Held (2020-09-29)}
Transparent conductors-nearly an oxymoron-are in pressing demand, as
ultra-thin-film technologies become ubiquitous commodities. As current
solutions rely on non-abundant elements, perovskites such as SrVO3 and SrNbO3
have been suggested as next generation transparent conductors. Our ab-initio
calculations and analytical insights show, however, that reducing the plasma
frequency below the visible spectrum by strong electronic correlations-a
recently proposed strategy-unavoidably comes at a price: an enhanced scattering
and thus a substantial optical absorption above the plasma edge. As a way out
of this dilemma we identify several perovskite transparent conductors, relying
on hole doping, somewhat larger bandwidths and separations to other bands.

\subsection*{\href{http://arxiv.org/abs/2009.13992v1}{Strongly-bound excitons and trions in anisotropic 2D semiconductors}}
\subsubsection*{Sangho Yoon, \dots, and Jonghwan Kim (2020-09-29)}
Monolayer and few-layer phosphorene are anisotropic quasi-two-dimensional
(quasi-2D) van der Waals (vdW) semiconductors with a linear-dichroic
light-matter interaction and a widely-tunable direct-band gap in the infrared
frequency range. Despite recent theoretical predictions of strongly-bound
excitons with unique properties, it remains experimentally challenging to probe
the excitonic quasiparticles due to the severe oxidation during device
fabrication. In this study, we report observation of strongly-bound excitons
and trions with highly-anisotropic optical properties in intrinsic bilayer
phosphorene, which are protected from oxidation by encapsulation with hexagonal
boron nitride (hBN), in a field-effect transistor (FET) geometry. Reflection
contrast and photoluminescence spectroscopy clearly reveal the linear-dichroic
optical spectra from anisotropic excitons and trions in the hBN-encapsulated
bilayer phosphorene. The optical resonances from the exciton Rydberg series
indicate that the neutral exciton binding energy is over 100 meV even with the
dielectric screening from hBN. The electrostatic injection of free holes
enables an additional optical resonance from a positive trion (charged exciton)
~ 30 meV below the optical bandgap of the charge-neutral system. Our work shows
exciting possibilities for monolayer and few-layer phosphorene as a platform to
explore many-body physics and novel photonics and optoelectronics based on
strongly-bound excitons with two-fold anisotropy.

\subsection*{\href{http://arxiv.org/abs/2009.13983v1}{Carrier diffusion in GaN -- a cathodoluminescence study. II: Ambipolar  vs. exciton diffusion}}
\subsubsection*{Oliver Brandt, \dots, and Uwe Jahn (2020-09-29)}
We determine the diffusion length of excess carriers in GaN by spatially
resolved cathodoluminescence spectroscopy utilizing a single quantum well as
carrier collector or carrier sink. Monochromatic intensity profiles across the
quantum well are recorded for temperatures between 10 and 300 K. A classical
diffusion model accounts for the profiles acquired between 120 and 300 K, while
for temperatures lower than 120 K, a quantum capture process has to be taken
into account in addition. Combining the diffusion length extracted from these
profiles and the effective carrier lifetime measured by time-resolved
photoluminescence experiments, we deduce the carrier diffusivity as a function
of temperature. The experimental values are found to be close to theoretical
ones for the ambipolar diffusivity of free carriers limited only by intrinsic
phonon scattering. This agreement is shown to be fortuitous. The high
diffusivity at low temperatures instead originates from an increasing
participation of excitons in the diffusion process.

\subsection*{\href{http://arxiv.org/abs/2009.13969v1}{Cold Atom Quantum Simulator for String and Hadron Dynamics in  Non-Abelian Lattice Gauge Theory}}
\subsubsection*{Raka Dasgupta and Indrakshi Raychowdhury (2020-09-29)}
We propose an analog quantum simulator for simulating real time dynamics of
$(1+1)$-d non-Abelian gauge theory well within the existing capacity of
ultracold atom experiments. The scheme calls for the realization of a two-state
ultracold fermionic system in a 1-dimensional bipartite lattice, and the
observation of subsequent tunneling dynamics. Being based on novel loop string
hadron formalism of SU(2) lattice gauge theory, this simulation technique is
completely SU(2) invariant and simulates accurate dynamics of physical
phenomena such as string breaking and/or pair production. The scheme is
scalable, and particularly effective in simulating the theory in weak coupling
regime, and also bulk limit of the theory in strong coupling regime up to
certain approximations. This paper also presents a numerical benchmark
comparison of exact spectrum and real time dynamics of lattice gauge theory to
that of the atomic Hamiltonian with experimentally realizable range of
parameters.

\subsection*{\href{http://arxiv.org/abs/2009.13966v1}{Influence of molecular beam effusion cell quality on optical and  electrical properties of quantum dots and quantum wells}}
\subsubsection*{G. N. Nguyen, \dots, and A. Ludwig (2020-09-29)}
Quantum dot heterostructures with excellent low-noise properties became
possible with high purity materials recently. We present a study on molecular
beam epitaxy grown quantum wells and quantum dots with a contaminated aluminum
evaporation cell, which introduced a high amount of impurities, perceivable in
anomalies in optical and electrical measurements. We describe a way of
addressing this problem and find that reconditioning the aluminum cell by
overheating can lead to a full recovery of the anomalies in photoluminescence
and capacitance-voltage measurements, leading to excellent low noise
heterostructures. Furthermore, we propose a method to sense photo-induced trap
charges using capacitance-voltage spectroscopy on self-assembled quantum dots.
Excitation energy-dependent ionization of defect centers leads to shifts in
capacitance-voltage spectra which can be used to determine the charge density
of photo-induced trap charges via 1D band structure simulations. This method
can be performed on frequently used quantum dot diode structures.

\subsection*{\href{http://arxiv.org/abs/2009.13945v1}{Ferrofluidic aqueous two-phase system with ultralow interfacial tension,  instabilities and pattern formation}}
\subsubsection*{Carlo Rigoni, \dots, and Jaakko V. I. Timonen (2020-09-29)}
Ferrofluids are strongly magnetic fluids consisting of magnetic nanoparticles
dispersed in a carrier fluid. Besides their technological applications, they
have a tendency to form beautiful and intriguing patterns when subjected to
external static and dynamic magnetic fields. Most of the patterns occur in
systems consisting of two fluids: one ferrofluidic and one non-magnetic (oil,
air, etc.), wherein the fluid-fluid interface deforms as a response to magnetic
fields. Usually, the fluids are completely immiscible and so the interfacial
energy in this systems is very large. Here we show that it is possible to
design a fully aqueous ferrofluid system by using phase separation of
incompatible polymers. This continuous aqueous system allows an ultralow
interfacial tension (down to 1 $\mu$N/m) and nearly vanishing pinning at three
phase contact lines. We demonstrate the normal-field instability with the
system and focus on the miniaturization of the pattern length from the typical
$\sim$10 mm size down to $\sim$200 $\mu$m. The normal-field instability is
characterized in glass capillaries of thickness comparable to the pattern
length. This system paves way towards interesting physics such as the
interaction between magnetic instabilities and thermal capillary waves and
offers a way to evaluate extremely small interfacial tensions.

\subsection*{\href{http://arxiv.org/abs/2009.13944v1}{Dispersive readout of reconfigurable ambipolar quantum dots in a  silicon-on-insulator nanowire}}
\subsubsection*{Jingyu Duan, \dots, and John J. L. Morton (2020-09-29)}
We report on ambipolar gate-defined quantum dots in silicon on insulator
(SOI) nanowires fabricated using a customised complementary
metal-oxide-semiconductor (CMOS) process. The ambipolarity was achieved by
extending a gate over an intrinsic silicon channel to both highly doped n-type
and p-type terminals. We utilise the ability to supply ambipolar carrier
reservoirs to the silicon channel to demonstrate an ability to reconfigurably
define, with the same electrodes, double quantum dots with either holes or
electrons. We use gate-based reflectometry to sense the inter-dot charge
transition(IDT) of both electron and hole double quantum dots, achieving a
minimum integration time of 160(100) $\mu$s for electrons (holes). Our results
present the opportunity to combine, in a single device, the long coherence
times of electron spins with the electrically controllable holes spins in
silicon.

\subsection*{\href{http://arxiv.org/abs/2009.13943v1}{Quantum mechanics of round magnetic electron lenses with Glaser and  power law models of $B(z)$}}
\subsubsection*{Sameen Ahmed Khan and Ramaswamy Jagannathan (2020-09-29)}
Scalar theory of quantum electron beam optics, at the single-particle level,
derived from the Dirac equation using a Foldy-Wouthuysen-like transformation
technique is considered. Round magnetic electron lenses with Glaser and power
law models for the axial magnetic field $B(z)$ are studied. Paraxial quantum
propagator for the Glaser model lens is obtained in terms of the well known
fundamental solutions of its paraxial equation of motion. In the case of lenses
with the power law model for $B(z)$ the well known fundamental solutions of the
paraxial equations, obtained by solving the differential equation, are
constructed using the Peano-Baker series also. Quantum mechanics of aberrations
is discussed briefly. Role of quantum uncertainties in aberrations, and in the
nonlinear part of the equations of motion for a nonparaxial beam, is pointed
out. The main purpose of this article is to understand the quantum mechanics of
electron beam optics though the influence of quantum effects on the optics of
present-day electron beam devices might be negligible.

\subsection*{\href{http://arxiv.org/abs/2009.13942v1}{Quantum reflection of a Bose-Einstein condensate from a rapidly varying  potential: the role of dark soliton}}
\subsubsection*{Dongmei Wang, and Tao Yang (2020-09-29)}
We study the dynamic behavior of a Bose-Einstein condensate (BEC) containing
a dark soliton separately reflected from potential drops and potential
barriers. It is shown that for a rapidly varying potential and in a certain
regime of incident velocity, the quantum reflection probability displays the
cosine of the deflection angle between the incident soliton and the reflected
soliton, i.e., $R(\theta) \sim \cos 2\theta$. For a potential drop, $R(\theta)$
is susceptible to the widths of potential drop up to the length of the dark
soliton and the difference of the reflection rates between the orientation
angle of the soliton $\theta=0$ and $\theta=\pi/2$, $\delta R_s$, displays
oscillating exponential decay with increasing potential widths. However, for a
barrier potential, $R(\theta)$ is insensitive for the potential width less than
the decay length of the matter wave and $\delta R_s$ presents an exponential
trend. This discrepancy of the reflectances in two systems is arisen from the
different behaviors of matter waves in the region of potential variation.

\subsection*{\href{http://arxiv.org/abs/2009.13936v1}{Band Depopulation of Graphene Nanoribbons Induced by Chemical Gating  with Amino Groups}}
\subsubsection*{Jingcheng Li, \dots, and Jose Ignacio Pascual (2020-09-29)}
The electronic properties of graphene nanoribbons (GNRs) can be precisely
tuned by chemical doping. Here we demonstrate that amino (NH$_2$) functional
groups attached at the edges of chiral GNRs (chGNRs) can efficiently gate the
chGNRs and lead to the valence band (VB) depopulation on a metallic surface.
The NH$_2$-doped chGNRs are grown by on-surface synthesis on Au(111) using
functionalized bianthracene precursors. Scanning tunneling spectroscopy
resolves that the NH$_2$ groups significantly up-shift the bands of chGNRs,
causing the Fermi level crossing of the VB onset of chGNRs. Through density
functional theory simulations we confirm that the hole-doping behavior is due
to an upward shift of the bands induced by the edge NH$_2$ groups.

\subsection*{\href{http://arxiv.org/abs/2009.13933v1}{Optical normal-mode induced phonon-sideband splitting in photon blockade  effect}}
\subsubsection*{Hong Deng, \dots, and Jie-Qiao Liao (2020-09-29)}
We study photon blockade effect in a loop-coupled double-cavity
optomechanical system consisting of two cavity modes and one mechanical mode.
Here, the mechanical mode is optomechanically coupled to the two cavity modes,
which are coupled with each other via a photon-hopping interaction. By treating
the photon-hopping interaction as a perturbation, we obtain the analytical
results of the eigenvalues and eigenstates of the system in the subspaces
associated with zero, one, and two photons. We find a phenomenon of optical
normal-mode induced phonon-sideband splitting in the photon blockade effect by
numerically calculating the second-order correlation function of the cavity
fields. This work not only presents a method to choose optimal driving
frequency of photon blockade by tuning the photon-hopping interaction, but also
provides a means to characterize the normal-mode splitting with quantum
statistics of cavity photons.

\subsection*{\href{http://arxiv.org/abs/2009.13932v1}{Proof of a nonequilibrium pattern-recognition phase transition in open  quantum multimode Dicke models}}
\subsubsection*{Federico Carollo and Igor Lesanovsky (2020-09-29)}
Understanding the interaction between light and matter in out-of-equilibrium
quantum systems is a problem of fundamental interest in physics. To investigate
such scenarios, paradigmatic theoretical models, so-called open (dissipative)
multimode Dicke systems, have been introduced. Albeit being structurally
simple, these models can show intriguing behavior. For instance, it was
suggested that open Dicke models may behave as neural networks, displaying
dynamics reminiscent of an associative memory. However, uncovering whether the
onset of such nonequilibrium behavior can be associated with a proper phase of
matter is challenging, since it requires the knowledge of the stationary state
of a many-body quantum system, with several interacting continuous and discrete
degrees of freedom. Here, we prove the existence of a nonequilibrium phase
transition towards a pattern-recognition phase in open multimode Dicke models.
To uncover this emergent behavior, we prove the validity of the mean-field
assumption for these systems. This general result, which to the best of our
knowledge was not yet rigorously established, implies that the semi-classical
treatment of open multimode Dicke models is exact -in the thermodynamic limit.

\subsection*{\href{http://arxiv.org/abs/2009.13930v1}{Simultaneous polydirectional transport of colloidal bipeds}}
\subsubsection*{Mahla Mirzaee-Kakhki, \dots, and Thomas M. Fischer (2020-09-29)}
Detailed control over the motion of colloidal particles is relevant in many
applications in colloidal science such as lab-on-a-chip devices. Here, we use
an external magnetic field to assemble paramagnetic colloidal spheres into
colloidal rods of several lengths. The rods reside above a square magnetic
pattern and are transported via modulation of the direction of the external
magnetic field. The rods behave like bipeds walking above the pattern.
Depending on their length, the bipeds perform topologically distinct classes of
protected walks above the pattern. We demonstrate that it is possible to design
parallel polydirectional modulation loops of the external field that command up
to six classes of bipeds to walk on distinct predesigned paths. We use such
parallel polydirectional loops to induce the collision of reactant bipeds,
their polymerization addition reaction to larger bipeds, the separation of
product bipeds from the educts, the sorting of different product bipeds, and
also the parallel writing of a word consisting of several different letters.

\subsection*{\href{http://arxiv.org/abs/2009.13915v1}{Fully autocompensating high-dimensional quantum cryptography by optical  phase conjugation}}
\subsubsection*{Jesús Liñares, \dots, and Gabriel M. Carral (2020-09-29)}
We present a bidirectional quantum communication system based on optical
phase conjugation for achieving fully autocompensating high-dimensional quantum
cryptography. We prove that random phase shifts and couplings among 2N spatial
and polarization optical modes described by SU(2N) transformations due to
perturbations are autocompensated after a single round trip between Alice and
Bob. Bob can use a source of single photons or, alternatively, coherent states
and then Alice attenuates them up to a single photon level, and thus
non-perturbated 1-qudit states are generated for high-dimensional QKD
protocols, as the BB84 one, of a higher security.

\subsection*{\href{http://arxiv.org/abs/2009.13910v1}{Selective-area van der Waals epitaxy of h-BN/graphene heterostructures  via He$^{+}$ irradiation-induced defect-engineering in 2D substrates}}
\subsubsection*{Martin Heilmann, \dots, and J. Marcelo J. Lopes (2020-09-29)}
The combination of two-dimensional (2D) materials into heterostructures
enabled the formation of atomically thin devices with designed properties. To
achieve a high density, bottom-up integration, the growth of these 2D
heterostructures via van der Waals epitaxy (vdWE) is an attractive alternative
to the currently mostly employed mechanical transfer, which is still
problematic in terms of scaling and reproducibility. However, controlling the
location of the nuclei formation remains a key challenge in vdWE. Here, we use
a focused He ion beam for a deterministic placement of defects in graphene
substrates, which act as preferential nucleation sites for the growth of
insulating, 2D hexagonal boron nitride (h-BN). We demonstrate a mask-free,
selective-area vdWE (SAvdWE), where nucleation yield and crystal quality of
h-BN is controlled by the ion beam parameter used for the defect formation.
Moreover, we show that h-BN grown via SAvdWE has electron tunneling
characteristics comparable to those of mechanically transferred layers, thereby
lying the foundation for a reliable, high density array fabrication of 2D
heterostructures for device integration via defect engineering in 2D
substrates.

\subsection*{\href{http://arxiv.org/abs/2009.13908v1}{What does the world look like according to superdeterminism?}}
\subsubsection*{Augustin Baas and Baptiste Le Bihan (2020-09-29)}
The violation of Bell inequalities seems to establish an important fact about
the world: that it is non-local. However, this result relies on the assumption
of the statistical independence of the measurement settings with respect to
potential past events that might have determined them. Superdeterminism refers
to the view that a local, and determinist, account of Bell inequalities
violations is possible, by rejecting this assumption of statistical
independence. We examine and clarify various problems with superdeterminism,
looking in particular at its consequences on the nature of scientific laws and
scientific reasoning. We argue that the view requires a neo-Humean account of
at least some laws, and creates a significant problem for the use of
statistical independence in other parts of physics and science more generally.

\subsection*{\href{http://arxiv.org/abs/2009.13873v1}{A map between time-dependent and time-independent quantum many-body  Hamiltonians}}
\subsubsection*{Oleksandr Gamayun and Oleg Lychkovskiy (2020-09-29)}
Given a time-independent Hamiltonian $\widetilde H$, one can construct a
time-dependent Hamiltonian $H_t$ by means of the gauge transformation $H_t=U_t
\widetilde H_t \, U^\dagger_t-i\, U_t\, \partial_t U_t^\dagger$. Here $U_t$ is
the unitary transformation that relates the solutions of the corresponding
Schrodinger equations. In the many-body case one is usually interested in
Hamiltonians with few-body (often, at most two-body) interactions. We refer to
such Hamiltonians as "physical". We formulate sufficient conditions on $U_t$
ensuring that $H_t$ is physical as long as $\widetilde H$ is physical (and vice
versa). This way we obtain a general method for finding such pairs of physical
Hamiltonians $H_t$, $\widetilde H$ that the driven many-body dynamics governed
by $H_t$ can be reduced to the quench dynamics due to the time-independent
$\widetilde H$. We apply this method to a number of many-body systems. First we
review the mapping of a spin system with isotropic Heisenberg interaction and
arbitrary time-dependent magnetic field to the time-independent system without
a magnetic field [F. Yan, L. Yang, B. Li, Phys. Lett. A 251, 289 (1999); Phys.
Lett. A 259, 207 (1999)]. Then we demonstrate that essentially the same gauge
transformation eliminates an arbitrary time-dependent magnetic field from a
system of interacting fermions. Further, we apply the method to the quantum
Ising spin system and a spin coupled to a bosonic environment. We also discuss
a more general situation where $\widetilde H = \widetilde H_t$ is
time-dependent but dynamically integrable.

\subsection*{\href{http://arxiv.org/abs/2009.13867v1}{Magnetic proximity effect on excitonic spin states in Mn-doped layered  hybrid perovskites}}
\subsubsection*{Timo Neumann, \dots, and Felix Deschler (2020-09-29)}
Materials combining the optoelectronic functionalities of semiconductors with
control of the spin degree of freedom are highly sought after for the
advancement of quantum technology devices. Here, we report the paramagnetic
Ruddlesden-Popper hybrid perovskite Mn:(PEA)2PbI4 (PEA = phenethylammonium) in
which the interaction of isolated Mn2+ ions with magnetically brightened
excitons leads to circularly polarized photoluminescence. Using a combination
of superconducting quantum interference device (SQUID) magnetometry and
magneto-optical experiments, we find that the Brillouin-shaped polarization
curve of the photoluminescence follows the magnetization of the material. This
indicates coupling between localized manganese magnetic moments and exciton
spins via a magnetic proximity effect. The saturation polarization of 15\% at 4
K and 6 T indicates a highly imbalanced spin population and demonstrates that
manganese doping enables efficient control of excitonic spin states in
Ruddlesden-Popper perovskites. Our finding constitutes the first example of
polarization control in magnetically doped hybrid perovskites and will
stimulate research on this highly tuneable material platform that promises
tailored interactions between magnetic moments and electronic states.

\subsection*{\href{http://arxiv.org/abs/2009.13865v1}{Quantum copy-protection of compute-and-compare programs in the quantum  random oracle model}}
\subsubsection*{Andrea Coladangelo, and Alexander Poremba (2020-09-29)}
Copy-protection allows a software distributor to encode a program in such a
way that it can be evaluated on any input, yet it cannot be "pirated" - a
notion that is impossible to achieve in a classical setting. Aaronson (CCC
2009) initiated the formal study of quantum copy-protection schemes, and
speculated that quantum cryptography could offer a solution to the problem
thanks to the quantum no-cloning theorem. In this work, we introduce a quantum
copy-protection scheme for a large class of evasive functions known as
"compute-and-compare programs" - a more expressive generalization of point
functions. A compute-and-compare program $\mathsf{CC}[f,y]$ is specified by a
function $f$ and a string $y$ within its range: on input $x$,
$\mathsf{CC}[f,y]$ outputs $1$, if $f(x) = y$, and $0$ otherwise. We prove that
our scheme achieves non-trivial security against fully malicious adversaries in
the quantum random oracle model (QROM), which makes it the first
copy-protection scheme to enjoy any level of provable security in a standard
cryptographic model. As a complementary result, we show that the same scheme
fulfils a weaker notion of software protection, called "secure software
leasing", introduced very recently by Ananth and La Placa (eprint 2020), with a
standard security bound in the QROM, i.e. guaranteeing negligible adversarial
advantage.

\subsection*{\href{http://arxiv.org/abs/2009.13850v1}{A new approach to evaluate the elastic modulus of metallic foils}}
\subsubsection*{C. O. W. Trost, \dots, and M. J. Cordill (2020-09-29)}
The accurate determination of the elastic properties is non-trivial for
metallic foils. The measured elastic modulus is often described in literature
as significantly smaller than the respective modulus of the bulk counterparts.
This paper describes a straight forward way to minimize the influence of the
measurement on the calculated values using a standard tensile testing device
combined with a laser speckle extensometer and advanced data evaluation. The
elasticmodulus obtained with a monotonic tensile procedure is compared to
values obtained frommultiple loading unloading cycles in the elastic regime.
The latter were found to lead to more reproducible and reasonable values with
significantly smaller standard deviations. To enable interpretation of the
modulus values electron backscatter diffraction and X ray diffraction were used
to emphasize the effects of texture and microstructure. Two different
electrodeposited copper foilswere characterized since copper iswidely used in
many industrial applications.

\subsection*{\href{http://arxiv.org/abs/2009.13842v1}{Three measures of fidelity for photon states}}
\subsubsection*{Iwo Bialynicki-Birula and Zofia Bialynicka-Birula (2020-09-29)}
We show that the standard method of introducing the quantum description of
the electromagnetic field -- by canonical field quantization -- is not the only
one. We have chosen here the relativistic quantum mechanics of the photon as
the starting point. The treatment of photons as elementary particles merges
smoothly with the description in terms of the quantized electromagnetic field
but it also reveals some essential differences. The most striking result is the
appearance of various measures of fidelity for quantum states of photons. These
measures are used to characterize the localization of photons.

\subsection*{\href{http://arxiv.org/abs/2009.13834v1}{Equilibrium orientation and adsorption of an ellipsoidal Janus particle  at a fluid-fluid interface}}
\subsubsection*{Florian Günther, and Jens Harting (2020-09-29)}
We investigate the equilibrium orientation and adsorption process of a
single, ellipsoidal Janus particle at a fluid-fluid interface. The particle
surface comprises equally sized parts that are hydrophobic or hydrophilic. We
present free energy models to predict the equilibrium orientation and compare
the theoretical predictions with lattice Boltzmann simulations. We find that
the deformation of the fluid interface strongly influences the equilibrium
orientation of the Janus ellipsoid. The adsorption process of the Janus
ellipsoid can lead to different final orientations determined by the interplay
of particle aspect ratio and particle wettablity contrast.

\subsection*{\href{http://arxiv.org/abs/2009.13822v1}{Evolution of the structural, magnetic and electronic properties of the  triple perovskite Ba$_{3}$CoIr$_{2}$O$_{9}$}}
\subsubsection*{Charu Garg, \dots, and Sunil Nair (2020-09-29)}
We report a comprehensive investigation of the triple perovskite iridate
Ba$_{3}$CoIr$_{2}$O$_{9}$. Stabilizing in the hexagonal $P6_{3}/mmc$ symmetry
at room temperature, this system transforms to a monoclinic $C2/c$ symmetry at
the magnetic phase transition. On further reduction in temperature, the system
partially distorts to an even lower symmetry ($P2/c$), with both these
structurally disparate phases coexisting down to the lowest measured
temperatures. The magnetic structure as determined from neutron diffraction
data indicates a weakly canted antiferromagnetic structure, which is also
supported by first-principles calculations. Theory indicates that the Ir$^{5+}$
carries a finite magnetic moment, which is also consistent with the neutron
data. This suggests that the putative $J=0$ state is avoided. Measurements of
heat capacity, electrical resistance noise and dielectric susceptibility all
point towards the stabilization of a highly correlated ground state in the
Ba$_{3}$CoIr$_{2}$O$_{9}$ system.

\subsection*{\href{http://arxiv.org/abs/2009.13785v1}{Effective Langevin equations leading to large deviation function of  time-averaged velocity for a nonequilibrium Rayleigh piston}}
\subsubsection*{Masato Itami, \dots, and Shin-ichi Sasa (2020-09-29)}
We study fluctuating dynamics of a freely movable piston that separates an
infinite cylinder into two regions filled with ideal gas particles at the same
pressure but different temperatures. To investigate statistical properties of
the time-averaged velocity of the piston in the long-time limit, we
perturbatively calculate the large deviation function of the time-averaged
velocity. Then, we derive an infinite number of effective Langevin equations
yielding the same large deviation function as in the original model. Finally,
we provide two possibilities for uniquely determining the form of the effective
model.

\subsection*{\href{http://arxiv.org/abs/2009.13776v1}{Spectral properties and enhanced superconductivity in renormalized  Migdal-Eliashberg theory}}
\subsubsection*{Benjamin Nosarzewski, and Thomas P. Devereaux (2020-09-29)}
Migdal-Eliashberg theory describes the properties of the normal and
superconducting states of electron-phonon mediated superconductors based on a
perturbative treatment of the electron-phonon interactions. It is necessary to
include both electron and phonon self-energies self-consistently in
Migdal-Eliashberg theory in order to match numerically exact results from
determinantal quantum Monte Carlo in the adiabatic limit. In this work we
provide a method to obtain the real-axis solutions of the Migdal-Eliashberg
equations with electron and phonon self-energies calculated self-consistently.
Our method avoids the typical challenge of computing cumbersome singular
integrals on the real axis and is numerically stable and exhibits fast
convergence. Analyzing the resulting real-frequency spectra and self-energies
of the two-dimensional Holstein model, we find that self-consistently including
the lowest-order correction to the phonon self-energy significantly affects the
solution of the Migdal-Eliashberg equations. The calculation captures the
broadness of the spectral function, renormalization of the phonon dispersion,
enhanced effective electron-phonon coupling strength, minimal increase in the
electron effective mass, and the enhancement of superconductivity which
manifests as a superconducting ground state despite strong competition with
charge-density-wave order. We discuss surprising differences in two common
definitions of the electron-phonon coupling strength derived from the electron
mass and the density of states, quantities which are accessible through
experiments such as angle-resolved photoemission spectroscopy and electron
tunneling. An approximate upper bound on $2\Delta / T_c$ for conventional
superconductors mediated by retarded electron-phonon interactions is proposed.

\subsection*{\href{http://arxiv.org/abs/2009.13753v1}{Spin splitting and spin Hall conductivity in buckled monolayers of the  group 14: first-principles calculations}}
\subsubsection*{S. M. Farzaneh and Shaloo Rakheja (2020-09-29)}
Elemental monolayers of the group 14 with a buckled honeycomb structure,
namely silicene, germanene, stanene, and plumbene, are known to demonstrate a
spin splitting as a result of an electric field parallel to their high symmetry
axis which is capable of tuning their topological phase between a quantum spin
Hall insulator and an ordinary band insulator. We perform first-principles
calculations based on the density functional theory to quantify the
spin-dependent band gaps and the spin splitting as a function of the applied
electric field and extract the main coefficients of the invariant Hamiltonian.
Using the linear response theory and the Wannier interpolation method, we
calculate the spin Hall conductivity in the monolayers and study its
sensitivity to an external electric field. Our results show that the spin Hall
conductivity is not quantized and in the case of silicene, germanene, and
stanene degrades significantly as the electric field inverts the band gap and
brings the monolayer into the trivial phase. The electric field induced band
gap does not close in the case of plumbene which shows a spin Hall conductivity
that is robust to the external electric field.

\subsection*{\href{http://arxiv.org/abs/2009.13708v1}{Extremely High Thermal Conductivity of Aligned Polyacetylene Predicted  using First-Principles-Informed United-Atom Force Field}}
\subsubsection*{Teng Zhang, and Tengfei Luo (2020-09-29)}
Molecular simulations of polymer rely on accurate force fields to describe
the inter-atomic interactions. In this work, we use first-principles density
functional theory (DFT) calculations to parameterize a united-atom force field
for polyacetylene (PA), a conjugated polymer potentially of high thermal
conductivity. Different electron correlation functionals in DFT have been
tested. Bonding interactions for the alternating single and double bonds in the
conjugated polymer backbone are explicitly described and Class II anharmonic
functions are separately parameterized. Bond angle and dihedral interactions
are also anharmonic and parameterized against DFT energy surfaces. The
established force field is then used in molecular dynamics (MD) simulations to
calculate the thermal conductivity of single PA chains and PA crystals with
different simulation domain lengths. It is found that the thermal conductivity
values of both PA single chains and crystals are very high and are
length-dependent. At 790 nm, their respective thermal conductivity values are
~480 Wm-1K-1 and ~320 Wm-1K-1, which are comparatively higher than those of
polyethylene (PE), the most thermally conductive polymer fiber measured
up-to-date.

\subsection*{\href{http://arxiv.org/abs/2009.13705v1}{Prediction of Intrinsic Triferroicity in Two-Dimensional Lattice}}
\subsubsection*{Shiying Shena, \dots, and Ying Dai (2020-09-29)}
Intrinsic triferroicity is essential and highly sought for novel device
applications, such as high-density multistate data storage. So far, the
intrinsic triferroicity has only been discussed in three-dimensional systems.
Herein on basis of first-principles, we report the intrinsic triferroicity in
two-dimensional lattice. Being exfoliatable from the layered bulk, single-layer
FeO2H is shown to be an intrinsically triferroic semiconductor, presenting
antiferromagnetism, ferroelasticity and ferroelectricity simultaneously.
Moreover, the directional control of its ferroelectric polarization is
achievable by 90$^{\circ}$ reversible ferroelastic switching. In addition,
single-layer FeO2H is identified to harbor in-plane piezoelectric effect. The
unveiled phenomena and mechanism of triferroics in this two-dimensional system
not only broaden the scientific and technological impact of triferroics but
also enable a wide range of nanodevice applications.

\subsection*{\href{http://arxiv.org/abs/2009.13703v1}{Frequency-tunable nano-oscillator based on Ovonic Threshold Switch (OTS)}}
\subsubsection*{Seon Jeong Kim, \dots, and Suyoun Lee (2020-09-29)}
Nano-oscillator devices are gaining more and more attention as a prerequisite
for developing novel energy-efficient computing systems based on coupled
oscillators. Here, we introduce a highly scalable, frequency-tunable
nano-oscillator consisting of one Ovonic threshold switch (OTS) and a
field-effect transistor (FET). It is presented that the proposed device shows
an oscillating behavior with a natural frequency (f\_{nat}) adjustable from 0.5
to 2 MHz depending on the gate voltage applied to the FET. In addition, under a
small periodic input, it is observed that the oscillating frequency (f\_{osc})
of the device is locked to the frequency (f\_{in}) of the input when f\_{in} ~
f\_{nat}, demonstrating the so-called synchronization phenomenon. It also shows
the phase lock of the combined oscillator network using circuit simulation,
where the phase relation between the oscillators can be controlled by the
coupling strength. These results imply that the proposed device is promising
for applications in oscillator-based computing systems.

\subsection*{\href{http://arxiv.org/abs/2009.13691v1}{Transition properties in dynamical and statistical features of drying  crack patterns}}
\subsubsection*{Shin-ichi Ito, and Satoshi Yukawa (2020-09-28)}
In this study, we experimentally investigated the time dependence of the
statistical properties of two-dimensional drying crack patterns to determine
the functional form of fragment size distribution. Experiments using a thin
layer of a magnesium carbonate hydroxide paste revealed a "dynamical scaling"
property in the time series of the fragment size distribution, which has been
predicted by theoretical and numerical studies. Further analysis results based
on Bayesian inference show the transition of the functional form of the
fragment size distribution from a log-normal distribution to a generalized
gamma distribution. The combination of a statistical model of the fragmentation
process and the dynamics of stress concentration of a drying thin layer of
viscoelastic material explains the origin of the transition.

\subsection*{\href{http://arxiv.org/abs/2009.13681v1}{Generalized Hamiltonian to describe imperfections in ion-light  interaction}}
\subsubsection*{Ming Li, \dots, and Yunseong Nam (2020-09-28)}
We derive a general Hamiltonian that governs the interaction between an
$N$-ion chain and an externally controlled laser field, where the ion motion is
quantized and the laser field is considered beyond the plane-wave
approximation. This general form not only explicitly includes terms that are
used to drive ion-ion entanglement, but also a series of unwanted terms that
can lead to quantum gate infidelity. We demonstrate the power of our
expressivity of the general Hamiltonian by singling out the effect of axial
mode heating and confirm this experimentally. We discuss pathways forward in
furthering the trapped-ion quantum computational quality, guiding hardware
design decisions.

\subsection*{\href{http://arxiv.org/abs/2009.13663v1}{Quantum Phase Diagrams of Matter-Field Hamiltonians II: Wigner Function  Analysis}}
\subsubsection*{Ramón López-Peña, \dots, and Octavio Castaños (2020-09-28)}
Non-classical states are of practical interest in quantum computing and
quantum metrology. These states can be detected through their Wigner function
negativity in some regions. In this paper, we calculate the ground state of the
three-level generalised Dicke model for a single atom and determine the
structure of its phase diagram using a fidelity criterion. We also calculate
the Wigner function of the electromagnetic modes of the ground state through
the corresponding reduced density matrix, and show in the phase diagram the
regions where entanglement is present. A finer classification is proposed for
the continuous phase transitions.

\subsection*{\href{http://arxiv.org/abs/2009.13661v1}{Infinite System of Random Walkers: Winners and Losers}}
\subsubsection*{P. L. Krapivsky (2020-09-28)}
We study an infinite system of particles initially occupying a half-line
$y\leq 0$ and undergoing random walks on the entire line. The right-most
particle is called a leader. Surprisingly, every particle except the original
leader may never achieve the leadership throughout the evolution. For the
equidistant initial configuration, the $k^{\text{th}}$ particle attains the
leadership with probability $e^{-2} k^{-1} (\ln k)^{-1/2}$ when $k\gg 1$. This
provides a quantitative measure of the correlation between earlier misfortune
(represented by $k$) and eternal failure. We also show that the winner defined
as the first walker overtaking the initial leader has label $k\gg 1$ with
probability decaying as $\exp\!\left[-\tfrac{1}{2}(\ln k)^2\right]$.

\subsection*{\href{http://arxiv.org/abs/2009.13660v1}{Spin-orbit coupling and linear crossings of dipolar magnons in van der  Waals antiferromagnets}}
\subsubsection*{Jie Liu, and Ka Shen (2020-09-28)}
A magnon spin-orbit coupling, induced by the dipole-dipole interaction, is
derived in monoclinic-stacked bilayer honeycomb spin lattice with perpendicular
magnetic anisotropy and antiferromagnetic interlayer coupling. Linear crossings
are predicted in the magnon spectrum around the band minimum in G valley, as
well as in the high frequency range around the zone boundary. The linear
crossings in K and K' valleys, which connect the acoustic and optical bands,
can be gapped when the intralayer dipole-dipole or Kitaev interactions exceed
the interlayer dipole-dipole interaction, resulting in a phase transition from
semimetal to insulator. Our results are useful for analyzing the magnon spin
dynamics and transport properties in van der Waals antiferromagnet.

\subsection*{\href{http://arxiv.org/abs/2009.13657v1}{Relaxation to Equilibrium in a Quantum Network}}
\subsubsection*{Jaroslav Novotný, \dots, and Igor Jex (2020-09-28)}
The approach to equilibrium of quantum mechanical systems is a topic as old
as quantum mechanics itself, but has recently seen a surge of interest due to
applications in quantum technologies, including, but not limited to, quantum
computation and sensing. The mechanisms by which a quantum system approaches
its long-time, limiting stationary state are fascinating and, sometimes, quite
different from their classical counterparts. In this respect, quantum networks
represent a mesoscopic quantum systems of interest. In such a case, the graph
encodes the elementary quantum systems (say qubits) at its vertices, while the
links define the interactions between them. We study here the relaxation to
equilibrium for a fully connected quantum network with CNOT gates representing
the interaction between the constituting qubits. We give a number of results
for the equilibration in these systems, including analytic estimates. The
results are checked using numerical methods for systems with up to 15-16
qubits. It is emphasized in which way the size of the network controls the
convergency.

\subsection*{\href{http://arxiv.org/abs/2009.13652v1}{Time-Resolved Detection of Photon-Surface-Plasmon Coupling at the  Single-Quanta Level}}
\subsubsection*{Chun-Yuan Cheng, \dots, and Chih-Sung Chuu (2020-09-28)}
The interplay of nonclassical light and surface plasmons has attracted
considerable attention due to fundamental interests and potential applications.
To gain more insight into the quantum nature of the photon-surface-plasmon
coupling, time-resolved detection of the interaction is invaluable. Here we
demonstrate the time-resolved detection of photon-surface-plasmon coupling by
exploiting single and entangled photons with long coherence time to excite
single optical plasmons. We examine the nonclassical correlation between the
single photons and single optical plasmons in such systems using the
time-resolved Cauchy-Schwarz inequality. We also realize single optical
plasmons with programmable temporal wavepacket by manipulating the waveform of
incident single photons. The time-resolved detection and coherent control of
single optical plasmons offer new opportunities to study and control the
light-matter interaction at the nanoscale.

\subsection*{\href{http://arxiv.org/abs/2009.13630v1}{Thermodynamic driving force in the formation of hexagonal-diamond Si and  Ge nanowires}}
\subsubsection*{E. Scalise, \dots, and L. Miglio (2020-09-28)}
The metastable hexagonal-diamond phase of Si and Ge (and of SiGe alloys)
displays superior optical properties with respect to the cubic-diamond one. The
latter is the most stable and popular one: growing hexagonal-diamond Si or Ge
without working at extreme conditions proved not to be trivial. Recently,
however, the possibility of growing hexagonal-diamond group-IV nanowires has
been demonstrated, attracting attention on such systems. Based on
first-principle calculations we show that the surface energy of the typical
facets exposed in Si and Ge nanowires is lower in the hexagonal-diamond phase
than in cubic ones. By exploiting a synergic approach based also on a recent
state-of-the-art interatomic potential and on a simple geometrical model, we
investigate the relative stability of nanowires in the two phases up to few
tens of nm in radius, highlighting the surface-related driving force and
discussing its relevance in recent experiments. We also explore the stability
of Si and Ge core-shell nanowires with hexagonal cores (made of GaP for Si
nanowires, of GaAs for Ge nanowires). In this case, the stability of the
hexagonal shell over the cubic one is also favored by the energy cost
associated with the interface linking the two phases. Interestingly, our
calculations indicate a critical radius of the hexagonal shell much lower than
the one reported in recent experiments, indicating the presence of a large
kinetic barrier allowing for the enlargement of the wire in a metastable phase.

\subsection*{\href{http://arxiv.org/abs/2009.13622v1}{A posteriori corrections to the Iterative Qubit Coupled Cluster method  to minimize the use of quantum resources in large-scale calculations}}
\subsubsection*{Ilya G. Ryabinkin, and Scott N. Genin (2020-09-28)}
The iterative qubit coupled cluster (iQCC) method is a systematic variational
approach to solve the electronic structure problem on universal quantum
computers. It is able to use arbitrarily shallow quantum circuits at expense of
iterative canonical transformation of the Hamiltonian and rebuilding a circuit.
Here we present a variety of a posteriori corrections to the iQCC energies to
reduce the number of iterations to achieve the desired accuracy. Our energy
corrections are based on a low-order perturbation theory series that can be
efficiently evaluated on a classical computer. Moreover, capturing a part of
the total energy perturbatively, allows us to formulate the qubit active-space
concept, in which only a subset of all qubits is treated variationally. As a
result, further reduction of quantum resource requirements is achieved. We
demonstrate the utility and efficiency of our approach numerically on the
examples of 10-qubit N$_2$ molecule dissociation, the 24-qubit H$_2$O symmetric
stretch, and 56-qubit singlet-triplet gap calculations for the technologically
important complex, tris-(2-phenylpyridine)iridium(III), Ir(ppy)$_3$.

\subsection*{\href{http://arxiv.org/abs/2009.13621v1}{Rydberg Magnetoexcitons in Cu$_2$O Quantum Wells}}
\subsubsection*{David Ziemkiewicz, \dots, and Sylwia Zielińska-Raczyńska (2020-09-28)}
We present theoretical approach that allows for calculation of optical
functions for Cu$_2$O Quantum Well (QW) with Rydberg excitons in an external
magnetic field of an arbitrary field strength. Both Faraday and Voigt
configurations are considered, in the energetic region of p-excitons. We use
the real density matrix approach and an effective e-h potential, which enable
to derive analytical expressions for the QW magneto-optical functions. For both
configurations, all three field regimes: weak, intermediate, and high field,
are considered and treated separately. With the help of the developed
approximeted method we are able to estimate the limits between the field
regimes. The obtained theoretical magneto-absorption spectra show a good
agreement with available experimental data.

\subsection*{\href{http://arxiv.org/abs/2009.13618v1}{Ternary and Binary Representation of Coordinate and Momentum in Quantum  Mechanics}}
\subsubsection*{M. G. Ivanov and A. Yu. Polushkin (2020-09-28)}
To simulate a quantum system with continuous degrees of freedom on a quantum
computer based on quantum digits, it is necessary to reduce continuous
observables (primarily coordinates and momenta) to discrete observables. We
consider this problem based on expanding quantum observables in series in
powers of two and three analogous to the binary and ternary representations of
real numbers. The coefficients of the series ("digits") are, therefore,
Hermitian operators. We investigate the corresponding quantum mechanical
operators and the relations between them and show that the binary and ternary
expansions of quantum observables automatically leads to renormalization of
some divergent integrals and series (giving them finite values).

\subsection*{\href{http://arxiv.org/abs/2009.13612v1}{Atomic Spectra in a Six-Level Scheme for Electromagnetically Induced  Transparency and Autler-Townes Splitting in Rydberg Atoms}}
\subsubsection*{Amy K. Robinson, \dots, and Christopher L. Holloway (2020-09-28)}
We investigate electromagnetically induced transparency (EIT) and
Autler-Townes (AT) splitting in Rydberg rubidium atoms for a six-level
excitation scheme. In this six-level system, one radio-frequency field
simultaneously couples to two high-laying Rydberg states and results in
interesting atomic spectra observed in the EIT lines. We present experimental
results for several excitation parameters. We also present two theoretical
models for this atomic system, where these two models capture different aspects
of the observed spectra. One is a six-level model used to predict dominant
spectral features and the other a more complex eight-level model used to
predict the full characteristics of this system. Both models shows very good
agreement with the experimental data.

\subsection*{\href{http://arxiv.org/abs/2009.13611v1}{Water diffusion in carbon nanotubes: interplay between confinement,  surface deformation and temperature}}
\subsubsection*{Bruno H. S. Mendonça, \dots, and Marcia C. Barbosa (2020-09-28)}
In this article we investigate through molecular dynamics simulations the
diffusion behavior of the TIP4P/2005 water when confined in pristine and
deformed carbon nanotubes (armchair and zigzag). To analyze different diffusive
mechanisms, the water temperature was varied from $210\leq T\leq 380$~K. The
results of our simulations reveal that water present a non-Arrhenius to
Arrhenius diffusion crossover. The confinement shifts the diffusion transition
to higher temperatures when compared with the bulk system. In addition, for
narrower nanotubes, water diffuses in a single line which leads to a mobility
independent of the activation energy.

\subsection*{\href{http://arxiv.org/abs/2009.13601v1}{Geometry of quantum hydrodynamics in theoretical chemistry}}
\subsubsection*{Michael S. Foskett (2020-09-28)}
This thesis investigates geometric approaches to quantum hydrodynamics (QHD)
in order to develop applications in theoretical quantum chemistry.
  Based upon the momentum map geometric structure of QHD and the associated
Lie-Poisson and Euler-Poincar\'e equations, alternative geometric approaches to
the classical limit in QHD are presented. These include a new regularised
Lagrangian which allows for singular solutions called 'Bohmions' as well as a
'cold fluid' classical closure quantum mixed states.
  The momentum map approach to QHD is then applied to the nuclear dynamics in a
chemistry model known as exact factorization. The geometric treatment extends
existing approaches to include unitary electronic evolution in the frame of the
nuclear flow, with the resulting dynamics carrying both Euler-Poincar\'e and
Lie-Poisson structures. A new mixed quantum-classical model is then derived by
considering a generalised factorisation ansatz at the level of the molecular
density matrix.
  A new alternative geometric formulation of QHD is then constructed.
Introducing a $\mathfrak{u}(1)$ connection as the new fundamental variable
provides a new method for incorporating holonomy in QHD, which follows from its
constant non-zero curvature. The fluid flow is no longer irrotational and
carries a non-trivial circulation theorem, allowing for vortex filament
solutions.
  Finally, non-Abelian connections are then considered in quantum mechanics.
The dynamics of the spin vector in the Pauli equation allows for the
introduction of an $\mathfrak{so}(3)$ connection whilst a more general
$\mathfrak{u}(\mathscr{H})$ connection is introduced from the unitary evolution
of a quantum system. This is used to provide a new geometric picture for the
Berry connection and quantum geometric tensor, whilst relevant applications to
quantum chemistry are then considered.

\subsection*{\href{http://arxiv.org/abs/2009.13599v1}{Tunable three-body loss in a nonlinear Rydberg medium}}
\subsubsection*{Dalia P. Ornelas Huerta, \dots, and J. V. Porto (2020-09-28)}
Long-range Rydberg interactions, in combination with electromagnetically
induced transparency (EIT), give rise to strongly interacting photons where the
strength, sign, and form of the interactions are widely tunable and
controllable. Such control can be applied to both coherent and dissipative
interactions, which provides the potential to generate novel few-photon states.
Recently it has been shown that Rydberg-EIT is a rare system in which
three-body interactions can be as strong or stronger than two-body
interactions. In this work, we study a three-body scattering loss for
Rydberg-EIT in a wide regime of single and two-photon detunings. Our numerical
simulations of the full three-body wavefunction and analytical estimates based
on Fermi's Golden Rule strongly suggest that the observed features in the
outgoing photonic correlations are caused by the resonant enhancement of the
three-body losses.

\subsection*{\href{http://arxiv.org/abs/2009.13585v1}{Some Recent Developments in Auxiliary-Field Quantum Monte Carlo for Real  Materials}}
\subsubsection*{Hao Shi and Shiwei Zhang (2020-09-28)}
The auxiliary-field quantum Monte Carlo (AFQMC) method is a general numerical
method for correlated many-electron systems, which is being increasingly
applied in lattice models, atoms, molecules, and solids. Here we introduce the
theory and algorithm of the method specialized for real materials, and present
several recent developments. We give a systematic exposition of the key steps
of AFQMC, closely tracking the framework of a modern software library we are
developing. The building of a Monte Carlo Hamiltonian, projecting to the ground
state, sampling two-body operators, phaseless approximation, and measuring
ground state properties are discussed in details. An advanced implementation
for multi-determinant trial wave functions is described which dramatically
speeds up the algorithm and reduces the memory cost. We propose a
self-consistent constraint for real materials, and discuss two flavors for its
realization, either by coupling the AFQMC calculation to an effective
independent-electron calculation, or via the natural orbitals of the computed
one-body density matrix.

\subsection*{\href{http://arxiv.org/abs/2009.13572v1}{Coherent control and spectroscopy of a semiconductor quantum dot Wigner  molecule}}
\subsubsection*{J. Corrigan, \dots, and M. A. Eriksson (2020-09-28)}
Multi-electron semiconductor quantum dots have found wide application in
qubits, where they enable readout and enhance polarizability. However, coherent
control in such dots has typically been restricted to only the lowest two
levels, and such control in the strongly interacting regime has not been
realized. Here we report quantum control of eight different resonances in a
silicon-based quantum dot. We use qubit readout to perform spectroscopy,
revealing a dense set of energy levels with characteristic spacing far smaller
than the single-particle energy. By comparing with full configuration
interaction calculations, we argue that the dense set of levels arises from
Wigner-molecule physics.

\subsection*{\href{http://arxiv.org/abs/2009.13568v1}{A coupled cluster framework for electrons and phonons}}
\subsubsection*{Alec F. White, \dots, and Garnet Kin-Lic Chan (2020-09-28)}
We describe a coupled cluster framework for coupled systems of electrons and
phonons. Neutral and charged excitations are accessed via the
equation-of-motion version of the theory. Benchmarks on the Hubbard-Holstein
model allow us to assess the strengths and weaknesses of different coupled
cluster approximations which generally perform well for weak to moderate
coupling. Finally, we report progress towards an implementation for {\it ab
initio} calculations on solids, and present some preliminary results on
finite-size models of diamond. We also report the implementation of
electron-phonon coupling matrix elements from crystalline Gaussian type
orbitals (cGTO) within the PySCF program package.

\subsection*{\href{http://arxiv.org/abs/2009.13557v1}{Uncertainty Quantification in Atomistic Modeling of Metals and its  Effect on Mesoscale and Continuum Modeling A Review}}
\subsubsection*{Joshua J. Gabriel, \dots, and Marius Stan (2020-09-28)}
The design of next-generation alloys through the Integrated Computational
Materials Engineering (ICME) approach relies on multi-scale computer
simulations to provide thermodynamic properties when experiments are difficult
to conduct. Atomistic methods such as Density Functional Theory (DFT) and
Molecular Dynamics (MD) have been successful in predicting properties of never
before studied compounds or phases. However, uncertainty quantification (UQ) of
DFT and MD results is rarely reported due to computational and UQ methodology
challenges. Over the past decade, studies have emerged that mitigate this gap.
These advances are reviewed in the context of thermodynamic modeling and
information exchange with mesoscale methods such as Phase Field Method (PFM)
and Calculation of Phase Diagrams (CALPHAD). The importance of UQ is
illustrated using properties of metals, with aluminum as an example, and
highlighting deterministic, frequentist and Bayesian methodologies. Challenges
facing routine uncertainty quantification and an outlook on addressing them are
also presented.

\subsection*{\href{http://arxiv.org/abs/2009.13556v1}{Excited states in variational Monte Carlo using a penalty method}}
\subsubsection*{Shivesh Pathak, \dots, and Lucas K. Wagner (2020-09-28)}
The authors present a technique using variational Monte Carlo to solve for
excited states of electronic systems. The technique is based on enforcing
orthogonality to lower energy states, which results in a simple variational
principle for the excited states. Energy optimization is then used to solve for
the excited states. An application to the well-characterized benzene molecule,
in which ~10,000 parameters are optimized for the first 12 excited
states.Agreement within approximately 0.15 eV is obtained with higher scaling
coupled cluster methods; small disagreements with experiment are likely due to
vibrational effects.

\subsection*{\href{http://arxiv.org/abs/2009.13551v1}{A degeneracy bound for homogeneous topological order}}
\subsubsection*{Jeongwan Haah (2020-09-28)}
We introduce a notion of homogeneous topological order, that is obeyed by
most, if not all, known examples of topological order including fracton phases
on quantum spins (qudits). The notion is a condition on the ground state
subspace, rather than on the Hamiltonian, and demands that given a collection
of ball-like regions, any linear transformation on the ground space be realized
by an operator that avoids the ball-like regions. We derive a bound on the
ground state degeneracy $\mathcal D$ for systems with homogeneous topological
order on an arbitrary closed Riemannian manifold of dimension $d$, which reads
\[ \log \mathcal D \le c (L/a)\^{d-2}.\] Here, $L$ is the diameter of the
system, $a$ is the lattice spacing, and $c$ is a constant that only depends on
the isometry class of the manifold and the density of degrees of freedom. This
bound is saturated up to constants by known examples.

\subsection*{\href{http://arxiv.org/abs/2009.13545v1}{The Meta-Variational Quantum Eigensolver (Meta-VQE): Learning energy  profiles of parameterized Hamiltonians for quantum simulation}}
\subsubsection*{Alba Cervera-Lierta, and Alán Aspuru-Guzik (2020-09-28)}
We present the meta-VQE, an algorithm capable to learn the ground state
energy profile of a parametrized Hamiltonian. By training the meta-VQE with a
few data points, it delivers an initial circuit parametrization that can be
used to compute the ground state energy of any parametrization of the
Hamiltonian within a certain trust region. We test this algorithm with an XXZ
spin chain and an electronic H$_{4}$ Hamiltonian. In all cases, the meta-VQE is
able to learn the shape of the energy functional and, in some cases, resulted
in improved accuracy in comparison to individual VQE optimization. The meta-VQE
algorithm introduces both a gain in efficiency for parametrized Hamiltonians,
in terms of the number of optimizations, and a good starting point for the
quantum circuit parameters for individual optimizations. The proposed algorithm
proposal can be readily mixed with other improvements in the field of
variational algorithms to shorten the distance between the current
state-of-the-art and applications with quantum advantage.

\subsection*{\href{http://arxiv.org/abs/2009.13535v1}{Detection of Kardar-Parisi-Zhang hydrodynamics in a quantum Heisenberg  spin-$1/2$ chain}}
\subsubsection*{A. Scheie, \dots, and D. A. Tennant (2020-09-28)}
Classical hydrodynamics is a remarkably versatile description of the
coarse-grained behavior of many-particle systems once local equilibrium has
been established. The form of the hydrodynamical equations is determined
primarily by the conserved quantities present in a system. Quantum spin chains
are known to possess, even in the simplest cases, a greatly expanded set of
conservation laws, and recent work suggests that these laws strongly modify
collective spin dynamics even at high temperature. Here, by probing the
dynamical exponent of the one-dimensional Heisenberg antiferromagnet KCuF$_3$
with neutron scattering, we find evidence that the spin dynamics are well
described by the dynamical exponent $z=3/2$, which is consistent with the
recent theoretical conjecture that the dynamics of this quantum system are
described by the Kardar-Parisi-Zhang universality class. This observation shows
that low-energy inelastic neutron scattering at moderate temperatures can
reveal the details of emergent quantum fluid properties like those arising in
non-Fermi liquids in higher dimensions.

\subsection*{\href{http://arxiv.org/abs/2009.13536v1}{Generalized Wigner crystallization in moiré materials}}
\subsubsection*{Bikash Padhi, and Philip W. Phillips (2020-09-28)}
Recent experiments on the twisted transition metal dichalcogenide (TMD)
material, $\rm WSe_2/WS_2$, have observed insulating states at fractional
occupancy of the moir\'e bands. Such states were conceived as generalized
Wigner crystals (GWCs). In this article, we investigate the problem of Wigner
crystallization in the presence of an underlying (moir\'e) lattice. Based on
the best estimates of the system parameters, we find a variety of homobilayer
and heterobilayer TMDs to be excellent candidates for realizing GWCs. In
particular, our analysis based on $r_{s}$ indicates that $\rm MoSe_{2}$ (among
the homobilayers) and $\rm MoSe_2/WSe_2$ or $\rm MoS_2/ WS_2$ (among the
heterobilayers) are the best candidates for realizing GWCs. We also establish
that due to larger effective mass of the valence bands, in general,
hole-crystals are easier to realize that electron-crystals as seen
experimentally. For completeness, we show that satisfying the Mott criterion
$n_{\rm Mott}^{1/2} a_{\ast} = 1$ requires densities nearly three orders of
magnitude larger than the maximal density for GWC formation. This indicates
that for the typical density of operation, HoM or HeM systems are far from the
Mott insulating regime. These crystals realized on a moir\'e lattice, unlike
the conventional Wigner crystals, are incompressible due the gap arising from
pinning with the lattice. Finally, we capture this many-body gap by
variationally renormalizing the dispersion of the vibration modes. We show
these low-energy modes, arising from coupling of the WC with the moir\'e
lattice, can be effectively modeled as a Sine-Gordon theory of fluctuations.

\subsection*{\href{http://arxiv.org/abs/2009.13530v1}{TBG IV: Exact Insulator Ground States and Phase Diagram of Twisted  Bilayer Graphene}}
\subsubsection*{Biao Lian, \dots, and B. Andrei Bernevig (2020-09-28)}
We derive the exact analytic insulator ground states of the projected
Hamiltonian of magic-angle twisted bilayer graphene (TBG) flat bands with
Coulomb interactions in various limits, and study the perturbations moving away
from these limits. We define the (first) chiral limit where the AA stacking
hopping is zero, and a flat limit with exactly flat single-particle bands. In
the chiral-flat limit, the TBG Hamiltonian has a U(4)$\times$U(4) symmetry, and
we find that the exact ground states at integer filling $-4\le \nu\le 4$
relative to charge neutrality are Chern insulators of Chern numbers
$\nu_C=4-|\nu|,2-|\nu|,\cdots,|\nu|-4$, all of which are degenerate. This
confirms recent experiments where Chern insulators are found to be competitive
low-energy states of TBG. When the chiral-flat limit is reduced to the
nonchiral-flat limit which has a U(4) symmetry, we find $\nu=0,\pm2$ has exact
ground states of Chern number $0$, while $\nu=\pm1,\pm3$ has perturbative
ground states of Chern number $\nu_C=\pm1$, all of which are U(4)
ferromagnetic. In the chiral-nonflat limit which has a different U(4) symmetry,
different Chern number states are degenerate up to second order perturbations.
When further reduced to the realistic nonchiral-nonflat case, we find the
perturbative ground states at all integer fillings $\nu$ to be in-plane valley
polarized (Chern) insulators, which are thus intervalley coherent. At certain
value of out-of-plane magnetic field $|B|>0$, a first-order phase transition
for $\nu=\pm1,\pm2$ from Chern number $\nu_C=\text{sgn}(\nu B)(2-|\nu|)$ to
$\nu_C=\text{sgn}(\nu B)(4-|\nu|)$ is expected, which agrees with recent
experimental observations. Lastly, we show that the TBG Hamiltonian reduces
into an extended Hubbard model in the stabilizer code limit.

\subsection*{\href{http://arxiv.org/abs/2009.13527v1}{To heat or not to heat: time crystallinity, scars, and finite-size  effects in clean Floquet systems}}
\subsubsection*{Andrea Pizzi, \dots, and Andreas Nunnenkamp (2020-09-28)}
A cornerstone assumption that most literature on discrete time crystals has
relied on is that homogeneous Floquet systems generally heat to a featureless
infinite temperature state, an expectation that motivated researchers in the
field to mostly focus on many-body localized systems. Challenging this belief,
an increasing number of works have however shown that the standard diagnostics
for time crystallinity apply equally well to clean settings without disorder.
This fact brought considerable confusion to the field: is an homogeneous
discrete time crystal possible, or do homogeneous systems always heat as
originally expected? Studying both a localized and an homogeneous model with
short-range interactions, we resolve this controversy showing explicitly the
key differences between the two cases. On the one hand, our careful scaling
analysis shows evidence that, in the thermodynamic limit and in contrast to
localized discrete time crystals, homogeneous systems indeed heat. On the other
hand, we show that, thanks to a mechanism reminiscent of quantum scars,
finite-size homogeneous systems can still exhibit very crisp signatures of time
crystallinity. A subharmonic response can in fact persist over timescales that
are much larger than those set by the integrability-breaking terms, with
thermalization possibly occurring only at very large system sizes (e.g., of
hundreds of spins). Beyond resolving the confusion in the field, our work casts
a spotlight on finite-size homogeneous systems as prime candidates for the
experimental implementation of nontrivial out-of-equilibrium physics.

\subsection*{\href{http://arxiv.org/abs/2009.13507v1}{Spin-1 spin-orbit- and Rabi-coupled Bose-Einstein condensate solver}}
\subsubsection*{Rajamanickam Ravisankar, \dots, and Sadhan K. Adhikari (2020-09-28)}
We present OpenMP versions of FORTRAN programs for solving the
Gross-Pitaevskii equation for a harmonically trapped three-component spin-1
spinor Bose-Einstein condensate (BEC) in one (1D) and two (2D) spatial
dimensions with or without spin-orbit (SO) and Rabi couplings. Several
different forms of SO coupling are included in the programs. We use the
split-step Crank-Nicolson discretization for imaginary- and real-time
propagation to calculate stationary states and BEC dynamics, respectively. The
imaginary-time propagation programs calculate the lowest-energy stationary
state. The real-time propagation programs can be used to study the dynamics.
The simulation input parameters are provided at the beginning of each program.
The programs propagate the condensate wave function and calculate several
relevant physical quantities. Outputs of the programs include the wave
function, energy, root-mean-square sizes, different density profiles (linear
density for the 1D program, linear and surface densities for the 2D program).
The imaginary- or real-time propagation can start with an analytic wave
function or a pre-calculated numerical wave function. The imaginary-time
propagation usually starts with an analytic wave function, while the real-time
propagation is often initiated with the previously calculated converged
imaginary-time wave function.

\subsection*{\href{http://arxiv.org/abs/2009.13493v1}{Hayden-Preskill decoding from noisy Hawking radiation}}
\subsubsection*{Ning Bao and Yuta Kikuchi (2020-09-28)}
In the Hayden-Preskill thought experiment, the Hawking radiation emitted
before a quantum state is thrown into the black hole is used along with the
radiation collected later for the purpose of decoding the quantum state. A
natural question is how the recoverability is affected if the stored early
radiation is damaged or subject to decoherence, and/or the decoding protocol is
imperfectly performed. We study the recoverability in the thought experiment in
the presence of decoherence or noise in the storage of early radiation.

\subsection*{\href{http://arxiv.org/abs/2009.13492v1}{In-situ tunable nonlinearity and competing signal paths in coupled  superconducting resonators}}
\subsubsection*{Michael Fischer, \dots, and Rudolf Gross (2020-09-28)}
We have fabricated and studied a system of two tunable and coupled nonlinear
superconducting resonators. The nonlinearity is introduced by galvanically
coupled dc-SQUIDs. We simulate the system response by means of a circuit model,
which includes an additional signal path introduced by the electromagnetic
environment. Furthermore, we present two methods allowing us to experimentally
determine the nonlinearity. First, we fit the measured frequency and flux
dependence of the transmission data to simulations based on the equivalent
circuit model. Second, we fit the power dependence of the transmission data to
a model that is predicted by the nonlinear equation of motion describing the
system. Our results show that we are able to tune the nonlinearity of the
resonators by almost two orders of magnitude via an external coil and two
on-chip antennas. The studied system represents the basic building block for
larger systems, allowing for quantum simulations of bosonic many-body systems
with a larger number of lattice sites.

\subsection*{\href{http://arxiv.org/abs/2009.13485v1}{Preparation of excited states on a quantum computer}}
\subsubsection*{Alessandro Roggero, \dots, and Thomas Papenbrock (2020-09-28)}
We study two different methods to prepare excited states on a quantum
computer, a key initial step to study dynamics within linear response theory.
The first method uses unitary evolution for a short time
$T=\mathcal{O}(\sqrt{1-F})$ to approximate the action of an excitation operator
$\hat{O}$ with fidelity $F$ and success probability $P\approx1-F$. The second
method probabilistically applies the excitation operator using the Linear
Combination of Unitaries (LCU) algorithm. We benchmark these techniques on
emulated and real quantum devices, using a toy model for thermal neutron-proton
capture. Despite its larger memory footprint, the LCU-based method is efficient
even on current generation noisy devices and can be implemented at a lower gate
cost than a naive analysis would suggest. These findings show that quantum
techniques designed to achieve good asymptotic scaling on fault tolerant
quantum devices might also provide practical benefits on devices with limited
connectivity and gate fidelity.

\subsection*{\href{http://arxiv.org/abs/2009.13476v1}{AOUP in the presence of Brownian noise: a perturbative approach}}
\subsubsection*{David Martin (2020-09-28)}
By working in the small-persistence-time limit, we determine the steady-state
distribution of an Active Ornstein Uhlenbeck Particle (AOUP) experiencing, in
addition to self-propulsion, a Gaussian white noise modelling a bath at
temperature T. This allows us to derive quantitative formulas for the spatial
probability density of a confined particle and for the current induced by an
asymmetric periodic potential. These formulas disentangle the respective roles
of the passive and active noises on the steady state of AOUPs, showing that
both the correction to the Gibbs-Boltzmann distribution and the ratchet current
are of order 1/T . Thus, signatures of nonequilibrium vanish in the limit of
large translational diffusion. We probe the range of validity of our analytical
derivations by numerical simulations. Finally, we explain how the method
presented here to tackle perturbatively an Ornstein Uhlenbeck (OU) noise could
be further generalized beyond the Brownian case.

\subsection*{\href{http://arxiv.org/abs/2009.13473v1}{Study of hydrogen atom described by a generalized wave equation: what  can we still learn about space dimensionality}}
\subsubsection*{Francisco Caruso, and Felipe Silveira (2020-09-28)}
Hydrogen atom is supposed to be described by a generalization of
Schr\"odinger equation, in which the Hamiltonian depends on an iterated
Laplacian and a Coulomb-like potential $r^{-\beta}$. Starting from previously
obtained solutions for this equation using the $1/N$ expansion method, it is
shown that new light can be shed on the problem of understanding the
dimensionality of the world as proposed by Paul Ehrenfest. A surprisingly new
result is obtained. Indeed, for the first time, we can understand that not only
the sign of energy but also the value of the ground state energy of hydrogen
atom is related to the threefold nature of space.

\subsection*{\href{http://arxiv.org/abs/2009.13471v1}{Neutrino Decoherence in Simple Open Quantum Systems}}
\subsubsection*{Bin Xu (2020-09-28)}
Neutrinos lose coherence as they propagate, which leads to the fading away of
oscillations. In this work, we model neutrino decoherence induced in open
quantum systems from their interaction with the environment. We present two
different models, in which the environment is modeled as forced harmonic
oscillators with white noise interactions, or two-level systems with stochastic
phase kicks. The exponential decay is obtained as a common feature for both
models, which shows the universality of the decoherence processes. We also
discuss connections to the GKSL master equation approach and give a clear
physical meaning of the Lindblad operators. We demonstrate that the
universality of exponential decay of coherence is based on the Born-Markov
approximation. The models in this work are suitable to be extended to describe
real physical processes that could be non-Markovian.

\subsection*{\href{http://arxiv.org/abs/2009.13465v1}{Electronic and magnetic characterization of epitaxial CrBr$_3$  monolayers}}
\subsubsection*{Shawulienu Kezilebieke, \dots, and Peter Liljeroth (2020-09-28)}
The ability to imprint a given material property to another through proximity
effect in layered two-dimensional materials has opened the way to the creation
of designer materials. Here, we use molecular-beam epitaxy (MBE) for a direct
synthesis of a superconductor-magnet hybrid heterostructure by combining
superconducting niobium diselenide (NbSe$_2$) with the monolayer ferromagnetic
chromium tribromide (CrBr$_3$). Using different characterization techniques and
density-functional theory (DFT) calculations, we have confirmed that the
CrBr$_3$ monolayer retains its ferromagnetic ordering with a magnetocrystalline
anisotropy favoring an out-of-plane spin orientation. Low-temperature scanning
tunneling microscopy (STM) measurements show a slight reduction of the
superconducting gap of NbSe$_2$ and the formation of a vortex lattice on the
CrBr$_3$ layer in experiments under an external magnetic field. Our results
contribute to the broader framework of exploiting proximity effects to realize
novel phenomena in 2D heterostructures.

\subsection*{\href{http://arxiv.org/abs/2009.13462v1}{Ultra-bright entangled-photon pair generation from an  AlGaAs-on-insulator microring resonator}}
\subsubsection*{Trevor J. Steiner, \dots, and Galan Moody (2020-09-28)}
Entangled-photon pairs are an essential resource for quantum information
technologies. Chip-scale sources of entangled pairs have been integrated with
various photonic platforms, including silicon, nitrides, indium phosphide, and
lithium niobate, but each has fundamental limitations that restrict the
photon-pair brightness and quality, including weak optical nonlinearity or high
waveguide loss. Here, we demonstrate a novel, ultra-low-loss
AlGaAs-on-insulator platform capable of generating time-energy entangled
photons in a $Q$ $>1$ million microring resonator with nearly 1,000-fold
improvement in brightness compared to existing sources. The
waveguide-integrated source exhibits an internal generation rate greater than
$20\times 10^9$ pairs sec$^{-1}$ mW$^{-2}$, emits near 1550 nm, produces
heralded single photons with $>99\%$ purity, and violates Bell's inequality by
more than 40 standard deviations with visibility $>97\%$. Combined with the
high optical nonlinearity and optical gain of AlGaAs for active component
integration, these are all essential features for a scalable quantum photonic
platform.

\subsection*{\href{http://arxiv.org/abs/2009.13440v1}{Magnetic domains and domain wall pinning in two-dimensional ferromagnets  revealed by nanoscale imaging}}
\subsubsection*{Qi-Chao Sun, \dots, and Jörg Wrachtrup (2020-09-28)}
Magnetic-domain structure and dynamics play an important role in
understanding and controlling the magnetic properties of two-dimensional
magnets, which are of interest to both fundamental studies and
applications[1-5]. However, the probe methods based on the spin-dependent
optical permeability[1,2,6] and electrical conductivity[7-10] can neither
provide quantitative information of the magnetization nor achieve nanoscale
spatial resolution. These capabilities are essential to image and understand
the rich properties of magnetic domains. Here, we employ cryogenic scanning
magnetometry using a single-electron spin of a nitrogen-vacancy center in a
diamond probe to unambiguously prove the existence of magnetic domains and
study their dynamics in atomically thin CrBr$_3$. The high spatial resolution
of this technique enables imaging of magnetic domains and allows to resolve
domain walls pinned by defects. By controlling the magnetic domain evolution as
a function of magnetic field, we find that the pinning effect is a dominant
coercivity mechanism with a saturation magnetization of about 26~$\mu_B$/nm$^2$
for bilayer CrBr$_3$. The magnetic-domain structure and pinning-effect
dominated domain reversal process are verified by micromagnetic simulation. Our
work highlights scanning nitrogen-vacancy center magnetometry as a quantitative
probe to explore two-dimensional magnetism at the nanoscale.

\subsection*{\href{http://arxiv.org/abs/2009.13439v2}{Open-destination measurement-device-independent quantum key distribution  network}}
\subsubsection*{Wen-Fei Cao, \dots, and Kai Chen (2020-09-28)}
Quantum key distribution (QKD) networks hold promise for sharing secure
randomness over multi-partities. Most existing QKD network schemes and
demonstrations are based on trusted relays or limited to point-to-point
scenario. Here, we propose a flexible and extensible scheme named as
open-destination measurement-device-independent QKD network. The scheme enjoys
security against untrusted relays and all detector side-channel attacks.
Particularly, any users can accomplish key distribution under assistance of
others in the network. As an illustration, we show in detail a four-user
network where two users establish secure communication and present realistic
simulations by taking into account imperfections of both sources and detectors.

\subsection*{\href{http://arxiv.org/abs/2009.13438v1}{Two-particle Interference in Double Twin-atom-beams}}
\subsubsection*{F. Borselli, \dots, and J. Schmiedmayer (2020-09-28)}
We demonstrate a highly efficient source for correlated pairs of atoms with
equal but opposite momenta (twin atoms) using an integrated matter-wave circuit
on an atom chip. The atom pairs are emitted into the ground state of two
parallel matter waveguides of an elongated one-dimensional double-well
potential (double twin-atom-beams). We characterize the state of the emitted
twin-atom beams by observing strong number-squeezing up to -10 dB in the
correlated modes of emission. We furthermore demonstrate genuine two-atom
interference in the normalized second-order correlation function $g^{(2)}$ in
the emitted atoms.

\subsection*{\href{http://arxiv.org/abs/2009.13434v1}{Extended Poisson-Kac theory: A unifying framework for stochastic  processes with finite propagation velocity}}
\subsubsection*{Massimiliano Giona, and Rainer Klages (2020-09-28)}
Stochastic processes play a key role for mathematically modeling a huge
variety of transport problems out of equilibrium. To formulate models of
stochastic dynamics the mainstream approach consists in superimposing random
fluctuations on a suitable deterministic evolution. These fluctuations are
sampled from probability distributions that are prescribed a priori, most
commonly as Gaussian or Levy. While these distributions are motivated by
(generalised) central limit theorems they are nevertheless unbounded. This
property implies the violation of fundamental physical principles such as
special relativity and may yield divergencies for basic physical quantities
like energy. It is thus clearly never valid in real-world systems by rendering
all these stochastic models ontologically unphysical. Here we solve the
fundamental problem of unbounded random fluctuations by constructing a
comprehensive theoretical framework of stochastic processes possessing finite
propagation velocity. Our approach is motivated by the theory of Levy walks,
which we embed into an extension of conventional Poisson-Kac processes. Our new
theory possesses an intrinsic flexibility that enables the modelling of many
different kinds of dynamical features, as we demonstrate by three examples. The
corresponding stochastic models capture the whole spectrum of diffusive
dynamics from normal to anomalous diffusion, including the striking Brownian
yet non Gaussian diffusion, and more sophisticated phenomena such as
senescence. Extended Poisson-Kac theory thus not only ensures by construction a
mathematical representation of physical reality that is ontologically valid at
all time and length scales. It also provides a toolbox of stochastic processes
that can be used to model potentially any kind of finite velocity dynamical
phenomena observed experimentally.

\subsection*{\href{http://arxiv.org/abs/2009.13429v1}{Quantum based machine learning of competing chemical reaction profiles}}
\subsubsection*{Stefan Heinen, and O. Anatole von Lilienfeld (2020-09-28)}
Kinetic and thermodynamic effects govern the outcome of competing chemical
reactions, and are key in organic synthesis. They are crucially influenced, if
not dominated, by the chemical composition of the reactants. For two competing
exemplary reactions, E2 and SN2, we show how to use quantum machine learning in
chemical compound space to rapidly predict outcome and respective transition
states for any new reactant. Machine learning model based predictions of
reactions in the chemical compound space of reactant candidates affords
numerical results suggesting that Hammond's postulate is valid for SN2, but not
to E2. The predictions are demonstrated to enable the construction of decision
trees for rational prospective experimental design efforts.

\subsection*{\href{http://arxiv.org/abs/2009.13427v1}{Large magnetoresistance observed in α-Sn/InSb heterostructures}}
\subsubsection*{Yuanfeng Ding, \dots, and Yan-Feng Chen (2020-09-28)}
In this study, we report the epitaxial growth of a series of {\alpha}-Sn
films on InSb substrate by molecular beam epitaxy (MBE) with thickness varying
from 10 nm to 400 nm. High qualities of the {\alpha}-Sn films are confirmed. An
enhanced large magnetoresistance (MR) over 450,000\% has been observed compared
to that of the bare InSb substrate. Thickness, angle and temperature dependent
MR are used to demonstrate the effects of {\alpha}-Sn films on the electrical
transport properties.

\subsection*{\href{http://arxiv.org/abs/2009.13424v1}{Magnetic-field dependence of low-energy magnons, anisotropic heat  conduction, and spontaneous relaxation of magnetic domains in the cubic  helimagnet ZnCr2Se4}}
\subsubsection*{D. S. Inosov, \dots, and J. L. Cohn (2020-09-28)}
Anisotropic low-temperature properties of the cubic spinel helimagnet
ZnCr2Se4 in the single-domain spin-spiral state are investigated by a
combination of neutron scattering, thermal conductivity, ultrasound velocity,
and dilatometry measurements. In an applied magnetic field, neutron
spectroscopy shows a complex and nonmonotonic evolution of the spin-wave
spectrum across the quantum-critical point that separates the spin-spiral phase
from the field-polarized ferromagnetic phase at high fields. A tiny spin gap of
the pseudo-Goldstone magnon mode, observed at wave vectors that are
structurally equivalent but orthogonal to the propagation vector of the spin
helix, vanishes at this quantum critical point, restoring the cubic symmetry in
the magnetic subsystem. The anisotropy imposed by the spin helix has only a
minor influence on the lattice structure and sound velocity but has a much
stronger effect on the heat conductivities measured parallel and perpendicular
to the magnetic propagation vector. The thermal transport is anisotropic at T <
2 K, highly sensitive to an external magnetic field, and likely results
directly from magnonic heat conduction. We also report long-time thermal
relaxation phenomena, revealed by capacitive dilatometry, which are due to
magnetic domain motion related to the destruction of the single-domain magnetic
state, initially stabilized in the sample by the application and removal of
magnetic field. Our results can be generalized to a broad class of helimagnetic
materials in which a discrete lattice symmetry is spontaneously broken by the
magnetic order.

\subsection*{\href{http://arxiv.org/abs/2009.13422v1}{Controlling correlations in NbSe$_2$ via quantum confinement}}
\subsubsection*{Somesh Chandra Ganguli, \dots, and Peter Liljeroth (2020-09-28)}
Transition metal dichalcogenides (TMDC) are a rich family of two-dimensional
materials displaying a multitude of different quantum ground states. In
particular, d$^3$ TMDCs are paradigmatic materials hosting a variety of
symmetry broken states, including charge density waves, superconductivity, and
magnetism. Among this family, NbSe$_2$ is one of the best-studied
superconducting materials down to the monolayer limit. Despite its
superconducting nature, a variety of results point towards strong electronic
repulsions in NbSe$_2$. Here, we control the strength of the interactions
experimentally via quantum confinement effects and use low-temperature scanning
tunneling microscopy (STM) and spectroscopy (STS) to demonstrate that NbSe$_2$
is in strong proximity to a correlated insulating state. This reveals the
coexistence of competing interactions in NbSe$_2$, creating a transition from a
superconducting to an insulating quantum correlated state by
confinement-controlled interactions. Our results demonstrate the dramatic role
of interactions in NbSe$_2$, establishing NbSe$_2$ as a correlated
superconductor with competing interactions.

\subsection*{\href{http://arxiv.org/abs/2009.13414v1}{Nonequilibrium induced by reservoirs: Physico-mathematical model and  numerical tests}}
\subsubsection*{Rupert Klein, \dots, and Luigi Delle Site (2020-09-28)}
A recently proposed mathematical model for open many particle systems (Delle
Site and Klein, J.Math.Phys. 61, 083102 (2020)) is extended to the case of two
reservoirs with different thermodynamic state points. It is assumed that (i)
particles interact via short-range pair interactions and that (ii) the states
of reservoir particles found within the interaction range of the open system
are statistically independent of the open system state. Under these conditions,
the model leads to a description of an open system out of equilibrium in which
the action of the reservoirs enters as a linear sum into the Liouville-type
evolution equations for the open system's statistics. This model is compared
with alternative open system models that include the linearity of the action of
the reservoirs as an a priori assumption. The possibility that nonlinear
effects arise within the present model under less restrictive conditions than
(i) and (ii) above is discussed together with a related numerical test. Even in
such a limiting case the approximation of a linear combination of actions turns
out to yield a description of the open system with sufficient accuracy for
practical purposes. The model presented can be used as a guiding protocol in
the development of simulation schemes for open systems out of equilibrium, thus
the results of this work are not only of conceptual but also of practical
interest.

\subsection*{\href{http://arxiv.org/abs/2009.13413v1}{Deformable and robust core-shell protein microcapsules templated by  liquid-liquid phase separated microdroplets}}
\subsubsection*{Yufan Xu, \dots, and Tuomas P. J. Knowles (2020-09-28)}
Microcapsules are a key class of microscale materials with applications in
areas ranging from personal care to biomedicine, and with increasing potential
to act as extracellular matrix (ECM) models of hollow organs or tissues. Such
capsules are conventionally generated from non-ECM materials including
synthetic polymers. Here, we fabricated robust microcapsules with controllable
shell thickness from physically- and enzymatically-crosslinked gelatin and
achieved a core-shell architecture by exploiting a liquid-liquid phase
separated aqueous dispersed phase system in a one-step microfluidic process.
Microfluidic mechanical testing revealed that the mechanical robustness of
thicker-shell capsules could be controlled through modulation of the shell
thickness. Furthermore, the microcapsules demonstrated
environmentally-responsive deformation, including buckling by osmosis and
external mechanical forces. Finally, a sequential release of cargo species was
obtained through the degradation of the capsules. Stability measurements showed
the capsules were stable at 37 $^{\circ}$C for more than two weeks. These smart
capsules are promising models of hollow biostructures, microscale drug
carriers, and building blocks or compartments for active soft materials and
robots.

\subsection*{\href{http://arxiv.org/abs/2009.13392v1}{Topological Inverse Faraday Effect in Weyl Semimetals}}
\subsubsection*{Yang Gao, and Di Xiao (2020-09-28)}
We demonstrate that in Weyl semimetals, the momentum-space helical spin
texture can couple to the chirality of the Weyl node to generate a
frequency-independent magnetization in response to circularly polarized light
through the inverse Faraday effect. This frequency-independence is rooted in
the topology of the Weyl node. Since the helicity and the chirality are always
locked for Weyl nodes, this effect is not subject to any symmetry constraint.
Finally, we show that the photoinduced frequency-independent magnetization is
robust against lattice effect and has a magnitude large enough to realize
ultrafast all-optical magnetization switching below picosecond.

\subsection*{\href{http://arxiv.org/abs/2009.13388v1}{Second-order coherence of fluorescence in multi-photon blockade}}
\subsubsection*{Th. K. Mavrogordatos and C. Lledó (2020-09-28)}
We calculate the second-order correlation function for the atomic
fluorescence in the two-photon resonance operation of a driven dissipative
Jaynes-Cummings oscillator. We employ a minimal four-level model comprising the
driven two-photon transition alongside two intermediate states visited in the
dissipative cascaded process, in the spirit of [S. S. Shamailov et al., Opt.
Commun. 283, 766 (2010)]. We point to the difference with ordinary resonance
fluorescence and reveal the quantum interference effect involving the
intermediate states, which is also captured in forwards photon scattering.

\subsection*{\href{http://arxiv.org/abs/2009.13385v1}{Fluorine-based color centers in diamond}}
\subsubsection*{S. Ditalia Tchernij, \dots, and J. Forneris (2020-09-28)}
We report on the creation and characterization of the luminescence properties
of high-purity diamond substrates upon F ion implantation and subsequent
thermal annealing. Their room-temperature photoluminescence emission consists
of a weak emission line at 558 nm and of intense bands in the 600 - 750 nm
spectral range. Characterization at liquid He temperature reveals the presence
of a structured set of lines in the 600 - 670 nm spectral range. We discuss the
dependence of the emission properties of F-related optical centers on different
experimental parameters such as the operating temperature and the excitation
wavelength. The correlation of the emission intensity with F implantation
fluence, and the exclusive observation of the afore-mentioned spectral features
in F-implanted and annealed samples provides a strong indication that the
observed emission features are related to a stable F-containing defective
complex in the diamond lattice.

\subsection*{\href{http://arxiv.org/abs/2009.13383v1}{Ising model as a $U(1)$ Lattice Gauge Theory with a $θ$-term}}
\subsubsection*{Tin Sulejmanpasic (2020-09-28)}
We discuss a gauged XY model a $\theta$-term on an arbitrary lattice in 1+1
dimensions, and show that the theory reduces exactly to the 2d Ising model on
the dual lattice in the limit of the strong gauge coupling, provided that the
topological term is defined via the Villain action. We discuss the phase
diagram by comparing the strong and weak gauge coupling limits, and perform
Monte Carlo simulations at intermediate couplings. We generalize the duality to
higher-dimensional Ising models using higher-form U(1) gauge field analogues.

\subsection*{\href{http://arxiv.org/abs/2009.13381v1}{A Green's function approach to the linear response of a driven  dissipative optomechanical system}}
\subsubsection*{Ali Motazedifard, and M. H. Naderi (2020-09-28)}
We discuss the generalization of the linear response theory (LRT) to
encompass the theory of open quantum systems and then apply the so-called
generalized LRT to investigate the linear response of a driven-dissipative
optomechanical system (OMS) to a weak time-dependent perturbation. To our
knowledge, there are elements of ambiguities in the literature in unification
of the standard LRT which has been basically formulated for closed systems and
the theory of open quantum systems. In this paper, we try to shed light on this
matter through the reformulation of the LRT of open quantum systems in the
Heisenberg picture. It is shown how the Green's function equations of motion of
a standard OMS as an open quantum system can be obtained from the quantum
Langevin equations (QLEs) in the Heisenberg picture. The obtained results
explain a wealth of phenomena, including the anti-resonance, normal mode
splitting and the optomechanically induced transparency (OMIT). Furthermore,
the reason why the Stokes or anti-Stokes sidebands are amplified or attenuated
in the red or blue detuning regimes is clearly determined which is in exact
coincidence, especially in the weak-coupling regime, with the Raman-scattering
picture.

\subsection*{\href{http://arxiv.org/abs/2009.13363v1}{Non-Hermitian BCS-BEC crossover of Dirac fermions}}
\subsubsection*{Takuya Kanazawa (2020-09-28)}
We investigate chiral symmetry breaking in a model of Dirac fermions with a
complexified coupling constant whose imaginary part represents dissipation. We
introduce a chiral chemical potential and observe that for real coupling a
relativistic BCS-BEC crossover is realized. We solve the model in the
mean-field approximation and construct the phase diagram as a function of the
complex coupling. It is found that the dynamical mass increases under
dissipation, although the chiral symmetry gets restored if dissipation exceeds
a threshold.

\subsection*{\href{http://arxiv.org/abs/2009.13362v1}{A Unique Perturbation Theory -- Adaptive Perturbation Theory}}
\subsubsection*{Xin Guo (2020-09-28)}
The idea of adaptive perturbation theory is to divide a Hamiltonian into a
solvable part and a perturbation part. The solvable part contains diagonal
elements of Fock space from the interacting terms, which is different from the
standard procedure of previous perturbation method. In this letter, we use the
adaptive perturbation theory to extract the solvable elements in the
two-particle system and obtain the energy spectrum of the solvable part. Then,
we diagonalize the Hamiltonian to obtain the numerical solution. Finally, we
compare two kind of values. The results show that the adaptive perturbation
theory can be well applied to the strong coupling regime of two-particle
system, and on the whole, with the increase of quantum number, two values are
closer.

\subsection*{\href{http://arxiv.org/abs/2009.13361v1}{Coupled-Double-Quantum-Dot Environmental Information Engines: A  Numerical Analysis}}
\subsubsection*{Katsuaki Tanabe (2020-09-28)}
We conduct numerical simulations for an autonomous information engine
comprising a set of coupled double quantum dots using a simple model. The
steady-state entropy production rate in each component, heat and electron
transfer rates are calculated via the probability distribution of the four
electronic states from the master transition-rate equations. We define an
information-engine efficiency based on the entropy change of the reservoir,
implicating power generators that employ the environmental order as a new
energy resource. We acquire device-design principles, toward the realization of
corresponding practical energy converters, including that (1) higher energy
levels of the detector-side reservoir than those of the detector dot provide
significantly higher work production rates by faster states' circulation, (2)
the efficiency is strongly dependent on the relative temperatures of the
detector and system sides and becomes high in a particular Coulomb-interaction
strength region between the quantum dots, and (3) the efficiency depends little
on the system dot's energy level relative to its reservoir but largely on the
antisymmetric relative amplitudes of the electronic tunneling rates.

\subsection*{\href{http://arxiv.org/abs/2009.13351v1}{Comment on: "On the influence of a Coulomb-like potential induced by the  Lorentz symmetry breaking effects on the harmonic oscillator''. Eur. Phys. J.  Plus (2012) \textbf{127}: 102}}
\subsubsection*{Francisco M. Fernández (2020-09-28)}
We analyze the calculation of bound states for a nonrelativistic spin-half
neutral particle under the influence of a Coulomb-like potential induced by
Lorentz symmetry breaking effects. We show that the truncation condition
proposed by the authors only provides one energy eigenvalue for a particular
model potential and misses all the other bound-state energies. The dependence
of the cyclotron frequency on the quantum numbers is a mere artifact of the
truncation condition that is by no means necessary for the existence of bound
states.

\subsection*{\href{http://arxiv.org/abs/2009.13350v1}{Approximating Lattice Gauge Theories on Superconducting Circuits:  Quantum Phase Transition and Quench Dynamics}}
\subsubsection*{Zi-Yong Ge, \dots, and Heng Fan (2020-09-28)}
We propose an implementation to approximate $\mathbb{Z}_2$ lattice gauge
theory (LGT) on superconducting quantum circuits, where the effective theory is
a mixture of a LGT and a gauge-broken term. Using matrix product state based
methods, both the ground state properties and quench dynamics are
systematically investigated. With an increase of the transverse (electric)
field, the system displays a quantum phase transition from a disordered phase
to a translational symmetry breaking phase. In the ordered phase, an
approximate Gaussian law of the $\mathbb{Z}_2$ LGT emerges in the ground state.
Moreover, to shed light on the experiments, we also study the quench dynamics,
where there is a dynamical signature of the spontaneous translational symmetry
breaking. The spreading of the single particle of matter degree is diffusive
under the weak transverse field, while it is ballistic with small velocity for
the strong field. Furthermore, due to the existence of an approximate Gaussian
law under the strong transverse field, the matter degree can also exhibit a
confinement which leads to a strong suppression of the nearest-neighbor
hopping. Our results pave the way for simulating the LGT on superconducting
circuits, including the quantum phase transition and quench dynamics.

\subsection*{\href{http://arxiv.org/abs/2009.13330v1}{Nanoscale transient magnetization gratings excited and probed by  femtosecond extreme ultraviolet pulses}}
\subsubsection*{D. Ksenzov, \dots, and C. Gutt (2020-09-28)}
We utilize coherent femtosecond extreme ultraviolet (EUV) pulses derived from
a free electron laser (FEL) to generate transient periodic magnetization
patterns with periods as short as 44 nm. Combining spatially periodic
excitation with resonant probing at the dichroic M-edge of cobalt allows us to
create and probe transient gratings of electronic and magnetic excitations in a
CoGd alloy. In a demagnetized sample, we observe an electronic excitation with
50 fs rise time close to the FEL pulse duration and ~0.5 ps decay time within
the range for the electron-phonon relaxation in metals. When the experiment is
performed on a sample magnetized to saturation in an external field, we observe
a magnetization grating, which appears on a sub-picosecond time scale as the
sample is demagnetized at the maxima of the EUV intensity and then decays on
the time scale of tens of picoseconds via thermal diffusion. The described
approach opens prospects for studying dynamics of ultrafast magnetic phenomena
on nanometer length scales.

\subsection*{\href{http://arxiv.org/abs/2009.13328v1}{Topological phase transition controlled by electric field in  two-dimensional ferromagnetic semiconductors}}
\subsubsection*{Jingyang You, \dots, and Gang Su (2020-09-28)}
To tune topological and magnetic properties of systems with band engineering
by applying an electric field is of vital important both in physics and in
practical applications. In this work, we find a topological phase transition
from topologically trivial to nontrivial states at an external electric field
of about 0.1 V/\AA\ in ferromagnetic semiconductor MnBi$_2$Te$_4$ monolayer. It
is shown that when electric field increases from 0 to 0.15 V/\AA, the magnetic
anisotropy energy (MAE) increases from 0.1 meV to about 5 meV, and the Curie
temperature Tc increases from 20 to about 70 K. The increased MAE mainly comes
from the enhanced spin-orbit coupling due to the applied electric field. The
enhanced Tc can be understood from the enhanced $p$-$d$ hybridization and
decreased energy difference between $p$ orbitals of Te atoms and $d$ orbitals
of Mn atoms. Moreover, we propose two novel Janus materials
MnBi$_2$Se$_2$Te$_2$ and MnBi$_2$S$_2$Te$_2$ monolayers with different internal
electric polarizations, which can realize quantum anomalous Hall effect (QAHE)
with Chern numbers $C$=1 and $C$=2, respectively. Our study not only exposes
the electric field induced exotic properties of MnBi$_2$Te$_4$ monolayer, but
also proposes novel materials to realize QAHE in ferromagnetic Janus
semiconductors with electric polarization.

\subsection*{\href{http://arxiv.org/abs/2009.13321v1}{Counter-propagating spectrally uncorrelated biphotons at 1550 nm  generated from periodically poled MTiOXO4 (M = K, Rb, Cs; X = P, As)}}
\subsubsection*{Wu-Hao Cai, \dots, and Rui-Bo Jin (2020-09-28)}
We theoretically investigated spectrally uncorrelated biphotons generated in
a counter-propagating spontaneous parametric downconversion (CP-SPDC) from
periodically poled MTiOXO4 (M = K, Rb, Cs; X = P, As) crystals. By numerical
calculation, it was found that the five crystals from the KTP family can be
used to generate heralded single photons with high spectral purity and wide
tunability. Under the type-0 phase-matching condition, the purity at 1550 nm
was between 0.91 and 0.92, and the purity can be maintained over 0.90 from 1500
nm to 2000 nm wavelength. Under the type-II phase-matching condition, the
purity at 1550 nm was 0.96, 0.97, 0.97, 0.98, and 0.98 for PPKTP, PPRTP, PPKTA,
PPRTA, and PPCTA, respectively; furthermore, the purity can be kept over 0.96
for more than 600 nm wavelength range. We also simulated the Hong-Ou-Mandel
interference between independent photon sources for PPRTP crystals at 1550 nm,
and interference visibility was 92\% (97\%) under type-0 (type-II) phase-matching
condition. This study may provide spectrally pure narrowband single-photon
sources for quantum memories and quantum networks at telecom wavelengths.

\subsection*{\href{http://arxiv.org/abs/2009.13313v1}{Triple helix vs. skyrmion lattice in two-dimensional non-centrosymmetric  magnets}}
\subsubsection*{V. E. Timofeev, and D. N. Aristov (2020-09-28)}
It is commonly assumed that a lattice of skyrmions, emerging in
two-dimensional non-centrosymmetric magnets in external magnetic fields, can be
represented as a sum of three magnetic helices. In order to test this
assumption we compare two approaches to a description of regular skyrmion
structure. We construct (i) a lattice of Belavin-Polyakov-like skyrmions within
the stereographic projection method, and (ii) a deformed triple helix defined
with the use of elliptic functions. The estimates for the energy density and
magnetic profiles show that these two ansatzes are nearly identical at zero
temperature for intermediate magnetic fields. However at higher magnetic
fields, near the transition to topologically trivial uniform phase, the
stereographic projection method is preferable, particularly, for the
description of disordered skyrmion liquid phase. We suggest to explore the
intensities of the secondary Bragg peaks to obtain the additional information
about the magnetic profile of individual skyrmions. We estimate these
intensities to be several percents of the main Bragg peak at high magnetic
fields.

\subsection*{\href{http://arxiv.org/abs/2009.13310v1}{Coverage Fluctuations and Correlations in Nanoparticle-Catalyzed  Diffusion-Influenced Bimolecular Reactions}}
\subsubsection*{Yi-Chen Lin, and Joachim Dzubiella (2020-09-28)}
The kinetic processes in nanoparticle-based catalysis are dominated by large
fluctuations and spatiotemporal heterogeneities, in particular for
diffusion-influenced reactions which are far from equilibrium. Here, we report
results from particle-resolved reaction-diffusion simulations of steady-state
bimolecular reactions catalyzed on the surface of a single, perfectly spherical
nanoparticle. We study various reactant adsorption and diffusion regimes, in
particular considering the crowding effects of the reaction products. Our
simulations reveal that fluctuations, significant coverage cross-correlations,
transient self-poisoning, related domain formation, and excluded-volume effects
on the nanoparticle surface lead to a complex kinetic behavior, sensitively
tuned by the balance between adsorption affinity, mixed 2D and 3D diffusion,
and chemical reaction propensity. The adsorbed products are found to influence
the correlations and fluctuations, depending on overall reaction speed, thereby
going beyond conventional steric (e.g., Langmuir-like) product inhibition
mechanisms. We summarize our findings in a state diagram depicting the
nonlinear kinetic regimes by an apparent surface reaction order in dependence
of the intrinsic reaction propensity and adsorption strength. Our study using a
simple, perfectly spherical, and inert nanocatalyst demonstrates that
spatiotemporal heterogeneities are intrinsic to the reaction-diffusion problem
and not necessarily caused by any dynamical surface effects from the catalyst
(e.g., dynamical surface reconstruction), as often argued.

\subsection*{\href{http://arxiv.org/abs/2009.13309v1}{Comment to Spatial Search by Quantum Walk is Optimal for Almost all  Graphs}}
\subsubsection*{Ryszard Kukulski and Adam Glos (2020-09-28)}
This comment is to correct the proof of optimality of quantum spatial search
for Erd\H{o}s-R\'enyi graphs presented in `Spatial Search by Quantum Walk is
Optimal for Almost all Graphs'
(https://doi.org/10.1103/PhysRevLett.116.100501). The authors claim that if
$p\geq \frac{\log^{3/2}(n)}{n}$, then the CTQW-based search is optimal for
almost all graphs. Below we point the issues found in the main paper, and
propose corrections, which in fact improve the result to $p=\omega(\log(n)/n)$
in case of transition rate $\gamma = 1/\lambda_1$. In the case of the proof for
simplified transition rate $1/(np)$ we pointed a possible issue with applying
perturbation theory.

\subsection*{\href{http://arxiv.org/abs/2009.13298v1}{Two-dimensional chiral stacking orders in quasi-one-dimensional charge  density waves}}
\subsubsection*{Sun-Woo Kim, \dots, and Tae-Hwan Kim (2020-09-28)}
Chirality manifests in various forms in nature. However, there is no evidence
of the chirality in one-dimensional charge density wave (CDW) systems. Here, we
have explored the chirality among quasi-one-dimensional CDW ground states with
the aid of scanning tunneling microscopy, symmetry analysis, and density
functional theory calculations. We discovered three distinct chiralities
emerging in the form of two-dimensional chiral stacking orders composed of
degenerate CDW ground states: right-, left-, and nonchiral stacking orders.
Such chiral stacking orders correspond to newly introduced chiral winding
numbers. Furthermore, we observed that these chiral stacking orders are
intertwined with chiral vortices and chiral domain walls, which play a crucial
role in engineering the chiral stacking orders. Our findings suggest that the
unexpected chiral stacking orders can open a way to investigate the chirality
in CDW systems, which can lead to diverse phenomena such as circular dichroism
depending on chirality.

\subsection*{\href{http://arxiv.org/abs/2009.13288v1}{Quantum-classical algorithms for skewed linear systems with optimized  Hadamard test}}
\subsubsection*{Bujiao Wu, \dots, and Patrick Rebentrost (2020-09-28)}
The solving of linear systems provides a rich area to investigate the use of
nearer-term, noisy, intermediate-scale quantum computers. In this work, we
discuss hybrid quantum-classical algorithms for skewed linear systems for
over-determined and under-determined cases. Our input model is such that the
columns or rows of the matrix defining the linear system are given via quantum
circuits of poly-logarithmic depth and the number of circuits is much smaller
than their Hilbert space dimension. Our algorithms have poly-logarithmic
dependence on the dimension and polynomial dependence in other natural
quantities. In addition, we present an algorithm for the special case of a
factorized linear system with run time poly-logarithmic in the respective
dimensions. At the core of these algorithms is the Hadamard test and in the
second part of this paper we consider the optimization of the circuit depth of
this test. Given an $n$-qubit and $d$-depth quantum circuit $\mathcal{C}$, we
can approximate $\langle 0|\mathcal{C}|0\rangle$ using $(n + s)$ qubits and
$O\left(\log s + d\log (n/s) + d\right)$-depth quantum circuits, where $s\leq
n$. In comparison, the standard implementation requires $n+1$ qubits and
$O(dn)$ depth. Lattice geometries underlie recent quantum supremacy experiments
with superconducting devices. We also optimize the Hadamard test for an
$(l_1\times l_2)$ lattice with $l_1 \times l_2 = n$, and can approximate
$\langle 0|\mathcal{C} |0\rangle$ with $(n + 1)$ qubits and $O\left(d \left(l_1
+ l_2\right)\right)$-depth circuits. In comparison, the standard depth is
$O\left(d n^2\right)$ in this setting. Both of our optimization methods are
asymptotically tight in the case of one-depth quantum circuits $\mathcal{C}$.

\subsection*{\href{http://arxiv.org/abs/2009.13286v1}{Electronic phase diagram of iron chalcogenide superconductors FeSe1-xSx  and FeSe1-yTey}}
\subsubsection*{Shaobo Liu, \dots, and Zhongxian Zhao (2020-09-28)}
Here we establish a combined electronic phase diagram of isoelectronic
FeSe1-xSx (0.19 > x > 0.0) and FeSe1-yTey (0.04 < y < 1.0) single crystals. The
FeSe1-yTey crystals with y = 0.04 - 0.30 are grown by a hydrothermal
ion-deintercalation (HID) method. Based on combined experiments of the specific
heat, electrical transport, and angle-resolved photoemission spectroscopy, no
signature of the tetragonal-symmetry-broken transition to orthorhombic
(nematic) phase is observed in the HID FeSe1-yTey samples, as compared with the
FeSe1-xSx samples showing this transition at Ts. A ubiquitous dip-like
temperature dependence of the Hall coefficient is observed around a
characteristic temperature T* in the tetragonal regimes, which is well above
the superconducting transition. More importantly, we find that the
superconducting transition temperature Tc is positively correlated with the
Hall-dip temperature T* across the FeSe1-xSx and FeSe1-yTey systems, suggesting
that the tetragonal background is a fundamental host for the superconductivity.

\subsection*{\href{http://arxiv.org/abs/2009.13277v1}{A continuous metal-insulator transition driven by spin correlations}}
\subsubsection*{Yejun Feng, \dots, and T. F. Rosenbaum (2020-09-28)}
Metal-insulator transitions involve a mix of charge, spin, and structural
degrees of freedom, and when strongly-correlated, can underlay the emergence of
exotic quantum states. Mott insulators induced by the opening of a Coulomb gap
are an important and well-recognized class of transitions, but insulators
purely driven by spin correlations are much less common, as the reduced energy
scale often invites competition from other degrees of freedom. Here we
demonstrate a clean example of a spin-correlation-driven metal-insulator
transition in the all-in-all-out pyrochlore antiferromagnet Cd2Os2O7, where the
lattice symmetry is fully preserved by the antiferromagnetism. After the
antisymmetric linear magnetoresistance from conductive, ferromagnetic domain
walls is carefully removed experimentally, the Hall coefficient of the bulk
reveals four Fermi surfaces, two of electron type and two of hole type,
sequentially departing the Fermi level with decreasing temperature below the
N\'eel temperature, T\_N. Contrary to the common belief of concurrent magnetic
and metal-insulator transitions in Cd2Os2O7, the charge gap of a continuous
metal-insulator transition opens only at T~10K, well below T\_N=227K. The
insulating mechanism resolved by the Hall coefficient parallels the Slater
picture, but without a folded Brillouin zone, and contrasts sharply with the
behavior of Mott insulators and spin density waves, where the electronic gap
opens above and at T\_N, respectively.

\subsection*{\href{http://arxiv.org/abs/2009.13259v1}{Transport properties of proximitized double quantum dots}}
\subsubsection*{G. Górski and K. Kucab (2020-09-28)}
We study the sub-gap spectrum and the transport properties of a double
quantum dot coupled to metallic and superconducting leads. The coupling of both
quantum dots to the superconducting lead induces a non-local pairing in both
quantum dots by the Andreev reflection processes. Additionally, we obtain two
channels of Cooper pair tunneling into a superconducting lead. In such a
system, the direct tunneling process (by one of two dots) or the crossed
tunneling process (by both quantum dots at the same time) is possible. We
consider the dependence of the Andreev transmittance on an inter-dot tunneling
amplitude and the coupling between a quantum dot and the superconducting lead.
We also consider the occurrence of interferometric Fano-type line shapes in the
linear Andreev conductance spectra.

\subsection*{\href{http://arxiv.org/abs/2009.13253v1}{Valley-dependent properties of monolayer MoSi$_{2}$N$_{4}$,  WSi$_{2}$N$_{4}$ and MoSi$_{2}$As$_{4}$}}
\subsubsection*{Si Li, \dots, and Shengyuan A. Yang (2020-09-28)}
In a recent work, new two-dimensional materials, the monolayer
MoSi$_{2}$N$_{4}$ and WSi$_{2}$N$_{4}$, have been successfully synthesized in
experiment, and several other monolayer materials with the similar structure,
such as MoSi$_{2}$As$_{4}$, have been predicted [{\color{blue}Science 369,
670-674 (2020)}]. Here, based on first-principles calculations and theoretical
analysis, we investigate the electronic and optical properties of monolayer
MoSi$_{2}$N$_{4}$, WSi$_{2}$N$_{4}$ and MoSi$_{2}$As$_{4}$. We show that these
materials are semiconductors, with a pair of Dirac-type valleys located at the
corners of the hexagonal Brillouin zone. Due to the broken inversion symmetry
and the effect of spin-orbit coupling, the valley fermions manifest spin-valley
coupling, valley-contrasting Berry curvature, and valley-selective optical
circular dichroism. We also construct the low-energy effective model for the
valleys, calculate the spin Hall conductivity and the permittivity, and
investigate the strain effect on the band structure. Our result reveals
interesting valley physics in monolayer MoSi$_{2}$N$_{4}$, WSi$_{2}$N$_{4}$ and
MoSi$_{2}$As$_{4}$, suggesting their great potential for valleytronics and
spintronics applications.

\subsection*{\href{http://arxiv.org/abs/2009.13244v1}{Dynamic self-consistent field approach for studying kinetic processes in  multiblock copolymer melts}}
\subsubsection*{Friederike Schmid and Bing Li (2020-09-28)}
The self-consistent field theory is a popular and highly successful
theoretical framework for studying equilibrium (co)polymer systems at the
mesoscopic level. Dynamic density functionals allow one to use this framework
for studying dynamical processes in the diffusive, non-inertial regime. The
central quantity in these approaches is the mobility function, which describes
the effect of chain connectivity on the nonlocal response of monomers to
thermodynamic driving fields. In a recent study [Mantha et al, Macromolecules
53, 3409 (2020)], we have developed a method to systematically construct
mobility functions from reference fine-grained simulations. Here we focus on
melts of linear chains in the Rouse regime and show how the mobility functions
can be calculated semi-analytically for multiblock copolymers with arbitrary
sequences without resorting to simulations. In this context, an accurate
approximate expression for the single-chain dynamic structure factor is
derived. Several limiting regimes are discussed. Then we apply the resulting
density functional theory to study ordering processes in a two-length scale
block copolymer system after instantaneous quenches into the ordered phase.
Different dynamical regimes in the ordering process are identified: At early
times, the ordering on short scales dominates; at late times, the ordering on
larger scales takes over. For large quench depths, the system does not
necessarily relax into the true equilibrium state. Our density functional
approach could be used for the computer-assisted design of quenching protocols
in order to create novel nonequilibrium materials.

\subsection*{\href{http://arxiv.org/abs/2009.13518v1}{Optimized recursion relation for the computation of partition functions  in the superconfiguration approach}}
\subsubsection*{Jean-Christophe Pain, and Brian G. Wilson (2020-09-28)}
Partition functions of a canonical ensemble of non-interacting bound
electrons are a key ingredient of the super-transition-array approach to the
computation of radiative opacity. A few years ago, we published a robust and
stable recursion relation for the calculation of such partition functions. In
this paper, we propose an optimization of the latter method and explain how to
implement it in practice. The formalism relies on the evaluation of elementary
symmetric polynomials, which opens the way to further improvements.

\subsection*{\href{http://arxiv.org/abs/2009.13229v1}{Some exact results for the statistical physics problem of  high-dimensional linear regression}}
\subsubsection*{Alexander Mozeika, \dots, and Anthony CC Coolen (2020-09-28)}
High-dimensional linear regression have become recently a subject of many
investigations which use statistical physics. The main bulk of this work relies
on powerful approximation techniques but also a more rigorous approaches are
becoming a more prominent. Considering Bayesian setting, we derive a number of
exact results for the inference and related statistical physics problems of the
linear regression.

\subsection*{\href{http://arxiv.org/abs/2009.13228v1}{Pressure-induced reconstructive phase transition in Cd$_3$As$_2$}}
\subsubsection*{Monika Gamża, \dots, and Sven Friedemann (2020-09-28)}
Cadmium arsenide Cd$_3$As$_2$ hosts massless Dirac electrons in its
ambient-conditions tetragonal phase. We report X-ray diffraction and electrical
resistivity measurements of Cd$_3$As$_2$ upon cycling pressure beyond the
critical pressure of the tetragonal phase and back to ambient conditions. We
find that at room temperature the transition between the low- and high-pressure
phases results in large microstrain and reduced crystallite size both on rising
and falling pressure. This leads to non-reversible electronic properties
including self-doping associated with defects and a reduction of the electron
mobility by an order of magnitude due to increased scattering. Our study
indicates that the structural transformation is sluggish and shows a sizable
hysteresis of over 1~GPa. Therefore, we conclude that the transition is
first-order reconstructive, with chemical bonds being broken and rearranged in
the high-pressure phase. Using the diffraction measurements we demonstrate that
annealing at ~200$^\circ$C greatly improves the crystallinity of the
high-pressure phase. We show that its Bragg peaks can be indexed as a primitive
orthorhombic lattice with a\_HP~8.68 A b\_HP~17.15 A and c\_HP~18.58 A. The
diffraction study indicates that during the structural transformation a new
phase with another primitive orthorhombic structure may be also stabilized by
deviatoric stress, providing an additional venue for tuning the unconventional
electronic states in Cd3As2.

\subsection*{\href{http://arxiv.org/abs/2009.13226v1}{Observation of pulse-width dependent saturable and reverse saturable  absorption in spinel ZnCo2O4 microflowers}}
\subsubsection*{Pritam Khan, \dots, and K. V. Adarsh (2020-09-28)}
We exploited intersystem crossing to demonstrate remarkably contrasting
optical nonlinearity in ZnCo2O4 (ZCO) microflowers. Ultrafast transient
absorption measurements reveal that intersystem crossing (ISC) from singlet to
triplet state takes place in 5 ps. When the pulse-width is shorter than ISC
lifetime for femtosecond excitation, saturable absorption (SA) takes place for
all intensities. On the contrary, when the pulse-width is longer than ISC for
nanosecond excitation, we observe transition from SA to reverse SA (RSA) at
higher intensities via excited-state absorption. We envisage that benefiting
from SA and RSA, ZCO emerges as potential candidate for mode locking and
optical limiting devices.

\subsection*{\href{http://arxiv.org/abs/2009.13224v1}{Collapse and Measures of Consciousness}}
\subsubsection*{Adrian Kent (2020-09-28)}
There has been an upsurge of interest lately in developing Wigner's
hypothesis that conscious observation causes collapse by exploring dynamical
collapse models in which some purportedly quantifiable aspect(s) of
consciousness resist superposition. Kremnizer-Ranchin, Chalmers-McQueen and
Okon-Sebasti\'an have explored the idea that collapse may be associated with a
numerical measure of consciousness. More recently, Chalmers-McQueen have argued
that any single measure is inadequate because it will allow superpositions of
distinct states of equal consciousness measure to persist. They suggest a
satisfactory model needs to associate collapse with a set of measures
quantifying aspects of consciousness, such as the "Q-shapes" defined by Tononi
et al. in their "integrated information theory" (IIT) of consciousness. I argue
here that Chalmers-McQueen's argument against associating a single measure with
collapse requires a precise symmetry between brain states associated with
different experiences and thus does not apply to the only case where we have
strong intuitions, namely human (or other terrestrial biological) observers. In
defence of Chalmers-McQueen's stance, it might be argued that idealized
artificial information processing networks could display such symmetries.
However, I argue that any theory (such as IIT) that postulates a map from
network states to mind states should assign identical mind states to isomorphic
network states (as IIT does). This suggests that, if such a map exists, no
familiar components of mind states, such as viewing different colours, or
experiencing pleasure or pain, are likely to be related by symmetries.

\subsection*{\href{http://arxiv.org/abs/2009.13221v1}{Fermions meet two bosons -- the heteronuclear Efimov effect revisited}}
\subsubsection*{Binh Tran, \dots, and Matthias Weidemüller (2020-09-28)}
In this article, we revisit the heteronuclear Efimov effect in a Bose-Fermi
mixture with large mass difference in the Born-Oppenheimer picture. As a
specific example, we consider the combination of bosonic $^{133}\mathrm{Cs}$
and fermionic $^6\mathrm{Li}$. In a system consisting of two heavy bosons and
one light fermion, the fermion-mediated potential between the two heavy bosons
gives rise to an infinite series of three-body bound states. The intraspecies
scattering length determines the three-body parameter and the scaling factor
between consecutive Efimov states. In a second scenario, we replace the single
fermion by an entire Fermi Sea at zero temperature. The emerging interaction
potential for the two bosons exhibits long-range oscillations leading to a
weakening of the binding and a breakup of the infinite series of Efimov states.
In this scenario, the binding energies follow a modified Efimov scaling law
incorporating the Fermi momentum. The scaling factor between deeply bound
states is governed by the intraspecies interaction, analogous to the Efimov
states in vacuum.

\subsection*{\href{http://arxiv.org/abs/2009.13212v1}{A femtotesla direct magnetic gradiometer using a single multipass cell}}
\subsubsection*{V. G. Lucivero, \dots, and M. V. Romalis (2020-09-28)}
We describe a direct gradiometer using optical pumping with opposite circular
polarization in two $^{87}$Rb atomic ensembles within a single multipass cell.
A far-detuned probe laser undergoes a near-zero paramagnetic Faraday rotation
due to the intrinsic subtraction of two contributions exceeding 3.5 rad from
the highly-polarized ensembles. We develop analysis methods for the novel
direct gradiometer signal and measure a gradiometer sensitivity of $10.1$
fT/cm$\sqrt{\mathrm{Hz}}$. We also demonstrate that the new multipass design,
in addition to increasing the optical depth, provides a fundamental advantage
due to the significantly reduced effect of atomic diffusion on the spin noise
time-correlation, in excellent agreement with theoretical estimate.

\subsection*{\href{http://arxiv.org/abs/2009.13205v1}{Pseudo-Waveform-Selective Metasurfaces and Their Limited Performance}}
\subsubsection*{Tomoyuki Nakasha, and Hiroki Wakatsuchi (2020-09-28)}
In recent years, metasurfaces composed of lumped circuit components,
including nonlinear Schottky diodes, have been reported to be capable of
sensing particular electromagnetic waves even at the same frequency depending
on their waveforms, or more specifically, their pulse widths. In this study, we
report analogous waveform-selective phenomena using only linear circuits and
linear media. Although such linear metasurfaces are analytically and
numerically demonstrated to exhibit variable absorption performance, it cannot
strictly be categorized as waveform-selective absorption. It is due to the fact
that the waveform-selective responses in the linear metasurfaces are originated
from the dispersion behaviors of the structures rather than the
frequency-conversion seen in nonlinear waveform-selective metasurfaces. We thus
refer to these linear structures as pseudo-waveform-selective metasurfaces.
Additionally, we show that the pseudo-waveform-selective metasurfaces have
limited performance unless nonlinearity is introduced. These results and
findings confirm the advantages of nonlinear waveform-selective metasurfaces,
which can be exploited to provide an additional degree of freedom to address
existing electromagnetic problems/challenges involving even waves at the same
frequency.

\subsection*{\href{http://arxiv.org/abs/2009.13203v1}{The LeClair-Mussardo series and nested Bethe Ansatz}}
\subsubsection*{Arthur Hutsalyuk, and Levente Pristyák (2020-09-28)}
We consider correlation functions in one dimensional quantum integrable
models related to the algebra symmetries $\mathfrak{gl}(2|1)$ and
$\mathfrak{gl}(3)$. Using the algebraic Bethe Ansatz approach we develop an
expansion theorem, which leads to an infinite integral series in the
thermodynamic limit. The series is the generalization of the LeClair-Mussardo
series to nested Bethe Ansatz systems, and it is applicable both to one-point
and two-point functions. As an example we consider the ground state
density-density correlator in the Gaudin-Yang model of spin-1/2 Fermi
particles. Explicit formulas are presented in a special large coupling and
large imbalance limit.

\subsection*{\href{http://arxiv.org/abs/2009.13201v1}{Optimal quantum-programmable projective measurements with coherent  states}}
\subsubsection*{Niraj Kumar, \dots, and Eleni Diamanti (2020-09-28)}
We consider a device which can be programmed using coherent states of light
to approximate a given projective measurement on an input coherent state. We
provide and discuss three practical implementations of this programmable
projective measurement device with linear optics, involving only balanced beam
splitters and single photon threshold detectors. The three schemes optimally
approximate any projective measurement onto a program coherent state in a
non-destructive fashion. We further extend these to the case where there are no
assumptions on the input state. In this setting, we show that our scheme
enables an efficient verification of an unbounded untrusted source with only
local coherent states, balanced beam splitters, and threshold detectors.
Exploiting the link between programmable measurements and generalised swap
test, we show as a direct application that our schemes provide an
asymptotically quadratic improvement in existing quantum fingerprinting
protocol to approximate the Euclidean distance between two unit vectors.

\subsection*{\href{http://arxiv.org/abs/2009.13197v1}{Bottom-up construction of dynamic density functional theories for  inhomogeneous polymer systems from microscopic simulations}}
\subsubsection*{Sriteja Mantha, and Friederike Schmid (2020-09-28)}
We propose and compare different strategies to construct dynamic density
functional theories (DDFTs) for inhomogeneous polymer systems close to
equilibrium from microscopic simulation trajectories. We focus on the
systematic construction of the mobility coefficient, $\Lambda(r,r')$, which
relates the thermodynamic driving force on monomers at position $r'$ to the
motion of monomers at position $r$. A first approach based on the Green-Kubo
formalism turns out to be impractical because of a severe plateau problem.
Instead, we propose to extract the mobility coefficient from an effective
characteristic relaxation time of the single chain dynamic structure factor. To
test our approach, we study the kinetics of ordering and disordering in diblock
copolymer melts. The DDFT results are in very good agreement with the data from
corresponding fine-grained simulations.

\subsection*{\href{http://arxiv.org/abs/2009.13195v1}{The ferromagnetic-electrodes-induced Hall effect in topological Dirac  semimetals}}
\subsubsection*{Koji Kobayashi and Kentaro Nomura (2020-09-28)}
We propose an unconventional type of Hall effect in a topological Dirac
semimetal with ferromagnetic electrodes. The topological Dirac semimetal itself
has time-reversal symmetry, whereas attached ferromagnetic electrodes break it,
causing the large Hall response. This induced Hall effect is a characteristic
of the helical surface states of topological Dirac semimetals and the helical
edge states of quantum spin Hall insulators. We compute the Hall
conductance/resistance and the Hall angle by using a lattice model with
four-terminal geometry. For topological Dirac semimetals with four electrodes,
the induced Hall effect occurs whether the current electrodes or the voltage
electrodes are ferromagnetic. When the spins in electrodes are almost fully
polarized, the Hall angle becomes as large as that of quantum Hall states or
ideal magnetic Weyl semimetals. We show the robustness of the induced Hall
effect against impurities and also discuss the spin injection and spin decay
problems. This Hall response can be used to detect whether the magnetizations
of the two ferromagnetic electrodes are parallel or antiparallel.

\subsection*{\href{http://arxiv.org/abs/2009.13191v1}{Phase diagram studies for the growth of (Mg,Zr):SrGa$_{12}$O$_{19}$  crystals}}
\subsubsection*{Detlef Klimm, \dots, and Matthias Bickermann (2020-09-28)}
By differential thermal analysis a concentration field suitable for the
growth of Zr, Mg codoped strontium hexagallate crystals was observed that
corresponds well with experimental results from Mateika and Laurien, J. Crystal
Growth 52 (1981) 566-572. It was shown that the melting point of doped crystal
is ca. 60 K higher than that of undoped crystals. This higher melting points
indicates hexagallate phase stabilization by Zr, Mg codoping, and increases the
growth window, compared to undoped SrO-Ga$_2$O$_3$ melts.

\subsection*{\href{http://arxiv.org/abs/2009.13187v1}{On estimating the Shannon entropy and (un)certainty relations for  design-structured POVMs}}
\subsubsection*{Alexey E. Rastegin (2020-09-28)}
Complementarity relations between various characteristics of a probability
distributions are at the core of information theory. In particular, lower and
upper bounds for the entropic function are of great importance. In applied
questions, we often deal with situations, where the sums of certain powers of
probabilities are known. The main question is to how convert the imposed
restrictions into two-sided estimates on the Shannon entropy. It is addressed
in two different ways. More intuitive of them is based on truncated expansions
of the Taylor type. Another method is based on the use of coefficients of the
shifted Chebyshev polynomials. We conjecture here a family of polynomials for
estimating the Shannon entropy from below. As a result, estimates are more
uniform in the sense that errors do not become too large in particular points.
The presented method is used for deriving uncertainty and certainty relations
for POVMs assigned to a quantum design. Quantum designs are currently the
subject of active researches due to potential applications in quantum
information science. Other ways to use the proposed estimates are briefly
discussed.

\subsection*{\href{http://arxiv.org/abs/2009.13186v1}{Electron Excess Induced Defects in Materials: the Case of Oxygen in  Wurtzite Aluminum Nitride}}
\subsubsection*{Piero Gasparotto, \dots, and Carlo Antonio Pignedoli (2020-09-28)}
Machine learning is changing how we design and interpret experiments in
materials science. In this work, we show how unsupervised learning, combined
with ab initio modeling, improves our understanding of structural metastability
in multicomponent alloys. We use the example case of Al-O-N alloys where the
formation of aluminum vacancies in wurtzite AlN upon the incorporation of
substitutional oxygen can be seen as a general mechanism of solids where
crystal symmetry is reduced to stabilize defects. The ideal AlN wurtzite
crystal structure occupation cannot be matched due to the presence of an
aliovalent hetero-element into the structure. The traditional interpretation of
the c-lattice shrinkage in sputter-deposited Al-O-N films from X-ray
diffraction (XRD) experiments suggests the existence of a solubility limit at
8at.\% oxygen content. Here we show that such naive interpretation is
misleading. We support XRD data with a machine learning analysis of ab initio
simulations and positron annihilation lifetime spectroscopy data, revealing no
signs of a possible solubility limit. Instead, the presence of a wide range of
non-equilibrium oxygen-rich defective structures emerging at increasing oxygen
contents suggests that the formation of grain boundaries is the most plausible
mechanism responsible for the lattice shrinkage measured in Al-O-N sputtered
films.

\subsection*{\href{http://arxiv.org/abs/2009.13185v1}{Precise control of $J_\mathrm{eff}=1/2$ magnetic properties in  Sr$_2$IrO$_4$ epitaxial thin films by variation of strain and thin film  thickness}}
\subsubsection*{Stephan Geprägs, \dots, and Dan Mannix (2020-09-28)}
We report on a comprehensive investigation of the effects of strain and film
thickness on the structural and magnetic properties of epitaxial thin films of
the prototypal $J_\mathrm{eff}=1/2$ compound Sr$_2$IrO$_4$ by advanced X-ray
scattering. We find that the Sr$_2$IrO$_4$ thin films can be grown fully
strained up to a thickness of 108 nm. By using X-ray resonant scattering, we
show that the out-of-plane magnetic correlation length is strongly dependent on
the thin film thickness, but independent of the strain state of the thin films.
This can be used as a finely tuned dial to adjust the out-of-plane magnetic
correlation length and transform the magnetic anisotropy from two-dimensional
(2D) to three-dimensional (3D) behavior by incrementing film thickness. These
results provide a clearer picture for the systematic control of the magnetic
degrees of freedom in epitaxial thin films of Sr$_2$IrO$_4$ and bring to light
the potential for a rich playground to explore the physics of $5d$-transition
metal compounds.

\subsection*{\href{http://arxiv.org/abs/2009.13183v1}{Symmetry sustained valley-pseudospin textures of the full-zone excitonic  bands of transition-metal dichalcogenide monolayers}}
\subsubsection*{Ping-Yuan Lo, \dots, and Shun-Jen Cheng (2020-09-28)}
Preserving high degree of valley polarization of photo-excited excitons in
transition-metal dichalcogenide monolayers (TMD-MLs) is desirable for the
valley-based photonic applications, but widely recognized as a hard task
hindered by the intrinsic electron-hole exchange interaction. In this study, we
present a comprehensive investigation of valley-polarized finite-momentum
excitons in WSe$_2$-MLs over the entire Brillouin zone by solving the
density-functional-theory(DFT)-based Bethe-Salpeter equation (BSE) under the
guidance of symmetry analysis. We reveal that finite-momentum excitons are
generally well immune form the exchange-induced valley depolarization, except
for those with specific exciton momenta directionally coincide with the axes of
the $3\sigma_v$ and $2C_2'$ symmetries in the $1H$ TMD-MLs. Sustained by the
symmetries, the valley pseudo-spin texture of the full-zone exciton band in the
momentum space is locally featured by individual skyrmion-like structures where
highly valley-polarized finite-momentum exciton states are centred.
Furthermore, we show that, under the assistance of phonons, the high degree of
valley polarizations of the finite-momentum exciton states are nearly fully
transferable to the optical polarization of the resulting indirect
photo-luminescence.

\subsection*{\href{http://arxiv.org/abs/2009.13182v1}{Three individual two-axis control of singlet-triplet qubits in a  micromagnet integrated quantum dot array}}
\subsubsection*{Wonjin Jang, \dots, and Dohun Kim (2020-09-28)}
We report individual confinement and two-axis qubit operations of two
electron spin qubits in GaAs gate-defined sextuple quantum dot array with
integrated micro-magnet. As a first step toward multiple qubit operations, we
demonstrate coherent manipulations of three singlet-triplet qubits showing
underdamped Larmor and Ramsey oscillations in all double dot sites. We provide
an accurate measure of site site-dependent field gradients and rms electric and
magnetic noise, and we discuss the adequacy of simple rectangular micro-magnet
for practical use in multiple quantum dot arrays. We also discuss current
limitations and possible strategies for realizing simultaneous multi
multi-qubit operations in extended linear arrays.

\subsection*{\href{http://arxiv.org/abs/2009.13179v1}{Towards Universal Sparse Gaussian Process Potentials: Application to  Lithium Diffusivity in Superionic Conducting Solid Electrolytes}}
\subsubsection*{Amir Hajibabaei, and Kwang S. Kim (2020-09-28)}
For machine learning of interatomic potentials the sparse Gaussian process
regression formalism is introduced with a data-efficient adaptive sampling
algorithm. This is applied for dozens of solid electrolytes. As a showcase,
experimental melting and glass-crystallization temperatures are reproduced for
Li7P3S11 and an unchartered infelicitous phase is revealed with much lower Li
diffusivity which should be circumvented. By hierarchical combinations of the
expert models universal potentials are generated, which pave the way for
modeling large-scale complexity by a combinatorial approach.

\subsection*{\href{http://arxiv.org/abs/2009.13176v1}{How to induce superconductivity in epitaxial graphene $via$ remote  proximity effect through an intercalated gold layer}}
\subsubsection*{Estelle Mazaleyrat, \dots, and Johann Coraux (2020-09-28)}
Graphene holds promises for exploring exotic superconductivity with
Dirac-like fermions. Making graphene a superconductor at large scales is
however a long-lasting challenge. A possible solution relies on
epitaxially-grown graphene, using a superconducting substrate. Such substrates
are scarce, and usually destroy the Dirac character of the electronic band
structure. Using electron diffraction (reflection high-energy, and low-energy),
scanning tunneling microscopy and spectroscopy, atomic force microscopy,
angle-resolved photoemission spectroscopy, Raman spectroscopy, and density
functional theory calculations, we introduce a strategy to induce
superconductivity in epitaxial graphene $via$ a remote proximity effect, from
the rhenium substrate through an intercalated gold layer. Weak graphene-Au
interaction, contrasting with the strong undesired graphene-Re interaction, is
demonstrated by a reduced graphene corrugation, an increased distance between
graphene and the underlying metal, a linear electronic dispersion and a
characteristic vibrational signature, both latter features revealing also a
slight $p$ doping of graphene. We also reveal that the main shortcoming of the
intercalation approach to proximity superconductivity is the creation of a high
density of point defects in graphene (10$^{14}$~cm$^{-2}$). Finally, we
demonstrate remote proximity superconductivity in graphene/Au/Re(0001), at low
temperature.

\subsection*{\href{http://arxiv.org/abs/2009.13150v1}{Atomic-scale control of graphene magnetism using hydrogen atoms}}
\subsubsection*{H. González-Herrero, \dots, and I. Brihuega (2020-09-28)}
Isolated hydrogen atoms absorbed on graphene are predicted to induce magnetic
moments. Here we demonstrate that the adsorption of a single hydrogen atom on
graphene induces a magnetic moment characterized by a ~20 meV spin-split state
at the Fermi energy. Our scanning tunneling microscopy (STM) experiments,
complemented by first-principles calculations, show that such a spin-polarized
state is essentially localized on the carbon sublattice complementary to the
one where the H atom is chemisorbed. This atomically modulated spin-texture,
which extends several nanometers away from the H atom, drives the direct
coupling between the magnetic moments at unusually long distances. Using the
STM tip to manipulate H atoms with atomic precision, we demonstrate the
possibility to tailor the magnetism of selected graphene regions.

\subsection*{\href{http://arxiv.org/abs/2009.13140v1}{Quantum computed moments correction to variational estimates}}
\subsubsection*{Harish J. Vallury, \dots, and Lloyd C. L. Hollenberg (2020-09-28)}
The variational principle of quantum mechanics is the backbone of hybrid
quantum computing for a range of applications. However, as the problem size
grows, quantum logic errors and the effect of barren plateaus overwhelm the
quality of the results. There is now a clear focus on strategies that require
fewer quantum circuit steps and are robust to device errors. Here we present an
approach in which problem complexity is transferred to dynamic quantities
computed on the quantum processor - Hamiltonian moments, $\langle H^n\rangle$.
From these quantum computed moments, estimates of the ground-state energy are
obtained using the ''infinum'' theorem from Lanczos cumulant expansions which
manifestly correct the associated variational calculation. With system dynamics
encoded in the moments the burden on the trial-state quantum circuit depth is
eased. The method is introduced and demonstrated on 2D quantum magnetism models
on lattices up to 5 $\times$ 5 (25 qubits) implemented on IBM Quantum
superconducting qubit devices. Moments were quantum computed to fourth order
with respect to a parameterised antiferromagnetic trial-state. A comprehensive
comparison with benchmark variational calculations was performed, including
over an ensemble of random coupling instances. The results showed that the
infinum estimate consistently outperformed the benchmark variational approach
for the same trial-state. These initial investigations suggest that the quantum
computed moments approach has a high degree of stability against trial-state
variation, quantum gate errors and shot noise, all of which bodes well for
further investigation and applications of the approach.

\subsection*{\href{http://arxiv.org/abs/2009.13137v1}{The background field method and critical vector models}}
\subsubsection*{Mikhail Goykhman, and Michael Smolkin (2020-09-28)}
We use the background field method to systematically derive CFT data for the
critical $\phi^6$ vector model in three dimensions, and the Gross-Neveu model
in dimensions $2\leq d \leq 4$. Specifically, we calculate the OPE coefficients
and anomalous dimensions of various operators, up to next-to-leading order in
the $1/N$ expansion.

\subsection*{\href{http://arxiv.org/abs/2009.13136v1}{Skyrmion Chirality Inversion in Ta/FeCoB/TaOx trilayers}}
\subsubsection*{Raj Kumar, \dots, and Hélène Béa (2020-09-28)}
Skyrmions are nontrivial spiral spin textures considered as potential
building blocks for ultrafast and power efficient spintronic memory and logic
devices. Controlling their chirality would provide an additional degree of
freedom and enable new functionalities in these devices. Achieving such control
requires adjusting the interfacial Dzyaloshinskii-Moriya interaction (DMI).
Thanks to Brillouin Light scattering measurements in Ta/FeCoB/TaOx trilayer, we
have evidenced a DMI sign crossover when tuning TaOx oxidation and suspected
another DMI sign crossover when tuning FeCoB thickness. Moreover, using polar
magneto-optical Kerr effect microscopy, we demonstrate skyrmion chirality
inversion through their opposite current induced motion direction either by
changing FeCoB thickness or TaOx oxidation rate. This chirality inversion
enables a more versatile manipulation of skyrmions, paving the way towards
multidirectional devices.

\subsection*{\href{http://arxiv.org/abs/2009.13115v1}{Prediction of giant and ideal Rashba-type splitting in ordered alloy  monolayers grown on a polar surface}}
\subsubsection*{Mingxing Chen and Feng Liu (2020-09-28)}
A large and ideal Rashba-type spin-orbit splitting is desired for the
applications of materials in spintronic devices and the detection of Majorana
Fermions in solids. Here, we propose an approach to achieve giant and ideal
spin-orbit splittings through a combination of ordered surface alloying and
interface engineering, that is, growing alloy monolayers on an insulating polar
surface. We illustrate this unique strategy by means of first-principles
calculations of buckled hexagonal monolayers of SbBi and PbBi supported on
Al$_2$O$_3$(0001). Both systems display ideal Rashba-type states with giant SO
splittings, characterized with energy offsets over 600 meV and momentum offsets
over 0.3 $\AA^{-1}$, respectively. Our study thus points to an effective way of
tuning spin-orbit splitting in low-dimensional materials to draw immediate
experimental interest.

\subsection*{\href{http://arxiv.org/abs/2009.13110v1}{Squeezing-induced Topological Gap Opening on Bosonic Bogoliubov  Excitations}}
\subsubsection*{Liang-Liang Wan, and Zhi-Fang Xu (2020-09-28)}
We investigate the role of squeezing interaction in inducing topological
Bogoliubov excitations of a bosonic system. We introduce a squeezing
transformation which is capable of reducing the corresponding Bogoliubov-de
Gennes Hamiltonian to an effective non-interacting one with the spectra and
topology unchanged. In the weak interaction limit, we apply the perturbation
theory to investigate the squeezing-induced topological gap opening on bosonic
Bogoliubov excitations and find that the squeezing interaction plays an
equivalent role as a spin-orbit or Zeeman-like coupling in the effective
Hamiltonian. We thus apply this formalism to two existed models for providing
deeper understandings of their topological structures. We also construct
minimal models based on the elegant Clifford algebra for realizing bosonic
topological Bogoliubov excitations. Our construction is potentially applicable
for experiments in bosonic systems.

\subsection*{\href{http://arxiv.org/abs/2009.13109v1}{Tuning the Ultrafast Response of Fano Resonances in Halide Perovskite  Nanoparticles}}
\subsubsection*{Paolo Franceschini, \dots, and Claudio Giannetti (2020-09-28)}
The full control of the fundamental photophysics of nanosystems at
frequencies as high as few THz is key for tunable and ultrafast nano-photonic
devices and metamaterials. Here we combine geometrical and ultrafast control of
the optical properties of halide perovskite nanoparticles, which constitute a
prominent platform for nanophotonics. The pulsed photoinjection of free
carriers across the semiconducting gap leads to a sub-picosecond modification
of the far-field electromagnetic properties that is fully controlled by the
geometry of the system. When the nanoparticle size is tuned so as to achieve
the overlap between the narrowband excitons and the geometry-controlled Mie
resonances, the ultrafast modulation of the transmittivity is completely
reversed with respect to what is usually observed in nanoparticles with
different sizes, in bulk systems and in thin films. The interplay between
chemical, geometrical and ultrafast tuning offers an additional control
parameter with impact on nano-antennas and ultrafast optical switches.

\subsection*{\href{http://arxiv.org/abs/2009.13106v2}{Extraction of Material Properties through Multi-fidelity Deep Learning  from Molecular Dynamics Simulation}}
\subsubsection*{Mahmudul Islam, \dots, and Mohammad Nasim Hasan (2020-09-28)}
Simulation of reasonable timescales for any long physical process using
molecular dynamics (MD) is a major challenge in computational physics. In this
study, we have implemented an approach based on multi-fidelity physics informed
neural network (MPINN) to achieve long-range MD simulation results over a large
sample space with significantly less computational cost. The fidelity of our
present multi-fidelity study is based on the integration timestep size of MD
simulations. While MD simulations with larger timestep produce results with
lower level of accuracy, it can provide enough computationally cheap training
data for MPINN to learn an accurate relationship between these low-fidelity
results and high-fidelity MD results obtained using smaller simulation
timestep. We have performed two benchmark studies, involving one and two
component LJ systems, to determine the optimum percentage of high-fidelity
training data required to achieve accurate results with high computational
saving. The results show that important system properties such as system energy
per atom, system pressure and diffusion coefficients can be determined with
high accuracy while saving 68\% computational costs. Finally, as a demonstration
of the applicability of our present methodology in practical MD studies, we
have studied the viscosity of argon-copper nanofluid and its variation with
temperature and volume fraction by MD simulation using MPINN. Then we have
compared them with numerous previous studies and theoretical models. Our
results indicate that MPINN can predict accurate nanofluid viscosity at a wide
range of sample space with significantly small number of MD simulations. Our
present methodology is the first implementation of MPINN in conjunction with MD
simulation for predicting nanoscale properties. This can pave pathways to
investigate more complex engineering problems that demand long-range MD
simulations.

\subsection*{\href{http://arxiv.org/abs/2009.13105v1}{Dynamical polarization and plasmons in noncentrosymmetric metals}}
\subsubsection*{Sonu Verma, and Tarun Kanti Ghosh (2020-09-28)}
We study the dynamical polarization function and plasmon modes for spin-orbit
coupled noncentrosymmetric metals (NCMs). These systems have different Fermi
surface topology for Fermi energies above and below the spin degenerate point
which is also known as the band touching point (BTP). We calculate the exact
dynamical polarization function numerically and also provide its analytical
expression in the long wavelength limit. We obtain the plasmon dispersion
within the framework of random phase approximation. In NCMs, there is a finite
energy gap in between intra and interband particle hole continuum (PHC) for
vanishing excitation wavevector. In the long wavelength limit, the width of
interband PHC behaves differently for Fermi energies below and above the BTP as
a clear signature of the Fermi surface topology change. We find a single
undamped optical plasmon mode lying in between the intra and interband PHC for
Fermi energies above and below the BTP. The plasmon mode below the BTP has
smaller velocity than that of above the BTP. It is interesting to find that as
we tune the Fermi energy around the BTP, the plasmon mode becomes damped within
a range of e-e interaction strength. For Fermi energies above and below the
BTP, we also obtain an approximate analytical result of plasma frequency and
plasmon dispersion which match well with their numerical counterparts in the
long wavelength limit. The plasmon dispersion is $\propto q^2$ with $q$ being
the wave vector for plasmon excitation in the long wavelength limit. We find
that varying the carrier density with fixed e-e interaction strength or vice
versa does not change the number of undamped plasmon mode, although damped
plasmon modes can be more in number for some values of these parameters. We
demonstrate our results by calculating the loss function and optical
conductivity which can be measured in experiments.

\subsection*{\href{http://arxiv.org/abs/2009.13099v1}{kMap.py: A Python program for simulation and data analysis in  photoemission tomography}}
\subsubsection*{Dominik Brandstetter, \dots, and Peter Puschnig (2020-09-28)}
For organic molecules adsorbed as well-oriented ultra-thin films on metallic
surfaces, angle-resolved photoemission spectroscopy has evolved into a
technique called photoemission tomography (PT). By approximating the final
state of the photoemitted electron as a free electron, PT uses the angular
dependence of the photocurrent, a so-called momentum map or k-map, and
interprets it as the Fourier transform of the initial state's molecular
orbital, thereby gains insights into the geometric and electronic structure of
organic/metal interfaces.
  In this contribution, we present kMap.py which is a Python program that
enables the user, via a PyQt-based graphical user interface, to simulate
photoemission momentum maps of molecular orbitals and to perform a one-to-one
comparison between simulation and experiment. Based on the plane wave
approximation for the final state, simulated momentum maps are computed
numerically from a fast Fourier transform of real space molecular orbital
distributions, which are used as program input and taken from density
functional calculations. The program allows the user to vary a number of
simulation parameters such as the final state kinetic energy, the molecular
orientation or the polarization state of the incident light field. Moreover,
also experimental photoemission data can be loaded into the program enabling a
direct visual comparison as well as an automatic optimization procedure to
determine structural parameters of the molecules or weights of molecular
orbitals contributions. With an increasing number of experimental groups
employing photoemission tomography to study adsorbate layers, we expect kMap.py
to serve as an ideal analysis software to further extend the applicability of
PT.

\subsection*{\href{http://arxiv.org/abs/2009.13094v1}{Improved generalization by noise enhancement}}
\subsubsection*{Takashi Mori and Masahito Ueda (2020-09-28)}
Recent studies have demonstrated that noise in stochastic gradient descent
(SGD) is closely related to generalization: A larger SGD noise, if not too
large, results in better generalization. Since the covariance of the SGD noise
is proportional to $\eta^2/B$, where $\eta$ is the learning rate and $B$ is the
minibatch size of SGD, the SGD noise has so far been controlled by changing
$\eta$ and/or $B$. However, too large $\eta$ results in instability in the
training dynamics and a small $B$ prevents scalable parallel computation. It is
thus desirable to develop a method of controlling the SGD noise without
changing $\eta$ and $B$. In this paper, we propose a method that achieves this
goal using ``noise enhancement'', which is easily implemented in practice. We
expound the underlying theoretical idea and demonstrate that the noise
enhancement actually improves generalization for real datasets. It turns out
that large-batch training with the noise enhancement even shows better
generalization compared with small-batch training.

\subsection*{\href{http://arxiv.org/abs/2009.13089v1}{Coupled dimer and bond-order-wave order in the quarter-filled  one-dimensional Kondo lattice model}}
\subsubsection*{Yixuan Huang, and C. S. Ting (2020-09-28)}
Motivated by the experiments on the organic compound
$(Per)_{2}[Pt(mnt)_{2}]$, we study the ground state of the one-dimensional
Kondo lattice model at quarter filling with the density matrix renormalization
group method. We show a coupled dimer and bond-order-wave (BOW) state in the
weak coupling regime for the localized spins and itinerant electrons,
respectively. The quantum phase transitions for the dimer and the BOW orders
occur at the same critical coupling parameter $J_{c}$, with the opening of a
charge gap. The emergence of the combination of dimer and BOW order agrees with
the experimental findings of the simultaneous Peierls and spin-Peierls
transitions at low temperatures, which provides a theoretical understanding of
such phase transition. We also show that the localized spins in this insulating
state have quasi-long ranged spin correlations with collinear configurations,
which resemble the classical dimer order in the absence of a magnetic order.

\subsection*{\href{http://arxiv.org/abs/2009.13078v1}{Dual topological characterization of non-Hermitian Floquet phases}}
\subsubsection*{Longwen Zhou, and Jiangbin Gong (2020-09-28)}
Non-Hermiticity is expected to add far more physical features to the already
rich Floquet topological phases of matter. Nevertheless, a systematic approach
to characterize non-Hermitian Floquet topological matter is still lacking. In
this work we introduce a dual scheme to characterize the topology of
non-Hermitian Floquet systems in momentum space and in real space, using a
piecewise quenched nonreciprocal Su-Schrieffer-Heeger model for our case
studies. Under the periodic boundary condition, topological phases are
characterized by a pair of experimentally accessible winding numbers that make
jumps between integers and half-integers. Under the open boundary condition, a
Floquet version of the so-called open boundary winding number is found to be
integers and can predict the number of pairs of zero and $\pi$ Floquet edge
modes coexisting with the non-Hermitian skin effect. Our results indicate that
a dual characterization of non-Hermitian Floquet topological matter is
necessary and also feasible because the formidable task of constructing the
celebrated generalized Brillouin zone for non-Hermitian Floquet systems with
multiple hopping length scales can be avoided. This work hence paves a way for
further studies of non-Hermitian physics in non-equilibrium systems.

\subsection*{\href{http://arxiv.org/abs/2009.13056v1}{A Facile Process to Make Phosphorus-doped Carbon Xerogel as Anode for  Sodium Ion Batteries}}
\subsubsection*{Changyu Deng and Wei Lu (2020-09-28)}
Sodium ion batteries become popular due to their low cost. Among possible
anode materials of sodium ion batteries, phosphorus has great potential owing
to its high theoretical capacity. Previous research that yielded high capacity
of phosphorus anode used very expensive materials such as black phosphorus (BP)
and phosphorene. To take advantage of the low cost of sodium ion batteries, we
proposed a new method to make anode: condensing red phosphorus (RP) on carbon
xerogel. Even with large particle size (~ 50 $\mu$m) and high mass loading (2
mg/cm$^2$), the composite cycled at 200 mA/g yielded a capacity of 242 mA/g, or
1993 mAh/g based on phosphorus after subtracting the contribution of carbon.
The average degradation rate is only 0.06\% in 80 cycles. The average columbic
efficiency is as high as 99.2\%. Our research provided an innovative approach to
synthesis of anodes for sodium ion batteries, which could promote their
commercialization.

\subsection*{\href{http://arxiv.org/abs/2009.13043v1}{Stability of Time-Reversal Symmetry Protected Topological Phases}}
\subsubsection*{Tian-Shu Deng, \dots, and Hui Zhai (2020-09-28)}
In a closed system, it is well known that the time-reversal symmetry can lead
to Kramers degeneracy and protect nontrivial topological states such as quantum
spin Hall insulator. In this letter we address the issue whether these effects
are stable against coupling to environment, provided that both environment and
the coupling to environment also respect the time-reversal symmetry. By
employing a non-Hermitian Hamiltonian with the Langevin noise term and
ultilizing the non-Hermitian linear response theory, we show that the spectral
functions for Kramers degenerate states can be split by dissipation, and the
backscattering between counter-propagating edge states can be induced by
dissipation. The latter leads to the absence of accurate quantization of
conductance in the case of quantum spin Hall effect. As an example, we
demonstrate this concretely with the Kane-Mele model. Our study could also be
extended to interacting topological phases protected by the time-reversal
symmetry.

\subsection*{\href{http://arxiv.org/abs/2009.13042v1}{Differential electron yield imaging with STXM}}
\subsubsection*{William A. Hubbard, \dots, and B. C. Regan (2020-09-28)}
Total electron yield (TEY) imaging is an established scanning transmission
X-ray microscopy (STXM) technique that gives varying contrast based on a
sample's geometry, elemental composition, and electrical conductivity. However,
the TEY-STXM signal is determined solely by the electrons that the beam ejects
from the sample. A related technique, X-ray beam-induced current (XBIC)
imaging, is sensitive to electrons and holes independently, but requires
electric fields in the sample. Here we report that multi-electrode devices can
be wired to produce differential electron yield (DEY) contrast, which is also
independently sensitive to electrons and holes, but does not require an
electric field. Depending on whether the region illuminated by the focused STXM
beam is better connected to one electrode or another, the DEY-STXM contrast
changes sign. DEY-STXM images thus provide a vivid map of a device's
connectivity landscape, which can be key to understanding device function and
failure. To demonstrate an application in the area of failure analysis, we
image a 100~nm, lithographically-defined aluminum nanowire that has failed
after being stressed with a large current density.

\subsection*{\href{http://arxiv.org/abs/2009.13041v1}{Cu-Doped KCl folded and unfolded band structure and optical properties  studied by DFT calculations}}
\subsubsection*{Jose Luis Cabellos, \dots, and Alvaro Posada-Amarillas (2020-09-28)}
We computed the optical properties and the folded and unfolded band structure
of Cu-doped KCl crystals. The calculations use the plane-wave pseudo-potential
approach implemented in the ABINIT electronic structure package within the
first-principles density-functional theory framework. Cu substitution into
pristine KCl crystals requires calculation by the supercell (SC) method from a
theoretical perspective. This procedure shrinks the Brillouin zone, resulting
in a folded band structure that is difficult to interpret. To solve this
problem and gain insight into the effect of cuprous ion (Cu+) on electronic
properties; We unfolded the band structure of SC KCl:Cu to directly compare
with the band structure of the primitive cell (PC) of pristine KCl. To
understand the effect of Cu substitution on optical absorption, we calculated
the imaginary part of the dielectric function of KCl:Cu through a
sum-over-states formalism and broke it down into different band contributions
by partially making an iterated cumulative sum (ICS) of selected valence and
conduction bands. As a result, we identified those interband transitions that
give rise to the absorption peaks due to the Cu ion. These transitions include
valence and conduction bands formed by the Cu-3d and Cu-4s electronic states.
To investigate the effects of doping position, we consider different doping
positions, where the Cu dopant occupies all the substitutional sites replacing
host K cations. Our results indicate that the doping position's effects give
rise to two octahedral shapes in the geometric structure. The distorted-twisted
octahedral square bipyramidal geometric-shape induces a difference in the
crystal field splitting energy compared to that of the perfect octahedral
square bipyramidal geometric-shape.

\subsection*{\href{http://arxiv.org/abs/2009.13034v1}{Tunable SSH model in ferromagnetic systems}}
\subsubsection*{Chi-Ho Cheung and Jinyu Zou (2020-09-28)}
It is well known that the topology of Su-Schrieffer-Heeger(SSH) model, which
belongs to AIII symmetry class, is protected by chiral symmetry. In this
article, instead of chiral symmetry, we constrain the bulk Hamiltonian by a
magnetic point group symmetry, which can be generated by a unitary symmetry and
an anti-unitary symmetry. Under these symmetries, a four-band model can be
block diagonalized into two 2-band models and each 2-band model is analogous to
an SSH model. As the two 2-band models are individual, we call the four-band
model double independent SSH (DISSH) model. Interestingly, since the symmetry
requirements of DISSH model can be fulfilled in ferromagnetic systems, the
discovery in this manuscript extends SSH model into ferromagnetic systems.
Furthermore, we presented an example of DISSH model with a set of reasonable
parameters to show that it is possible to manipulate the topological phase of
DISSH model by tuning the magnetic moment in experiment.

\subsection*{\href{http://arxiv.org/abs/2009.13031v1}{Mechanical properties of lateral transition metal dichalcogenide  heterostructures}}
\subsubsection*{Sadegh Imani Yengejeh, and Yun Wang (2020-09-28)}
Transition metal dichalcogenide (TMD) monolayers attract great attention due
to their specific structural, electronic and mechanical properties. The
formation of their lateral heterostructures allows a new degree of flexibility
in engineering electronic and optoelectronic devices. However, the mechanical
properties of the lateral heterostructures are rarely investigated. In this
study, a comparative investigation on the mechanical characteristics of 1H, 1T'
and 1H/1T' heterostructure phases of different TMD monolayers including
molybdenum disulfide (MoS2) molybdenum diselenide (MoSe2), Tungsten disulfide
(WS2), and Tungsten diselenide (WSe2) was conducted by means of density
functional theory (DFT) calculations. Our results indicate that the lateral
heterostructures have a relatively weak mechanical strength for all the TMD
monolayers. The significant correlation between the mechanical properties of
the TMD monolayers and their structural phases can be used to tune their
stiffness of the materials. Our findings, therefore, suggest a novel strategy
to manipulate the mechanical characteristics of TMDs by engineering their
structural phases for their practical applications.

\subsection*{\href{http://arxiv.org/abs/2009.13022v1}{Coexistence of intrinsic piezoelectricity and ferromagnetism induced by  small biaxial strain in septuple-atomic-layer $\mathrm{VSi_2P_4}$}}
\subsubsection*{San-Dong Guo, \dots, and Xing-Qiu Chen (2020-09-28)}
The septuple-atomic-layer $\mathrm{VSi_2P_4}$ with the same structure of
experimentally synthesized $\mathrm{MoSi_2N_4}$ is predicted to be a
spin-gapless semiconductor (SGS). In this work, the biaxial strain is applied
to tune electronic properties of $\mathrm{VSi_2P_4}$, and it spans a wide range
of properties upon the increasing strain from ferromagnetic metal (FMM) to SGS
to ferromagnetic semiconductor (FMS) to SGS to ferromagnetic half-metal (FMHM).
Due to broken inversion symmetry, the coexistence of ferromagnetism and
piezoelectricity can be achieved in FMS $\mathrm{VSi_2P_4}$ with strain range
of 0\% to 4\%. The calculated piezoelectric strain coefficients $d_{11}$ for
1\%, 2\% and 3\% strains are 4.61 pm/V, 4.94 pm/V and 5.27 pm/V, respectively,
which are greater than or close to a typical value of 5 pm/V for bulk
piezoelectric materials. Finally, similar to $\mathrm{VSi_2P_4}$, the
coexistence of piezoelectricity and ferromagnetism can be realized by strain in
the $\mathrm{VSi_2N_4}$ monolayer. Our works show that $\mathrm{VSi_2P_4}$ in
FMS phase with intrinsic piezoelectric properties can have potential
applications in spin electronic devices.

\subsection*{\href{http://arxiv.org/abs/2009.13014v1}{Rolling Waves with Non-Paraxial Phonon Spins}}
\subsubsection*{Peng Zhang, \dots, and Pai Wang (2020-09-28)}
We demonstrate a new class of elastic waves in the bulk: When longitudinal
and transverse components propagate at the same speed, rolling waves with a
spin that is not parallel to the wave vector can emerge. First, we give a
general definition of spin for traveling waves. Then, since rolling waves
cannot exist in isotropic solids, we derive conditions for anisotropic media
and proceed to design architected materials capable of hosting rolling waves.
Numerically, we show spin manipulations by reflection. Structures reported in
this work can be fabricated using available techniques, opening new
possibilities for spin technologies in acoustics, mechanics and phononics.

\subsection*{\href{http://arxiv.org/abs/2009.13007v1}{High Fidelity Entangling Gates in a 3D Ion Crystal under Micromotion}}
\subsubsection*{Y. -K. Wu, \dots, and L. -M. Duan (2020-09-28)}
Ion trap is one of the most promising candidates for quantum computing.
Current schemes mainly focus on a linear chain of up to about one hundred ions
in a Paul trap. To further scale up the qubit number, one possible direction is
to use 2D or 3D ion crystals (Wigner crystals). In these systems, ions are
generally subjected to large micromotion due to the strong fast-oscillating
electric field, which can significantly influence the performance of entangling
gates. In this work, we develop an efficient numerical method to design
high-fidelity entangling gates in a general 3D ion crystal. We present
numerical algorithms to solve the equilibrium configuration of the ions and
their collective normal modes. We then give mathematical description of the
micromotion and use it to generalize the gate scheme for linear ion chains into
a general 3D crystal. The involved time integral of highly oscillatory
functions is expanded into a fast-converging series for accurate and efficient
evaluation and optimization. As a numerical example, we show a high-fidelity
entangling gate design between two ions in a 100-ion crystal, with a
theoretical fidelity of 99.9\%.

\subsection*{\href{http://arxiv.org/abs/2009.12998v1}{Axion electrodynamics in $p+is$ superconductors}}
\subsubsection*{Chao Xu and Wang Yang (2020-09-28)}
We perform a systematic study of axion electrodynamics in $p+is$
superconductors. Unlike the superconducting Dirac/Weyl systems, the induced
electric field does not enter into the axion action. Furthermore, in addition
to the usual axion angle which is defined as the phase difference between the
superconducting phases on the two Fermi surfaces of different helicities, the
axion field contains an additional sinusoidal term. Our work reveals the
differences for axion electrodynamics between the relativistic cases and the
$p+is$ superconductors.

\subsection*{\href{http://arxiv.org/abs/2009.12995v1}{Thermal Magnetic Fluctuations of a Ferroelectric Quantum Critical Point}}
\subsubsection*{Alexander Khaetskii, and Alexander V. Balatsky (2020-09-28)}
Entanglement of two different quantum orders is of an interest of the modern
condensed matter physics. One of the examples is the dynamical multiferroicity,
where fluctuations of electric dipoles lead to magnetization. We investigate
this effect at finite temperature and demonstrate an elevated magnetic response
of a ferroelectric near the ferroelectric quantum critical point (FE QCP). We
calculate the magnetic susceptibility of a bulk sample on the paraelectric side
of the FE QCP at finite temperature and find enhanced magnetic susceptibility
near the FE QCP. We propose quantum paraelectric strontium titanate (STO) as a
candidate material to search for dynamic multiferroicity. We estimate the
magnitude of the magnetic susceptibility for this material and find that it is
detectable experimentally.

\subsection*{\href{http://arxiv.org/abs/2009.12985v1}{Two-dimensional Janus van der Waals heterojunctions: a review of recent  research progresses}}
\subsubsection*{Lin Ju, \dots, and Liangzhi Kou (2020-09-28)}
Two-dimensional Janus van der Waals (vdW) heterojunctions, referring to the
junction containing at least one Janus material, are found to exhibit tuneable
electronic structures, wide light adsorption spectra, controllable contact
resistance, and sufficient redox potential due to the intrinsic polarization
and unique interlayer coupling. These novel structures and properties are
promising for the potential applications in electronics and energy conversion
devices. To provide a comprehensive picture about the research progress and
guide the following investigations, here we summarize their fundamental
properties of different types of two-dimensional Janus vdW heterostructures
including electronic structure, interface contact and optical properties, and
discuss the potential applications in electronics and energy conversion
devices. The further challenges and possible research directions of the novel
heterojunctions are discussed at the end of this review.

\subsection*{\href{http://arxiv.org/abs/2009.12982v1}{Quantum soundness of the classical low individual degree test}}
\subsubsection*{Zhengfeng Ji, \dots, and Henry Yuen (2020-09-27)}
Low degree tests play an important role in classical complexity theory,
serving as basic ingredients in foundational results such as $\mathsf{MIP} =
\mathsf{NEXP}$ [BFL91] and the PCP theorem [AS98,ALM+98]. Over the last ten
years, versions of these tests which are sound against quantum provers have
found increasing applications to the study of nonlocal games and the complexity
class~$\mathsf{MIP}^*$. The culmination of this line of work is the result
$\mathsf{MIP}^* = \mathsf{RE}$ [JNV+20].
  One of the key ingredients in the first reported proof of $\mathsf{MIP}^* =
\mathsf{RE}$ is a two-prover variant of the low degree test, initially shown to
be sound against multiple quantum provers in [Vid16]. Unfortunately a mistake
was recently discovered in the latter result, invalidating the main result of
[Vid16] as well as its use in subsequent works, including [JNV+20].
  We analyze a variant of the low degree test called the low individual degree
test. Our main result is that the two-player version of this test is sound
against quantum provers. This soundness result is sufficient to re-derive
several bounds on~$\mathsf{MIP}^*$ that relied on [Vid16], including
$\mathsf{MIP}^* = \mathsf{RE}$.

\subsection*{\href{http://arxiv.org/abs/2009.12972v1}{Multiple polarization orders in individual twinned colloidal  nanocrystals of centrosymmetric HfO2}}
\subsubsection*{Hongchu Du, \dots, and Joachim Mayer (2020-09-27)}
Spontaneous polarization is essential for ferroelectric functionality in
non-centrosymmetric crystals. High-integration-density ferroelectric devices
require the stabilization of ferroelectric polarization in small volumes. Here,
atomic-resolution transmission electron microscopy imaging reveals that
twinning-induced symmetry breaking in colloidal nanocrystals of centrosymmetric
HfO2 leads to the formation of multiple polarization orders, which are
associated with sub-nanometer ferroelectric and antiferroelectric phases. The
minimum size limit of the ferroelectric phase is found to be ~4 nm3. Density
functional theory calculations indicate that transformations between the
ferroelectric and antiferroelectric phases can be modulated by lattice strain
and are energetically possible in either direction. The results of this work
provide a route towards applications of HfO2 nanocrystals in information
storage at densities that are more than an order of magnitude higher than the
scaling limit defined by the nanocrystal size.

\subsection*{\href{http://arxiv.org/abs/2009.12946v1}{Effects of spatially-varying substrate anchoring on instabilities and  dewetting of thin Nematic Liquid Crystal films}}
\subsubsection*{M. Lam, and L. Cummings (2020-09-27)}
Partially wetting nematic liquid crystal (NLC) films on substrates are
unstable to dewetting-type instabilities due to destablizing solid/NLC
interaction forces. These instabilities are modified by the nematic nature of
the films, which influences the effective solid/NLC interaction. In this work,
we focus on the influence of imposed substrate anchoring on the instability
development. The analysis is carried out within a long-wave formulation based
on the Leslie-Ericksen description of NLC films. Linear stability analysis of
the resulting equations shows that some features of the instability, such as
emerging wavelengths, may not be influenced by the imposed substrate anchoring.
Going further into the nonlinear regime, considered via large-scale GPU-based
simulations, shows however that nonlinear effects may play an important role,
in particular in the case of strong substrate anchoring anisotropy. Our
simulations show that instability of the film develops in two stages: the first
stage involves formation of ridges that are perpendicular to the local
anchoring direction; and the second involves breakup of these ridges and
formation of drops, whose final distribution is influenced by the anisotropy
imposed by the substrate. Finally, we show that imposing more complex substrate
anisotropy patterns allows us to reach basic understanding of the influence of
substrate-imposed defects in director orientation on the instability evolution.

\subsection*{\href{http://arxiv.org/abs/2009.12943v1}{Key role of the moire potential for the quasi-condensation of interlayer  excitons in van der Waals heterostructures}}
\subsubsection*{Camille Lagoin and Francois Dubin (2020-09-27)}
Interlayer excitons confined in bilayer heterostructures of transition metal
dichalcogenides (TMDs) offer a promising route to implement two-dimensional
dipolar superfluids. Here, we study the experimental conditions necessary for
the realisation of such collective state. Particularly, we show that the moire
potential inherent to TMD bilayers yields an exponential increase of the
excitons effective mass. To allow for exciton superfluidity at sizeable
temperatures it is then necessary to intercalate a high-k dielectric between
the monolayers confining electrons and holes. Thus the moire lattice depth is
sufficiently weak for a superfluid phase to theoretically emerge below a
critical temperature of around 10 K. Importantly, for realistic experimental
parameters interlayer excitons quasi-condense in a state with finite momentum,
so that the superfluid is optically inactive and flows spontaneously.

\subsection*{\href{http://arxiv.org/abs/2009.12941v1}{Theoretical study on the electric field effect on magnetism of Pd/Co/Pt  thin films}}
\subsubsection*{Eszter Simon, \dots, and Hubert Ebert (2020-09-27)}
Based on first principles calculations we investigate the electronic and
magnetic properties of Pt layers in Pd$(001)$/Co/Pt thin film structures
exposed to an external electric field. Due to the Co underlayer, the surface Pt
layers have induced moments that are modified by an external electric field.
The field induced changes can be explained by the modified spin-dependent
orbital hybridization that varies non-linearly with the field strength. We
calculate the x-ray absorption and the x-ray magnetic circular dichroism
spectra for an applied external electric field and examine its impact on the
spectra in the Pt layer around the L$_{2}$ and L$_{3}$ edges. We also determine
the layer dependent magneto-crystalline anisotropy and show that the anisotropy
can be tuned easily in the different layers by the external electric field.

\subsection*{\href{http://arxiv.org/abs/2009.12939v1}{Strong replica symmetry for high-dimensional disordered log-concave  Gibbs measures}}
\subsubsection*{Jean Barbier, and Manuel Sáenz (2020-09-27)}
We consider a generic class of log-concave, possibly random, (Gibbs)
measures. Using a new type of perturbation we prove concentration of an
infinite family of order parameters called multioverlaps. These completely
parametrise the quenched Gibbs measure of the system, so that their
self-averaging behavior implies a simple representation of asymptotic Gibbs
measures, as well as decoupling of the variables at hand in a strong sense. Our
concentration results may prove themselves useful in several contexts. In
particular in machine learning and high-dimensional inference, log-concave
measures appear in convex empirical risk minimisation, maximum a-posteriori
inference or M-estimation. We believe that our results may be applicable in
establishing some type of "replica symmetric formulas" for the free energy,
inference or generalisation error in such settings.

\subsection*{\href{http://arxiv.org/abs/2009.12934v1}{Logarithmic superdiffusivity of the 2-dimensional anisotropic KPZ  equation}}
\subsubsection*{Giuseppe Cannizzaro, and Fabio Toninelli (2020-09-27)}
We study an anisotropic variant of the two-dimensional Kardar-Parisi-Zhang
equation, that is relevant to describe growth of vicinal surfaces and has
Gaussian, logarithmically rough, stationary states. While the folklore belief
(based on one-loop Renormalization Group) is that the equation has the same
scaling behaviour as the (linear) Edwards-Wilkinson equation, we prove that, on
the contrary, the non-linearity induces the emergence of a logarithmic
super-diffusivity. This phenomenon is similar in flavour to the
super-diffusivity for two-dimensional fluids and driven particle systems.

\subsection*{\href{http://arxiv.org/abs/2009.12925v1}{Dynamical spin susceptibility in La2CuO4 studied by resonant inelastic  x-ray scattering}}
\subsubsection*{H. C. Robarts, \dots, and Ke-Jin Zhou (2020-09-27)}
Resonant inelastic X-ray scattering (RIXS) is a powerful probe of elementary
excitations in solids. It is now widely applied to study magnetic excitations.
However, its complex cross-section means that RIXS has been more difficult to
interpret than inelastic neutron scattering (INS). Here we report
high-resolution RIXS measurements of magnetic excitations of La2CuO4, the
antiferromagnetic parent of one system of high-temperature superconductors. At
high energies (~2 eV), the RIXS spectra show angular-dependent dd orbital
excitations which are found to be in good agreement with single-site multiplet
calculations. At lower energies (<0.3 eV), we show that the
wavevector-dependent RIXS intensities are proportional to the product of the
single-ion spin-flip cross section and the dynamical susceptibility of the
spin-wave excitations. When the spin-flip crosssection is dividing out, the
RIXS magnon intensities show a remarkable resemblance to INS data. Our results
show that RIXS is a quantitative probe the dynamical spin susceptibility in
cuprate and therefore should be used for quantitative investigation of other
correlated electron materials.

\subsection*{\href{http://arxiv.org/abs/2009.12917v1}{Changing the universality class of the three-dimensional  Edwards-Anderson spin-glass model by selective bond dilution}}
\subsubsection*{F. Romá (2020-09-27)}
The three-dimensional Edwards-Anderson spin-glass model present strong
spatial heterogeneities well characterized by the called "backbone", a magnetic
structure that arises as a consequence of the properties of the ground state
and the low-excitation levels of such a frustrated Ising system. Using
extensive Monte Carlo simulations and finite size scaling, we study how these
heterogeneities affect the phase transition of the model. Although, we do not
detect any significant difference between the critical behavior displayed by
the whole system and that observed inside and outside the backbone,
surprisingly, a selective bond dilution of the complement of this magnetic
structure induces a change of the universality class, whereas no change is
noted when the backbone is fully diluted. This finding suggests that the region
surrounding the backbone plays a more relevant role in determining the physical
properties of the Edwards-Anderson spin-glass model than previously thought.
Furthermore, we show that when a selective bond dilution changes the
universality class of the phase transition, the ground state of the model does
not undergo any change. The opposite case is also valid, i. e., a dilution that
does not change the critical behavior significantly affects the fundamental
level.

\subsection*{\href{http://arxiv.org/abs/2009.12902v1}{Quantum-mechanics free subsystem with mechanical oscillators}}
\subsubsection*{Laure Mercier de Lépinay, \dots, and Mika A. Sillanpää (2020-09-27)}
Quantum mechanics sets a limit for the precision of measurement of the
position of an oscillator. The quantum noise associated with the measurement of
a quadrature of the motion imprints a backaction on the orthogonal quadrature,
which feeds back to the measured observable in the case of a continuous
measurement. In a quantum backaction evading measurement, the added noise can
be confined in the orthogonal quadrature. Here we show how it is possible to
evade this limitation and measure an oscillator without backaction by
constructing one effective oscillator from two physical oscillators. This
facilitates detection of weak forces and the creation and measurement of
nonclassical motional states of the oscillators. We realize the proposal using
two micromechanical oscillators, and show the measurements of two collective
quadratures while evading the quantum backaction by $8$ decibels on both of
them, obtaining a total noise within a factor two from the full quantum limit.
Moreover, by modifying the measurement we directly verify the quantum
entanglement of the two oscillators by measuring the Duan quantity $1.3$
decibels below the separability bound.

\subsection*{\href{http://arxiv.org/abs/2009.12891v1}{Intrinsic Andreev $π$-reflection and Josephson $π$-junction for  centrosymmetric spin-triplet superconductors}}
\subsubsection*{Han-Bing Leng, and Xin Liu (2020-09-27)}
In this work, we systematically study two phases, called Andreev $\pi$-phase
and orbital-phase, and their influence on the Josephson effect. When the system
is time-reversal invariant and centrosymmetric, these two phases only appear in
the odd-parity pairings. The Andreev $\pi$-phase has nothing to do with the
specific form of the odd-parity pairings and means an intrinsic $\pi$-phase
between the spin-triplet Cooper pairs entering and leaving CTSCs in the Andreev
reflections. The orbital-phase corresponds to the phase difference between the
spin-triplet Cooper pairs with opposite spin polarization and depends on the
specific form of the odd-parity gap functions. When the normal region of the
Josephson junction contacts the same side of the CTSCs with some specific
odd-parity parings, the competition between the two phases can lead to the
Josephson $\pi$-junction. Note that this junction is different from that of the
conventional Josephson junction (JJ) and is dubbed a U-shaped junction
according to its geometry. Meanwhile, in a conventional JJ, the interplay of
these two phases causes their impact on the CPR to be completely canceled out.
Therefore no matter what kind of pairing symmetries the CTSC has, it will lead
to Josephson 0-junction in this case. We obtain our results based on the model
of the M$_{x}$Bi$_2$Se$_3$ family where M may be Cu, Sr, or Nb. Therefore, we
propose to detect the pairing symmetry of M$_{x}$Bi$_2$Se$_3$ through a
superconducting quantum interference device containing a U-shaped Josephson
junction.

\subsection*{\href{http://arxiv.org/abs/2009.12890v1}{Giant Anomalous Hall Effect due to Double-Degenerate Quasi Flat Bands}}
\subsubsection*{Wei Jiang, \dots, and Tony Low (2020-09-27)}
We propose a novel approach to achieve giant AHE in materials with flat bands
(FBs). FBs are accompanied by small electronic bandwidths, which consequently
increases the momentum separation ($K$) within pair of Weyl points and thus the
integrated Berry curvature. Starting from a simple model with a single pair of
Weyl nodes, we demonstrated the increase of $K$ and AHE by decreasing
bandwidth. It is further expanded to a realistic pyrochlore lattice model with
characteristic double degenerated FBs, where we discovered a giant AHE while
maximizing the $K$ with nearly vanishing band dispersion of FBs. We identify
that such model system can be realized in both pyrochlore and spinel compounds
based on first-principles calculations, validating our theoretical model and
providing a feasible platform for experimental exploration.

\subsection*{\href{http://arxiv.org/abs/2009.12888v1}{Squeezed comb states}}
\subsubsection*{Namrata Shukla, and Barry C. Sanders (2020-09-27)}
Continuous-variable codes are an expedient solution for quantum information
processing and quantum communication involving optical networks. Here we
characterize the squeezed comb, a finite superposition of equidistant squeezed
coherent states on a line, and its properties as a continuous-variable encoding
choice for a logical qubit. The squeezed comb is a realistic approximation to
the ideal code proposed by Gottesman, Kitaev and Preskill, which is fully
protected against errors caused by the paradigmatic types of quantum noise in
continuous-variable systems: damping and diffusion. This is no longer the case
for the code space of finite squeezed combs, and noise robustness depends
crucially on the encoding parameters. We analyze finite squeezed comb states in
phase space, highlighting their complicated interference features and
characterizing their dynamics when exposed to amplitude damping and Gaussian
diffusion noise processes. We find that squeezed comb state are more suitable
and less error-prone when exposed to damping, which speaks against standard
error correction strategies that employ linear amplification to convert damping
into easier-to-describe isotropic diffusion noise.

\subsection*{\href{http://arxiv.org/abs/2009.12885v2}{Exploring In$_2$(Se$_{1-x}$Te$_x$)$_3$ alloys as photovoltaic materials}}
\subsubsection*{Wei Li, \dots, and Anderson Janotti (2020-09-27)}
In$_2$Se$_3$ in the three-dimensional (3D) hexagonal crystal structure with
space group $P6_1$ ($\gamma$-In$_2$Se$_3$) has a direct band gap of $\sim$1.8
eV and high absorption coefficient, making it a promising semiconductor
material for optoelectronics. Incorporating Te allows for tuning the band gap,
adding flexibility to device design and extending the application range. Here
we report the growth and characterization of $\gamma$-In$_2$Se$_3$ thin films,
and results of hybrid density functional theory calculations to assess the
electronic and optical properties of $\gamma$-In$_2$Se$_3$ and
$\gamma$-In$_2$(Se$_{1-x}$Te$_x$)$_3$ alloys. The calculated band gap of 1.84
eV for $\gamma$-In$_2$Se$_3$ is in good agreement with data from the absorption
spectrum, and the absorption coefficient is found to be as high as that of
direct band gap conventional III-V and II-VI semiconductors. Incorporation of
Te in the form of $\gamma$-In$_2$(Se$_{1-x}$Te$_x$)$_3$ alloys is an effective
way to tune the band gap from 1.84 eV down to 1.23 eV, thus covering the
optimal band gap range for solar cells. We also discuss band gap bowing and
mixing enthalpies, aiming at adding $\gamma$-In$_2$Se$_3$ and
$\gamma$-In$_2$(Se$_{1-x}$Te$_x$)$_3$ alloys to the available toolbox of
materials for solar cells and other optoelectronic devices.

\subsection*{\href{http://arxiv.org/abs/2009.12883v1}{Mode II fracture of an MMA adhesive layer: theory versus experiment}}
\subsubsection*{Sina Askarinejad, \dots, and Norman A. Fleck (2020-09-27)}
Thick adhesive layers have potential structural application in ship
construction for the joining of a composite superstructure to a steel hull. The
purpose of this study is to develop a mechanics model for the adhesive fracture
of such lap joints under shear loading. Modified Thick-Adherend-Shear-Test
(TAST) specimens made from a MMA-based adhesive and steel adherents are
designed and fabricated. Crack initiation and growth of these joints is
measured and monitored by Digital Image Correlation (DIC). An attempt is made
to use a cohesive zone model to predict the magnitude of shear strain across
the adhesive layer both at crack initiation and at peak load, and to predict
the extent of crack growth as a function of shear strain across the adhesive
layer. The ability of a cohesive zone model to predict several features of
specimen failure is assessed for the case of an adhesive layer of high shear
ductility.

\subsection*{\href{http://arxiv.org/abs/2009.12882v1}{Kernels for noninteracting fermions via a Green's function approach with  applications to step potentials}}
\subsubsection*{David S. Dean, \dots, and Naftali R. Smith (2020-09-27)}
The quantum correlations of $N$ noninteracting spinless fermions in their
ground state can be expressed in terms of a two-point function called the
kernel. Here we develop a general and compact method for computing the kernel
in a general trapping potential in terms of the Green's function for the
corresponding single particle Schr\"odinger equation. For smooth potentials the
method allows a simple alternative derivation of the local density
approximation for the density and of the sine kernel in the bulk part of the
trap in the large $N$ limit. It also recovers the density and the kernel of the
so-called {\em Airy gas} at the edge. This method allows to analyse the quantum
correlations in the ground state when the potential has a singular part with a
fast variation in space. For the square step barrier of height $V_0$, we derive
explicit expressions for the density and for the kernel. For large Fermi energy
$\mu>V_0$ it describes the interpolation between two regions of different
densities in a Fermi gas, each described by a different sine kernel. Of
particular interest is the {\em critical point} of the square well potential
when $\mu=V_0$. In this critical case, while there is a macroscopic number of
fermions in the lower part of the step potential, there is only a finite $O(1)$
number of fermions on the shoulder, and moreover this number is independent of
$\mu$. In particular, the density exhibits an algebraic decay $\sim 1/x^2$,
where $x$ is the distance from the jump. Furthermore, we show that the critical
behaviour around $\mu = V_0$ exhibits universality with respect with the shape
of the barrier. This is established (i) by an exact solution for a smooth
barrier (the Woods-Saxon potential) and (ii) by establishing a general relation
between the large distance behavior of the kernel and the scattering amplitudes
of the single-particle wave-function.

\subsection*{\href{http://arxiv.org/abs/2009.12880v1}{Diffraction-free beam propagation at the exceptional point of  non-Hermitian Glauber Fock lattices}}
\subsubsection*{Cem Yuce and Hamidreza Ramezani (2020-09-27)}
We construct localized beams that are located at the edge of a non-Hermitian
Glauber Fock (NGF) lattice made of coupled waveguides and can propagate for a
long distance without almost no diffraction. Specifically, we calculate the
closed-form of the eigenstates at the exceptional point of a semi-infinite NGF
lattice composed of waveguides which are coupled in a unidirectional manner. We
use the closed-form solution to construct the non-diffracting beams in finite
NGF lattices. We provide a numerical approach to find other lattices that are
capable of supporting non-diffracting beams at an exceptional point.

\subsection*{\href{http://arxiv.org/abs/2009.12876v1}{Non-diffracting states at exceptional points}}
\subsubsection*{Cem Yuce and Hamidreza Ramezani (2020-09-27)}
We propose to use exceptional points (EPs) to construct diffraction-free beam
propagation and localized power oscillation in lattices. Specifically, here we
propose two systems to utilize EPs for diffraction-free beam propagation, one
in synthetic gauge lattices and the other one, in unidirectionally coupled
resonators where each resonator individually is capable of creating orbital
angular momentum beams (OAM). In the second system, we introduce the concept of
robust and tunable OAM beam propagation in discrete lattices. We show that one
can create robust OAM beams in an arbitrary number of sites of a photonic
lattice. Furthermore, we report power oscillation at the EP of a non-Hermitian
lattice. Our study widens the study and application of EPs in different
photonic systems including the OAM beams and their associated dynamics in
discrete lattices.

\subsection*{\href{http://arxiv.org/abs/2009.12868v1}{Tiling with triangles: parquet and $GWγ$ methods unified}}
\subsubsection*{Friedrich Krien, and Karsten Held (2020-09-27)}
The parquet formalism and Hedin's $GW\gamma$ approach are unified into a
single theory of vertex corrections, corresponding to a bosonization of the
parquet equations. The method has no drawbacks compared to previous parquet
solvers but has the significant advantage that the vertex functions decay
quickly with frequencies and with respect to distances in the real space. These
properties coincide with the respective separation of the length and energy
scales of the two-particle correlations into long/short-ranged and
high/low-energetic.

\subsection*{\href{http://arxiv.org/abs/2009.12865v1}{Non-equilibrium Effects in Dissipative Strongly Correlated Systems}}
\subsubsection*{Jiajun Li (2020-09-27)}
Novel physics arises when strongly correlated system is driven out of
equilibrium by external fields. Dramatic changes in physical properties, such
as conductivity, are empirically observed in strongly correlated materials
under high electric field. In particular, electric-field driven metal-insulator
transitions are well-known as the resistive switching effect in a variety of
materials, such as VO$_2$, V$_2$O$_3$ and other transition metal oxides. To
satisfactorily explain both the phenomenology and its underlying mechanism, it
is required to model microscopically the out-of-equilibrium dissipative lattice
system of interacting electrons. In this thesis, we developed a systematic
method of modeling non-equilibrium steady states for dissipative lattice
systems by means of Non-equilibrium Green's function and Dynamical Mean Field
Theory. We firstly establish a "minimum model" to formulate the strong-field
transport in non-interacting dissipative electron lattice. This model is
exactly soluble and convenient for discussing energy dissipation and
steady-state properties. The formalism is then combined with Dynamical Mean
Field Theory to provide a systematic framework describing the nonequilibrium
steady-state of correlated materials. We use the formalism to study the
strong-field transport properties of correlated materials, Mott insulators and
Dirac electrons in graphene. We concentrate on the microscopic description of
resistive switching. Of particular interest is the filament formation during
the dynamical phase transition, which has been interpreted as a result of the
delicate interplay between dissipation and Mott physics. We will also examine
$IV$ characteristics and particularly the current saturation of Dirac electrons
in graphene. The arXiv version has been updated with minor modifications and
corrections.

\subsection*{\href{http://arxiv.org/abs/2009.12860v1}{Observation of Blackbody Radiation Enhanced Superradiance in ultracold  Rydberg Gases}}
\subsubsection*{Liping Hao, \dots, and Suotang Jia (2020-09-27)}
An ensemble of excited atoms can synchronize emission of light collectively
in a process known as superradiance when its characteristic size is smaller
than the wavelength of emitted photons. The underlying superradiance depends
strongly on electromagnetic (photon) fields surrounding the ensemble. Here we
report observation of superradiance of ultracold Rydberg atoms embedded in a
bath of room-temperature photons. High mode densities of microwave photons from
$300$K blackbody radiation (BBR) significantly enhance decay rates of Rydberg
states to a neighbouring state, enabling superradiance that is otherwise not
possible with barely vacuum induced spontaneous decay. We measure directly
temporal evolution of superradiant decay in Rydberg state $|nD\rangle$ to
$|(n+1)P\rangle$ transition of Cs atoms ($n$ the principal quantum number).
Decay speed of the ensemble increases with larger number of Rydberg atoms.
Importantly, we find the scaling of the Rydberg superradiance is strongly
modified by van der Waals interactions in Rydberg states. Theoretical
simulations of the many-body dynamics confirm the BBR enhanced superradiance in
the Rydberg ensemble and agree with the experimental observation. Our study
provides insights into the many-body dynamics of interacting atoms coupled to
thermal BBR, and might open a route to the design of blackbody thermometry at
microwave frequencies via collective photon-atom interactions.

\subsection*{\href{http://arxiv.org/abs/2009.12859v1}{Quantum Kinetics of the Magneto Photo Galvanic Effect}}
\subsubsection*{Dieter Hornung and Ralph von Baltz (2020-09-27)}
By using the Keldysh--technique we derive a set of quasiclassical equations
for the shift-- and ballistic--photogalvanic effects for a system of
Bloch--electrons in external electric and magnetic fields. Explicit results are
presented for the photogalvanic current for linear and circular polarized light
and a magnetic field. The basic equations are analogous to the
Semiconductor--Bloch--Equations and describe the electrical transport in
noncentrosymmetric crystals. Our approach may be useful for the development of
novel ferroelectric solar cell materials and Weyl--semimetals, in addition, we
disprove existing statements, that the shift--photogalvanic effect does not
contribute to the photo Hall current.

\subsection*{\href{http://arxiv.org/abs/2009.12851v1}{Time dependent rationally extended Poschl-Teller potential and some of  their properties}}
\subsubsection*{D. Nath and P. Roy (2020-09-27)}
We examine time dependent Schrodinger equation with oscillating boundary
condition. More specifically, we use separation of variable technique to
construct time dependent rationally extended Poschl-Teller potential (whose
solutions are given by in terms of X1 Jacobi exceptional orthogonal
polynomials) and its supersymmetric partner, namely the Poschl-Teller
potential. We have obtained exact solutions of the Schrodinger equation with
the above mentioned potentials subjected to some boundary conditions of the
oscillating type. A number of physical quantities like the average energy,
probability density, expectation values etc. have also been computed for both
the systems and compared with each other.

\subsection*{\href{http://arxiv.org/abs/2009.12827v1}{Realizing a quantum generative adversarial network using a programmable  superconducting processor}}
\subsubsection*{Kaixuan Huang, \dots, and Heng Fan (2020-09-27)}
Generative adversarial networks are an emerging technique with wide
applications in machine learning, which have achieved dramatic success in a
number of challenging tasks including image and video generation. When equipped
with quantum processors, their quantum counterparts--called quantum generative
adversarial networks (QGANs)--may even exhibit exponential advantages in
certain machine learning applications. Here, we report an experimental
implementation of a QGAN using a programmable superconducting processor, in
which both the generator and the discriminator are parameterized via layers of
single- and multi-qubit quantum gates. The programmed QGAN runs automatically
several rounds of adversarial learning with quantum gradients to achieve a Nash
equilibrium point, where the generator can replicate data samples that mimic
the ones from the training set. Our implementation is promising to scale up to
noisy intermediate-scale quantum devices, thus paving the way for experimental
explorations of quantum advantages in practical applications with near-term
quantum technologies.

\subsection*{\href{http://arxiv.org/abs/2009.12822v1}{Towards Chirality Control of Graphene Nanoribbons Embedded in Hexagonal  Boron Nitride}}
\subsubsection*{Hui Shan Wang, \dots, and Xiaoming Xie (2020-09-27)}
The integrated inplane growth of two dimensional materials with similar
lattices, but distinct electrical properties, could provide a promising route
to achieve integrated circuitry of atomic thickness. However, fabrication of
edge specific GNR in the lattice of hBN still remains an enormous challenge for
present approaches. Here we developed a two step growth method and successfully
achieved sub 5 nm wide zigzag and armchair GNRs embedded in hBN, respectively.
Further transport measurements reveal that the sub 7 nm wide zigzag GNRs
exhibit openings of the band gap inversely proportional to their width, while
narrow armchair GNRs exhibit some fluctuation in the bandgap width
relationship.This integrated lateral growth of edge specific GNRs in hBN brings
semiconducting building blocks to atomically thin layer, and will provide a
promising route to achieve intricate nanoscale electrical circuits on high
quality insulating hBN substrates.

\subsection*{\href{http://arxiv.org/abs/2009.12816v1}{Timescales and contribution of heating and helicity effect in  helicity-dependent all-optical switching}}
\subsubsection*{Guanqi Li, \dots, and Yongbing Xu (2020-09-27)}
The manipulation of the magnetic direction by using the ultrafast laser pulse
is attractive for its great advantages in terms of speed and energy efficiency
for information storage applications. However, the heating and helicity effects
induced by circularly polarized laser excitation are entangled in the
helicity-dependent all-optical switching (HD-AOS), which hinders the
understanding of magnetization dynamics involved. Here, by applying a dual-pump
laser excitation, first with a linearly polarized (LP) laser pulse followed by
a circularly polarized (CP) laser pulse, we identify the timescales and
contribution from heating and helicity effects in HD-AOS with a Pt/Co/Pt triple
layer. When the sample is preheated by the LP laser pulses to a nearly fully
demagnetized state, CP laser pulses with a much-reduced power switches the
sample's magnetization. By varying the time delay between the two pump pulses,
we show that the helicity effect, which gives rise to the deterministic
helicity induced switching, onsets instantly upon laser excitation, and only
exists for less than 0.2 ps close to the laser pulse duration of 0.15 ps. The
results reveal that that the transient magnetization state upon which CP laser
pulses impinge is the key factor for achieving HD-AOS, and importantly, the
tunability between heating and helicity effects with the unique dual-pump laser
excitation approach will enable HD-AOS in a wide range of magnetic material
systems for the potential ultrafast spintronics applications.

\subsection*{\href{http://arxiv.org/abs/2009.12813v1}{Read-out of Quasi-periodic Systems using Qubits}}
\subsubsection*{Madhumita Saha, and B. Prasanna Venkatesh (2020-09-27)}
We develop a theoretical scheme to perform a read-out of the properties of a
quasi-periodic system by coupling it to one or two qubits. We show that the
decoherence dynamics of a single qubit coupled via a pure dephasing type term
to a 1D quasi-periodic system with a potential given by the
Andr\'e-Aubry-Harper (AAH) model and its generalized versions (GAAH model) is
sensitive to the nature of the single particle eigenstates (SPEs). More
specifically, we can use the non-markovianity of the qubit dynamics as
quantified by the backflow of information to clearly distinguish the localized,
delocalized, and mixed regimes with a mobility edge of the AAH and GAAH model
and evidence the transition between them. By attaching two qubits at distinct
sites of the system, we demonstrate that the transport property of the
quasi-periodic system is encoded in the scaling of the threshold time to
develop correlations between the qubits with the distance between the qubits.
This scaling can also be used to distinguish and infer different regimes of
transport such as ballistic, diffusive and no transport engendered by SPEs that
are delocalized, critical and localized respectively. When there is a mobility
edge allowing the coexistence of different kinds of SPEs in the spectrum, such
as the coexistence of localized and delocalized states in the GAAH models, we
find that the transport behaviour and the scaling of the threshold time with
qubit separation is governed by the fastest spreading states.

\subsection*{\href{http://arxiv.org/abs/2009.12808v1}{In situ functionalization of graphene}}
\subsubsection*{Kyrylo Greben, \dots, and Kirill I. Bolotin (2020-09-27)}
While the basal plane of graphene is inert, defects in it are centers of
chemical activity. An attractive application of such defects is towards
controlled functionalization of graphene with foreign molecules. However, the
interaction of the defects with reactive environment, such as ambient,
decreases the efficiency of functionalization and makes it poorly controlled.
Here, we report a novel approach to generate, monitor with time resolution, and
functionalize the defects $\textit{in situ}$ without ever exposing them to the
ambient. The defects are generated by an energetic Argon plasma and their
properties are monitored using $\textit{in situ}$ Raman spectroscopy. We find
that these defects are functional, very reactive, and strongly change their
density from $\approx 1\cdot10^{13} cm^{-2}$ to $\approx 5\cdot10^{11} cm^{-2}$
upon exposure to air. We perform the proof of principle $\textit{in situ}$
functionalization by generating defects using the Argon plasma and
functionalizing them $\textit{in situ}$ using Ammonia functional. The
functionalization induces the n-doping with a carrier density up to
$5\cdot10^{12} cm^{-2}$ in graphene and remains stable in ambient conditions.

\subsection*{\href{http://arxiv.org/abs/2009.12805v1}{Transition from band insulator to excitonic insulator via alloying Se  into Monolayer TiS$_3$: A Computational Study}}
\subsubsection*{Shan Dong and Yuanchang Li (2020-09-27)}
First-principles density functional theory plus Bethe-Salpeter equation
calculations are employed to investigate the electronic and excitonic
properties of monolayer titanium trichalcogenide alloys TiS$_{3-x}$Se$_x$
($x$=1 and 2). It is found that bandgap and exciton binding energy display
asymmetric dependence on the substitution of Se for S. While the bandgap can be
significantly decreased as compared to that of pristine TiS$_3$, the exciton
binding energy just varies a little, regardless of position and concentration
of the Se substitution. A negative exciton formation energy is found when the
central S atoms are replaced by Se atoms, suggesting a many-body ground state
with the spontaneous exciton condensation. Our work thus offers a new insight
for engineering an excitonic insulator.

\subsection*{\href{http://arxiv.org/abs/2009.12790v1}{Effective-field theory for high-$T_c$ cuprates}}
\subsubsection*{A. S. Moskvin and Yu. D. Panov (2020-09-27)}
Starting with a minimal model for the CuO$_2$ planes with the on-site Hilbert
space reduced to only three effective valence centers [CuO$_4$]$^{7-,6-,5-}$
(nominally Cu$^{1+,2+,3+}$) with different conventional spin and different
orbital symmetry we propose a unified non-BCS model that allows one to describe
the main features of the phase diagrams of doped cuprates within the framework
of a simple effective field theory.

\subsection*{\href{http://arxiv.org/abs/2009.12767v1}{A Hybrid Framework Using a QUBO Solver For Permutation-Based  Combinatorial Optimization}}
\subsubsection*{Siong Thye Goh, \dots, and Hoong Chuin Lau (2020-09-27)}
In this paper, we propose a hybrid framework to solve large-scale
permutation-based combinatorial problems effectively using a high-performance
quadratic unconstrained binary optimization (QUBO) solver. To do so,
transformations are required to change a constrained optimization model to an
unconstrained model that involves parameter tuning. We propose techniques to
overcome the challenges in using a QUBO solver that typically comes with
limited numbers of bits. First, to smooth the energy landscape, we reduce the
magnitudes of the input without compromising optimality. We propose a machine
learning approach to tune the parameters for good performance effectively. To
handle possible infeasibility, we introduce a polynomial-time projection
algorithm. Finally, to solve large-scale problems, we introduce a
divide-and-conquer approach that calls the QUBO solver repeatedly on small
sub-problems. We tested our approach on provably hard Euclidean Traveling
Salesman (E-TSP) instances and Flow Shop Problem (FSP). Optimality gap that is
less than $10\%$ and $11\%$ are obtained respectively compared to the
best-known approach.

\subsection*{\href{http://arxiv.org/abs/2009.12759v1}{The relationship between costs for quantum error mitigation and  non-Markovian measures}}
\subsubsection*{Hideaki Hakoshima, and Suguru Endo (2020-09-27)}
Quantum error mitigation (QEM) has been proposed as an alternative method of
quantum error correction (QEC) to compensate errors in quantum systems without
qubit overhead. While Markovian gate errors on digital quantum computers are
mainly considered previously, it is indispensable to discuss a relationship
between QEM and non-Markovian errors because non-Markovian noise effects
inevitably exist in most of the solid state systems. In this work, we
investigate the QEM for non-Markovian noise, and show that there is a clear
relationship between costs for QEM and non-Markovian measures. We exemplify
several non-Markovian noise models to bridge a gap between our theoretical
framework and concrete physical systems. This discovery may help designing
better QEM strategies for realistic quantum devices with non-Markovian
environments.

\subsection*{\href{http://arxiv.org/abs/2009.13955v1}{Distance Matrix based Crystal Structure Prediction using Evolutionary  Algorithms}}
\subsubsection*{Jianjun Hu, and Edirisuriya M. Dilanga Siriwardane (2020-09-27)}
Crystal structure prediction (CSP) for inorganic materials is one of the
central and most challenging problems in materials science and computational
chemistry. This problem can be formulated as a global optimization problem in
which global search algorithms such as genetic algorithms (GA) and particle
swarm optimization have been combined with first principle free energy
calculations to predict crystal structures given only a material composition or
only a chemical system. These DFT based ab initio CSP algorithms are
computationally demanding and can only be used to predict crystal structures of
relatively small systems. The vast coordinate space plus the expensive DFT free
energy calculations make it inefficient and ineffective. On the other hand, a
similar structure prediction problem has been intensively investigated in
parallel in the protein structure prediction community of bioinformatics, in
which the dominating predictors are knowledge based approaches including
homology modeling and threading that exploit known protein structures. Herein
we explore whether known geometric constraints such as the pairwise atomic
distances of a target crystal material can help predict/reconstruct its
structure given its space group and lattice information. We propose DMCrystal,
a genetic algorithm based crystal structure reconstruction algorithm based on
predicted atomic pairwise distances. Based on extensive experiments, we show
that the predicted distance matrix can dramatically help to reconstruct the
crystal structure and usually achieves much better performance than CMCrystal,
an atomic contact map based crystal structure prediction algorithm. This
implies that knowledge of atomic interaction information learned from existing
materials can be used to significantly improve the crystal structure prediction
in terms of both speed and quality.

\subsection*{\href{http://arxiv.org/abs/2009.12750v1}{Power-Law Decay Exponents of Nambu-Goldstone Transverse Correlations}}
\subsubsection*{Tohru Koma (2020-09-27)}
We study a quantum antiferromagnetic Heisenberg model on a hypercubic lattice
in three or higher dimensions $d\ge 3$. When a phase transition occurs with the
continuous symmetry breaking, the nonvanishing spontaneous magnetization which
is obtained by applying the infinitesimally weak symmetry breaking field is
equal to the maximum spontaneous magnetization at zero or non-zero low
temperatures. In addition, the transverse correlation in the infinite-volume
limit exhibits a Nambu-Goldstone-type slow decay. In this paper, we assume that
the transverse correlation decays by power law with distance. Under this
assumption, we prove that the power is equal to $2-d$ at non-zero low
temperatures, while it is equal to $1-d$ at zero temperature. The method is
applied also to a quantum XY model and a classical Heisenberg model at non-zero
low temperatures in three or higher dimensions. The resulting power is given by
the same $2-d$ at non-zero low temperatures.

\subsection*{\href{http://arxiv.org/abs/2009.12741v1}{The role of solvent microstructure on the aging dynamics and rheology of  aqueous suspensions of a soft colloidal clay}}
\subsubsection*{Chandeshwar Misra, and Ranjini Bandyopadhyay (2020-09-27)}
The influence of solvent microstructure on the microscopic dynamics and
rheology of aging colloidal smectite clay suspensions is investigated by
performing dynamic light scattering and rheology experiments. Additives that
can either induce or break-up the ordering of water molecules in an aqueous
medium are incorporated in aqueous Laponite clay suspensions. While the
addition of sodium chloride, glucose and potassium chloride accelerates the
aging dynamics of the Laponite particles in suspension and leads to rapid
dynamical arrest, the presence of N,N-Dimethylformamide disrupts the aging and
jamming dynamics of the particles. An increase in temperature leads to the
breaking of hydrogen bonds in an aqueous suspension medium. Our experiments
reveal an accelerated approach to dynamical arrest when the temperature of
Laponite suspensions is raised. The aging dynamics and rheology of the samples
are correlated with their microstructural details visualized using cryogenic
electron microscopy. Our data demonstrate that the microscopic dynamics of
aging Laponite suspensions show self-similar time-evolution, while their
long-time aging behavior and nonlinear rheological responses are sensitive to
the temperature of the suspension medium and the presence of additive
molecules.

\subsection*{\href{http://arxiv.org/abs/2009.12734v1}{On the Origin of Precipitation of Transition Metals Implanted in MgO}}
\subsubsection*{Debolina Misra and Satyesh K. Yadav (2020-09-27)}
Transition metals implanted in single crystal MgO can precipitate out at
grain boundaries or remain embedded in bulk. Using first-principles
calculations based on density functional theory we have calculated the
thermodynamic stability and diffusion coefficients of the implanted ions to
explain Fe and Ni precipitation in MgO. Experimentally it has been observed
that some of the Fe atoms precipitate out, while few Fe atoms in 2+ and 3+
charge states remain embedded in the lattice. Our simulation shows that at 600
K (typical annealing temperature) while neutral iron in MgO would migrate 1
$\mu$m in few microseconds, it takes several years for the charged Fe ions to
migrate the same distance. On the other hand, Ni ions in all its charge states
(neutral, 1+, 2+, and 3+) would migrate 1 $\mu$m in just few microseconds, at
600 K. This explains the experimental observation that implanted Ni always
precipitates out. Our study paves a way forward to predict if ions implanted in
stable oxide will be stable or will precipitate out.

\subsection*{\href{http://arxiv.org/abs/2009.12730v1}{Lasing in the Space Charge-Limited Current Regime}}
\subsubsection*{Alex J. Grede and Noel C. Giebink (2020-09-27)}
We introduce an analytical model for ideal organic laser diodes based on the
argument that their intrinsic active layers necessitate operation in the
bipolar space charge-limited current regime. Expressions for the threshold
current and voltage agree well with drift-diffusion modeling of complete p-i-n
devices and an analytical bound is established for laser operation in the
presence of annihilation and excited-state absorption losses. These results
establish a foundation for the development of organic laser diode technology.

\subsection*{\href{http://arxiv.org/abs/2009.12720v1}{Novel soliton in dipolar BEC caused by the quantum fluctuations}}
\subsubsection*{Pavel A. Andreev (2020-09-27)}
Solitons in the extended hydrodynamic model of the dipolar Bose-Einstein
condensate with quantum fluctuations are considered. This model includes the
continuity equation for the scalar field of concentration, the Euler equation
for the vector field of velocity, the pressure evolution equation for the
second rank tensor of pressure, and the evolution equation for the third rank
tensor. Large amplitude soliton solution caused by the dipolar part of quantum
fluctuations is found. It appears as the bright soliton. Hence, it is the area
of compression of the number of particles. Moreover, it exists for the
repulsive short-range interaction.

\subsection*{\href{http://arxiv.org/abs/2009.12709v1}{Exchange and exclusion in the non-abelian anyon gas}}
\subsubsection*{Douglas Lundholm and Viktor Qvarfordt (2020-09-26)}
We review and develop the many-body spectral theory of ideal anyons, i.e.
identical quantum particles in the plane whose exchange rules are governed by
unitary representations of the braid group on $N$ strands. Allowing for
arbitrary rank (dependent on $N$) and non-abelian representations, and letting
$N \to \infty$, this defines the ideal non-abelian many-anyon gas. We compute
exchange operators and phases for a common and wide class of representations
defined by fusion algebras, including the Fibonacci and Ising anyon models.
Furthermore, we extend methods of statistical repulsion (Poincar\'e and Hardy
inequalities) and a local exclusion principle (also implying a Lieb-Thirring
inequality) developed for abelian anyons to arbitrary geometric anyon models,
i.e. arbitrary sequences of unitary representations of the braid group, for
which two-anyon exchange is nontrivial.

\subsection*{\href{http://arxiv.org/abs/2009.12700v1}{SO(5) non-Fermi liquid in a Coulomb box device}}
\subsubsection*{Andrew K. Mitchell, \dots, and Ian Affleck (2020-09-26)}
Non-Fermi liquid (NFL) physics can be realized in quantum dot devices where
competing interactions frustrate the exact screening of dot spin or charge
degrees of freedom. We show that a standard nanodevice architecture, involving
a dot coupled to both a quantum box and metallic leads, can host an exotic
SO(5) symmetry Kondo effect, with entangled dot and box charge and spin. This
NFL state is surprisingly robust to breaking channel and spin symmetry, but
destabilized by particle-hole asymmetry. By tuning gate voltages, the SO(5)
state evolves continuously to a spin and then "flavor" two-channel Kondo state.
The expected experimental conductance signatures are highlighted.

\subsection*{\href{http://arxiv.org/abs/2009.12679v1}{Sparse-Hamiltonian approach to the time evolution of molecules on  quantum computers}}
\subsubsection*{Christina Daniel, \dots, and James K. Freericks (2020-09-26)}
Quantum chemistry has been viewed as one of the potential early applications
of quantum computing. Two techniques have been proposed for electronic
structure calculations: (i) the variational quantum eigensolver and (ii) the
phase-estimation algorithm. In both cases, the complexity of the problem
increases for basis sets where either the Hamiltonian is not sparse, or it is
sparse, but many orbitals are required to accurately describe the molecule of
interest. In this work, we explore the possibility of mapping the molecular
problem onto a sparse Hubbard-like Hamiltonian, which allows a
Green's-function-based approach to electronic structure via a hybrid
quantum-classical algorithm. We illustrate the time-evolution aspect of this
methodology with a simple four-site hydrogen ring.

\subsection*{\href{http://arxiv.org/abs/2009.12676v1}{Possibility of coherent electron transport in a nanoscale circuit}}
\subsubsection*{Mark J. Hagmann (2020-09-26)}
Others have solved the Schr\"odinger equation to estimate the tunneling
current between two electrodes at specified potentials, or the transmission
through a potential barrier, assuming that an incident wave causes one
reflected wave and one transmitted wave. However, this may not be appropriate
in some nanoscale circuits because the electron mean-free path may be as long
as 68 nm in metals. Thus, the wavefunction may be coherent throughout the metal
components in a circuit if the interaction of the electrons with the surface of
conductors and grain boundaries, which reduces the mean-free path, is reduced.
We consider the use of single-crystal wires, and include a tunneling junction
to focus and collimate the electrons near the axis, to further reduce their
interaction with the surface of the wire. Our simulations suggest that, in
addition to the incoherent phenomena, there are extremely sharply-defined
coherent modes in nanoscale circuits. Algorithms are presented with examples to
determine the sets of the parameters for these modes. Other algorithms are
presented to determine the normalized coefficients in the wavefunction and the
distribution of current in the circuits. This is done using only algebra with
calculus for analytical solutions of the Schr\"odinger equation.

\subsection*{\href{http://arxiv.org/abs/2009.12668v1}{Thermodynamic curvature of the binary van der Waals fluid}}
\subsubsection*{George Ruppeiner and Alex Seftas (2020-09-26)}
The thermodynamic Ricci curvature scalar $R$ has been applied in a number of
contexts, mostly for systems characterized by 2D thermodynamic geometries.
Calculations of $R$ in thermodynamic geometries of dimension three or greater
have been very few, especially in the fluid regime. In this paper, we calculate
$R$ for two examples involving binary fluid mixtures: a binary mixture of a van
der Waals (vdW) fluid with only repulsive interactions, and a binary vdW
mixture with attractive interactions added. In both these examples, we evaluate
$R$ for full 3D thermodynamic geometries. Our finding is that basic physical
patterns found for $R$ in the pure fluid are reproduced to a large extent for
the binary fluid.

\subsection*{\href{http://arxiv.org/abs/2009.12655v1}{Nondisturbing Quantum Measurement Models}}
\subsubsection*{Stan Gudder (2020-09-26)}
A measurement model is a framework that describes a quantum measurement
process. In this article we restrict attention to $MM$s on finite-dimensional
Hilbert spaces. Suppose we want to measure an observable $A$ whose outcomes
$A_x$ are represented by positive operators (effects) on a Hilbert Space $H$.
We call $H$ the base or object system. We interact $H$ with a probe system on
another Hilbert space $K$ by means of a quantum channel. The probe system
contains a probe (or meter or pointer) observable $F$ whose outcomes $F_x$ are
measured by an apparatus that is frequently (but need not be) classical in
practice. The $MM$ protocol gives a method for determining the probability of
an outcome $A_x$ for any state of $H$ in terms of the outcome $F_x$. The
interaction channel usually entangles this state with an initial probe state of
$K$ that can be quite complicated. However, if the channel is nondisturbing in
a sense that we describe, then the entanglement is considerably simplified. In
this article, we give formulas for observables and instruments measured by
nondisturbing $MM$s. We begin with a general discussion of nondisturbing
operators relative to a quantum context. We present two examples that
illustrate this theory in terms of unitary nondisturbing channels.

\subsection*{\href{http://arxiv.org/abs/2009.13952v1}{Superconductivity enhanced by $d$-band filling in La$Tr_2$Al$_{20}$ with  $Tr$ = Mo and W}}
\subsubsection*{Rumika Miyawaki, \dots, and Yuji Aoki (2020-09-26)}
Electrical resistivity, magnetic susceptibility, and specific heat
measurements on single crystals of La$Tr_2$Al$_{20}$ with $Tr$ = Mo and W
revealed that these compounds exhibit superconductivity with transition
temperatures $T_c$ = 3.22 and 1.81 K, respectively, achieving the highest
values in the reported La$Tr_2$Al$_{20}$ compounds. There appears a positive
correlation between $T_c$ and the electronic specific heat coefficient, which
increases with increasing the number of $4d$- and $5d$-electrons. This finding
indicates that filling of the upper $e_g$ orbitals in the $4d$ and $5d$ bands
plays an essential role for the significant enhancement of the superconducting
condensation energy. Possible roles played by the $d$ electrons in the strongly
correlated electron phenomena appearing in $RTr_{2}$Al$_{20}$ are discussed.

\subsection*{\href{http://arxiv.org/abs/2009.12644v1}{Emergence and Breaking of Duality Symmetry in Thermodynamic Behavior:  Repeated Measurements and Macroscopic Limit}}
\subsubsection*{Zhiyue Lu and Hong Qian (2020-09-26)}
Thermodynamic laws are limiting behavior of the statistics of repeated
measurements of an arbitrary system with a priori probability distribution. A
duality symmetry arises, between Massieu-Guggenheim entropy and Gibbs entropy,
in the limit of large number of measurements. This yields the fundamental
thermodynamic relation and Hill-Gibbs-Duhem (HGD) equation as a dual pair. We
show if the system itself has a second macroscopic limit that satisfies
Callen's postulate that entropy being an Eulerian homogeneous function of all
extensive variables, the symmetry is lost: the HGD equation reduces to the
Gibbs-Duhem equation. This theory provides better logic to textbook
thermodynamics, a clarification on nanothermodynamics, as well as novel ideas
for a thermodynamic-like framework for single-cell biology.

\subsection*{\href{http://arxiv.org/abs/2009.12642v1}{Retrodiction beyond the Heisenberg uncertainty relation}}
\subsubsection*{Han Bao, \dots, and Yanhong Xiao (2020-09-26)}
In quantum mechanics, the Heisenberg uncertainty relation presents an
ultimate limit to the precision by which one can predict the outcome of
position and momentum measurements on a particle. Heisenberg explicitly stated
this relation for the prediction of "hypothetical future measurements", and it
does not describe the situation where knowledge is available about the system
both earlier and later than the time of the measurement. We study what happens
under such circumstances with an atomic ensemble containing $10^{11}$
$^{87}\text{Rb}$ atoms, initiated nearly in the ground state in presence of a
magnetic field. The collective spin observables of the atoms are then well
described by canonical position and momentum observables, $\hat{x}_A$ and
$\hat{p}_A$ that satisfy $[\hat{x}_A,\hat{p}_A]=i\hbar$. Quantum non-demolition
measurements of $\hat{p}_A$ before and of $\hat{x}_A$ after time $t$ allow
precise estimates of both observables at time $t$. The capability of assigning
precise values to multiple observables and to observe their variation during
physical processes may have implications in quantum state estimation and
sensing.

\subsection*{\href{http://arxiv.org/abs/2009.12640v1}{Majorana Zero Modes in Cylindrical Semiconductor Quantum Wire}}
\subsubsection*{Chao Lei, \dots, and Allan H. MacDonald (2020-09-26)}
We study Majorana zero modes properties in cylindrical cross-section
semiconductor quantum wires based on the $k \cdot p$ theory and a discretized
lattice model. Within this model, the influence of disorder potentials in the
wire and amplitude and phase fluctuations of the superconducting
order-parameter are discussed. We find that for typical wire geometries,
pairing potentials, and spin-orbit coupling strengths, coupling between
quasi-one-dimensional sub-bands is weak, low-energy quasiparticles near the
Fermi energy are nearly completely spin-polarized, and the number of electrons
in the active sub-bands of topological states is small.

\subsection*{\href{http://arxiv.org/abs/2009.12595v1}{Bose condensation and spin superfluidity of magnons in a perpendicularly  magnetized film of yttrium iron garnet}}
\subsubsection*{P. M. Vetoshko, \dots, and Yu. M. Bunkov (2020-09-26)}
The formation of a Bose condensate of magnons in a perpendicularly magnetized
film of yttrium iron garnet under radio-frequency pumping in a strip line is
studied experimentally. The characteristics of nonlinear magnetic resonance and
the spatial distribution of the Bose condensate of magnons in the magnetic
field gradient are investigated. In these experiments, the Bosonic system of
magnons behaves similarly to the Bose condensate of magnons in the
antiferromagnetic superfluid 3He-B, which was studied in detail earlier.
Magnonic BEC forms a coherently precessing state with the properties of
magnonic superfluidity. Its stability is determined by the repulsive potential
between excited magnons, which compensates for the inhomogeneity of the
magnetic field.

\subsection*{\href{http://arxiv.org/abs/2009.12587v1}{Heat vortexes of ballistic, diffusive and hydrodynamic phonon transport  in two-dimensional materials}}
\subsubsection*{Chuang Zhang, and Zhaoli Guo (2020-09-26)}
In this work, the heat vortexes in two-dimensional porous or ribbon
structures are investigated based on the phonon Boltzmann transport equation
(BTE) under the Callaway model. First, the separate thermal effects of normal
(N) scattering and resistive (R) scattering are investigated with
frequency-independent assumptions. And then the heat vortexes in graphene are
studied as a specific example. It is found that the heat vortexes can appear in
both ballistic (rare R/N scattering) and hydrodynamic (N scattering dominates)
regimes but disappear in the diffusive (R scattering dominates) regime. As long
as there is not sufficient R scattering, the heat vortexes can appear in
present simulations.

\subsection*{\href{http://arxiv.org/abs/2009.12584v1}{Wetting boundary conditions for multicomponent pseudopotential lattice  Boltzmann}}
\subsubsection*{Rodrigo C. V. Coelho, \dots, and Nuno A. M. Araújo (2020-09-26)}
The implementation of boundary conditions is among the most challenging parts
of modeling fluid flow through channels and complex media. Here, we show that
the existing methods to deal with liquid-wall interactions using multicomponent
Lattice Boltzmann are accurate when the wall is aligned with the main axes of
the lattice but fail otherwise. To solve this problem, we extend a strategy
previously developed for multiphase models. As an example, we study the
coalescence of two droplets on a curved surface in two dimensions. The strategy
proposed here is of special relevance for binary flows in complex geometries.

\subsection*{\href{http://arxiv.org/abs/2009.12578v1}{Superconductivity emerging from a stripe charge order in IrTe2  nanoflakes}}
\subsubsection*{Sungyu Park, \dots, and Jun Sung Kim (2020-09-26)}
Superconductivity in the vicinity of a competing electronic order often
manifests itself with a superconducting dome, centred at a presumed quantum
critical point in the phase diagram. This common feature, found in many
unconventional superconductors, has supported a prevalent scenario that
fluctuations or partial melting of a parent order are essential for inducing or
enhancing superconductivity. Here we present a contrary example, found in IrTe2
nanoflakes of which the superconducting dome is identified well inside the
parent stripe charge ordering phase in the thickness-dependent phase diagram.
The coexisting stripe charge order in IrTe2 nanoflakes significantly increases
the out-of-plane coherence length and the coupling strength of
superconductivity, in contrast to the doped bulk IrTe2. These findings clarify
that the inherent instabilities of the parent stripe phaseare sufficient to
induce superconductivity in IrTe2 without its complete or partial melting. Our
study highlights the thickness control as an effective means to unveil
intrinsic phase diagrams of correlated vdW materials.

\subsection*{\href{http://arxiv.org/abs/2009.12573v1}{Expressibility of comb tensor network states (CTNS) for the P-cluster  and the FeMo-cofactor of nitrogenase}}
\subsubsection*{Zhendong Li (2020-09-26)}
Polynuclear transition metal complexes such as the P-cluster and the
FeMo-cofactor of nitrogenase with eight transition metal centers represent a
great challenge for current electronic structure methods. In this work, we
initiated the use of comb tensor network states (CTNS), whose underlying
topology has a one-dimensional backbone and several one-dimensional branches,
as a many-body wavefunction ansatz to tackle such challenging systems. As an
important first step, we explored the expressive power of CTNS with different
underlying topologies. To this end, we presented an algorithm to express a
configuration interaction (CI) wavefunction into CTNS based on the Schmidt
decomposition. The algorithm was illustrated for representing selected CI
wavefunctions for the P-cluster and the FeMo-cofactor into CTNS with three
chemically meaningful comb structures, which successively group orbitals
belonging to the same atom into branches. The conventional matrix product
states (MPS) representation can be obtained as a special case. We also
discussed the insights gained from such decompositions, which shed some light
on the future developments of efficient numerical tools for polynuclear
transition metal complexes.

\subsection*{\href{http://arxiv.org/abs/2009.12572v2}{Kerr nonlinearity effect on light transmission in one dimensional  photonic crystal}}
\subsubsection*{Daoud Mansour and Khaled Senouci (2020-09-26)}
We investigate numerically the effect of Kerr nonlinearity on the
transmission spectrum of a one dimensional $\delta$-function photonic crystal.
It is found that the photonic band gap (PBG) width either increases or
decreases depending on both sign and strength of Kerr nonlinearity. We found
that any amount of self-focusing nonlinearity $(\alpha >0)$ leads to an
increase of the PBG width leading to light localization. However, for
defocusing nonlinearity, we found a range of non-linearity strengths for which
the photonic band gap width decreases when the nonlinearity strength increases
and a critical non-linearity strength $|\alpha_{c}|$ above which the behaviour
is reversed. At this critical value the photonic crystal become transparent and
the photonic band gap is suppressed. We have also studied the dependence on the
angle of incidence and polarization in the transmission spectrum of our
one-dimensional photonic crystal. We found that the minimum of the transmission
increases with incident angle but seems to be polarization-independent. We also
found that position of the photonic band gap (PBG) shifts to lower wavelengths
when the angle of incidence increases for TE mode while it shifts to longer
wavelengths for TM mode.

\subsection*{\href{http://arxiv.org/abs/2009.12568v1}{Quantum measurements with, and yet without an Observer}}
\subsubsection*{Dmitri Sokolovski (2020-09-26)}
It is argued that Feynman's rules for evaluating probabilities, combined with
von Neumann's principle of psycho-physical parallelism, help avoid
inconsistencies, often associated with quantum theory. The former allows one to
assign probabilities to entire sequences of hypothetical Observers'
experiences, without mentioning the problem of wave function collapse.The
latter limits the Observer's (e.g., Wigner's friend's) participation in a
measurement to the changes produced in material objects,thus leaving his/her
consciousness outside the picture.

\subsection*{\href{http://arxiv.org/abs/2009.12563v1}{Electroferrofluids with non-equilibrium voltage-controlled magnetism,  interfaces, and patterns}}
\subsubsection*{Tomy Cherian, \dots, and Jaakko V. I. Timonen (2020-09-26)}
Materials with continuous dissipation can exhibit responses and
functionalities that are not possible in thermodynamic equilibrium. While this
concept is well-known, a major challenge has been the implementation: how to
rationally design materials with functional non-equilibrium states and quantify
the dissipation? Here we address these questions for the widely used colloidal
nanoparticles that convey several functionalities. We propose that useful
non-equilibrium states can be realised by creating and maintaining steady-state
nanoparticle concentration gradients by continuous injection and dissipation of
energy. We experimentally demonstrate this with superparamagnetic iron oxide
nanoparticles that in thermodynamic equilibrium form a homogeneous functional
fluid with a strong magnetic response (a ferrofluid). To create non-equilibrium
functionalities, we charge the nanoparticles with anionic charge control agents
to create electroferrofluids where nanoparticles act as charge carriers that
can be driven with electric fields and current to non-homogeneous dissipative
steady-states. The dissipative steady-states exhibit voltage-controlled
magnetic properties and emergent diffuse interfaces. The diffuse interfaces
respond strongly to external magnetic fields, leading to dissipative patterns
that are not possible in the equilibrium state. We identify the closest
non-dissipative analogues of these dissipative patterns, discuss the
differences, and highlight how pattern formation in electroferrofluids is
linked to dissipation that can be directly quantified. Beyond electrically
controlled ferrofluids and patterns, we foresee that the concept can be
generalized to other functional nanoparticles to create various scientifically
and technologically relevant non-equilibrium states with optical, electrical,
catalytic, and mechanical responses that are not possible in thermodynamic
equilibrium.

\subsection*{\href{http://arxiv.org/abs/2009.12561v1}{First-Principles Prediction of Graphene-Like XBi (X=Si, Ge, Sn)  Nanosheets}}
\subsubsection*{A. Bafekry, \dots, and B. Mortazavi (2020-09-26)}
Research progress on single-layer group III monochalcogenides have been
increasing rapidly owing to their interesting physics. Herein, we predict the
dynamically stable single-layer forms of XBi (X=Ge, Si, or Sn) by using density
functional theory calculations. Phonon band dispersion calculations and
ab-initio molecular dynamics simulations reveal the dynamical and thermal
stability of predicted nanosheets. Raman spectra calculations indicate the
existence of 5 Raman active phonon modes 3 of which are prominent and can be
observed in a possible Raman measurement. Electronic band structures of the XBi
single-layers investigated with and without spin-orbit coupling effects (SOC).
Our results show that XBi single-layers show semiconducting property with the
narrow band gap values without SOC. However only the single-layer SiBi is an
indirect band gap semiconductor while GeBi and SnBi exhibit metallic behaviors
by adding spin-orbit coupling effects. In addition, the calculated
linear-elastic parameters indicate the soft nature of predicted monolayers.
Moreover, our predictions for the thermoelectric properties of single-layer XBi
reveal that SiBi is a good thermoelectric material with increasing temperature.
Overall, it is proposed that single-layer XBi structures can be alternative,
stable 2D single-layers with their varying electronic and thermoelectric
properties.

\subsection*{\href{http://arxiv.org/abs/2009.12554v2}{Quantum dark solitons in ultracold one-dimensional Bose and Fermi gases}}
\subsubsection*{Andrzej Syrwid (2020-09-26)}
Solitons are ubiquitous phenomena that appear, among others, in the
description of tsunami waves, fiber-optic communication and ultracold atomic
gases. The latter systems turned out to be an excellent playground for
investigations of matter-wave solitons in a quantum world. This Tutorial
provides a general overview of the ultracold contact interacting Bose and Fermi
systems in a one-dimensional space that can be described by the renowned
Lieb-Liniger and Yang-Gaudin models. Both the quantum many-body systems are
exactly solvable by means of the Bethe ansatz technique, granting us a
possibility for investigations of quantum nature of solitonic excitations. We
discuss in details a specific class of quantum many-body excited eigenstates
called yrast states and show that they are strictly related to quantum dark
solitons in the both considered Bose and Fermi systems.

\subsection*{\href{http://arxiv.org/abs/2009.12543v1}{Deterministically fabricated strain-tunable quantum dot single-photon  sources emitting in the telecom O-band}}
\subsubsection*{Nicole Srocka, \dots, and Stephan Reitzenstein (2020-09-26)}
Most quantum communication schemes aim at the long-distance transmission of
quantum information. In the quantum repeater concept, the transmission line is
subdivided into shorter links interconnected by entanglement distribution via
Bell-state measurements to overcome inherent channel losses. This concept
requires on-demand single-photon sources with a high degree of multi-photon
suppression and high indistinguishability within each repeater node. For a
successful operation of the repeater spectral matching of remote quantum light
sources is essential. We present a spectrally tunable single-photon source
emitting in the telecom O-band with the potential to function as a building
block of a quantum communication network based on optical fibers. A thin
membrane of GaAs embedding InGaAs quantum dots (QDs) is attached onto a
piezoelectric actuator via gold thermocompression bonding. Here the thin gold
layer acts simultaneously as an electrical contact, strain transmission medium
and broadband backside mirror for the QD-micromesa. The nanofabrication of the
QD-micromesa is based on in-situ electron-beam lithography, which makes it
possible to integrate pre-selected single QDs deterministically into the center
of monolithic micromesa structures. The QD pre-selection is based on distinct
single-QD properties, signal intensity and emission energy. In combination with
strain-induced fine tuning this offers a robust method to achieve spectral
resonance in the emission of remote QDs. We show that the spectral tuning has
no detectable influence on the multi-photon suppression with $g^{(2)}(0)$ as
low as 2-4\% and that the emission can be stabilized to an accuracy of 4 $\mu$eV
using a closed-loop optical feedback.

\subsection*{\href{http://arxiv.org/abs/2009.12540v1}{Growth rate of 3D Tetris}}
\subsubsection*{M. V. Tamm, and S. Nechaev (2020-09-26)}
We consider configurational statistics of three-dimensional heaps of $N$
pieces ($N\gg 1$) on a simple cubic lattice in a large 3D bounding box of base
$n \times n$, and calculate the growth rate, $\Lambda(n)$, of the corresponding
partition function, $Z_N\sim N^{\theta}[\Lambda(n)]^N$, at $n\gg 1$. Our
computations rely on a theorem of G.X. Viennot, which connects the generating
function of a $(D+1)$-dimensional heap of pieces to the generating function of
projection of these pieces onto a $D$-dimensional subspace. The growth rate of
a heap of cubic blocks, which cannot touch each other by vertical faces, is
thus related to the position of zeros of the partition function describing 2D
lattice gas of hard squares. We study the corresponding partition function
exactly at low densities on finite $n\times n$ lattice of arbitrary $n$, and
extrapolate its behavior to the jamming transition density. This allows us to
estimate the limiting growth rate, $\Lambda =\lim_{n\to\infty}\Lambda(n)\approx
9.5$. The same method works for any graph above the underlying 2D lattice and
for various shapes of pieces: flat vertical squares, mapped to an ensemble of
repulsive dimers, dominoes mapped to an ensemble of rectangles with hard-core
repulsion, etc.

\subsection*{\href{http://arxiv.org/abs/2009.12538v1}{Lie transformation on shortcut to adiabaticity in parametric driving  quantum system}}
\subsubsection*{Jian-jian Cheng, and Lin Zhang (2020-09-26)}
Shortcut to adiabaticity (STA) is a speed way to produce the same final state
that would result in an adiabatic, infinitely slow process. Two typical
techniques to engineer STA are developed by either introducing auxiliary
counterdiabatic fields or finding new Hamiltonians that own dynamical
invariants to constraint the system into the adiabatic paths. In this paper, a
consistent method is introduced to naturally connect the above two techniques
with a unified Lie algebraic framework, which neatly removes the requirements
of finding instantaneous states in the transitionless driving method and the
invariant quantities in the invariant-based inverse engineering approach. The
general STA schemes for different potential expansions are concisely achieved
with the aid of this method.

\subsection*{\href{http://arxiv.org/abs/2009.12536v1}{Direct visualization of electro-thermal filament formation in a Mott  system}}
\subsubsection*{Matthias Lange, \dots, and Dieter Koelle (2020-09-26)}
The high power consumption caused by Joule heating is one reason for the
emergence of the research area of neuromorphic computing. However, Joule
heating is not only detrimental. In a specific class of devices considered for
emulating firing of neurons, the formation of an electro-thermal filament
sustained by locally confined Joule heating accompanies resistive switching.
Here, the resistive switching in a V2O3-thin-film device is visualized via
high-resolution wide-field microscopy. Although the formation and destruction
of electro-thermal filaments dominate the resistive switching, the
strain-induced coupling of the structural and electronic degrees of freedom
leads to various unexpected effects like oblique filaments, filament splitting,
memory effect, and a hysteretic current-voltage relation with saw-tooth like
jumps at high currents.

\subsection*{\href{http://arxiv.org/abs/2009.12535v1}{Double C-NOT attack on a single-state semi-quantum key distribution  protocol and its improvement}}
\subsubsection*{Jun Gu and Tzonelih Hwang (2020-09-26)}
Recently, Zhang et al. proposed a single-state semi-quantum key distribution
protocol (Int. J. Quantum Inf, 18, 4, 2020) to help a quantum participant to
share a secret key with a classical participant. However, this study shows that
an eavesdropper can use a double C-NOT attack to obtain parts of the final
shared key without being detected by the participants. To avoid this problem, a
modification is proposed here.

\subsection*{\href{http://arxiv.org/abs/2009.12530v1}{Bekenstein bound and uncertainty relations}}
\subsubsection*{Luca Buoninfante, \dots, and Fabio Scardigli (2020-09-26)}
The non-zero value of Planck constant $\hbar$ underlies the emergence of
several inequalities that must be satisfied in the quantum realm, the most
prominent one being the Heisenberg Uncertainty Principle. Among these
inequalities, the Bekenstein bound provides a universal limit on the entropy
that can be contained in a localized quantum system of given size and total
energy. In this letter, we explore how the Bekenstein bound is affected when
the Heisenberg uncertainty relation is deformed so as to accommodate
gravitational effects at the Planck scale (Generalized Uncertainty Principle).
By resorting to very general arguments, we derive in this way a "generalized
Bekenstein bound". Physical implications of this result are discussed for both
cases of positive and negative values of the deformation parameter.

\subsection*{\href{http://arxiv.org/abs/2009.12522v1}{Structure and Isotropy of Lattice Pressure Tensors for Multi-range  Potentials}}
\subsubsection*{Matteo Lulli, \dots, and Xiaowen Shan (2020-09-26)}
We systematically analyze the tensorial structure of the lattice pressure
tensors for a class of multiphase lattice Boltzmann models (LBM) with
multi-range interactions. Due to lattice discrete effects, we show that the
built-in isotropy properties of the lattice interaction forces are not
necessarily mirrored in the corresponding lattice pressure tensor. We therefore
outline a new procedure to retrieve the desired isotropy in the lattice
pressure tensors via a suitable choice of multi-range potentials. The newly
obtained LBM forcing schemes are tested via numerical simulations of non-ideal
equilibrium interfaces and are shown to yield weaker and less spatially
extended spurious currents with respect to forcing schemes obtained by forcing
isotropy requirements only.

\subsection*{\href{http://arxiv.org/abs/2009.12520v1}{Orientational quantum revivals induced by a single-cycle terahertz pulse}}
\subsubsection*{Chuan-Cun Shu, \dots, and Niels E. Henriksen (2020-09-26)}
The phenomenon of quantum revivals resulting from the self-interference of
wave packets has been observed in several quantum systems and utilized widely
in spectroscopic applications. Here, we present a combined analytical and
numerical study on the generation of orientational quantum revivals (OQRs)
exclusively using a single-cycle THz pulse. As a proof of principle, we examine
the scheme in the linear polar molecule HCN with experimentally accessible
pulse parameters and obtain strong field-free OQR without requiring the
condition of the sudden-impact limit. To visualize the involved quantum
mechanism, we derive a three-state model using the Magnus expansion of the
time-evolution operator. Interestingly, the THz pulse interaction with the
electric-dipole moment can activate direct multiphoton processes, leading to
OQR enhancements beyond that induced by a rotational ladder-climbing mechanism
from the rotational ground state. This work provides an explicit and feasible
approach toward quantum control of molecular rotation, which is at the core of
current research endeavors with potential applications in atomic and molecular
physics, photochemistry, and quantum information science.

\subsection*{\href{http://arxiv.org/abs/2009.12519v1}{High-Throughput Design of Peierls and Charge Density Wave Phases in Q1D  Organometallic Materials}}
\subsubsection*{Prakriti Kayastha and Raghunathan Ramakrishnan (2020-09-26)}
Understanding periodic crystal distortions driven by Peierls or charge
density wave mechanisms in quasi one-dimensional systems is of abiding
importance from a materials design perspective. Dispersions of a soft phonon
mode in the dynamically unstable, translationally symmetric phase encode a
material's preference for a less symmetric phase lower in energy. Guided by
this dynamical tenet as a design criterion, we examine 1199 materials
combinatorially generated with B/N/S-substituted cyclopentadienyl anion and
monovalent metal cations. Our unbiased enumeration comprehensively covers all
stable, planar 5-membered rings. We identify materials stabilizing in
undistorted, Peierls and CDW phases validating the approach. Our results are in
accord with experimentally observed polydecker/zigzag arrangements in Cp-metal
complexes. As case studies, we present Peierls candidates exhibiting
gap-opening as well as an uncommon gap-closing transition. Further, we zero-in
on the illustrative CDW prototype, B$_3$NSH$_3$Cu undergoing a trimerization.

\subsection*{\href{http://arxiv.org/abs/2009.12505v1}{Efficient Hybrid Density Functional Calculations for Large Periodic  Systems Using Numerical Atomic Orbitals}}
\subsubsection*{Peize Lin, and Lixin He (2020-09-26)}
We present an efficient, linear-scaling implementation for building the
(screened) Hartree-Fock exchange (HFX) matrix for periodic systems within the
framework of numerical atomic orbital (NAO) basis functions. Our implementation
is based on the localized resolution of the identity approximation by which
two-electron Coulomb repulsion integrals can be obtained by only computing
two-center quantities -- a feature that is highly beneficial to NAOs. By
exploiting the locality of basis functions and efficient prescreening of the
intermediate three- and two-index tensors, one can achieve a linear scaling of
the computational cost for building the HFX matrix with respect to the system
size. Our implementation is massively parallel, thanks to a MPI/OpenMP hybrid
parallelization strategy for distributing the computational load and memory
storage. All these factors add together to enable highly efficient hybrid
functional calculations for large-scale periodic systems. In this work we
describe the key algorithms and implementation details for the HFX build as
implemented in the ABACUS code package. The performance and scalability of our
implementation with respect to the system size and the number of CPU cores are
demonstrated for selected benchmark systems up to 4096 atoms.

\subsection*{\href{http://arxiv.org/abs/2009.12502v1}{Discovery of a weak topological insulating state and van Hove  singularity in triclinic RhBi2}}
\subsubsection*{Kyungchan Lee, \dots, and Adam Kaminski (2020-09-26)}
Time reversal symmetric (TRS) invariant topological insulators (TIs) fullfil
a paradigmatic role in the field of topological materials, standing at the
origin of its development. Apart from TRS protected 'strong' TIs, it was
realized early on that more confounding weak topological insulators (WTI)
exist. WTIs depend on translational symmetry and exhibit topological surface
states only in certain directions making it significantly more difficult to
match the experimental success of strong TIs. We here report on the discovery
of a WTI state in RhBi2 that belongs to the optimal space group P1, which is
the only space group where symmetry indicated eigenvalues enumerate all
possible invariants due to absence of additional constraining crystalline
symmetries. Our ARPES, DFT calculations, and effective model reveal topological
surface states with saddle points that are located in the vicinity of a Dirac
point resulting in a van Hove singularity (VHS) along the (100) direction close
to the Fermi energy. Due to the combination of exotic features, this material
offers great potential as a material platform for novel quantum effects.

\subsection*{\href{http://arxiv.org/abs/2009.12497v1}{$k$-uniform states and quantum information masking}}
\subsubsection*{Fei Shi, \dots, and Xiande Zhang (2020-09-26)}
A pure state of $N$ parties with local dimension $d$ is called a $k$-uniform
state if all the reductions to $k$ parties are maximally mixed. Based on the
connections among $k$-uniform states, orthogonal arrays and linear codes, we
give general constructions for $k$-uniform states. We show that when $d\geq
4k-2$ (resp. $d\geq 2k-1$) is a prime power, there exists a $k$-uniform state
for any $N\geq 2k$ (resp. $2k\leq N\leq d+1$). Specially, we give the existence
of $4,5$-uniform states for almost every $N$-qudits. Further, we generalize the
concept of quantum information masking in bipartite systems given by [Modi
\emph{et al.} Phys. Rev. Lett. \textbf{120}, 230501 (2018)] to $k$-uniform
quantum information masking in multipartite systems, and we show that
$k$-uniform states can be used for $k$-uniform quantum information masking.

\subsection*{\href{http://arxiv.org/abs/2009.12490v1}{Structural and Optoelectronic Behaviour of Copper Doped Cs2AgInCl6  Double Perovskite: A DFT Investigation}}
\subsubsection*{I. B. Ogunniranye, and O. E. Oyewande (2020-09-26)}
Recently, direct bandgap double perovskites are becoming more popular among
photovoltaic research community owing to their potential to address issues of
lead (Pb) toxicity and structural instability inherent in lead halide (simple)
perovskites. In this study, In-Ag based direct bandgap double perovskite,
Cs2AgInCl6 (CAIC), is treated with transition metal doping to improve the
optoelectronic properties of the material. Investigations of structural and
optoelectronic properties of Cu-doped CAIC, Cs2Ag(1-x)CuxInCl6, are done using
ab-initio calculations with density functional theory (DFT) and virtual crystal
approximation (VCA). Our calculations show that with increasing Cu content, the
optimized lattice parameter and direct bandgap of Cs2Ag(1-x)CuxInCl6 decrease
following linear and quadratic functions respectively, while the bulk modulus
increases following a quadratic function. The photo-absorption coefficient,
optical conductivity and other optical parameters of interest are also
computed, indicating enhanced absorption and conductivity for higher Cu
contents. Based on our results, transition metal (Cu) doping is a viable means
of treating double perovskites - by tuning their optoelectronic properties
suitable for an extensive range of photovoltaics, solar cells and
optoelectronics.

\subsection*{\href{http://arxiv.org/abs/2009.12486v1}{Suppressing The Ferroelectric Switching Barrier in Hybrid Improper  Ferroelectrics}}
\subsubsection*{Shutong Li and Turan Birol (2020-09-26)}
Integration of ferroelectric materials into novel technological applications
requires low coercive field materials, and consequently, design strategies to
reduce the ferroelectric switching barriers. In this first principles study, we
show that biaxial strain, which has a strong effect on the ferroelectric ground
states, can also be used to tune the switching barrier of hybrid improper
ferroelectric Ruddlesden-Popper oxides. We identify the region of the strain --
tolerance factor phase diagram where this intrinsic barrier is suppressed, and
show that it can be explained in relation to strain induced phase transitions
to nonpolar phases.

\subsection*{\href{http://arxiv.org/abs/2009.12479v1}{Performance benefits of increased qubit connectivity in quantum  annealing 3-dimensional spin glasses}}
\subsubsection*{Andrew D. King and William Bernoudy (2020-09-26)}
An important challenge in superconducting quantum computing is the need to
physically couple many devices using quasi-two-dimensional fabrication
processes. Recent advances in the design and fabrication of quantum annealing
processors have enabled an increase in pairwise connectivity among thousands of
qubits. One benefit of this is the ability to minor-embed optimization problems
using fewer physical qubits for each logical spin. Here we demonstrate the
benefit of this progress in the problem of minimizing the energy of
three-dimensional spin glasses. Comparing the previous generation D-Wave 2000Q
system to the new Advantage system, we observe improved scaling of solution
time and improved consistency over multiple graph embeddings.

\subsection*{\href{http://arxiv.org/abs/2009.12476v1}{Magnetization Plateau Observed by Ultra-High Field Faraday Rotation in a  Kagomé Antiferromagnet Herbertsmithite}}
\subsubsection*{Ryutaro Okuma, and Shojiro Takeyama (2020-09-26)}
To capture the high-field magnetization process of herbertsmithite
(ZnCu3(OH)6Cl2), Faraday rotation (FR) measurements were carried out on a
single crystal in magnetic fields of up to 190 T. The magnetization data
evaluated from the FR angle exhibited a saturation behavior above 150 T at low
temperatures, which was attributed to the 1/3 magnetization plateau. The
overall behavior of the magnetization process was reproduced by theoretical
models based on the nearest-neighbor Heisenberg model. This suggests that
herbertsmithite is a proximate kagome antiferromagnet hosting an ideal quantum
spin liquid in the ground state. A distinguishing feature is the superlinear
magnetization increase, which is in contrast to the Brillouin function-type
increase observed by conventional magnetization measurements and indicates a
reduced contribution from free spins located at the Zn sites to the FR signal.

\subsection*{\href{http://arxiv.org/abs/2009.12472v1}{How will quantum computers provide an industrially relevant  computational advantage in quantum chemistry?}}
\subsubsection*{V. E. Elfving, \dots, and A. Bochevarov (2020-09-25)}
Numerous reports claim that quantum advantage, which should emerge as a
direct consequence of the advent of quantum computers, will herald a new era of
chemical research because it will enable scientists to perform the kinds of
quantum chemical simulations that have not been possible before. Such
simulations on quantum computers, promising a significantly greater accuracy
and speed, are projected to exert a great impact on the way we can probe
reality, predict the outcomes of chemical experiments, and even drive design of
drugs, catalysts, and materials. In this work we review the current status of
quantum hardware and algorithm theory and examine whether such popular claims
about quantum advantage are really going to be transformative. We go over
subtle complications of quantum chemical research that tend to be overlooked in
discussions involving quantum computers. We estimate quantum computer resources
that will be required for performing calculations on quantum computers with
chemical accuracy for several types of molecules. In particular, we directly
compare the resources and timings associated with classical and quantum
computers for the molecules H$_2$ for increasing basis set sizes, and Cr$_2$
for a variety of complete active spaces (CAS) within the scope of the CASCI and
CASSCF methods. The results obtained for the chromium dimer enable us to
estimate the size of the active space at which computations of non-dynamic
correlation on a quantum computer should take less time than analogous
computations on a classical computer. Using this result, we speculate on the
types of chemical applications for which the use of quantum computers would be
both beneficial and relevant to industrial applications in the short term.

\subsection*{\href{http://arxiv.org/abs/2009.12451v1}{Emergent dynamics from entangled mixed states}}
\subsubsection*{A. Valdés-Hernández, \dots, and A. R. Plastino (2020-09-25)}
Entanglement is at the core of quantum physics, playing a central role in
quantum phenomena involving composite systems. According to the timeless
picture of quantum dynamics, entanglement may also be essential for
understanding the very origins of dynamical evolution and the flow of time.
Within this point of view, the Universe is regarded as a bipartite entity
comprising a clock $C$ and a system $R$ (or "rest of the Universe") jointly
described by a global stationary state, and the dynamical evolution of $R$ is
construed as an emergent phenomena arising from the entanglement between $C$
and $R$. In spite of substantial recent efforts, many aspects of this approach
remain unexplored, particularly those involving mixed states. In the present
contribution we investigate the timeless picture of quantum dynamics for mixed
states of the clock-system composite, focusing on quantitative relations
linking the clock-system entanglement with the emerging dynamical evolution
experienced by the system.

\subsection*{\href{http://arxiv.org/abs/2009.12449v1}{Periodic Sudden Freezing and Thawing of Entanglement in Multiparty  Pure-State Quantum Systems}}
\subsubsection*{Yi Ding, and Joseph H. Eberly (2020-09-25)}
Within the one-excitation context of two identical two-level atoms
interacting with a common cavity, we are concerned with the dynamics of all
bipartite one-to-other entanglements between each qubit and the remaining part
of the whole system, from the perspective of resource sharing. We find a new
non-analytic "sudden" dynamical behavior of entanglement. Specifically, the sum
of the three one-to-other entanglements of the system can be suddenly frozen at
its maximal value or can be suddenly thawed from this value in a periodic
manner. We calculate the onset timing of sudden freezing and sudden thawing
under several different initial conditions. The phenomenon of the permanent
freezing of entanglement is also found. Further analyses about freezing and
thawing processes reveal quantitative and qualitative laws of resource sharing.

\subsection*{\href{http://arxiv.org/abs/2009.12445v1}{Quantum Fluctuations and New Instantons I: Linear Unbounded Potential}}
\subsubsection*{Viatcheslav Mukhanov, and Alexander Sorin (2020-09-25)}
We consider the decay of a false vacuum in circumstances where the methods
suggested by Coleman run into difficulties. We find that in these cases quantum
fluctuations play a crucial role. Namely, they naturally induce both an
ultraviolet and infrared cutoff scales, determined by the parameters of the
classical solution, beyond which this solution cannot be trusted anymore. This
leads to the appearance of a broad class of new $O(4)$ invariant instantons,
which would have been singular in the absence of an ultraviolet cutoff. We
apply our results to a case where the potential is unbounded from below in a
linear way and in particular show how the problem of small instantons is
resolved by taking into account the inevitable quantum fluctuations.

\subsection*{\href{http://arxiv.org/abs/2009.12444v1}{Quantum Fluctuations and New Instantons II: Quartic Unbounded Potential}}
\subsubsection*{Viatcheslav Mukhanov, and Alexander Sorin (2020-09-25)}
We study the fate of a false vacuum in the case of a potential that contains
a portion which is quartic and unbounded. We first prove that an $O(4)$
invariant instanton with the Coleman boundary conditions does not exist in this
case. This, however, does not imply that the false vacuum does not decay. We
show how the quantum fluctuations may regularize the singular classical
solutions. This gives rise to a new class of $O(4)$ invariant regularized
instantons which describe the vacuum instability in the absence of the Coleman
instanton. We derive the corresponding solutions and calculate the decay rate
they induce.

\subsection*{\href{http://arxiv.org/abs/2009.12439v1}{Spin-selective resonant tunneling induced by Rashba spin-orbit  interaction in semiconductor nanowire}}
\subsubsection*{J. Pawłowski, \dots, and S. Bednarek (2020-09-25)}
We consider a single electron confined within a quantum wire in a system of
two electrostatically-induced QDs defined by nearby gates. The time-varying
electric field, of single GHz frequency, perpendicular to the quantum wire, is
used to induce the Rashba coupling and enable spin-dependent resonant tunneling
of the electron between two adjacent potential wells with fidelity over 99.5\%.
This effect can be used for the high fidelity all-electrical electron-spin
initialization or readout in the spin-based quantum computer. In contrast to
other spin initialization methods, our technique can be performed adiabatically
without increase in the energy of the electron. Our simulations are supported
by a realistic self-consistent time-dependent Poisson-Schroedinger
calculations.

\subsection*{\href{http://arxiv.org/abs/2009.12438v1}{Quantum Enhanced Precision Estimation with Bright Squeezed Light}}
\subsubsection*{G. S. Atkinson, \dots, and J. C. F. Matthews (2020-09-25)}
Squeezed light has lower quantum noise in amplitude or phase than the quantum
noise limit (QNL) of classical light. This enables enhancing sensitivity --
quantified by the signal-to-noise ratio (SNR) -- to beyond the QNL in optical
techniques such as spectroscopy, gravitational wave detection, magnetometry and
imaging. Precision -- the variance of repeated estimates -- has also been
enhanced beyond the QNL using squeezed vacuum to estimate optical phase and
transmission. However, demonstrations have been limited to femtowatts of probe
power. Here we demonstrate simultaneous precision and sensitivity beyond the
QNL for estimating modulated transmission using a squeezed amplitude probe of
0.2 mW average (25 W peak) power. This corresponds to 8 orders of magnitude
above the power limitations of previous sub-QNL precision measurements. Our
theory and experiment show precision enhancement scales with the amount of
squeezing and increases with the resolution bandwidth of detection. We conclude
that quantum enhanced precision in estimating a modulated transmission
increases when observing dynamics at $\sim$ kHz and above. This opens the way
to performing measurements that compete with the optical powers of current
classical techniques, but have superior precision and sensitivity beyond the
classical limit.

\subsection*{\href{http://arxiv.org/abs/2009.12435v1}{Efficient matrix-product-state preparation of highly entangled trial  states: Weak Mott insulators on the triangular lattice revisited}}
\subsubsection*{Amir M Aghaei, \dots, and Ryan V. Mishmash (2020-09-25)}
Using tensor network states to unravel the physics of quantum spin liquids in
minimal, yet generic microscopic spin or electronic models remains notoriously
challenging. A prominent open question concerns the nature of the insulating
ground state of two-dimensional half-filled Hubbard-type models on the
triangular lattice in the vicinity of the Mott metal-insulator transition, a
regime which can be approximated microscopically by a spin-1/2 Heisenberg model
supplemented with additional "ring-exchange" interactions. Using a novel and
efficient state preparation technique whereby we initialize full density matrix
renormalization group (DMRG) calculations with highly entangled
Gutzwiller-projected Fermi surface trial wave functions, we show -- contrary to
previous works -- that the simplest triangular lattice $J$-$K$ spin model with
four-site ring exchange likely does not harbor a fully gapless U(1) spinon
Fermi surface (spin Bose metal) phase on four- and six-leg wide ladders. Our
methodology paves the way to fully resolve with DMRG other controversial
problems in the fields of frustrated quantum magnetism and strongly correlated
electrons.

\subsection*{\href{http://arxiv.org/abs/2009.12429v1}{Isotope effects in x-ray absorption spectra of liquid water}}
\subsubsection*{Chunyi Zhang, \dots, and Xifan Wu (2020-09-25)}
The isotope effects in x-ray absorption spectra of liquid water are studied
by a many-body approach within electron-hole excitation theory. The molecular
structures of both light and heavy water are modeled by path-integral molecular
dynamics based on the advanced deep-learning technique. The neural network is
trained on ab initio data obtained with SCAN density functional theory. The
experimentally observed isotope effect in x-ray absorption spectra is
reproduced semiquantitatively in theory. Compared to the spectrum in normal
water, the blueshifted and less pronounced pre- and main-edge in heavy water
reflect that the heavy water is more structured at short- and
intermediate-range of the hydrogen-bond network. In contrast, the isotope
effect on the spectrum is negligible at post-edge, which is consistent with the
identical long-range ordering in both liquids as observed in the diffraction
experiment.

\subsection*{\href{http://arxiv.org/abs/2009.12428v1}{Enhancement of superconductivity with external phonon squeezing}}
\subsubsection*{Andrey Grankin, and Victor M. Galitski (2020-09-25)}
Squeezing of phonons due to the non-linear coupling to electrons is a way to
enhance superconductivity as theoretically studied in a recent work [Kennes et
al. Nature Physics 13, 479 (2017)]. We study quadratic electron-phonon
interaction in the presence of phonon pumping and an additional external
squeezing. Interference between these two driving sources induces a
phase-sensitive enhancement of electron-electron attraction, which we find as a
generic mechanism to enhance any boson-mediated interactions. The strongest
enhancement of superconductivity is shown to be on the boundary with the
dynamical lattice instabilities caused by driving. We propose several
experimental platforms to realize our scheme.

\subsection*{\href{http://arxiv.org/abs/2009.12403v1}{Universal behavior of the bosonic metallic ground state in a  two-dimensional superconductor}}
\subsubsection*{Zhuoyu Chen, \dots, and Harold Y. Hwang (2020-09-25)}
Anomalous metallic behavior, marked by a saturating finite resistivity much
lower than the Drude estimate, has been observed in a wide range of
two-dimensional superconductors. Utilizing the electrostatically gated
LaAlO3/SrTiO3 interface as a versatile platform for superconductor-metal
quantum phase transitions, we probe variations in the gate, magnetic field, and
temperature to construct a phase diagram crossing from superconductor,
anomalous metal, vortex liquid, to Drude metal states, combining longitudinal
and Hall resistivity measurements. We find that the anomalous metal phases
induced by gating and magnetic field, although differing in symmetry, are
connected in the phase diagram and exhibit similar magnetic field response
approaching zero temperature. Namely, within a finite regime of the anomalous
metal state, the longitudinal resistivity linearly depends on field while the
Hall resistivity diminishes, indicating an emergent particle-hole symmetry. The
universal behavior highlights the uniqueness of the quantum bosonic metallic
state, distinct from bosonic insulators and vortex liquids.

\subsection*{\href{http://arxiv.org/abs/2009.12398v1}{Andreev-Coulomb Drag in Coupled Quantum Dots}}
\subsubsection*{S. Mojtaba Tabatabaei, \dots, and Rafael Sanchez (2020-09-25)}
The Coulomb drag effect has been observed as a tiny current induced by both
electron-hole asymmetry and interactions in normal coupled quantum dot devices.
In the present work we show that the effect can be boosted by replacing one of
the normal electrodes by a superconducting one. Moreover, we show that at low
temperatures and for sufficiently strong coupling to the superconducting lead,
the Coulomb drag is dominated by Andreev processes, is robust against details
of the system parameters and can be controlled with a single gate voltage. This
mechanism can be distinguished from single-particle contributions by a sign
inversion of the drag current.

\subsection*{\href{http://arxiv.org/abs/2009.12396v1}{Gel rupture in a dynamic environment}}
\subsubsection*{Kelsey-Ann Leslie, \dots, and Michelle M. Driscoll (2020-09-25)}
Hydrogels have had a profound impact in the fields of tissue engineering,
drug delivery, and materials science as a whole. Due to the network
architecture of these materials, imbibement with water often results in uniform
swelling and isotropic expansion which scales with the degree of cross-linking.
However, the development of internal stresses during swelling can have dramatic
consequences, leading to surface instabilities as well as rupture or bursting
events. To better understand hydrogel behavior, macroscopic mechanical
characterization techniques (e.g.\ tensile testing, rheometry) are often used,
however most commonly these techniques are employed on samples that are in two
distinct states: (1) unswollen and without any solvent, or (2) in an
equilibrium swelling state where the maximum amount of water has been imbibed.
Rarely is the dynamic process of swelling studied, especially in samples where
rupture or failure events are observed. To address this gap, here we focus on
rupture events in poly(ethylene glycol)-based networks that occur in response
to swelling with water. Rupture events were visualized using high-speed
imaging, and the influence of swelling on material properties was characterized
using dynamic mechanical analysis. We find that rupture events follow a
three-stage process that includes a waiting period, a slow fracture period, and
a final stage in which a rapid increase in the velocity of crack propagation is
observed. We describe this fracture behavior based on changes in material
properties that occur during swelling, and highlight how this rupture behavior
can be controlled by straight-forward modifications to the hydrogel network
structure.

\subsection*{\href{http://arxiv.org/abs/2009.12393v1}{Valence-Bond Order in a Honeycomb Antiferromagnet Coupled to Quantum  Phonons}}
\subsubsection*{Manuel Weber (2020-09-25)}
We use exact quantum Monte Carlo simulations to demonstrate that the N\'eel
ground state of an antiferromagnetic SU(2) spin-$\frac{1}{2}$ Heisenberg model
on the honeycomb lattice can be destroyed by a coupling to quantum phonons. We
find a clear first-order transition to a valence-bond-solid state with Kekul\'e
order instead of a deconfined quantum critical point. However, quantum lattice
fluctuations can drive the transition towards weakly first-order, revealing a
tunability of the transition by the retardation of the interaction. In contrast
to the one-dimensional case, our phase diagram in the adiabatic regime is
qualitatively different from the frustrated $J_1$-$J_2$ model. Our results
suggest that a coupling to bond phonons can induce Kekul\'e order in Dirac
systems.

\subsection*{\href{http://arxiv.org/abs/2009.12389v1}{Entanglement and weak interaction driven mobility of small molecules in  polymer networks}}
\subsubsection*{Rajarshi Guha, \dots, and Jennifer L. Ross (2020-09-25)}
Diffusive transport of small molecules within the internal structures of
biological and synthetic material systems is complex because the crowded
environment presents chemical and physical barriers to mobility. We explored
this mobility using a synthetic experimental system of small dye molecules
diffusing within a polymer network at short time scales. We find that the
diffusion of inert molecules is inhibited by the presence of the polymers.
Counter-intuitively, small, hydrophobic molecules display smaller reduction in
mobility and also able to diffuse faster through the system by leveraging
crowding specific parameters. We explained this phenomenon by developing a de
novo model and using these results, we hypothesized that non-specific
hydrophobic interactions between the molecules and polymer chains could
localize the molecules into compartments of overlapped and entangled chains
where they experience microviscosity, rather than macroviscosity. We introduced
a characteristic interaction time parameter to quantitatively explain
experimental results in the light of frictional effects and molecular
interactions. Our model is in good agreement with the experimental results and
allowed us to classify molecules into two different mobility categories solely
based on interaction. By changing the surface group, polymer molecular weight,
and by adding salt to the medium, we could further modulate the mobility and
mean square displacements of interacting molecules. Our work has implications
in understanding intracellular diffusive transport in microtubule networks and
other systems with macromolecular crowding and could lead to transport
enhancement in synthetic polymer systems.

\subsection*{\href{http://arxiv.org/abs/2009.12387v1}{Stopping and reversing sound via dynamic dispersion tuning in a phononic  metamaterial}}
\subsubsection*{Pragalv Karki and Jayson Paulose (2020-09-25)}
Slowing down, stopping, and reversing a signal is a core functionality for
information processing. Here, we show that this functionality can be realized
by tuning the dispersion of a periodic system through a dispersionless, or
flat, band. Specifically, we propose a new class of phononic metamaterials
based on plate resonators, in which the phonon band dispersion can be changed
from acoustic to optical character by modulating a uniform prestress. The
switch is enabled by the change in sign of an effective coupling between
fundamental modes, which generically leads to a nearly dispersion-free band at
the transition point. We demonstrate how adiabatic tuning of the band
dispersion can immobilize and reverse the propagation of a sound pulse in
simulations of a one-dimensional resonator chain. Our study relies on the basic
principles of thin-plate elasticity independently of any specific material,
making our results applicable across varied length scales and experimental
platforms. More broadly, our approach could be replicated for signal
manipulation in photonic metamaterials and electronic heterostructures.

\subsection*{\href{http://arxiv.org/abs/2009.12378v1}{Phonon transmittance of one dimensional quasicrystals}}
\subsubsection*{Junmo Jeon and SungBin Lee (2020-09-25)}
In quasicrystals, special tiling patterns could give rise to unique physical
phenomena such as critical states distinct from periodic systems. In this
paper, we study how quasi-periodicity in aperiodic systems results in anomalous
phonon modes, especially focusing on thermal transmittance in one-dimensional
quasicrystals. Unlike periodic or compeletly random systems, we classify
certain quasicrystals could host critical phonon modes whose transport
properties are topologically protected based on their pattern equivariant
cohomology group of supertilings. Starting from discussing general rule to find
such critical phonon modes, we discuss classification of topologically distinct
thermal transmittance in quasiperiodic systems. To be more specific, we
exemplify (decorated) metallic-mean tilings and Cantor tiling, and derive
universal features for resonant and decaying phonon modes as a function of
quasi-periodic strength. Our study paves a new way to understand thermal
transmittance of quasi-periodic systems based on the topological classification
and offers quasicrystals as strong candidates to control drastic phonon modes.

\subsection*{\href{http://arxiv.org/abs/2009.12376v1}{TBG III: Interacting Hamiltonian and Exact Symmetries of Twisted Bilayer  Graphene}}
\subsubsection*{B. Andrei Bernevig, \dots, and Biao Lian (2020-09-25)}
We derive the explicit Hamiltonian of twisted bilayer graphene (TBG) with
Coulomb interaction projected into the flat bands, and study the symmetries of
the Hamiltonian. First, we show that all projected TBG Hamiltonians can be
written as Positive Semidefinite Hamiltonian, the first example of which was
found in [PRL 122, 246401]. We then prove that the interacting TBG Hamiltonian
exhibits an exact U(4) symmetry in the exactly flat band (nonchiral-flat)
limit. We further define, besides a first chiral limit where the AA stacking
hopping is zero, a new second chiral limit where the AB/BA stacking hopping is
zero. In the first chiral-flat limit (or second chiral-flat limit) with exactly
flat bands, the TBG is enhanced to have an exact U(4)$\times$U(4) symmetry,
whose generators are different between the two chiral limits. While in the
first chiral limit and in the non-chiral case these symmetries have been found
in [PRX 10, 031034] for the $8$ lowest bands, we here prove that they are valid
for projection into any $8 n_\text{max}$ particle-hole symmetric TBG bands,
with $n_\text{max}>1$ being the practical case for small twist angles
$<1^\circ$. Furthermore, in the first or second chiral-nonflat limit without
flat bands, an exact U(4) symmetry still remains. We also elucidate the link
between the U(4) symmetry presented here and the similar but different U(4) of
[PRL 122, 246401]. Furthermore, we show that our projected Hamiltonian can be
viewed as the normal-ordered Coulomb interaction plus a Hartree-Fock term from
passive bands, and exhibits a many-body particle-hole symmetry which renders
the physics symmetric around charge neutrality. We also provide an efficient
parameterization of the interacting Hamiltonian. The existence of two chiral
limits, with an enlarged symmetry suggests a possible duality of the model yet
undiscovered.

\subsection*{\href{http://arxiv.org/abs/2009.12366v1}{Controlling the macroscopic electrical properties of reduced graphene  oxide by nanoscale writing of electronic channels}}
\subsubsection*{Arijit Kayal, \dots, and J. Mitra (2020-09-25)}
The allure of all carbon electronics stems from the spread in physical
properties, across all its allotropes. The scheme also harbours unique
challenges, like tunability of band-gap, variability of doping and defect
control. Here, we explore the technique of scanning probe tip induced nanoscale
reduction of graphene oxide (GO), which nucleates conducting, sp2 rich
graphitic regions on the insulating GO background. Flexibility of direct
writing is supplemented with control over degree of reduction and tunability of
bandgap, through macroscopic control parameters. The fabricated reduced - GO
channels and ensuing devices are investigated via spectroscopic, and
temperature and bias dependent electrical transport and correlated with
spatially resolved electronic properties, using surface potentiometry. Presence
of carrier localization effects, induced by the phase-separated sp2/sp3
domains, and large local electric field fluctuations are reflected in the
non-linear transport across the channels. Together the results indicate a
complex transport phenomena which may be variously dominated by tunnelling,
variable range hopping or activated depending on the electronic state of the
material.

\subsection*{\href{http://arxiv.org/abs/2009.12364v1}{Sound in a system of chiral one-dimensional fermions}}
\subsubsection*{K. A. Matveev (2020-09-25)}
We consider a system of one-dimensional fermions moving in one direction,
such as electrons at the edge of a quantum Hall system. At sufficiently long
time scales the system is brought to equilibrium by weak interactions between
the particles, which conserve their total number, energy, and momentum. Time
evolution of the system near equilibrium is described by hydrodynamics based on
the three conservation laws. We find that the system supports three sound
modes. In the low temperature limit one mode is a pure oscillation of particle
density, analogous to the ordinary sound. The other two modes involve
oscillations of both particle and entropy densities. In the presence of
disorder, the first sound mode is strongly damped at frequencies below the
momentum relaxation rate, whereas the other two modes remain weakly damped.

\subsection*{\href{http://arxiv.org/abs/2009.12361v2}{Hardware-efficient variational quantum algorithms for time evolution}}
\subsubsection*{Marcello Benedetti, and Michael Lubasch (2020-09-25)}
Parameterized quantum circuits are a promising technology for achieving a
quantum advantage. An important application is the variational simulation of
time evolution of quantum systems. To make the most of quantum hardware,
variational algorithms need to be as hardware-efficient as possible. Here we
present alternatives to the time-dependent variational principle that are
hardware-efficient and do not require matrix inversion. In relation to
imaginary time evolution, our approach significantly reduces the hardware
requirements. With regards to real time evolution, where high precision can be
important, we present algorithms of systematically increasing accuracy and
hardware requirements. We numerically analyze the performance of our algorithms
using quantum Hamiltonians with local interactions.

\subsection*{\href{http://arxiv.org/abs/2009.12354v1}{Low Energy Magneto-optics of Tb$_{2}$Ti$_{2}$O$_{7}$ in [111] Magnetic  Field}}
\subsubsection*{Xinshu Zhang, \dots, and N. P. Armitage (2020-09-25)}
The pyrochlore magnet Tb$_{2}$Ti$_{2}$O$_{7}$ shows a lack of magnetic order
to low temperatures and is considered to be a quantum spin liquid candidate. We
perform time-domain THz spectroscopy on high quality Tb$_{2}$Ti$_{2}$O$_{7}$
crystal and study the low energy excitations as a function of [111] magnetic
field with high energy resolution. The low energy crystal field excitations
change their energies anomalously under magnetic field. Despite several sharp
field dependent changes, we show that the material's spectrum can be described
not by a phase transitions, but by field dependent hybridization between the
low energy crystal field levels. We highlight the strong coupling between spin
and lattice degrees of freedom in Tb$_{2}$Ti$_{2}$O$_{7}$ as evidenced by the
magnetic field tunable crystal field environment. Calculations based on single
ion physics with field induced symmetry reduction of the crystal field
environment can reproduce our data.

\subsection*{\href{http://arxiv.org/abs/2009.12352v1}{Driving a low critical current Josephson junction array with a  mode-locked laser}}
\subsubsection*{J. Nissila, \dots, and A. Kemppinen (2020-09-25)}
We demonstrate the operation of Josephson junction arrays (JJA) driven by
optical pulses generated by a mode-locked laser and an optical time-division
multiplexer. A commercial photodiode converts the optical pulses into
electrical ones in liquid helium several cm from the JJA. The performance of
our custom-made mode-locked laser is sufficient for driving a JJA with low
critical current at multiple Shapiro steps. Our optical approach is a potential
enabler for fast and energy-efficient pulse drive without expensive
high-bandwidth electrical pulse pattern generator, and without high-bandwidth
electrical cabling crossing temperature stages. Our measurements and
simulations motivate an improved integration of photodiodes and JJAs using,
e.g., flip-chip techniques, in order to improve both the understanding and
fidelity of pulse-driven Josephson Arbitrary Waveform Synthesizers (JAWS).

\subsection*{\href{http://arxiv.org/abs/2009.14030v1}{Antiferromagnetic phase diagram of the cuprate superconductors}}
\subsubsection*{Lizardo H. C. M. Nunes, and E. C. Marino (2020-09-25)}
Taking the spin-fermion model as the starting point for describing the
cuprate superconductors, we obtain an effective nonlinear sigma-field
hamiltonian, which takes into account the effect of doping in the system. We
obtain an expression for the spin-wave velocity as a function of the chemical
potential. For appropriate values of the parameters we determine the
antiferromagnetic phase diagram for the YBa$_2$Cu$_3$O$_{6+x}$ compound as a
function of the dopant concentration in good agreement with the experimental
data. Furthermore, our approach provides a unified description for the phase
diagrams of the hole-doped and the electron doped compounds, which is
consistent with the remarkable similarity between the phase diagrams of these
compounds, since we have obtained the suppression of the antiferromagnetic
phase as the modulus of the chemical potential increases. The aforementioned
result then follows by considering positive values of the chemical potential
related to the addition of holes to the system, while negative values
correspond to the addition of electrons.

\subsection*{\href{http://arxiv.org/abs/2009.12331v1}{Frequency fluctuations in nanomechanical silicon nitride string  resonators}}
\subsubsection*{Pedram Sadeghi, \dots, and Silvan Schmid (2020-09-25)}
High quality factor ($Q$) nanomechanical resonators have received a lot of
attention for sensor applications with unprecedented sensitivity. Despite the
large interest, few investigations into the frequency stability of high-$Q$
resonators have been reported. Such resonators are characterized by a linewidth
significantly smaller than typically employed measurement bandwidths, which is
the opposite regime to what is normally considered for sensors. Here, the
frequency stability of high-$Q$ silicon nitride string resonators is
investigated both in open-loop and closed-loop configurations. The stability is
here characterized using the Allan deviation. For open-loop tracking, it is
found that the Allan deviation gets separated into two regimes, one limited by
the thermomechanical noise of the resonator and the other by the detection
noise of the optical transduction system. The point of transition between the
two regimes is the resonator response time, which can be shown to have a linear
dependence on $Q$. Laser power fluctuations from the optical readout is found
to present a fundamental limit to the frequency stability. Finally, for
closed-loop measurements, the response time is shown to no longer be
intrinsically limited but instead given by the bandwidth of the closed-loop
tracking system. Computed Allan deviations based on theory are given as well
and found to agree well with the measurements. These results are of importance
for the understanding of fundamental limitations of high-$Q$ resonators and
their application as high performance sensors.

\subsection*{\href{http://arxiv.org/abs/2009.12328v1}{Enhanced violation of Leggett-Garg Inequality in three flavour neutrino  oscillations via non-standard interactions}}
\subsubsection*{Sheeba Shafaq and Poonam Mehta (2020-09-25)}
Neutrino oscillations occur due to non-zero masses and mixings and most
importantly they are believed to maintain quantum coherence even over
astrophysical length scales. Here, we study the quantumness of three flavour
neutrino oscillations by studying the extent of violation of Leggett-Garg
inequalities (LGI) if non-standard interactions are taken into account. We
report an enhancement in violation of LGI with respect to the standard scenario
for certain choice of NSI parameters.

\subsection*{\href{http://arxiv.org/abs/2009.12327v1}{Alternation of Magnetic Anisotropy Accompanied by Metal-Insulator  Transition in Strained Ultrathin Manganite Heterostructures}}
\subsubsection*{Masaki Kobayashi, \dots, and Atsushi Fujimori (2020-09-25)}
Fundamental understanding of interfacial magnetic properties in ferromagnetic
heterostructures is essential to utilize ferromagnetic materials for spintronic
device applications. In this paper, we investigate the interfacial magnetic and
electronic structures of epitaxial single-crystalline LaAlO$_3$
(LAO)/La$_{0.6}$Sr$_{0.4}$MnO$_3$ (LSMO)/Nb:SrTiO$_3$ (Nb:STO) heterostructures
with varying LSMO-layer thickness, in which the magnetic anisotropy strongly
changes depending on the LSMO thickness due to the delicate balance between the
strains originating from both the Nb:STO and LAO layers, using x-ray magnetic
circular dichroism (XMCD) and photoemission spectroscopy (PES). We successfully
detect the clear change of the magnetic behavior of the Mn ions concomitant
with the thickness-dependent metal-insulator transition (MIT). Our results
suggest that double-exchange interaction induces the ferromagnetism in the
metallic LSMO film under tensile strain caused by the SrTiO$_3$ substrate,
while superexchange interaction determines the magnetic behavior in the
insulating LSMO film under compressive strain originating from the top LAO
layer. Based on those findings, the formation of a magnetic dead layer near the
LAO/LSMO interface is attributed to competition between the superexchange
interaction via Mn 3$d_{3z^2-r^2}$ orbitals under compressive strain and the
double-exchange interaction via the 3$d_{x^2-y^2}$ orbitals.

\subsection*{\href{http://arxiv.org/abs/2009.12322v1}{Temperature vs. doping phase diagram of cuprate superconductors}}
\subsubsection*{Lizardo H. C. M. Nunes, and E. C. Marino (2020-09-25)}
Starting from a spin-fermion model for the cuprate superconductors, we obtain
an effective interaction for the charge carriers by integrating out the spin
degrees of freedom. Our model predicts a quantum critical point for the
superconducting interaction coupling, which sets up a threshold for the onset
of superconductivity in the system. We show that the physical value of this
coupling is below this threshold, thus explaining why there is no
superconducting phase for the undoped system. Then, by including doping, we
find a dome-shaped dependence of the critical temperature as charge carriers
are added to the system, in agreement with the experimental phase diagram. The
superconducting critical temperature is calculated without adjusting any free
parameter and yields, at optimal doping $ T_c \sim $ 45 K, which is comparable
to the experimental data.

\subsection*{\href{http://arxiv.org/abs/2009.12321v1}{A Unified Description of Spin Transport, Weak Antilocalization and  Triplet Superconductivity in Systems with Spin-Orbit Coupling}}
\subsubsection*{Stefan Ilić, and F. Sebastián Bergeret (2020-09-25)}
The Eilenberger equation is a standard tool in the description of
superconductors with an arbitrary degree of disorder. It can be generalized to
systems with linear-in-momentum spin-orbit coupling (SOC), by exploiting the
analogy of SOC with a non-abelian background field. Such field mixes singlet
and triplet components and yields the rich physics of magnetoelectric
phenomena. In this work we show that the application of this equation extends
further, beyond superconductivity. In the normal state, the linearized
Eilenberger equation describes the coupled spin-charge dynamics. Moreover, its
resolvent corresponds to the so called Cooperons, and can be used to calculate
the weak localization corrections. Specifically, we show how to solve this
equation for any source term and provide a closed-form solution for the case of
Rashba SOC. We use this solution to address several problems of interest for
spintronics and superconductivity. Firstly, we study spin injection from
ferromagnetic electrodes in the normal state, and describe the spatial
evolution of spin density in the sample, and the complete crossover from the
diffusive to the ballistic limit. Secondly, we address the so-called
superconducting Edelstein effect, and generalize the previously known results
to arbitrary disorder. Thirdly, we study weak localization correction beyond
the diffusive limit, which can be a valuable tool in experimental
characterization of materials with very strong SOC. We also address the
so-called pure gauge case where the persistent spin helices form. Our work
establishes the linearized Eilenberger equation as a powerful and a very
versatile method for the study of materials with spin-orbit coupling, which
often provides a simpler and more intuitive picture compared to alternative
methods.

\subsection*{\href{http://arxiv.org/abs/2009.12317v1}{Anomalous Non-Hydrogenic Exciton Series in 2D Materials on High-$κ$  Dielectric Substrates}}
\subsubsection*{Anders C. Riis-Jensen, \dots, and Kristian S. Thygesen (2020-09-25)}
Engineering of the dielectric environment represents a powerful strategy to
control the electronic and optical properties of two-dimensional (2D) materials
without compromising their structural integrity. Here we show that the recent
development of high-$\kappa$ 2D materials present new opportunities for
dielectric engineering. By solving a 2D Mott-Wannier exciton model for WSe$_2$
on different substrates using a screened electron-hole interaction obtained
from first principles, we demonstrate that the exciton Rydberg series changes
qualitatively when the dielectric screening within the 2D semiconductor becomes
dominated by the substrate. In this regime, the distance dependence of the
screening is reversed and the effective screening increases with exciton
radius, which is opposite to the conventional 2D screening regime.
Consequently, higher excitonic states become underbound rather than overbound
as compared to the Hydrogenic Rydberg series. Finally, we derive a general
analytical expression for the exciton binding energy of the entire 2D Rydberg
series

\subsection*{\href{http://arxiv.org/abs/2009.12315v1}{Scientific instrument for creation of effective Cooper pair mass  spectroscopy}}
\subsubsection*{Todor M. Mishonov and Albert M. Varonov (2020-09-25)}
We describe electronic instruments for creation of effective Cooper pair
spectroscopy. The suggested spectroscopy requires study of electric field
effects on the surface of cleaved superconductors. The electronic instrument
reacquires low noise amplifier with 10$^6$ amplitude amplification which we
have formerly used for study of Johnson-Nyquist and Schottky noises. The
nonspecific amplifier is followed by high-Q tunable resonance filter based on
schematics of general impedance converter topology which is also and innovative
device. The work of the device is based on the Manhattan equation of
operational amplifier. After a final nonspecific amplification the total
amplification can exceed 10$^9$ and in such a way sub-nano-volt signals can be
reliably detected. In short the observation of new effects in condensed matter
physics leads to creation of new generation of electronic equipment.

\subsection*{\href{http://arxiv.org/abs/2009.12308v1}{Reduced Conformal Symmetry}}
\subsubsection*{Andreas Karch and Amir Raz (2020-09-25)}
We construct field theories in $2+1$ dimensions with multiple conformal
symmetries acting on only one of the spatial directions. These can be
considered a conformal extension to "subsystem scale invariances", borrowing
the language often used for fractons.

\subsection*{\href{http://arxiv.org/abs/2009.12304v1}{Dynamical Entanglement}}
\subsubsection*{Gilad Gour and Carlo Maria Scandolo (2020-09-25)}
Unlike the entanglement of quantum states, very little is known about the
entanglement of bipartite channels, called dynamical entanglement. Here we work
with the partial transpose of a superchannel, and use it to define computable
measures of dynamical entanglement, such as the negativity. We show that a
version of it, the max-logarithmic negativity, represents the exact asymptotic
dynamical entanglement cost. We discover a family of dynamical entanglement
measures that provide necessary and sufficient conditions for bipartite channel
simulation under local operations and classical communication and under
operations with positive partial transpose.

\subsection*{\href{http://arxiv.org/abs/2009.12302v1}{Non-thermal transport of energy driven by photoexcited carriers in  switchable solid states of GeTe}}
\subsubsection*{R. Gu, \dots, and P. Ruello (2020-09-25)}
Phase change alloys have seen widespread use from rewritable optical discs to
the present day interest in their use in emerging neuromorphic computing
architectures. In spite of this enormous commercial interest, the physics of
carriers in these materials is still not fully understood. Here, we describe
the time and space dependence of the coupling between photoexcited carriers and
the lattice in both the amorphous and crystalline states of one phase change
material, GeTe. We study this using a time-resolved optical technique called
picosecond acoustic method to investigate the \textit{in situ} thermally
assisted amorphous to crystalline phase transformation in GeTe. Our work
reveals a clear evolution of the electron-phonon coupling during the phase
transformation as the spectra of photoexcited acoustic phonons in the amorphous
($a$-GeTe) and crystalline ($\alpha$-GeTe) phases are different. In particular
and surprisingly, our analysis of the photoinduced acoustic pulse duration in
crystalline GeTe suggests that a part of the energy deposited during the
photoexcitation process takes place over a distance that clearly exceeds that
defined by the pump light skin depth. In the opposite, the lattice
photoexcitation process remains localized within that skin depth in the
amorphous state. We then demonstrate that this is due to supersonic diffusion
of photoexcited electron-hole plasma in the crystalline state. Consequently
these findings prove the existence of a non-thermal transport of energy which
is much faster than lattice heat diffusion.

\subsection*{\href{http://arxiv.org/abs/2009.12289v1}{Performance-based screening of porous materials for carbon capture}}
\subsubsection*{Stefano Brandani, \dots, and Lev Sarkisov (2020-09-25)}
Multiscale computational screening methods have been accelerating materials
discovery and technology deployment in many areas from batteries to alloys. In
this review, we focus on post-combustion carbon capture using adsorption in
porous materials. Prompted by the recent unprecedented developments in material
science, researchers in material engineering, molecular simulations, and
process modelling have been interested in finding the best porous materials for
carbon capture which would offer a less energy demanding alternative to the
current technologies. Recent efforts have been directed towards development of
new screening approaches where molecular-level simulation techniques are
combined with process modelling into performance-based multiscale screening
workflows. The idea of such workflows envisages being able to go from structure
of a porous material to equilibrium and transport properties, and eventually to
the performance of the material in the actual process, predicted by process
modelling and optimization. Development of these methods requires stepping into
a highly interdisciplinary field where scientists from very different
backgrounds should work together to address the multitude of technical
problems. The objective of this review is to facilitate this process. We
provide a complete and systematic overview of the methods and elements required
for the implementation of multiscale screening workflows, a comprehensive and
single source of references combining information about available materials
databases, state-of-the-art molecular simulation and process modelling tools,
and the full list of data and parameters required for performance-based
materials screening. We review recent developments, identify key existing
challenges, pose new questions, and propose directions for the future.

\subsection*{\href{http://arxiv.org/abs/2009.12288v1}{Star topology increases ballistic resistance in thin polymer films}}
\subsubsection*{Andrea Giuntoli, and Sinan Keten (2020-09-25)}
Polymeric films with greater impact and ballistic resistance are highly
desired for numerous applications, but molecular configurations that best
address this need remain subject to debate. We study the resistance to
ballistic impact of thin polymer films using coarse-grained molecular dynamics
simulations, investigating melts of linear polymer chains and star polymers
with varying number 2<=f<=16 and degree of polymerization 10<=M<=50 of the
arms. We show that increasing the number of arms f or the length of the arms M
both result in greater specific penetration energy within the parameter ranges
studied. Greater interpenetration of chains in stars with larger f allows
energy to be dissipated predominantly through rearrangement of the stars
internally, rather than chain sliding. During film deformation, stars with
large f show higher energy absorption rates soon after contact with the
projectile, whereas stars with larger M have a delayed response where
dissipation arises primarily from chain sliding, which results in significant
back face deformation. Our results suggest that stars may be advantageous for
tuning energy dissipation mechanisms of ultra-thin films. These findings set
the stage for a topology-based strategy for the design of impact-resistant
polymer films.

\subsection*{\href{http://arxiv.org/abs/2009.12287v1}{Large deviations for Markov processes with stochastic resetting :  analysis via the empirical density and flows or via excursions between resets}}
\subsubsection*{Cecile Monthus (2020-09-25)}
Markov processes with stochastic resetting towards the origin produce
non-equilibrium steady-states. Long dynamical trajectories can be thus analyzed
via the large deviations at level 2.5 for the joint probability of the
empirical density and the empirical flows, or via the large deviations of
semi-Markov processes for the empirical density of excursions between
consecutive resets. The general formalism is described for the three possible
frameworks, namely discrete-time/discrete-space Markov chains,
continuous-time/discrete-space Markov jump processes, and
continuous-time/continuous-space diffusion processes, and is illustrated with
examples based on the Sisyphus Random Walk.

\subsection*{\href{http://arxiv.org/abs/2009.12284v1}{Reply to a "Comment on 'Physics without determinism: Alternative  interpretations of classical physics' "}}
\subsubsection*{Flavio Del Santo and Nicolas Gisin (2020-09-25)}
In this short note we reply to a comment by Callegaro et al. [1]
(arXiv:2009.11709) that points out some weakness of the model of
indeterministic physics that we proposed in Ref. [2] (Physical Review A,
100(6), p.062107), based on what we named "finite information quantities"
(FIQs). While we acknowledge the merit of their criticism, we maintain that it
applies only to a concrete example that we discussed in [2], whereas the main
concept of FIQ remains valid and suitable for describing indeterministic
physical models. We hint at a more sophisticated way to define FIQs which,
taking inspiration from intuitionistic mathematics, would allow to overcome the
criticisms in [1].

\subsection*{\href{http://arxiv.org/abs/2009.12279v1}{An extensible lattice Boltzmann method for viscoelastic flows: complex  and moving boundaries in Oldroyd-B fluids}}
\subsubsection*{Michael Kuron, \dots, and Christian Holm (2020-09-25)}
Most biological fluids are viscoelastic, meaning that they have elastic
properties in addition to the dissipative properties found in Newtonian fluids.
Computational models can help us understand viscoelastic flow, but are often
limited in how they deal with complex flow geometries and suspended particles.
Here, we present a lattice Boltzmann solver for Oldroyd-B fluids that can
handle arbitrarily-shaped fixed and moving boundary conditions, which makes it
ideally suited for the simulation of confined colloidal suspensions. We
validate our method using several standard rheological setups, and additionally
study a single sedimenting colloid, also finding good agreement with
literature. Our approach can readily be extended to constitutive equations
other than Oldroyd-B. This flexibility and the handling of complex boundaries
holds promise for the study of microswimmers in viscoelastic fluids.

\subsection*{\href{http://arxiv.org/abs/2009.12278v1}{The Hybrid Topological Longitudinal Transmon Qubit}}
\subsubsection*{Alec Dinerstein, and Eugene F. Dumitrescu (2020-09-25)}
We introduce a new hybrid qubit consisting of a Majorana qubit interacting
with a transmon longitudinally coupled to a resonator. To do so, we equip the
longitudinal transmon qubit with topological quasiparticles, supported by an
array of heterostructure nanowires, and derive charge- and phase-based
interactions between the Majorana qubit and the resonator and transmon degrees
of freedom. Inspecting the charge coupling, we demonstrate that the Majorana
self-charging can be eliminated by a judicious choice of charge offset, thereby
maintaining the Majorana degeneracy regardless of the quasiparticles spatial
arrangement and parity configuration. We perform analytic and numerical
calculations to derive the effective qubit-qubit interaction elements and
discuss their potential utility for state readout and quantum error correction.
Further, we find that select interactions depend strongly on the overall
superconducting parity, which may provide a direct mechanism to characterize
deleterious quasiparticle poisoning processes.

\subsection*{\href{http://arxiv.org/abs/2009.12265v1}{Development of interatomic potential appropriate for simulation of  dislocation migration in fcc Fe}}
\subsubsection*{Mikhail I. Mendelev and Valery Borovikov (2020-09-25)}
Molecular dynamics (MD) simulation of dislocation migration requires
semi-empirical potentials of the interatomic interaction. While there are many
reliable semi-empirical potentials for the bcc Fe, the number of the available
potentials for the fcc is very limited. In the present study we tested three
EAM potentials for the fcc Fe (ABCH97 [Phil. Mag. A, 75, 713-732 (1997)], BCT13
[MSMSE 21, 085004 (2013)] and ZFS18 [J. Comp. Chem. 39, 2420-2431 (2018)]) from
literature. It was found that the ABCH97 potential does not provide that the
fcc phase is the most stable at any temperature. On the other hand, the fcc
phase is always more stable than the bcc phase for the BCT13, ZFS18 potentials.
The hcp phase is the most stable phase for the BCT13 potential at any
temperature. In order to fix these problems we developed two new EAM potentials
(MB1 and MB2). The fcc phase is still more stable than the bcc phase for the
MB1 potential but the MB2 potential provides that the bcc phase is the most
stable phase from the upper fcc-bcc transformation temperature, T\_gamma-delta,
to the melting temperature, Tm, and the fcc phase is the most stable phase
below T\_gamma-delta. This potential also leads to an excellent agreement with
the experimental data on the fcc elastic constants and reasonable stacking
fault energy which makes it the best potential for the simulation of the
dislocation migration in the fcc Fe among all semi-empirical potentials
considered in the present study. The MD simulation demonstrated that only the
ZFS18, MB1 and MB2 potentials are actually suitable for the simulation of the
dislocation migration in the fcc Fe. They lead to the same orders of magnitude
for the dislocation velocities and all of them show that the edge dislocation
is faster than the screw dislocation. However, the actual values of the
dislocation velocities do depend on the employed semi-empirical potential.

\subsection*{\href{http://arxiv.org/abs/2009.12262v1}{Strain-engineering of the charge and spin-orbital interactions in  Sr2IrO4}}
\subsubsection*{Eugenio Paris, \dots, and Thorsten Schmitt (2020-09-25)}
In the high spin-orbit coupled Sr2IrO4, the high sensitivity of the ground
state to the details of the local lattice structure shows a large potential for
the manipulation of the functional properties by inducing local lattice
distortions. We use epitaxial strain to modify the Ir-O bond geometry in
Sr2IrO4 and perform momentum-dependent Resonant Inelastic X-ray Scattering
(RIXS) at the metal and at the ligand sites to unveil the response of the low
energy elementary excitations. We observe that the pseudospin-wave dispersion
for tensile-strained Sr2IrO4 films displays large softening along the [h,0]
direction, while along the [h,h] direction it shows hardening. This evolution
reveals a renormalization of the magnetic interactions caused by a
strain-driven crossover from anisotropic to isotropic interactions between the
magnetic moments. Moreover, we detect dispersive electron-hole pair excitations
which shift to lower (higher) energies upon compressive (tensile) strain,
manifesting a reduction (increase) in the size of the charge gap. This behavior
shows an intimate coupling between charge excitations and lattice distortions
in Sr2IrO4, originating from the modified hopping elements between the t2g
orbitals. Our work highlights the central role played by the lattice degrees of
freedom in determining both the pseudospin and charge excitations of Sr2IrO4
and provides valuable information towards the control of the ground state of
complex oxides in the presence of high spin-orbit coupling.

\subsection*{\href{http://arxiv.org/abs/2009.12259v1}{Gadolinium as a single atom catalyst in a single molecule magnet}}
\subsubsection*{Aram Kostanyan, \dots, and Thomas Greber (2020-09-25)}
Endohedral fullerenes are perfect nanolaboratories for the study of
magnetism. The substitution of a diamagnetic scandium atom in Dy2ScN@C80 with
gadolinium decreases the stability of a given magnetization and demonstrates Gd
to act as a single atom catalyst that accelerates the reaching of thermal
equilibrium. X-ray magnetic circular dichroism at the M4,5 edges of Gd and Dy
shows that the Gd magnetic moment follows the sum of the external and the
dipolar magnetic field of the two Dy ions and compared to Dy2ScN@C80 a lower
exchange barrier is found between the ferromagnetic and the antiferromagnetic
Dy configuration. The Arrhenius equilibration barrier as obtained from
superconducting quantum interference device magnetometry is more than one order
of magnitude larger, though a much smaller prefactor imposes faster
equilibration in Dy2GdN@C80. This sheds light on the importance of the angular
momentum balance in magnetic relaxation.

\subsection*{\href{http://arxiv.org/abs/2009.12249v1}{A numerical extrapolation method for complex conductivity of disordered  metals}}
\subsubsection*{S. Kern, \dots, and M. Grajcar (2020-09-25)}
Recently, quantum corrections to optical conductivity of disordered metals up
to the UV region were observed. Although this increase of conductivity with
frequency, also called anti-Drude behaviour, should disappear at the electron
collision frequency, such transition has never been observed or described
theoretically. Thus, the knowledge of optical conductivity in a wide frequency
range is of great interest. It is well known that the extrapolation of complex
conductivity is ill-posed - a solution of the analytic continuation problem is
not unique for data with finite accuracy. However, we show that assuming
physically appropriate properties of the searched function $\sigma(\omega)$,
such as: symmetry, smoothness, and asymptotic solution for low and high
frequencies, one can significantly restrict the set of solutions. We present a
simple numerical method utilizing the radial basis function approximation and
simulated annealing, which reasonably extrapolates the optical conductivity
from visible frequency range down to far infrared and up to ultraviolet region.
Extrapolation obtained on MoC and NbN thin films was checked by transmission
measurement across a wide frequency range.

\subsection*{\href{http://arxiv.org/abs/2009.12239v1}{Fast quantum imaginary time evolution}}
\subsubsection*{Kok Chuan Tan (2020-09-25)}
A fast implementation of the quantum imaginary time evolution (QITE)
algorithm called Fast QITE is proposed. The algorithmic cost of QITE typically
scales exponentially with the number of particles it nontrivially acts on in
each Trotter step. In contrast, a Fast QITE implementation reduces this to only
a linear scaling. It is shown that this speed up leads to a quantum advantage
when sampling diagonal elements of a matrix exponential, which cannot be
achieved using the standard implementation of the QITE algorithm. Finally the
cost of implementing Fast QITE for finite temperature simulations is also
discussed.

\subsection*{\href{http://arxiv.org/abs/2009.12233v1}{Magnetic-field-driven topological phase transition and topological Hall  effect in AF-skyrmion lattice}}
\subsubsection*{M. Tomé and H. D. Rosales (2020-09-25)}
The topological Hall effect (THE), given by a composite of electric and
topologically non-trivial spin texture is commonly observed in magnetic
skyrmion crystals. Here we present a study of the THE of electrons coupled to
antiferromagnetic Skyrmion lattices (AF-SkX). We show that, in the strong Hund
coupling limit, topologically non-trivial phases emerge at specific fillings.
Interestingly, at low filling an external field controlling the magnetic
texture, drives the system from a conventional insulator phase to a phase
exhibiting THE. Such behavior suggests the occurrence of a topological
transition which is confirmed by a closing of the bulk-gap that is followed by
its reopening, appearing simultaneously with a single pair of helical edge
states. This transition is further verified by the calculation of the the Chern
numbers and Berry curvature. We also compute a variety of observables in order
to quantify the THE, namely: Hall conductivity (HC) and the orbital
magnetization (OM) of electrons moving in the AF-SkX texture.

\subsection*{\href{http://arxiv.org/abs/2009.12194v1}{Exciton-polariton mediated interaction between two nitrogen-vacancy  color centers in diamond using two-dimensional transition metal  dichalcogenides}}
\subsubsection*{J. C. G. Henriques, and N. M. R. Peres (2020-09-25)}
In this paper, starting from a quantum master equation, we discuss the
interaction between two negatively charged Nitrogen-vacancy color centers in
diamond via exciton-polaritons propagating in a two-dimensional transition
metal dichalcogenide layer in close proximity to a diamond crystal. We focus on
the optical 1.945 eV transition and model the Nitrogen-vacancy color centers as
two-level (artificial) atoms. We find that the interaction parameters and the
energy levels renormalization constants are extremely sensitive to the distance
of the Nitrogen-vacancy centers to the transition metal dichalcogenide layer.
Analytical expressions are obtained for the spectrum of the exciton-polaritons
and for the damping constants entering the Lindblad equation. The conditions
for occurrence of exciton mediated superradiance are discussed.

\subsection*{\href{http://arxiv.org/abs/2009.12182v1}{Influence of lateral confinement on granular flows: comparison between  shear-driven and gravity-driven flows}}
\subsubsection*{Patrick Richard, \dots, and Renaud Delannay (2020-09-25)}
The properties of confined granular flows are studied through discrete
numerical simulations. Two types of flows with different boundaries are
compared: (i) gravity-driven flows topped with a free surface and over a base
where erosion balances accretion (ii) shear-driven flows with a constant
pressure applied at their top and a bumpy bottom moving at constant velocity.
In both cases we observe shear localization over or/and under a creep zone. We
show that, although the different boundaries induce different flow properties
(e.g. shear localization of transverse velocity profiles), the two types of
flow share common properties like (i) a power law relation between the granular
temperature and the shear rate (whose exponent varies from 1 for dense flows to
2 for dilute flows) and (ii) a weakening of friction at the sidewalls which
gradually decreases with the depth within the flow.

\subsection*{\href{http://arxiv.org/abs/2009.12163v1}{First-principles modeling of plasmons in aluminum under ambient and  extreme conditions}}
\subsubsection*{Kushal Ramakrishna, \dots, and Jan Vorberger (2020-09-25)}
The numerical modeling of plasmon behavior is crucial for an accurate
interpretation of inelastic scattering diagnostics in many experiments. We
highlight the utility of linear-response time-dependent density functional
theory (LR-TDDFT) as an appropriate first-principles framework for a consistent
modeling of plasmon properties. We provide a comprehensive analysis of plasmons
from ambient throughout warm dense conditions and assess typical properties
such as the dynamical structure factor, the plasmon dispersion, and the plasmon
width. We compare them with experimental measurements in aluminum accessible
via x-ray Thomson scattering and with other dielectric models such as the
Lindhard model, the Mermin approach based on parametrized collision
frequencies, and the dielectric function obtained using static local field
corrections of the uniform electron gas parametrized from path integral Monte
Carlo simulations both at the ground state and at finite temperature. We
conclude with the remark that the common practice of extracting and employing
plasmon dispersion relations and widths is an insufficient procedure to capture
the complicated physics contained in the dynamic structure factor in its full
breadth.

\subsection*{\href{http://arxiv.org/abs/2009.12162v1}{Active particles with polar alignment in ring-shaped confinement}}
\subsubsection*{Zahra Fazli and Ali Naji (2020-09-25)}
We study steady-state properties of a suspension of active, nonchiral and
chiral, Brownian particles with polar alignment and steric interactions
confined within a ring-shaped (annulus) confinement in two dimensions.
Exploring possible interplays between polar interparticle alignment, geometric
confinement and the surface curvature, being incorporated here on minimal
levels, we report a surface-population reversal effect, whereby active
particles migrate from the outer concave boundary of the annulus to accumulate
on its inner convex boundary. This contrasts the conventional picture, implying
stronger accumulation of active particles on concave boundaries relative to the
convex ones. The population reversal is caused by both particle alignment and
surface curvature, disappearing when either of these factors is absent. We
explore the ensuing consequences for the chirality-induced current and swim
pressure of active particles and analyze possible roles of system parameters,
such as the mean number density of particles and particle self-propulsion,
chirality and alignment strengths.

\subsection*{\href{http://arxiv.org/abs/2009.12151v1}{Two-dimensional simulated tempering for the isobaric-isothermal ensemble  with fast on-the-fly weight determination}}
\subsubsection*{Hiromune Wada and Yuko Okamoto (2020-09-25)}
We propose a method to extend the fast on-the-fly weight determination scheme
for simulated tempering to two-dimensional space including not only temperature
but also pressure. During the simulated tempering simulation, weight parameters
for temperature-update and pressure-update are self-updated independently
according to the trapezoidal rule. In order to test the effectiveness of the
algorithm, we applied our proposed method to a peptide, chignolin, in explicit
water. After setting all weight parameters to zero, the weight parameters were
quickly determined during the simulation. The simulation realised a uniform
random walk in the entire temperature-pressure space.

\subsection*{\href{http://arxiv.org/abs/2009.12147v1}{Many-body collisional dynamics of impurities injected into a double-well  trapped Bose-Einstein condensate}}
\subsubsection*{Friethjof Theel, \dots, and Peter Schmelcher (2020-09-25)}
We unravel the many-body dynamics of a harmonically trapped impurity
colliding with a bosonic medium confined in a double-well upon quenching the
initially displaced harmonic trap to the center of the double-well. We reveal
that the emerging correlation dynamics crucially depends on the impurity-medium
interaction strength allowing for a classification into different dynamical
response regimes. For strong attractive impurity-medium couplings the impurity
is bound to the bosonic bath, while for intermediate attractions it undergoes
an effective tunneling. In the case of weak attractive or repulsive couplings
the impurity penetrates the bosonic bath and performs a dissipative oscillatory
motion. Further increasing the impurity-bath repulsion results in the pinning
of the impurity between the density peaks of the bosonic medium, a phenomenon
that is associated with a strong impurity-medium entanglement. For strong
repulsions the impurity is totally reflected by the bosonic medium. To unravel
the underlying microscopic excitation processes accompanying the dynamics we
employ an effective potential picture. We extend our results to the case of two
bosonic impurities and demonstrate the existence of a qualitatively similar
impurity dynamics.

\subsection*{\href{http://arxiv.org/abs/2009.12138v2}{Fermi arcs and surface criticality in dirty Dirac materials}}
\subsubsection*{Eric Brillaux and Andrei A. Fedorenko (2020-09-25)}
We study the effects of disorder on semi-infinite Weyl and Dirac semimetals
where the presence of a boundary leads to the formation of either Fermi
arcs/rays or Dirac surface states. Using a local version of the self-consistent
Born approximation, we calculate the profile of the local density of states and
the surface group velocity. This allows us to explore the full phase diagram as
a function of boundary conditions and disorder strength. While in all cases we
recover the sharp criticality in the bulk, we unveil a critical behavior at the
surface of Dirac semimetals, which is smoothed out by Fermi arcs in Weyl
semimetals.

\subsection*{\href{http://arxiv.org/abs/2009.12137v1}{Polaronic Contributions to Friction in a Manganite Thin Film}}
\subsubsection*{Niklas A. Weber, \dots, and Prof. Cynthia A. Volkert (2020-09-25)}
Despite the huge importance of friction in regulating movement in all natural
and technological processes, the mechanisms underlying dissipation at a sliding
contact are still a matter of debate. Attempts to explain the dependence of
measured frictional losses at nanoscale contacts on the electronic degrees of
freedom of the surrounding materials have so far been controversial. Here, it
is proposed that friction can be explained by considering damping of stick-slip
pulses in a sliding contact. Based on friction force microscopy studies of
La$_{(1-x)}$Sr$_x$MnO$_3$ films at the ferromagnetic-metallic to
paramagnetic-polaronic conductor phase transition, it is confirmed that the
sliding contact generates thermally-activated slip pulses in the nanoscale
contact, and argued that these are damped by direct coupling into phonon bath.
Electron-phonon coupling leads to the formation of Jahn-Teller polarons and a
clear increase in friction in the high temperature phase. There is no evidence
for direct electronic drag on the atomic force microscope tip nor any
indication of contributions from electrostatic forces. This intuitive scenario,
that friction is governed by the damping of surface vibrational excitations,
provides a basis for reconciling controversies in literature studies as well as
suggesting possible tactics for controlling friction.

\subsection*{\href{http://arxiv.org/abs/2009.12126v1}{Numerical Study of the Thermodynamic Uncertainty Relation for the  KPZ-Equation}}
\subsubsection*{Oliver Niggemann and Udo Seifert (2020-09-25)}
A general framework for the field-theoretic thermodynamic uncertainty
relation was recently proposed and illustrated with the $(1+1)$ dimensional
Kardar-Parisi-Zhang equation. In the present paper, the analytical results
obtained there in the weak coupling limit are tested via a direct numerical
simulation of the KPZ equation with good agreement. The accuracy of the
numerical results varies with the respective choice of discretization of the
KPZ non-linearity. Whereas the numerical simulations strongly support the
analytical predictions, an inherent limitation to the accuracy of the
approximation to the total entropy production is found. In an analytical
treatment of a generalized discretization of the KPZ non-linearity, the origin
of this limitation is explained and shown to be an intrinsic property of the
employed discretization scheme.

\subsection*{\href{http://arxiv.org/abs/2009.12118v1}{Signing Information in the Quantum Era}}
\subsubsection*{K. Longmate, \dots, and R. J. Young (2020-09-25)}
Signatures are primarily used as a mark of authenticity, to demonstrate that
the sender of a message is who they claim to be. In the current digital age,
signatures underpin trust in the vast majority of information that we exchange,
particularly on public networks such as the internet. However, schemes for
signing digital information which are based on assumptions of computational
complexity are facing challenges from advances in mathematics, the capability
of computers, and the advent of the quantum era. Here we present a review of
digital signature schemes, looking at their origins and where they are under
threat. Next, we introduce post-quantum digital schemes, which are being
developed with the specific intent of mitigating against threats from quantum
algorithms whilst still relying on digital processes and infrastructure.
Finally, we review schemes for signing information carried on quantum channels,
which promise provable security metrics. Signatures were invented as a
practical means of authenticating communications and it is important that the
practicality of novel signature schemes is considered carefully, which is kept
as a common theme of interest throughout this review.

\subsection*{\href{http://arxiv.org/abs/2009.12104v1}{Macroscopic Quantum Electrodynamics and Density Functional Theory  Approaches to Dispersion Interactions between Fullerenes}}
\subsubsection*{Saunak Das, \dots, and Martin Presselt (2020-09-25)}
The processing and material properties of commercial organic semiconductors,
for e.g. fullerenes is largely controlled by their precise arrangements,
specially intermolecular symmetries, distances and orientations, more
specifically, molecular polarisabilities. These supramolecular parameters
heavily influence their electronic structure, thereby determining molecular
photophysics and therefore dictating their usability as n-type semiconductors.
In this article we evaluate van der Waals potentials of a fullerene dimer model
system using two approaches: a) Density Functional Theory and, b) Macroscopic
Quantum Electrodynamics, which is particularly suited for describing long-range
van der Waals interactions. Essentially, we determine and explain the model
symmetry, distance and rotational dependencies on binding energies and spectral
changes. The resultant spectral tuning is compared using both methods showing
correspondence within the constraints placed by the different model
assumptions. We envision that the application of macroscopic methods and
structure/property relationships laid forward in this article will find use in
fundamental supramolecular electronics.

\subsection*{\href{http://arxiv.org/abs/2009.12089v1}{External control of GaN band bending using phosphonate self-assembled  monolayers}}
\subsubsection*{T. Auzelle, \dots, and S. Fernández-Garrido (2020-09-25)}
We report on the optoelectronic properties of GaN$(0001)$ and $(1\bar{1}00)$
surfaces after their functionalization with phosphonic acid derivatives. To
analyze the possible correlation between the acid's electronegativity and the
GaN surface band bending, two types of phosphonic acids, n-octylphosphonic acid
(OPA) and 1H,1H,2H,2H-perfluorooctanephosphonic acid (PFOPA), are grafted on
oxidized GaN$(0001)$ and GaN$(1\bar{1}00)$ layers as well as on GaN nanowires.
The resulting hybrid inorganic/organic heterostructures are investigated by
X-ray photoemission and photoluminescence spectroscopy. The GaN work function
is changed significantly by the grafting of phosphonic acids, evidencing the
formation of dense self-assembled monolayers. Regardless of the GaN surface
orientation, both types of phosphonic acids significantly impact the GaN
surface band bending. A dependence on the acids' electronegativity is, however,
only observed for the oxidized GaN$(1\bar{1}00)$ surface, indicating a
relatively low density of surface states and a favorable band alignment between
the surface oxide and acids' electronic states. Regarding the optical
properties, the covalent bonding of PFOPA and OPA on oxidized GaN layers and
nanowires significantly affect their external quantum efficiency, especially in
the nanowire case due to the large surface-to-volume ratio. The variation in
the external quantum efficiency is related to the modication of both the
internal electric fields and surface states. These results demonstrate the
potential of phosphonate chemistry for the surface functionalization of GaN,
which could be exploited for selective sensing applications.

\subsection*{\href{http://arxiv.org/abs/2009.12083v2}{Unidirectional Quantum Transport in Optically Driven $V$-type Quantum  Dot Chains}}
\subsubsection*{Oliver Kaestle, \dots, and Alexander Carmele (2020-09-25)}
We predict a mechanism for achieving complete population inversion in a
continuously driven InAs/GaAs semiconductor quantum dot featuring $V$-type
transitions. This highly nonequilibrium steady state is enabled by the
interplay between $V$-type interband transitions and a non-Markovian
decoherence mechanism, introduced by acoustic phonons. The population trapping
mechanism is generalized to a chain of coupled emitters. Exploiting the
population inversion, we predict unidirectional excitation transport from one
end of the chain to the other without external bias, independent of the unitary
interdot coupling mechanism.

\subsection*{\href{http://arxiv.org/abs/2009.12082v2}{Probing the transition from dislocation jamming to pinning by machine  learning}}
\subsubsection*{Henri Salmenjoki, and Mikko J. Alava (2020-09-25)}
Collective motion of dislocations is governed by the obstacles they
encounter. In pure crystals, dislocations form complex structures as they
become jammed by their anisotropic shear stress fields. On the other hand,
introducing disorder to the crystal causes dislocations to pin to these
impeding elements and, thus, leads to a competition between
dislocation-dislocation and dislocation-disorder interactions. Previous studies
have shown that, depending on the dominating interaction, the mechanical
response and the way the crystal yields change.
  Here we employ three-dimensional discrete dislocation dynamics simulations
with varying density of fully coherent precipitates to study this phase
transition $-$ from jamming to pinning $-$ using unsupervised machine learning.
By constructing descriptors characterizing the evolving dislocation
configurations during constant loading, a confusion algorithm is shown to be
able to distinguish the systems into two separate phases. These phases agree
well with the observed changes in the relaxation rate during the loading. Our
results also give insights on the structure of the dislocation networks in the
two phases.

\subsection*{\href{http://arxiv.org/abs/2009.12079v1}{Resource reduction for simultaneous generation of two types of  continuous variable nonclassical states}}
\subsubsection*{Long Tian, \dots, and Kunchi Peng (2020-09-25)}
We demonstrate experimentally the simultaneous generation and detection of
two types of continuous variable nonclassical states from one type-0
phase-matching optical parametric amplification (OPA) and subsequent two ring
filter cavities (RFCs). The output field of the OPA includes the baseband
{\omega}0 and sideband modes {\omega}0+/-n{\omega}f subjects to the cavity
resonance condition, which are separated by two cascaded RFCs. The first RFC
resonates with half the pump wavelength {\omega}0 and the transmitted baseband
component is a squeezed state. The reflected fields of the first RFC, including
the sideband modes {\omega}0+/-{\omega}f, are separated by the second RFC,
construct Einstein-Podolsky-Rosen entangled state. All freedoms, including the
filter cavities for sideband separation and relative phases for the
measurements of these sidebands, are actively stabilized. The noise variance of
squeezed states is 10.2 dB below the shot noise limit (SNL), the correlation
variances of both quadrature amplitude-sum and quadrature phase-difference for
the entanglement state are 10.0 dB below the corresponding SNL.

\subsection*{\href{http://arxiv.org/abs/2009.12077v1}{A new mechanism for gain in time dependent media}}
\subsubsection*{J. B. Pendry, and P. A. Huidobro (2020-09-25)}
Time dependent systems do not in general conserve energy invalidating much of
the theory developed for static systems and turning our intuition on its head.
This is particularly acute in luminal space time crystals where the structure
moves at or close to the velocity of light. Conventional Bloch wave theory no
longer applies, energy grows exponentially with time, and a new perspective is
required to understand the phenomenology. In this letter we identify a new
mechanism for pulse amplification: the compression of lines of force that are
nevertheless conserved in number.

\subsection*{\href{http://arxiv.org/abs/2009.12073v1}{Survey of temperature dependence of the damping parameter in the  ferrimagnet Gd$_3$Fe$_5$O$_{12}$}}
\subsubsection*{Isaac Ng, and Qiming Shao (2020-09-25)}
The damping parameter ${\alpha}_{\text{FM}}$ in ferrimagnets defined by
following the conventional practice for ferromagnets is known to be strongly
temperature dependent and diverge at the angular momentum compensation
temperature, where the net angular momentum vanishes. However, recent
theoretical and experimental developments suggest that the damping parameter
can be defined in such a way, which we denote by ${\alpha}_{\text{FiM}}$, that
it is free of the diverging anomaly at the angular momentum compensation point
and is little dependent of temperature. To further understand the temperature
dependence of the damping parameter in ferrimagnets, we analyze several data
sets from literature for gadolinium iron garnet (Gd$_3$Fe$_5$O$_{12}$) by using
the two different definitions of the damping parameter. Using two methods to
estimate the individual sublattice magnetizations, which yield results
consistent with each other, we found that in all the used data sets, the
damping parameter ${\alpha}_{\text{FiM}}$ does not increase at the angular
compensation temperature and shows no anomaly whereas the conventionally
defined ${\alpha}_{\text{FM}}$ is strongly dependent on the temperature.

\subsection*{\href{http://arxiv.org/abs/2009.12071v1}{Mott transition and electronic excitation of the Gutzwiller wavefunction}}
\subsubsection*{Masanori Kohno (2020-09-25)}
The Mott transition is usually considered as resulting from the divergence of
the effective mass of the quasiparticle in the Fermi-liquid theory; the
dispersion relation around the Fermi level is considered to become flat towards
the Mott transition. Here, to clarify the characterization of the Mott
transition under the assumption of a Fermi-liquid-like ground state, the
electron-addition excitation from the Gutzwiller wavefunction in the $t$-$J$
model is investigated on a chain, ladder, square lattice, and bilayer square
lattice in the single-mode approximation using a Monte Carlo method. The
numerical results demonstrate that an electronic mode that is continuously
deformed from a non-interacting band at zero electron density loses its
spectral weight and gradually disappears towards the Mott transition. It
exhibits essentially the magnetic dispersion relation shifted by the Fermi
momentum in the small-doping limit as indicated by recent studies for the
Hubbard and $t$-$J$ models, even if the ground state is assumed to be a
Fermi-liquid-like state exhibiting gradual disappearance of the quasiparticle
weight. This implies that, rather than as the divergence of the effective mass
or disappearance of the carrier density that is expected in conventional
single-particle pictures, the Mott transition can be better understood as
freezing of the charge degrees of freedom while the spin degrees of freedom
remain active, even if the ground state is like a Fermi liquid.

\subsection*{\href{http://arxiv.org/abs/2009.12060v1}{Note: Quantitative approximation scheme of density of states near  jamming}}
\subsubsection*{Harukuni Ikeda and Masanari Shimada (2020-09-25)}
The vibrational density of states $D(\omega)$ plays a central role to
characterize the low-temperature properties of solids. Here, we propose a
simple yet accurate mean field like approximation to calculate $D(\omega)$ of
harmonic spheres near the jamming transition point. We compare our results with
previous numerical simulations in several spatial dimensions $d=3$, $5$, and
$9$. Near the jamming transition point, we find a good agreement even in $d=3$,
and obtain better agreements in larger $d$, suggesting that our theory may be
exact in the limit of large spatial dimensions.

\subsection*{\href{http://arxiv.org/abs/2009.12058v1}{Conserved current of nonconserved quantities}}
\subsubsection*{Cong Xiao and Qian Niu (2020-09-25)}
We provide a unified semiclassical theory for the conserved current of
locally nonconserved quantities. The attained current vanishes and circulates
in uniform and nonuniform equilibrium systems, respectively, thus can be used
to characterize the corresponding orbital magnetization and transport of
nonconserved quantities. We manifest the theory in two entirely different
physical contexts: the spin transport of Bloch electrons and the charge orbital
magnetization in superconductors. Several longstanding puzzles that enormously
limit the playground of the conserved spin current of electrons are solved,
paving the way for material related studies. A Berry phase formula is
established for the orbital magnetization in unconventional pairing
superconductors, consisting of both the localized and global charge circuits of
bogolons.

\subsection*{\href{http://arxiv.org/abs/2009.12050v1}{Viscosity of Cohesive Granular Matter}}
\subsubsection*{Matthew Macaulay and Pierre Rognon (2020-09-25)}
Cohesive granular materials such as wet sand, snow, and powders can flow like
a viscous liquid. However, the elementary mechanisms of momentum transport in
such athermal particulate fluids are elusive. As a result, existing models for
cohesive granular viscosity remain phenomenological and debated. Here we use
discrete element simulations of plane shear flows to measure the viscosity of
cohesive granular materials, while tuning the intensity of inter-particle
adhesion. We establish that two adhesion-related, dimensionless numbers control
their viscosity. These numbers compare the force and energy required to break a
bond to the characteristic stress and kinetic energy in the flow. This
progresses the commonly accepted view that only one dimensionless number could
control the effect of adhesion. The resulting scaling law captures strong,
non-Newtonian variations in viscosity, unifying several existing viscosity
models. We then directly link these variations in viscosity to adhesion-induced
modifications in the flow micro-structure and contact network. This analysis
reveals the existence of two modes of momentum transport, involving either
grain micro-acceleration or balanced contact forces, and shows that adhesion
only affects the later. This advances our understanding of rheological models
for granular materials and other soft materials such as emulsions and
suspensions, which may also involve inter-particle adhesive forces.

\subsection*{\href{http://arxiv.org/abs/2009.12048v1}{Magnetoactive elastomer based on superparamagnetic nanoparticles with  Curie point close to room temperature}}
\subsubsection*{Yu. I. Dzhezherya, \dots, and G. G. Levchenko (2020-09-25)}
A magnetoactive elastomer (MAE) consisting of single-domain La0.8Ag0.2Mn1.2O3
nanoparticles with a Curie temperature close to room temperature (TC = 308 K)
in a silicone matrix has been prepared and comprehensively studied. It has been
found that at room temperature and above, MAE particles are magnetized
superparamagnetically with a low coercivity below 10 Oe, and the influence of
magnetic anisotropy on the appearance of a torque is justified. A coupling
between magnetization and magnetoelasticity has been also established. The
mechanisms of the appearance of magnetoelasticity, including the effect of MAE
rearrangement and MAE compression by magnetized particles, have been revealed.
It has been found that the magnetoelastic properties of MAE have critical
features near TC. The magnetoelastic properties of MAE disappear at T > TC and
are restored at T < TC. This makes it possible to use MAE at room temperature
as a smart material for devices with self-regulating magnetoelastic properties.

\subsection*{\href{http://arxiv.org/abs/2009.12036v1}{High-order Dirac and Weyl points in screw-symmetric materials}}
\subsubsection*{Peng-Jen Chen, and Ting-Kuo Lee (2020-09-25)}
High-order topological charge is of intensive interest in the field of
topological matters. In real materials, cubic Dirac point is rare and the
chiral charge of one Weyl point (WP) is generally limited to |C|<= 3 for
spin-1/2 electronic systems. In this work, we argue that a cubic Dirac point
can result in one quadruple WP (|C| = 4 with double band degeneracy) when
time-reversal symmetry is broken, provided that this cubic Dirac point is away
from the high-symmetry points and involves coupling of eight bands, rather than
four bands that were thought to be sufficient to describe a Dirac point. The
eight-band manifold can be realized in materials with screw symmetry. Near the
zone boundary along the screw axis, the folded bands are coupled to their
"parent" bands, resulting in doubling dimension of the Hilbert space. Indeed,
in \epsilon-TaN (space group 194 with screw symmetry) we find a quadruple WP
when applying a Zeeman field along the screw axis. This quadruple WP away from
high symmetry points is distinct from highly degenerate nodes at the
high-symmetry points already reported. We further find that such a high chiral
charge might be related to the parity fixing of bands with high degeneracy,
which in turn alters the screw eigenvalues and the resulted chiral charge.

\subsection*{\href{http://arxiv.org/abs/2009.12032v1}{Bounding the finite-size error of quantum many-body dynamics simulations}}
\subsubsection*{Zhiyuan Wang, and Kaden R. A. Hazzard (2020-09-25)}
Finite-size errors, the discrepancy between an observable in a finite-size
system and in the thermodynamic limit, are ubiquitous in numerical simulations
of quantum many body systems. Although a rough estimate of these errors can be
obtained from a sequence of finite-size results, a strict, quantitative bound
on the magnitude of finite-size error is still missing. Here we derive rigorous
upper bounds on the finite-size error of local observables in real time quantum
dynamics simulations initiated from a product state and evolved under a general
Hamiltonian in arbitrary spatial dimension. In locally interacting systems with
a finite local Hilbert space, our bound implies $ |\langle
\hat{S}(t)\rangle_L-\langle \hat{S}(t)\rangle_\infty|\leq C(2v t/L)^{cL-\mu}$,
with $v$ the Lieb-Robinson (LR) speed and $C, c,\mu $ constants independent of
$L$ and $t$, which we compute explicitly. For periodic boundary conditions
(PBC), the constant $c$ is twice as large as that for open boundary conditions
(OBC), suggesting that PBC have smaller finite-size error than OBC at early
times. Our bounds are practically useful in determining the validity of
finite-size results, as we demonstrate in simulations of the one-dimensional
quantum Ising and Fermi-Hubbard models.

\subsection*{\href{http://arxiv.org/abs/2009.13263v1}{A simple practical quantum bit commitment protocol}}
\subsubsection*{Muqian Wen (2020-09-25)}
We proposed a practical quantum bit commitment protocol that possibly
requires less technological limitations on non-demolition measurements and
long-term quantum memories and moves closer to be unconditionally secure.

\subsection*{\href{http://arxiv.org/abs/2009.12026v1}{Entanglement-Assisted Absorption Spectroscopy}}
\subsubsection*{Haowei Shi, \dots, and Quntao Zhuang (2020-09-25)}
Spectroscopy is an important tool for probing the properties of materials,
chemicals and biological samples. We design a practical transmitter-receiver
system that exploits entanglement to achieve a provable quantum advantage over
all spectroscopic schemes based on classical sources. To probe the absorption
spectra, modelled as pattern of transmissivities among different frequency
modes, we employ broad-band signal-idler pairs in two-mode squeezed vacuum
states. At the receiver side, we apply photodetection after optical parametric
amplification. Finally, we perform a maximal-likehihood decision test on the
measurement results, achieving orders-of-magnitude-lower error probability than
the optimum classical systems in various examples, including `wine-tasting' and
`drug-testing' where real molecules are considered. In detecting the presence
of an absorption line, our quantum scheme achieves the optimum performance
allowed by quantum mechanics. The quantum advantage in our system is robust
against noise and loss, which makes near-term experimental demonstration
possible.

\subsection*{\href{http://arxiv.org/abs/2009.12017v1}{Ising Matter as Bulky: An Identity Described by Many Models}}
\subsubsection*{Martin H. Krieger (2020-09-25)}
The two-dimensional Ising model of a ferromagnet allows for many ways of
computing its partition function and other properties. Each way reveals
surprising features of what we might call Ising Matter. Moreover, the various
ways would appear to analogize with the mathematical threefold analogy of
analysis, algebra, and arithmetic, due to R. Dedekind and H. Weber, 1882, and
more recently described by A. Weil.

\subsection*{\href{http://arxiv.org/abs/2009.12016v1}{Legerdemain in Mathematical Physics: Structure, Tricks, and Lacunae in  Derivations of the Partition Function of the Two-Dimensional Ising Model and  in Proofs of The Stability of Matter}}
\subsubsection*{Martin H. Krieger (2020-09-25)}
We review various derivations of the partition function of the
two-dimensional Ising Model of ferromagnetism and proofs of the stability of
matter, paying attention to passages where there would appear to be a lacuna
between steps or where the structure of the argument is not so straightforward.
Authors cannot include all the intermediate steps, but sometimes most readers
and especially students will be mystified by such a transition. Moreover,
careful consideration of such lacunae points to interesting physics and not
only mathematical technology. Also, when reading the original papers, the
structure of the physics argument may be buried by the technical moves.
Improvements in the derivations, in subsequent papers by others, may well be
clearer and more motivated. But, there is remarkably little written and
published about how to read some of the original papers, and the subsequent
ones, yet students and their teachers would often benefit from such guidance. I
should note that much of the discussion below will benefit from having those
papers in front of you.

\subsection*{\href{http://arxiv.org/abs/2009.12006v1}{Gate-tunable cross-plane heat dissipation in single-layer transition  metal dichalcogenides}}
\subsubsection*{Zhun-Yong Ong, \dots, and Yong-Wei Zhang (2020-09-25)}
Efficient heat dissipation to the substrate is crucial for optimal device
performance in nanoelectronics. We develop a theory of electronic thermal
boundary conductance (TBC) mediated by remote phonon scattering for the
single-layer transition metal dichalcogenide (TMD) semiconductors MoS$_{2}$ and
WS$_{2}$, and model their electronic TBC with different dielectric substrates
(SiO$_{2}$, HfO$_{2}$ and Al$_{2}$O$_{3}$). Our results indicate that the
electronic TBC is strongly dependent on the electron density, suggesting that
it can be modulated by the gate electrode in field-effect transistors, and this
effect is most pronounced with Al$_{2}$O$_{3}$. Our work paves the way for the
design of novel thermal devices with gate-tunable cross-plane heat-dissipative
properties.

\subsection*{\href{http://arxiv.org/abs/2009.12000v1}{SeQUeNCe: A Customizable Discrete-Event Simulator of Quantum Networks}}
\subsubsection*{Xiaoliang Wu, \dots, and Martin Suchara (2020-09-25)}
Recent advances in quantum information science enabled the development of
quantum communication network prototypes and created an opportunity to study
full-stack quantum network architectures. This work develops SeQUeNCe, a
comprehensive, customizable quantum network simulator. Our simulator consists
of five modules: Hardware models, Entanglement Management protocols, Resource
Management, Network Management, and Application. This framework is suitable for
simulation of quantum network prototypes that capture the breadth of current
and future hardware technologies and protocols. We implement a comprehensive
suite of network protocols and demonstrate the use of SeQUeNCe by simulating a
photonic quantum network with nine routers equipped with quantum memories. The
simulation capabilities are illustrated in three use cases. We show the
dependence of quantum network throughput on several key hardware parameters and
study the impact of classical control message latency. We also investigate
quantum memory usage efficiency in routers and demonstrate that redistributing
memory according to anticipated load increases network capacity by 69.1\% and
throughput by 6.8\%. We design SeQUeNCe to enable comparisons of alternative
quantum network technologies, experiment planning, and validation and to aid
with new protocol design. We are releasing SeQUeNCe as an open source tool and
aim to generate community interest in extending it.

\subsection*{\href{http://arxiv.org/abs/2009.11980v1}{Large Bulk Piezophotovoltaic Effect of Monolayer Transition Metal  Dichalcogenides}}
\subsubsection*{Aaron M. Schankler, and Andrew M. Rappe (2020-09-24)}
The bulk photovoltaic effect in noncentrosymmetric materials is an intriguing
physical phenomenon that holds potential for high-efficiency energy harvesting.
Here, we study the shift current bulk photovoltaic effect in the transition
metal dichalcogenide MoS$_2$. We present a simple automated method to guide
materials design and use it to uncover a distortion to monolayer $2H$-MoS$_2$
that dramatically enhances the integrated shift current. Using this distortion,
we show that overlap in the Brillouin zone of the distributions of the shift
vector (a quantity measuring the net displacement in real space of coherent
wave packets during excitation) and the transition intensity is crucial for
increasing the shift current. The distortion pattern is related to the material
polarization and can be realized through an applied electric field via the
converse piezoelectric effect. This finding suggests an additional method to
engineer the shift current response of materials to augment previously reported
methods using mechanical strain.

\subsection*{\href{http://arxiv.org/abs/2009.11970v1}{Integer Programming from Quantum Annealing and Open Quantum Systems}}
\subsubsection*{Chia Cheng Chang, \dots, and Jim Ostrowski (2020-09-24)}
While quantum computing proposes promising solutions to computational
problems not accessible with classical approaches, due to current hardware
constraints, most quantum algorithms are not yet capable of computing systems
of practical relevance, and classical counterparts outperform them. To
practically benefit from quantum architecture, one has to identify problems and
algorithms with favorable scaling and improve on corresponding limitations
depending on available hardware. For this reason, we developed an algorithm
that solves integer linear programming problems, a classically NP-hard problem,
on a quantum annealer, and investigated problem and hardware-specific
limitations. This work presents the formalism of how to map ILP problems to the
annealing architectures, how to systematically improve computations utilizing
optimized anneal schedules, and models the anneal process through a simulation.
It illustrates the effects of decoherence and many body localization for the
minimum dominating set problem, and compares annealing results against
numerical simulations of the quantum architecture. We find that the algorithm
outperforms random guessing but is limited to small problems and that annealing
schedules can be adjusted to reduce the effects of decoherence. Simulations
qualitatively reproduce algorithmic improvements of the modified annealing
schedule, suggesting the improvements have origins from quantum effects.

\subsection*{\href{http://arxiv.org/abs/2009.11962v1}{Phonon hydrodynamics in crystalline GeTe at low temperature}}
\subsubsection*{Kanka Ghosh, and Jean-Luc Battaglia (2020-09-24)}
A first-principles density functional method along with the direct solution
of linearized Boltzmann transport equations are employed to systematically
analyze the low-temperature thermal transport in crystalline GeTe. The
extensive thermal transport simulations, ranging from room temperature to
cryogenic temperatures, reveal the emergence of a phonon hydrodynamic regime in
GeTe at low temperature. The reduction of grain boundary scattering is found to
play a crucial role along with the divergent trend of umklapp and normal
scattering at low temperatures in accommodating the hydrodynamic regime.
Average scattering rates for normal, umklapp, and other resistive processes are
distinguished for a wide range (4-300 K) of temperatures and used for
identifying various phonon transport regimes. Therefore, the variations of
lattice thermal conductivity, phonon propagation length, and thermal
diffusivity with temperature, related to these transport regimes (ballistic,
hydrodynamic, and kinetic), have been thoroughly investigated. The modewise
decomposition of lattice thermal conductivity and the distinction of thermal
diffusivity according to different scattering processes reveal rich information
on the dominant phonon modes and phonon scattering processes in GeTe at low
temperature. Further, the kinetic-collective model is used to elucidate the
hydrodynamic behavior of phonon scattering through the relative study of
collective and kinetic contributions to the thermal transport properties. In
this context, the Knudsen number is estimated through the characteristic
nonlocal length and the grain size, which further quantifies the consistent
hydrodynamic behavior of phonon thermal transport for GeTe. Finally,
phonon-vacancy scattering for GeTe is realized, and vacancies are found
strongly to influence the hydrodynamic window while incorporating the other
resistive scattering mechanisms.

\subsection*{\href{http://arxiv.org/abs/2009.11960v1}{Magnetic Field Induced Phase Transition in Spinel GeNi2O4}}
\subsubsection*{T. Basu, \dots, and X. Ke (2020-09-24)}
Cubic spinel GeNi2O4 exhibits intriguing magnetic properties with two
successive antiferromagnetic phase transitions (TN1 12.1 and TN2 11.4 K) with
the absence of any structural transition. We have performed detailed heat
capacity and magnetic measurements in different crystallographic orientations.
A new magnetic phase in presence of magnetic field (H > 4 T) along the [111]
direction is revealed, which is not observed when the magnetic field is applied
along the [100] and [110] directions. High field neutron powder diffraction
measurements confirm such a change in magnetic phase, which could be ascribed
to a spin reorientation in the presence of magnetic field. A strong magnetic
anisotropy and competing magnetic interactions play a crucial role on the
complex magnetic behavior in this cubic system.

\subsection*{\href{http://arxiv.org/abs/2009.11866v1}{Universal recovery and p-fidelity in von Neumann algebras}}
\subsubsection*{Marius Junge and Nicholas LaRacuente (2020-09-24)}
Scenarios ranging from quantum error correction to high energy physics use
recovery maps, which try to reverse the effects of generally irreversible
quantum channels. The decrease in quantum relative entropy between two states
under the same channel quantifies information lost. A small decrease in
relative entropy often implies recoverability via a universal map depending
only the second argument to the relative entropy. We find such a universal
recovery map for arbitrary channels on von Neumann algebras, and we generalize
to p-fidelity via subharmonicity of a logarithmic p-fidelity of recovery.
Furthermore, we prove that non-decrease of relative entropy is equivalent to
the existence of an L 1 -isometry implementing the channel on both input
states. Our primary technique is a reduction method by Haagerup, approximating
a non-tracial, type III von Neumann algebra by a finite algebra. This technique
has many potential applications in porting results from quantum information
theory to high energy settings.

\subsection*{\href{http://arxiv.org/abs/2009.11949v1}{Power-Law Stretching of Associating Polymers in Steady-State Extensional  Flow}}
\subsubsection*{Charley Schaefer and Tom C. B. McLeish (2020-09-24)}
We present a tube model for the Brownian dynamics of associating polymers in
extensional flow. In linear response, the model confirms the analytical
predictions for the sticky diffusivity by Leibler- Rubinstein-Colby theory.
Although a single-mode DEMG approximation accurately describes the transient
stretching of the polymers above a 'sticky' Weissenberg number (product of the
strain rate with the sticky-Rouse time), the pre-averaged model fails to
capture a remarkable development of a power-law distribution of stretch in
steady-state extensional flow: while the mean stretch is finite, the
fluctuations in stretch may diverge. We present an analytical model that shows
how strong stochastic forcing drive the long tail of the distribution, gives
rise to rare events of reaching a threshold stretch and constitutes a framework
within which nucleation rates of flow-induced crystallization may understood in
systems of associating polymers under flow. The model also exemplifies a wide
class of driven systems possessing strong, and scaling, fluctuations.

\subsection*{\href{http://arxiv.org/abs/2009.13069v1}{Exciton-Trion-Polaritons in Two-Dimensional Materials}}
\subsubsection*{Farhan Rana, \dots, and Christina Manolatou (2020-09-24)}
We present a many-body theory for exciton-trion-polaritons in doped
two-dimensional materials. Exciton-trion-polaritons are robust coherent hybrid
excitations involving excitons, trions, and photons. Signatures of these
polaritons have been recently seen in experiments. In these polaritons, the
2-body exciton states are coupled to the material ground state via
exciton-photon interaction and the 4-body trion states are coupled to the
exciton states via Coulomb interaction. The trion states are not directly
optically coupled to the material ground state. The energy-momentum dispersion
of these polaritons exhibit three bands. We calculate the energy band
dispersions and the compositions of polaritons at different doping densities
using Green's functions. The energy splittings between the polariton bands, as
well as the spectral weights of the polariton bands, depend on the strength of
the Coulomb coupling between the excitons and the trions and which in turn
depends on the doping density. The excitons are Coulomb coupled to both bound
and unbound trion states. The latter are exciton-electron scattering states and
their inclusion is necessary to capture the spectral weight transfer among the
polariton bands as a function of the doping density.

\subsection*{\href{http://arxiv.org/abs/2009.11944v1}{GdBO$_3$ and YBO$_3$ Nanocrystals under Compression}}
\subsubsection*{Robin Turnbull, \dots, and Francisco Javier Manjón (2020-09-24)}
High-pressure X-ray diffraction studies on nanocrystals of the
pseudo-vaterite-type borates GdBO$_3$ and YBO$_3$ are herein reported up to
17.4(2) and 13.4(2) GPa respectively. The subsequent determination of the
room-temperature pressure-volume equations of state is presented and discussed
in the context of contemporary publications which contradict the findings of
this work. In particular, the isothermal bulk moduli of GdBO$_3$ and YBO$_3$
are found to be 170(13) and 163(13) GPa respectively, almost 50\% smaller than
recent findings. Our experimental results provide an accurate revision of the
high-pressure compressibility behaviour of GdBO$_3$ and YBO$_3$ which is
consistent with the known systematics in isomorphic borates and previous ab
initio calculations. Finally we discuss how experimental/analytical errors
could have led to unreliable conclusions reported elsewhere.

\subsection*{\href{http://arxiv.org/abs/2009.11941v1}{Topological phases in $α$-Li$_{\rm 3}$N-type crystal structure of  light-element compounds}}
\subsubsection*{Ali Ebrahimian, and Mehrdad Dadsetani (2020-09-24)}
Materials with tunable topological features, simple crystal structure and
flexible synthesis, are in extraordinary demand towards technological
exploitation of unique properties of topological nodal points. The controlled
design of the lattice geometry of light elements is determined by utilizing
density functional theory and the effective Hamiltonian model together with the
symmetry analysis. This provides an intriguing venue for reasonably achieving
various distinct types of novel fermions. We, therefore, show that a nodal line
(type-I and II), Dirac fermion, and triple point (TP) fermionic excitation can
potentially appear as a direct result of a band inversion in group-I nitrides
with $\alpha$-Li$_{\rm 3}$N-type crystal structure. The imposed strain is
exclusively significant for these compounds, and it invariably leads to the
considerable modification of the nodal line type. Most importantly, a type-II
nodal loop can be realized in the system under strain. These unique
characteristics make $\alpha$-Li$_{\rm 3} $N-type crystal structure an ideal
playground to achieve various types of novel fermions well-suited for
technological applications.

\subsection*{\href{http://arxiv.org/abs/2009.11938v1}{Zero forcing number of graphs with a power law degree distribution}}
\subsubsection*{Alexei Vazquez (2020-09-24)}
The zero forcing number is the minimum number of black vertices that can turn
a white graph black following a single neighbour colour forcing rule. The zero
forcing number provides topological information about linear algebra on graphs,
with applications to the controllability of linear dynamical systems and
quantum walks on graphs among other problems. Here, I investigate the zero
forcing number of undirected simple graphs with a power law degree distribution
$p_k\sim k^{-\gamma}$. For graphs generated by the preferential attachment
model, with a diameter scaling logarithmically with the graph size, the zero
forcing number approaches the graph size when $\gamma\rightarrow2$. In
contrast, for graphs generated by the deactivation model, with a diameter
scaling linearly with the graph size, the zero forcing number is smaller than
the graph size independently of $\gamma$. Therefore the scaling of the graph
diameter with the graph size is another factor determining the controllability
of dynamical systems.

\subsection*{\href{http://arxiv.org/abs/2009.13068v1}{On a proper-time approach to localization}}
\subsubsection*{E. R. F. Taillebois and A. T. Avelar (2020-09-24)}
The causality issues concerning Hegerfeldt's paradox and the localization of
relativistic quantum systems are addressed through a proper-time formalism of
single-particle operators. The proposed description does not depend on
classical parameters connected to an external observer and results in a
single-particle formalism in which localization is described by explicitly
covariant four-vector operators associated with POVM measurements parametrized
by the system's proper-time. As a consequence, it is shown that physically
acceptable states are necessarily associated with the existence of a temporal
uncertainty and their proper-time evolution is not subject to the causality
violation predicted by Hegerfeldt.

\subsection*{\href{http://arxiv.org/abs/2009.11922v1}{Comment on "Colossal Pressure-Induced Softening in Scandium Fluoride"}}
\subsubsection*{I. A. Zaliznyak, and A. V. Tkachenko (2020-09-24)}
The results reported by Wei et al. [Phys. Rev. Lett. 124, 255502 (2020)] can
be confronted with predictive, quantitative theories of negative thermal
expansion (NTE) and pressure-induced softening, allowing to corroborate, or
invalidate certain approaches. Motivated to corroborate the quantitative
predictions of the recent Coulomb Floppy Network (CFN) microscopic theory of
vibrational and thermomechanical properties of empty perovskite crystals
[Tkachenko and Zaliznyak, arXiv:1908.11643 (2019)], we compared theory
prediction for the mean-squared transverse displacement of the F atoms,
U$_{perp}$, with that reported in Fig. 5 of Wei et al. and observed a marked
discrepancy (an order-of-magnitude larger than the error bar). We then compared
these results with the previously published Xray diffraction data of Greve, et
al. [JACS 132, 15496 (2010)] and the neutron diffraction data of Wendt, et al.
[Science Advances 5 (2019), 10.1126/sciadv.aay2748]. We found the latter two
data sets to be in a good agreement with each other, as well as with the
prediction of CFN theory. We thus conclude that U$_{perp}$ values reported in
Fig. 5 of Wei et al. are substantially incorrect. The purpose of this Comment
is twofold: (i) to caution the researchers against using the U$_{perp}$ data of
Wei et al. for quantitative comparisons with theory, and (ii) to encourage Wei
et al. to reconsider their analysis and obtain a reliable U$_{perp}$ data by
better accounting for the beam transmission and attenuation effects.

\subsection*{\href{http://arxiv.org/abs/2009.11916v1}{Electronic and Thermoelectric Properties of Half-Heusler Alloys NiTZ}}
\subsubsection*{Dhurba R. Jaishi, \dots, and Madhav Prasad Ghimire (2020-09-24)}
We have investigated the electronic and thermoelectric properties of
half-Heusler alloys NiTZ (T = Sc, and Ti; Z = P, As, Sn, and Sb) having 18
valence electron. Calculations are performed by means of density functional
theory and Boltzmann transport equation with constant relaxation time
approximation, validated by NiTiSn. The chosen half-Heuslers are found to be an
indirect band gap semiconductor, and the lattice thermal conductivity is
comparable with the state-of-the-art thermoelectric materials. The estimated
power factor for NiScP, NiScAs, and NiScSb reveals that their thermoelectric
performance can be enhanced by appropriate doping rate. The value of $ZT$ found
for NiScP, NiScAs, and NiScSb are 0.46, 0.35, and 0.29, respectively at 1200 K.

\subsection*{\href{http://arxiv.org/abs/2009.11908v2}{Hydrodynamic stabilization of self-organized criticality in a driven  Rydberg gas}}
\subsubsection*{K. Klocke, \dots, and M. Buchhold (2020-09-24)}
Signatures of self-organized criticality (SOC) have recently been observed in
an ultracold atomic gas under continuous laser excitation to
strongly-interacting Rydberg states [S. Helmrich et al., Nature, 577, 481--486
(2020)]. This creates a unique possibility to study this intriguing dynamical
phenomenon, e.g., to probe its robustness and universality, under controlled
experimental conditions. Here we examine the self-organizing dynamics of a
driven ultracold gas and identify an unanticipated feedback mechanism, which is
especially important for systems coupled to thermal baths. It sustains an
extended critical region in the trap center for a notably long time via
hydrodynamic transport of particles from the flanks of the cloud toward the
center. This compensates the avalanche-induced atom loss and leads to a
characteristic flat-top density profile, providing an additional experimental
signature for SOC and minimizing effects of inhomogeneity on the SOC features.

\subsection*{\href{http://arxiv.org/abs/2009.11887v1}{Thermal effects in non-Fermi liquid superconductivity}}
\subsubsection*{Jeremias Aguilera Damia, and Gonzalo Torroba (2020-09-24)}
We revisit the interplay between superconductivity and quantum criticality
when thermal effects from virtual static bosons are included. These
contributions, which arise from an effective theory compactified on the thermal
circle, strongly affect field-theoretic predictions even at small temperatures.
We argue that they are ubiquitous in a wide variety of models of non-Fermi
liquid behavior, and generically produce a parametric suppression of
superconducting instabilities. We apply these ideas to non-Fermi liquids in
$d=2$ space dimensions, obtained by coupling a Fermi surface to a Landau-damped
soft boson. Extending previous methods developed for $d=3-\epsilon$ dimensions,
we determine the dynamics and phase diagram. It features a naked quantum
critical point, separated by a continuous infinite order transition from a
superconducting phase with strong non-Fermi liquid corrections. We also
highlight the relevance of these effects for (numerical) experiments on
non-Fermi liquids.

\subsection*{\href{http://arxiv.org/abs/2009.11885v1}{Bosonic condensation of exciton-polaritons in an atomically thin crystal}}
\subsubsection*{Carlos Anton-Solanas, \dots, and Christian Schneider (2020-09-24)}
The emergence of two-dimensional crystals has revolutionized modern
solid-state physics. From a fundamental point of view, the enhancement of
charge carrier correlations has sparked enormous research activities in the
transport- and quantum optics communities. One of the most intriguing effects,
in this regard, is the bosonic condensation and spontaneous coherence of
many-particle complexes. Here, we find compelling evidence of bosonic
condensation of exciton-polaritons emerging from an atomically thin crystal of
MoSe2 embedded in a dielectric microcavity under optical pumping. The formation
of the condensate manifests itself in a sudden increase of luminescence
intensity in a threshold-like manner, and a significant spin-polarizability in
an externally applied magnetic field. Spatial coherence is mapped out via
highly resolved real-space interferometry, revealing a spatially extended
condensate. Our device represents a decisive step towards the implementation of
coherent light-sources based on atomically thin crystals, as well as
non-linear, valleytronic coherent devices.

\subsection*{\href{http://arxiv.org/abs/2009.11884v1}{Local optimization on pure Gaussian state manifolds}}
\subsubsection*{Bennet Windt, \dots, and Lucas Hackl (2020-09-24)}
We exploit insights into the geometry of bosonic and fermionic Gaussian
states to develop an efficient local optimization algorithm to extremize
arbitrary functions on these families of states. The method is based on notions
of gradient descent attuned to the local geometry which also allows for the
implementation of local constraints. The natural group action of the symplectic
and orthogonal group enables us to compute the geometric gradient efficiently.
While our parametrization of states is based on covariance matrices and linear
complex structures, we provide compact formulas to easily convert from and to
other parametrization of Gaussian states, such as wave functions for pure
Gaussian states, quasiprobability distributions and Bogoliubov transformations.
We review applications ranging from approximating ground states to computing
circuit complexity and the entanglement of purification that have both been
employed in the context of holography. Finally, we use the presented methods to
collect numerical and analytical evidence for the conjecture that Gaussian
purifications are sufficient to compute the entanglement of purification of
arbitrary mixed Gaussian states.

\subsection*{\href{http://arxiv.org/abs/2009.11883v1}{Turbulent relaxation after a quench in the Heisenberg model}}
\subsubsection*{Joaquin F. Rodriguez-Nieva (2020-09-24)}
We predict the emergence of turbulent scaling in the quench dynamics of the
two-dimensional Heisenberg model for a wide range of initial conditions and
model parameters. In the isotropic Heisenberg model, we find that the spin-spin
correlation function exhibits universal scaling consistent with a turbulent
energy cascade. When the spin rotational symmetry is broken with an easy-plane
anisotropic exchange, we find a dual cascade of energy and quasiparticles. The
scaling is shown to be robust to quantum fluctuations, which tend to inhibit
the turbulent relaxation when the spin number $S$ is small. The universal
character of the cascade, insensitive to microscopic details or the initial
condition, suggests that turbulence in spin systems can be broadly realized in
cold atom and solid-state experiments.

\subsection*{\href{http://arxiv.org/abs/2009.11881v1}{Entanglement and Complexity of Purification in (1+1)-dimensional free  Conformal Field Theories}}
\subsubsection*{Hugo A. Camargo, \dots, and Bennet Windt (2020-09-24)}
Finding pure states in an enlarged Hilbert space that encode the mixed state
of a quantum field theory as a partial trace is necessarily a challenging task.
Nevertheless, such purifications play the key role in characterizing quantum
information-theoretic properties of mixed states via entanglement and
complexity of purifications. In this article, we analyze these quantities for
two intervals in the vacuum of free bosonic and Ising conformal field theories
using, for the first time, the~most general Gaussian purifications. We provide
a comprehensive comparison with existing results and identify universal
properties. We further discuss important subtleties in our setup: the massless
limit of the free bosonic theory and the corresponding behaviour of the mutual
information, as well as the Hilbert space structure under the Jordan-Wigner
mapping in the spin chain model of the Ising conformal field theory.

\subsection*{\href{http://arxiv.org/abs/2009.11880v1}{Constraints on beta functions in field theories}}
\subsubsection*{Han Ma and Sung-Sik Lee (2020-09-24)}
The $\beta$-functions describe how couplings run under the renormalization
group flow in field theories. In general, all couplings allowed by symmetry and
locality are generated under the renormalization group flow, and the exact
renormalization group flow takes place in the infinite dimensional space of
couplings. In this paper, we show that the renormalization group flow is highly
constrained so that the $\beta$-functions defined in a measure zero subspace of
couplings completely determine the $\beta$-functions in the entire space of
couplings. We provide a quantum renormalization group-based algorithm for
reconstructing the full $\beta$-functions from the $\beta$-functions defined in
the subspace. The general prescription is applied to two simple examples.

\subsection*{\href{http://arxiv.org/abs/2009.11874v1}{Cobordism invariants from BPS q-series}}
\subsubsection*{Sergei Gukov, and Pavel Putrov (2020-09-24)}
Many BPS partition functions depend on a choice of additional structure:
fluxes, Spin or Spin$^c$ structures, etc. In a context where the BPS generating
series depends on a choice of Spin$^c$ structure we show how different limits
with respect to the expansion variable $q$ and different ways of summing over
Spin$^c$ structures produce different invariants of homology cobordisms out of
the BPS $q$-series.

\subsection*{\href{http://arxiv.org/abs/2009.11872v1}{TBG II: Stable Symmetry Anomaly in Twisted Bilayer Graphene}}
\subsubsection*{Zhi-Da Song, \dots, and Andrei B. Bernevig (2020-09-24)}
We show that the entire continuous model of twisted bilayer graphene (TBG)
(and not just the two active bands) with particle-hole symmetry is anomalous
and hence incompatible with a lattice model. Previous works, e.g., [Phys. Rev.
Lett. 123, 036401], [Phys. Rev. X 9, 021013], [Phys. Rev. B 99, 195455], and
others [1-4] found that the two flat bands in TBG possess a fragile topology
protected by the $C_{2z}T$ symmetry. [Phys. Rev. Lett. 123, 036401] also
pointed out an approximate particle-hole symmetry ($\mathcal{P}$) in the
continuous model of TBG. In this work, we numerically confirm that
$\mathcal{P}$ is indeed a good approximation for TBG and show that the fragile
topology of the two flat bands is enhanced to a $\mathcal{P}$-protected stable
topology. This stable topology implies $4l+2$ ($l\in\mathbb{N}$) Dirac points
between the middle two bands. The $\mathcal{P}$-protected stable topology is
robust against arbitrary gap closings between the middle two bands the other
bands. We further show that, remarkably, this $\mathcal{P}$-protected stable
topology, as well as the corresponding $4l + 2$ Dirac points, cannot be
realized in lattice models that preserve both $C_{2z}T$ and $\mathcal{P}$
symmetries. In other words, the continuous model of TBG is anomalous and cannot
be realized on lattices. Two other topology related topics, with consequences
for the interacting TBG problem, i.e., the choice of Chern band basis in the
two flat bands and the perfect metal phase of TBG in the so-called second
chiral limit, are also discussed.

\subsection*{\href{http://arxiv.org/abs/2009.11864v1}{Dynamical control of the conductivity of an atomic Josephson junction}}
\subsubsection*{Beilei Zhu, \dots, and Ludwig Mathey (2020-09-24)}
We propose to dynamically control the conductivity of a Josephson junction
composed of two weakly coupled one dimensional condensates of ultracold atoms.
A current is induced by a periodically modulated potential difference between
the condensates, giving access to the conductivity of the junction. By using
parametric driving of the tunneling energy, we demonstrate that the
low-frequency conductivity of the junction can be enhanced or suppressed,
depending on the choice of the driving frequency. The experimental realization
of this proposal provides a quantum simulation of optically enhanced
superconductivity in pump-probe experiments of high temperature
superconductors.

\subsection*{\href{http://arxiv.org/abs/2009.11863v1}{Spectral statistics in constrained many-body quantum chaotic systems}}
\subsubsection*{Sanjay Moudgalya, \dots, and Amos Chan (2020-09-24)}
We study the spectral statistics of spatially-extended many-body quantum
systems with on-site Abelian symmetries or local constraints, focusing
primarily on those with conserved dipole and higher moments. In the limit of
large local Hilbert space dimension, we find that the spectral form factor
$K(t)$ of Floquet random circuits can be mapped exactly to a classical Markov
circuit, and, at late times, is related to the partition function of a
frustration-free Rokhsar-Kivelson (RK) type Hamiltonian. Through this mapping,
we show that the inverse of the spectral gap of the RK-Hamiltonian lower bounds
the Thouless time $t_\mathrm{Th}$ of the underlying circuit. For systems with
conserved higher moments, we derive a field theory for the corresponding
RK-Hamiltonian by proposing a generalized height field representation for the
Hilbert space of the effective spin chain. Using the field theory formulation,
we obtain the dispersion of the low-lying excitations of the RK-Hamiltonian in
the continuum limit, which allows us to extract $t_\mathrm{Th}$. In particular,
we analytically argue that in a system of length $L$ that conserves the
$m^{th}$ multipole moment, $t_\mathrm{Th}$ scales subdiffusively as
$L^{2(m+1)}$. Our work therefore provides a general approach for studying
spectral statistics in constrained many-body chaotic systems.

\subsection*{\href{http://arxiv.org/abs/2009.11860v1}{Custom fermionic codes for quantum simulation}}
\subsubsection*{Riley W. Chien and James D. Whitfield (2020-09-24)}
Simulating a fermionic system on a quantum computer requires encoding the
anti-commuting fermionic variables into the operators acting on the qubit
Hilbert space. The most familiar of which, the Jordan-Wigner transformation,
encodes fermionic operators into non-local qubit operators. As non-local
operators lead to a slower quantum simulation, recent works have proposed ways
of encoding fermionic systems locally. In this work, we show that locality may
in fact be too strict of a condition and the size of operators can be reduced
by encoding the system quasi-locally. We give examples relevant to lattice
models of condensed matter and systems relevant to quantum gravity such as SYK
models. Further, we provide a general construction for designing codes to suit
the problem and resources at hand and show how one particular class of
quasi-local encodings can be thought of as arising from truncating the state
preparation circuit of a local encoding. We end with a discussion of designing
codes in the presence of device connectivity constraints.

\subsection*{\href{http://arxiv.org/abs/2009.11856v1}{Cavity-induced quantum spin liquids}}
\subsubsection*{Alessio Chiocchetta, \dots, and Sebastian Diehl (2020-09-24)}
Quantum spin liquids provide paradigmatic examples of highly entangled
quantum states of matter. Frustration is the key mechanism to favor spin
liquids over more conventional magnetically ordered states. Here we propose to
engineer frustration by exploiting the coupling of quantum magnets to the
quantized light of an optical cavity. The interplay between the quantum
fluctuations of the electro-magnetic field and the strongly correlated
electrons results in a tunable long-range interaction between localized spins.
This cavity-induced frustration robustly stabilizes spin liquid states, which
occupy an extensive region in the phase diagram spanned by the range and
strength of the tailored interaction. Remarkably, this occurs even in
originally unfrustrated systems, as we showcase for the Heisenberg model on the
square lattice.

\subsection*{\href{http://arxiv.org/abs/2009.13247v2}{Quantum circuits of CNOT gates}}
\subsubsection*{Marc Bataille (2020-09-24)}
We study in details the algebraic structure underlying quantum circuits
generated by CNOT gates. Our results allow us to propose polynomial heuristics
to reduce the number of gates used in a given CNOT gates circuit and we also
give algorithms to optimize this type of circuits in some particular cases.
Finally we show how to create some usefull entangled states using a CNOT gates
circuit acting on a fully factorized state.

\subsection*{\href{http://arxiv.org/abs/2009.11833v1}{Quantum spin torque driven transmutation of antiferromagnetic Mott  insulator}}
\subsubsection*{Marko D. Petrovic, and Branislav K. Nikolic (2020-09-24)}
The standard model of spin-transfer torque (STT) in antiferromagnetic
spintronics considers exchange of angular momentum between quantum spins of
flowing conduction electrons and noncollinear-to-them localized spins treated
as classical vectors. These classical vectors are assumed to realize N\'{e}el
order in equilibrium, $\uparrow \downarrow \ldots \uparrow \downarrow$, such
that their STT-driven dynamics is then described by the Landau-Lifshitz-Gilbert
(LLG) equation. However, many experimentally employed materials (such as
prototypical NiO) are strongly correlated antiferromagnetic Mott insulators
(AFMI) where localized spins form a ground state quite different from the
unentangled N\'{e}el state $|\!\! \uparrow \downarrow \ldots \uparrow
\downarrow \rangle$. The true ground state is entangled by quantum spin
fluctuations, so that the expectation value of each localized spin is zero and
the classical LLG picture of rotating localized spins driven by STT is
inapplicable. Instead, a fully quantum treatment of both conduction electrons
and localized spins is necessary to capture exchange of spin angular momentum
between them, denoted as {\em quantum STT}. We use a very recently developed
time-dependent density matrix renormalization group (tDMRG) approach to quantum
STT to demonstrate how injection of a spin-polarized current pulse into a
normal metallic layer proximity coupled to AFMI overlayer via exchange
interaction and/or small interlayer hopping, where such setup mimics
topological-insulator/NiO bilayer employed in recent experiments, will induce
nonzero expectation values of localized spins within AFMI. They are collinear
but spatially inhomogeneous with zigzag profile, where the total spin angular
momentum absorbed by AFMI increases with Coulomb repulsion in AFMI, as well as
when the two layers do not exchange any charge.

\subsection*{\href{http://arxiv.org/abs/2009.11830v1}{Measuring interfacial Dzyaloshinskii-Moriya interaction in ultra thin  films}}
\subsubsection*{Michaela Kuepferling, \dots, and Giovanni Carlotti (2020-09-24)}
The Dzyaloshinskii-Moriya interaction (DMI), being one of the origins for
chiral magnetism, is currently attracting huge attention in the research
community focusing on applied magnetism and spintronics. For future
applications an accurate measurement of its strength is indispensable. In this
work, we present a review of the state of the art of measuring the coefficient
$D$ of the Dzyaloshinskii-Moriya interaction, the DMI constant, focusing on
systems where the interaction arises from the interface between two materials.
The measurement techniques are divided into three categories: a) domain wall
based measurements, b) spin wave based measurements and c) spin orbit torque
based measurements. We give an overview of the experimental techniques as well
as their theoretical background and models for the quantification of the DMI
constant $D$. We analyze the advantages and disadvantages of each method and
compare $D$ values in different stacks. The review aims to obtain a better
understanding of the applicability of the different techniques to different
stacks and of the origin of apparent disagreement of literature values.

\subsection*{\href{http://arxiv.org/abs/2009.11824v1}{Efficient sampling from shallow Gaussian quantum-optical circuits with  local interactions}}
\subsubsection*{Haoyu Qi, \dots, and Nicolás Quesada (2020-09-24)}
We prove that a classical computer can efficiently sample from the
photon-number probability distribution of a Gaussian state prepared by using an
optical circuit that is shallow and local. Our work generalizes previous known
results for qubits to the continuous-variable domain. The key to our proof is
the observation that the adjacency matrices characterizing the Gaussian states
generated by shallow and local circuits have small bandwidth. To exploit this
structure, we devise fast algorithms to calculate loop hafnians of banded
matrices. Since sampling from deep optical circuits with exponential-scaling
photon loss is classically simulable, our results pose a challenge to the
feasibility of demonstrating quantum supremacy on photonic platforms with local
interactions.

\subsection*{\href{http://arxiv.org/abs/2009.11819v1}{Isoelectronic perturbations to $f$-$d$-electron hybridization and the  enhancement of hidden order in URu$_2$Si$_2$}}
\subsubsection*{C. T. Wolowiec, \dots, and M. B. Maple (2020-09-24)}
Electrical resistivity measurements were performed on single crystals of
URu$_2-x$Os$_x$Si$_2$ up to $x$ = 0.28 under hydrostatic pressure up to $P$ = 2
GPa. As the Os concentration, $x$ , is increased, (1) the lattice expands,
creating an effective negative chemical pressure $P_{ch}$($x$), (2) the hidden
order (HO) phase is enhanced and the system is driven toward a large-moment
antiferromagnetic (LMAFM) phase, and (3) less external pressure $P_{c}$ is
required to induce the HO to LMAFM phase transition. We compare the $T(x)$,
$T(P)$ phase behavior reported here for the URu$_2-x$Os$_x$Si$_2$ system with
previous reports of enhanced HO in URu$_2$Si$_2$ upon tuning with $P$, or
similarly in URu$_2-x$Fe$_x$Si$_2$ upon tuning with positive $P_{ch}$($x$). It
is noted that pressure, Fe substitution, and Os substitution are the only known
perturbations that enhance the HO phase and induce the first order transition
to the LMAFM phase in URu$_2$Si$_2$. We present a scenario in which the
application of pressure or the isoelectronic substitution of Fe and Os ions for
Ru results in an increase in the hybridization of the U-5$f$- and transition
metal $d$-electron states which leads to electronic instability in the
paramagnetic phase and a concurrent stability of HO (and LMAFM) in
URu$_2$Si$_2$. Calculations in the tight binding approximation are included to
determine the strength of hybridization between the U-5$f$ electrons and each
of the isoelectronic transition metal $d$-electron states of Fe, Ru, and Os.

\subsection*{\href{http://arxiv.org/abs/2009.11818v1}{Enhancing secure key rates of satellite QKD using a quantum dot  single-photon source}}
\subsubsection*{Poompong Chaiwongkhot, \dots, and Thomas Jennewein (2020-09-24)}
Global quantum secure communication can be achieved using quantum key
distribution (QKD) with orbiting satellites. Established techniques use
attenuated lasers as weak coherent pulse (WCP) sources, with so-called
decoy-state protocols, to generate the required single-photon-level pulses.
While such approaches are elegant, they come at the expense of attainable final
key due to inherent multi-photon emission, thereby constraining secure key
generation over the high-loss, noisy channels expected for satellite
transmissions. In this work we improve on this limitation by using true
single-photon pulses generated from a semiconductor quantum dot (QD) embedded
in a nanowire, possessing low multi-photon emission ($<10^{-6}$) and an
extraction system efficiency of -15 dB (or 3.1\%). Despite the limited
efficiency, the key generated by the QD source is greater than that generated
by a WCP source under identical repetition rate and link conditions
representative of a satellite pass. We predict that with realistic improvements
of the QD extraction efficiency to -4.0 dB (or 40\%), the quantum-dot QKD
protocol outperforms WCP-decoy-state QKD by almost an order of magnitude.
Consequently, a QD source could allow generation of a secure key in conditions
where a WCP source would simply fail, such as in the case of high channel
losses. Our demonstration is the first specific use case that shows a clear
benefit for QD-based single-photon sources in secure quantum communication, and
has the potential to enhance the viability and efficiency of satellite-based
QKD networks.

\subsection*{\href{http://arxiv.org/abs/2009.11817v1}{The modified logarithmic Sobolev inequality for quantum spin systems:  classical and commuting nearest neighbour interactions}}
\subsubsection*{Ángela Capel, and Daniel Stilck França (2020-09-24)}
Given a uniform, frustration-free family of local Lindbladians defined on a
quantum lattice spin system in any spatial dimension, we prove a strong
exponential convergence in relative entropy of the system to equilibrium under
a condition of spatial mixing of the stationary Gibbs states and the rapid
decay of the relative entropy on finite-size blocks. Our result leads to the
first examples of the positivity of the modified logarithmic Sobolev inequality
for quantum lattice spin systems independently of the system size. Moreover, we
show that our notion of spatial mixing is a consequence of the recent quantum
generalization of Dobrushin and Shlosman's complete analyticity of the
free-energy at equilibrium. The latter typically holds above a critical
temperature.
  Our results have wide applications in quantum information processing. As an
illustration, we discuss three of them: first, using techniques of quantum
optimal transport, we show that a quantum annealer subject to a finite range
classical noise will output an energy close to that of the fixed point after
constant annealing time. Second, we prove a finite blocklength refinement of
the quantum Stein lemma for the task of asymmetric discrimination of two Gibbs
states of commuting Hamiltonians satisfying our conditions. In the same
setting, our results imply the existence of a local quantum circuit of
logarithmic depth to prepare Gibbs states of a class of commuting Hamiltonians.

\subsection*{\href{http://arxiv.org/abs/2009.11810v1}{Deep learning enabled design of complex transmission matrices for  universal optical components}}
\subsubsection*{Nicholas J. Dinsdale, \dots, and Otto L. Muskens (2020-09-24)}
Recent breakthroughs in photonics-based quantum, neuromorphic and analogue
processing have pointed out the need for new schemes for fully programmable
nanophotonic devices. Universal optical elements based on interferometer meshes
are underpinning many of these new technologies, however this is achieved at
the cost of an overall footprint that is very large compared to the limited
chip real estate, restricting the scalability of this approach. Here, we
propose an ultracompact platform for low-loss programmable elements using the
complex transmission matrix of a multi-port multimode waveguide. Our approach
allows the design of arbitrary transmission matrices using patterns of weakly
scattering perturbations, which is successfully achieved by means of a deep
learning inverse network. The demonstrated platform allows control over both
the intensity and phase in a multiport device at a four orders reduced device
footprint compared to conventional technologies, thus opening the door for
large-scale integrated universal networks.

\subsection*{\href{http://arxiv.org/abs/2009.11809v1}{Three-terminal nonlocal conductance in Majorana nanowires:  distinguishing topological and trivial in realistic systems with disorder and  inhomogeneous potential}}
\subsubsection*{Haining Pan, and S. Das Sarma (2020-09-24)}
We develop a theory for the three-terminal nonlocal conductance in Majorana
nanowires as existing in the superconductor-semiconductor hybrid structures in
the presence of superconducting proximity, spin-orbit coupling, and Zeeman
splitting. The key question addressed is whether such nonlocal conductance can
decisively distinguish between trivial and topological Majorana scenarios in
the presence of chemical potential inhomogeneity and random impurity disorder.
We calculate the local electrical as well as nonlocal electrical, and thermal
conductance of the pristine nanowire (good zero-bias conductance peaks), the
nanowire in the presence of quantum dots and inhomogeneous potential (bad
zero-bias conductance peaks), and the nanowire in the presence of large
disorder (ugly zero-bias conductance peaks). The local conductance by itself is
incapable of distinguishing the trivial states from the topological states
since zero-bias conductance peaks are generic in the presence of disorder and
inhomogeneous potential. The nonlocal conductance, which in principle is
capable of providing the bulk gap closing and reopening information at the
topological quantum phase transition, is found to be far too weak in magnitude
to be particularly useful in the presence of disorder and inhomogeneous
potential. Therefore, we focus on the question of whether the combination of
the local, nonlocal electrical and the thermal conductance can separate the
good, bad, and ugly zero-bias conductance peaks in finite-length wires. Our
paper aims to provide a guide to future experiments, and we conclude that a
combination of all three measurements would be necessary for a decisive
demonstration of topological Majorana zero modes in nanowires -- positive
signals corresponding to just one kind of measurements are likely to be false
positives arising from disorder and inhomogeneous potential.

\subsection*{\href{http://arxiv.org/abs/2009.11804v1}{Transition between chaotic and stochastic universality classes of  kinetic roughening}}
\subsubsection*{E. Rodriguez-Fernandez and R Cuerno (2020-09-24)}
The dynamics of non-equilibrium spatially extended systems are often
dominated by fluctuations, due to e.g.\ deterministic chaos or to intrinsic
stochasticity. This reflects into generic scale invariant or kinetic roughening
behavior that can be classified into universality classes defined by critical
exponent values and by the probability distribution function (PDF) of field
fluctuations. Suitable geometrical constraints are known to change secondary
features of the PDF while keeping the values of the exponents unchanged,
inducing universality subclasses. Working on the Kuramoto-Sivashinsky equation
as a paradigm of spatiotemporal chaos, we show that the physical nature of the
prevailing fluctuations (chaotic or stochastic) can also change the
universality class while respecting the exponent values, as the PDF is
substantially altered. This transition takes place at a non-zero value of the
stochastic noise amplitude and may be suitable for experimental verification.

\subsection*{\href{http://arxiv.org/abs/2009.11802v1}{Search for Efficient Formulations for Hamiltonian Simulation of  non-Abelian Lattice Gauge Theories}}
\subsubsection*{Zohreh Davoudi, and Andrew Shaw (2020-09-24)}
Hamiltonian formulation of lattice gauge theories (LGTs) is the most natural
framework for the purpose of quantum simulation, an area of research that is
growing with advances in quantum-computing algorithms and hardware. It,
therefore, remains an important task to identify the most accurate, while
computationally economic, Hamiltonian formulation(s) in such theories,
considering the necessary truncation imposed on the Hilbert space of gauge
bosons with any finite computing resources. This paper is a first step toward
addressing this question in the case of non-Abelian LGTs, which further require
the imposition of non-Abelian Gauss's laws on the Hilbert space, introducing
additional computational complexity. Focusing on the case of SU(2) LGT in 1+1 D
coupled to matter, a number of different formulations of the original
Kogut-Susskind framework are analyzed with regard to the dependence of the
dimension of the physical Hilbert space on boundary conditions, system's size,
and the cutoff on the excitations of gauge bosons. The impact of such
dependencies on the accuracy of the spectrum and dynamics is examined, and the
(classical) computational-resource requirements given these considerations are
studied. Besides the well-known angular-momentum formulation of the theory, the
cases of purely fermionic and purely bosonic formulations (with open boundary
conditions), and the Loop-String-Hadron formulation are analyzed, along with a
brief discussion of a Quantum Link Model of the same theory. Clear advantages
are found in working with the Loop-String-Hadron framework which implements
non-Abelian Gauss's laws a priori using a complete set of gauge-invariant
operators. Although small lattices are studied in the numerical analysis of
this work, and only the simplest algorithms are considered, a range of
conclusions will be applicable to larger systems and potentially to higher
dimensions.

\subsection*{\href{http://arxiv.org/abs/2009.11790v1}{Single-shot error correction of three-dimensional homological product  codes}}
\subsubsection*{Armanda O. Quintavalle, \dots, and Earl T. Campbell (2020-09-24)}
Single-shot error correction corrects data noise using only a single round of
noisy measurements on the data qubits, removing the need for intensive
measurement repetition. We introduce a general concept of confinement for
quantum codes, which roughly stipulates qubit errors cannot grow without
triggering more measurement syndromes. We prove confinement is sufficient for
single-shot decoding of adversarial errors. Further to this, we prove that all
three-dimensional homological product codes exhibit confinement in their
$X$-components and are therefore single-shot for adversarial phase-flip noise.
For stochastic phase-flip noise, we numerically explore these codes and again
find evidence of single-shot protection. Our Monte-Carlo simulations indicate
sustainable thresholds of $3.08(4)\%$ and $2.90(2)\%$ for 3D surface and toric
codes respectively, the highest observed single-shot thresholds to date. To
demonstrate single-shot error correction beyond the class of topological codes,
we also run simulations on a randomly constructed 3D homological product code.

\subsection*{\href{http://arxiv.org/abs/2009.11784v1}{Bath-induced Zeno localization in driven many-body quantum systems}}
\subsubsection*{Thibaud Maimbourg, \dots, and Alberto Rosso (2020-09-24)}
We study a quantum interacting spin system subject to an external drive and
coupled to a thermal bath of spatially localized vibrational modes, serving as
a model of Dynamic Nuclear Polarization. We show that even when the many-body
eigenstates of the system are ergodic, a sufficiently strong coupling to the
bath may effectively localize the spins due to many-body quantum Zeno effect,
as manifested by the hole-burning shape of the electron paramagnetic resonance
spectrum. Our results provide an explanation of the breakdown of the thermal
mixing regime experimentally observed above 4 - 5 Kelvin.

\subsection*{\href{http://arxiv.org/abs/2009.11781v1}{Self-organized criticality in neural networks from activity-based  rewiring}}
\subsubsection*{Stefan Landmann, and Stefan Bornholdt (2020-09-24)}
Neural systems process information in a dynamical regime between silence and
chaotic dynamics. This has lead to the criticality hypothesis which suggests
that neural systems reach such a state by self-organizing towards the critical
point of a dynamical phase transition. Here, we study a minimal neural network
model that exhibits self-organized criticality in the presence of stochastic
noise using a rewiring rule which only utilizes local information. For network
evolution, incoming links are added to a node or deleted, depending on the
node's average activity. Based on this rewiring-rule only, the network evolves
towards a critcal state, showing typical power-law distributed avalanche
statistics. The observed exponents are in accord with criticality as predicted
by dynamical scaling theory, as well as with the observed exponents of neural
avalanches. The critical state of the model is reached autonomously without
need for parameter tuning, is independent of initial conditions, is robust
under stochastic noise, and independent of details of the implementation as
different variants of the model indicate. We argue that this supports the
hypothesis that real neural systems may utilize similar mechanisms to
self-organize towards criticality especially during early developmental stages.

\subsection*{\href{http://arxiv.org/abs/2009.11773v1}{Evolution of Spin-Orbital Entanglement with Increasing Ising Spin-Orbit  Coupling}}
\subsubsection*{Dorota Gotfryd, \dots, and Andrzej M. Oles (2020-09-24)}
Several realistic spin-orbital models for transition metal oxides go beyond
the classical expectations and could be understood only by employing the
quantum entanglement. Experiments on these materials confirm that spin-orbital
entanglement has measurable consequences. Here, we capture the essential
features of spin-orbital entanglement in complex quantum matter utilizing 1D
spin-orbital model which accommodates SU(2)xSU(2) symmetric Kugel-Khomskii
superexchange as well as the Ising on-site spin-orbit coupling. Building on the
results obtained for full and effective models in the regime of strong
spin-orbit coupling, we address the question whether the entanglement found on
superexchange bonds always increases when the Ising spin-orbit coupling is
added. We show that (i) quantum entanglement is amplified by strong spin-orbit
coupling and, surprisingly, (ii) almost classical disentangled states are
possible. We complete the latter case by analyzing how the entanglement
existing for intermediate values of spin-orbit coupling can disappear for
higher values of this coupling.

\subsection*{\href{http://arxiv.org/abs/2009.11768v1}{Many-body Quantum Geometry in Superconductor-Quantum Dot Chains}}
\subsubsection*{Raffael L. Klees, \dots, and Gianluca Rastelli (2020-09-24)}
Multiterminal Josephson junctions constitute engineered topological systems
in arbitrary synthetic dimensions defined by the superconducting phases.
Microwave spectroscopy enables the measurement of the quantum geometric tensor,
a fundamental quantity describing both the quantum geometry and the topology of
the emergent Andreev bound states in a unified manner. In this work we propose
an experimentally feasible multiterminal setup of $N$ quantum dots connected to
$N+1$ superconducting leads to study nontrivial topology in terms of the
many-body Chern number of the ground state. Moreover, we generalize the
microwave spectroscopy scheme to the multiband case and show that the elements
of the quantum geometric tensor of the noninteracting ground state can be
experimentally accessed from the measurable oscillator strengths at low
temperature.

\subsection*{\href{http://arxiv.org/abs/2009.11756v1}{Hydro-chemical interactions in dilute phoretic suspensions: from  individual particle properties to collective organization}}
\subsubsection*{Tullio Traverso and Sebastien Michelin (2020-09-24)}
Janus phoretic colloids (JPs) self-propel as a result of self-generated
chemical gradients and exhibit spontaneous nontrivial dynamics within phoretic
suspensions, on length scales much larger than the microscopic swimmer size.
Such collective dynamics arise from the competition of (i) the self-propulsion
velocity of the particles, (ii) the attractive/repulsive chemically-mediated
interactions between particles and (iii) the flow disturbance they introduce in
the surrounding medium. These three ingredients are directly determined by the
shape and physico-chemical properties of the colloids' surface. Owing to such
link, we adapt a recent and popular kinetic model for dilute suspensions of
chemically-active JPs where the particles' far-field hydrodynamic and chemical
signatures are intrinsically linked and explicitly determined by the design
properties. Using linear stability analysis, we show that self-propulsion can
induce a wave-selective mechanism for certain particles' configurations
consistent with experimental observations. Numerical simulations of the
complete kinetic model are further performed to analyze the relative importance
of chemical and hydrodynamic interactions in the nonlinear dynamics. Our
results show that regular patterns in the particle density are promoted by
chemical signaling but prevented by the strong fluid flows generated
collectively by the polarized particles, regardless of their chemotactic or
antichemotactic nature (i.e. for both puller and pusher swimmers).

\subsection*{\href{http://arxiv.org/abs/2009.11755v1}{Impulsively Excited Gravitational Quantum States: Echoes and  Time-resolved Spectroscopy}}
\subsubsection*{I. Tutunnikov, \dots, and I. Sh. Averbukh (2020-09-24)}
We theoretically study an impulsively excited quantum bouncer (QB) - a
particle bouncing off a surface in the presence of gravity. A pair of
time-delayed pulsed excitations is shown to induce a wave-packet echo effect -
a partial rephasing of the QB wave function appearing at twice the delay
between pulses. In addition, an appropriately chosen observable [here, the
population of the ground gravitational quantum state (GQS)] recorded as a
function of the delay is shown to contain the transition frequencies between
the GQSs, their populations, and partial phase information about the wave
packet quantum amplitudes. The wave-packet echo effect is a promising candidate
method for precision studies of GQSs of ultra-cold neutrons, atoms, and
anti-atoms confined in closed gravitational traps.

\subsection*{\href{http://arxiv.org/abs/2009.11745v1}{Integrability of $1D$ Lindbladians from operator-space fragmentation}}
\subsubsection*{Fabian H. L. Essler and Lorenzo Piroli (2020-09-24)}
We introduce families of one-dimensional Lindblad equations describing open
many-particle quantum systems that are exactly solvable in the following sense:
$(i)$ the space of operators splits into exponentially many (in system size)
subspaces that are left invariant under the dissipative evolution; $(ii)$ the
time evolution of the density matrix on each invariant subspace is described by
an integrable Hamiltonian. The prototypical example is the quantum version of
the asymmetric simple exclusion process (ASEP) which we analyze in some detail.
We show that in each invariant subspace the dynamics is described in terms of
an integrable spin-1/2 XXZ Heisenberg chain with either open or twisted
boundary conditions. We further demonstrate that Lindbladians featuring
integrable operator-space fragmentation can be found in spin chains with
arbitrary local physical dimension.

\subsection*{\href{http://arxiv.org/abs/2009.11740v1}{Design using randomness: a new dimension for metallurgy}}
\subsubsection*{Wolfram Georg Nöhring and W. A. Curtin (2020-09-24)}
High entropy alloys add a new dimension, atomic-scale randomness and the
associated scale-dependent composition fluctuations, to the traditional
metallurgical axes of time-temperature-composition-microstructure. Alloy
performance is controlled by the energies and motion of defects (dislocations,
grain boundaries, vacancies, cracks, ...). Randomness at the atomic scale can
introduce new length and energy scales that can control defect behavior, and
hence control alloy properties. The axis of atomic-scale randomness combined
with the huge compositional space in multicomponent alloys thus enables, in
tandem with still-valid traditional principles, a new broader alloy design
strategy that may help achieve the multi-performance requirements of many
engineering applications.

\subsection*{\href{http://arxiv.org/abs/2009.11733v1}{Incipient antiferromagnetism in the Eu-doped topological insulator  Bi$_2$Te$_3$}}
\subsubsection*{Abdul Tcakaev, \dots, and Vladimir Hinkov (2020-09-24)}
Rare earth ions typically exhibit larger magnetic moments than transition
metal ions and thus promise the opening of a wider exchange gap in the Dirac
surface states of topological insulators. Yet, in a recent photoemission study
of Eu-doped Bi$_2$Te$_3$ films, the spectra remained gapless down to $T =
20\;\text{K}$. Here, we scrutinize whether the conditions for a substantial gap
formation in this system are present by combining spectroscopic and bulk
characterization methods with theoretical calculations. For all studied Eu
doping concentrations, our atomic multiplet analysis of the $M_{4,5}$ x-ray
absorption and magnetic circular dichroism spectra reveals a Eu$^{2+}$ valence
and confirms a large magnetic moment, consistent with a $4f^7 \; {^8}S_{7/2}$
ground state. At temperatures below $10\;\text{K}$, bulk magnetometry indicates
the onset of antiferromagnetic (AFM) ordering. This is in good agreement with
density functional theory, which predicts AFM interactions between the Eu
impurities. Our results support the notion that antiferromagnetism can coexist
with topological surface states in rare-earth doped Bi$_2$Te$_3$ and call for
spectroscopic studies in the kelvin range to look for novel quantum phenomena
such as the quantum anomalous Hall effect.

\subsection*{\href{http://arxiv.org/abs/2009.11730v1}{Grating-graphene metamaterial as a platform for terahertz nonlinear  photonics}}
\subsubsection*{Jan-Christoph Deinert, \dots, and Klaas-Jan Tielrooij (2020-09-24)}
Nonlinear optics is an increasingly important field for scientific and
technological applications, owing to its relevance and potential for optical
and optoelectronic technologies. Currently, there is an active search for
suitable nonlinear material systems with efficient conversion and small
material footprint. Ideally, the material system should allow for
chip-integration and room-temperature operation. Two-dimensional materials are
highly interesting in this regard. Particularly promising is graphene, which
has demonstrated an exceptionally large nonlinearity in the terahertz regime.
Yet, the light-matter interaction length in two-dimensional materials is
inherently minimal, thus limiting the overall nonlinear-optical conversion
efficiency. Here we overcome this challenge using a metamaterial platform that
combines graphene with a photonic grating structure providing field
enhancement. We measure terahertz third-harmonic generation in this
metamaterial and obtain an effective third-order nonlinear susceptibility with
a magnitude as large as 3$\cdot$10$^{-8}$m$^2$/V$^2$, or 21 esu, for a
fundamental frequency of 0.7 THz. This nonlinearity is 50 times larger than
what we obtain for graphene without grating. Such an enhancement corresponds to
third-harmonic signal with an intensity that is three orders of magnitude
larger due to the grating. Moreover, we demonstrate a field conversion
efficiency for the third harmonic of up to $\sim$1\% using a moderate field
strength of $\sim$30 kV/cm. Finally we show that harmonics beyond the third are
enhanced even more strongly, allowing us to observe signatures of up to the
9$^{\rm th}$ harmonic. Grating-graphene metamaterials thus constitute an
outstanding platform for commercially viable, CMOS compatible, room
temperature, chip-integrated, THz nonlinear conversion applications.

\subsection*{\href{http://arxiv.org/abs/2009.11711v1}{Mechanistic insights of dissolution and mechanical breakdown of FeCO3  corrosion films}}
\subsubsection*{Adriana Matamoros-Veloza, \dots, and Anne Neville (2020-09-24)}
To understand the behavior of corrosion films on X65 C-steel under CO2
conditions is paramount to identify the formation and transformations of
corrosion products. This work presents the chemical changes and mechanical
effects produced by pH and flow on corrosion films through the combination of
molecular techniques with imaging. Siderite, wustite and magnetite were
identified as corrosion products at neutral pH, which dissolved and mechanical
damaged at low pH by a 1m/s brine flow with a crystal size reduction of ~80\%.
In contrast, at pH 7 and 1m/s flow facilitated the removal of entire crystals
from the film.

\subsection*{\href{http://arxiv.org/abs/2009.11709v1}{Comment on "Physics without determinism: Alternative interpretations of  classical physics", Phys. Rev. A, 100:062107, Dec 2019}}
\subsubsection*{Luca Callegaro, and Walter Bich (2020-09-24)}
The paper "Physics without determinism: Alternative interpretations of
classical physics" [Phys. Rev. A, 100:062107, Dec 2019] defines finite
information quantities (FIQ). A FIQ expresses the available information about
the value of a physical quantity. We show that a change in the measurement unit
does not preserve the information carried by a FIQ, and therefore that the
definition provided in the paper is not complete.

\subsection*{\href{http://arxiv.org/abs/2009.11695v1}{Correlation of microdistortions with misfit volumes in High Entropy  Alloys}}
\subsubsection*{Wolfram Georg Nöhring and W. A. Curtin (2020-09-24)}
The yield strengths of High Entropy Alloys have recently been correlated with
measured picometer-scale atomic distortions. Here, the root mean square
microdistortion in a multicomponent alloy is shown to be nearly proportional to
the misfit-volume parameter that enters into a predictive model of solute
strengthening. Analysis of two model ternary alloy families, face-centered
cubic Cr-Fe-Ni and body-centered cubic Nb-Mo-V, demonstrates the correlation
over a wide composition space. The reported correlation of yield strength with
microdistortion is thus a consequence of the correlation between
microdistortion and misfit parameter and the derived dependence of yield
strength on the misfit parameter.

\subsection*{\href{http://arxiv.org/abs/2009.11691v1}{Experimentally friendly approach towards nonlocal correlations in  multisetting N -partite Bell scenarios}}
\subsubsection*{Artur Barasiński, \dots, and Jan Soubusta (2020-09-24)}
In this work, we study a recently proposed operational measure of nonlocality
which describes the probability of violation of local realism under randomly
sampled observables, and the strength of such violation as described by
resistance to white noise admixture. While our knowledge concerning these
quantities is well established from a theoretical point of view, the
experimental counterpart is a considerably harder task and very little has been
done in this field. It is caused by the lack of complete knowledge about the
local polytope required for the analysis. In this paper, we propose a simple
procedure towards experimentally determining both quantities, based on the
incomplete set of tight Bell inequalities. We show that the imprecision arising
from this approach is of similar magnitude as the potential measurement errors.
We also show that even with both a randomly chosen N -qubit pure state and
randomly chosen measurement bases, a violation of local realism can be detected
experimentally almost 100\% of the time. Among other applications, our work
provides a feasible alternative for the witnessing of genuine multipartite
entanglement without aligned reference frames.

\subsection*{\href{http://arxiv.org/abs/2009.11687v1}{Radiation Eigenmodes of Dicke Superradiance}}
\subsubsection*{Benjamin Lemberger and Klaus Mølmer (2020-09-24)}
We calculate the field eigenmodes of the superradiant emission from an
ensemble of $N$ two-level atoms. While numerical techniques are effective due
to the symmetry of the problem, we develop also an analytical method to
approximates the modes in the limit of a large number of emitters. We find that
Dicke superradiant emission is restricted to a small number of modes, with a
little over 90\% of the photons emitted in a single dominant mode.

\subsection*{\href{http://arxiv.org/abs/2009.11683v1}{Magnetic anisotropy and exchange paths for octa- and tetrahedrally  coordinated Mn$^{2+}$ ions in the honeycomb multiferroic Mn$_2$Mo$_3$O$_8$}}
\subsubsection*{D. Szaller, \dots, and I. Kézsmárki (2020-09-24)}
We investigated the static and dynamic magnetic properties of the polar
ferrimagnet Mn$_2$Mo$_3$O$_8$ in three magnetically ordered phases via
magnetization, magnetic torque, and THz absorption spectroscopy measurements.
The observed magnetic field dependence of the spin-wave resonances, including
Brillouin zone-center and zone-boundary excitations, magnetization, and torque,
are well described by an extended two-sublattice antiferromagnetic classical
mean-field model. In this orbitally quenched system, the competing weak
easy-plane and easy-axis single-ion anisotropies of the two crystallographic
sites are determined from the model and assigned to the tetra- and octahedral
sites, respectively, by ab initio calculations.

\subsection*{\href{http://arxiv.org/abs/2009.11672v1}{Accurate characterization of tip-induced potential using electron  interferometry}}
\subsubsection*{A. Iordanescu, \dots, and B. Brun (2020-09-24)}
Using the tip of a scanning probe microscope as a local electrostatic gate
gives access to real space information on electrostatics as well as charge
transport at the nanoscale, provided that the tip-induced electrostatic
potential is well known. Here, we focus on the accurate characterization of the
tip potential, in a regime where the tip locally depletes a two-dimensional
electron gas (2DEG) hosted in a semiconductor heterostructure. Scanning the tip
in the vicinity of a quantum point contact defined in the 2DEG, we observe
Fabry-P\'erot interference fringes at low temperature in maps of the device
conductance. We exploit the evolution of these fringes with the tip voltage to
measure the change in depletion radius by electron interferometry. We find that
a semi-classical finite-element self-consistent model taking into account the
conical shape of the tip reaches a faithful correspondence with the
experimental data.

\subsection*{\href{http://arxiv.org/abs/2009.11671v1}{Double Parameters Modulated Optical Lattice Clock: Interference and  Topology}}
\subsubsection*{Xiao-Tong Lu, \dots, and Hong Chang (2020-09-24)}
The quantum system under periodical modulation is the simplest path to
understand the quantum non-equilibrium system, because it can be well described
by the effective static Floquet Hamiltonian. Under the stroboscopic
measurement, the initial phase is usually irrelevant. However, if two
uncorrelated parameters are modulated, their relative phase can not be gauged
out, so that it can dramatically change physics. Here, we simultaneously
modulate the frequency of lattice laser and Rabi frequency in the optical
lattice clock (OLC) system. Thanks to ultra-high precision and ultra-stability
of OLC, the relative phase could be fine-tuned. As a smoking gun, we observed
the interference between two Floquet channels. At last, we discuss the relation
between effective Floquet Hamiltonian and 1-D topological insulator with high
winding numbers. Our experiment not only provides a direction for detecting the
phase effect, but also paves a way in simulating quantum topological phase in
OLC platform.

\subsection*{\href{http://arxiv.org/abs/2009.11645v1}{Universal moiré nematic phase in twisted graphitic systems}}
\subsubsection*{Carmen Rubio-Verdú, \dots, and Abhay N. Pasupathy (2020-09-24)}
Graphene moir\'e superlattices display electronic flat bands. At integer
fillings of these flat bands, energy gaps due to strong electron-electron
interactions are generally observed. However, the presence of other
correlation-driven phases in twisted graphitic systems at non-integer fillings
is unclear. Here, we report scanning tunneling microscopy (STM) measurements
that reveal the existence of threefold rotational (C3) symmetry breaking in
twisted double bilayer graphene (tDBG). Using spectroscopic imaging over large
and uniform areas to characterize the direction and degree of C3 symmetry
breaking, we find it to be prominent only at energies corresponding to the flat
bands and nearly absent in the remote bands. We demonstrate that the C3
symmetry breaking cannot be explained by heterostrain or the displacement
field, and is instead a manifestation of an interaction-driven electronic
nematic phase, which emerges even away from integer fillings. Comparing our
experimental data with a combination of microscopic and phenomenological
modeling, we show that the nematic instability is not associated with the local
scale of the graphene lattice, but is an emergent phenomenon at the scale of
the moir\'e lattice, pointing to the universal character of this ordered state
in flat band moir\'e materials.

\subsection*{\href{http://arxiv.org/abs/2009.11624v1}{Properties of self gravitating quasi-stationary states}}
\subsubsection*{Francesco Sylos Labini and Roberto Capuzzo-Dolcetta (2020-09-24)}
Initially far out-of-equilibrium self-gravitating systems form, through a
collisionless relaxation dynamics, quasi-stationary states (QSS). These may
arise from a bottom-up aggregation of structures or in a top-down frame; their
quasi-equilibrium properties are well described by the Jeans equation and are
not universal, i.e. they depend on initial conditions. To understand the origin
of such dependence, we present results of numerical experiments of initially
cold and spherical systems characterized by various choices of the spectrum of
initial density fluctuations. The amplitude of such fluctuations determines
whether the system relaxes in a top-down or a bottom-up manner. We find that
statistical properties of the resulting QSS mainly depend upon the amount of
energy exchanged during the formation process. In particular, in the violent
top-down collapses the energy exchange is large and the QSS show an inner core
with an almost flat density profile and a quasi Maxwell-Boltzmann (isotropic)
velocity distribution, while their outer regions display a density profile
$\rho(r) \propto r^{-\alpha}$ ($\alpha >0$) with radially elongated orbits. We
analytically show that $\alpha=4$ in agreement with numerical experiments. In
the less violent bottom-up dynamics, the energy exchange is much smaller, the
orbits are less elongated and $0< \alpha(r) \le 4$, with a a density profile
well fitted by the Navarro-Frenk-White behavior. Such a dynamical evolution is
shown by both non-uniform spherical isolated systems and by halos extracted
from cosmological simulations. We consider the relation of these results with
the core-cusp problem concluding that this is naturally solved if galaxies form
through a monolithic collapse.

\subsection*{\href{http://arxiv.org/abs/2009.11594v1}{Feedback-type thermoelectric effect in correlated solids}}
\subsubsection*{Yugo Onishi and Naoto Nagaosa (2020-09-24)}
Thermoelectric effect using a Maxwell's demon-like feedback mechanism is
proposed. A specific model for the thermoelectric effect in solid is formulated
in the framework of stochastic thermodynamics and the response to the electric
field and temperature gradient is calculated. The results show that relatively
high $ZT$ can be achieved within a range of realistic parameters.

\subsection*{\href{http://arxiv.org/abs/2009.11591v1}{Calculation of the Residual Entropy of Ice Ih by Monte Carlo simulation  with the Combination of the Replica-Exchange Wang-Landau algorithm and  Multicanonical Replica-Exchange Method}}
\subsubsection*{Takuya Hayashi, and Yuko Okamoto (2020-09-24)}
We estimated the residual entropy of ice Ih by the recently developed
simulation protocol, namely, the combination of Replica-Exchange Wang-Landau
algorithm and Multicanonical Replica-Exchange Method. We employed a model with
the nearest neighbor interactions on the three-dimensional hexagonal lattice,
which satisfied the ice rules in the ground state. The results showed that our
estimate of the residual entropy is found to be within 0.038 \% of series
expansion estimate by Nagle and within 0.000077 \% of PEPS algorithm by
Vanderstraeten. In this article, we not only give our latest estimate of the
residual entropy of ice Ih but also discuss the importance of the uniformity of
a random number generator in MC simulations.

\subsection*{\href{http://arxiv.org/abs/2009.11586v1}{Poissonian twin beam states and the effect of symmetrical photon  subtraction in loss estimations}}
\subsubsection*{N. Samantaray, and J. G. Rarity (2020-09-24)}
We have devised an experimentally realizable model generating twin beam
states whose individual beam photon statistics are varied from thermal to
Poissonian keeping the non-classical mode correlation intact. We have studied
the usefulness of these states for loss measurement by considering three
different estimators, comparing with the correlated thermal twin beam states
generated from spontaneous parametric down conversion or four-wave mixing. We
then incorporated the photon subtraction operation into the model and
demonstrate their advantage in loss estimations with respect to un-subtracted
states at both fixed squeezing and per photon exposure of the absorbing sample.
For instance, at fixed squeezing, for two photon subtraction, up to three times
advantage is found. In the latter case, albeit the advantage due to photon
subtraction mostly subsides in standard regime, an unexpected result is that in
some operating regimes the photon subtraction scheme can also give up to 20\%
advantage over the correlated Poisson beam result. We have also made a
comparative study of these estimators for finding the best measurement for loss
estimations. We present results for all the values of the model parameters
changing the statistics of twin beam states from thermal to Poissonian.

\subsection*{\href{http://arxiv.org/abs/2009.11585v1}{Large magnetoresistance in the iron-free pnictide superconductor  LaRu$_2$P$_2$}}
\subsubsection*{Marta Fernández-Lomana, \dots, and Isabel Guillamón (2020-09-24)}
The magnetoresistance of iron pnictide superconductors is often dominated by
electron-electron correlations and deviates from the H$^2$ or saturating
behaviors expected for uncorrelated metals. Contrary to similar Fe-based
pnictide systems, the superconductor LaRu$_2$P$_2$ (T$_c$ = 4 K) shows no
enhancement of electron-electron correlations. Here we report a non-saturating
magnetoresistance deviating from the H$^2$ or saturating behaviors in
LaRu$_2$P$_2$. We have grown and characterized high quality single crystals of
LaRu$_2$P$_2$ and measured a magnetoresistance following H$^{1.3}$ up to 22 T.
We discuss our result by comparing the bandstructure of LaRu$_2$P$_2$ with Fe
based pnictide superconductors. The different orbital structures of Fe and Ru
leads to a 3D Fermi surface with negligible bandwidth renormalization in
LaRu$_2$P$_2$, that contains a large open sheet over the whole Brillouin zone.
We show that the large magnetoresistance in LaRu$_2$P$_2$ is unrelated to the
one obtained in materials with strong electron-electron correlations and that
it is compatible instead with conduction due to open orbits on the rather
complex Fermi surface structure of LaRu$_2$P$_2$.

\subsection*{\href{http://arxiv.org/abs/2009.11584v1}{Beyond Graphene: Low-Symmetry and Anisotropic 2D Materials}}
\subsubsection*{Salvador Barraza-Lopez, \dots, and Han Wang (2020-09-24)}
Low-symmetry 2D materials---such as ReS$_2$ and ReSe$_2$ monolayers, black
phosphorus monolayers, group-IV monochalcogenide monolayers, borophene, among
others---have more complex atomistic structures than the honeycomb lattices of
graphene, hexagonal boron nitride, and transition metal dichalcogenides. The
reduced symmetries of these emerging materials give rise to inhomogeneous
electron, optical, valley, and spin responses, as well as entirely new
properties such as ferroelasticity, ferroelectricity, magnetism, spin-wave
phenomena, large nonlinear optical properties, photogalvanic effects, and
superconductivity. Novel electronic topological properties, nonlinear elastic
properties, and structural phase transformations can also take place due to low
symmetry. The "Beyond Graphene: Low-Symmetry and Anisotropic 2D Materials"
Special Topic was assembled to highlight recent experimental and theoretical
research on these emerging materials.

\subsection*{\href{http://arxiv.org/abs/2009.11574v1}{Phase diagram of multi-layer ferromagnet system with dipole-dipole  interaction}}
\subsubsection*{Taichi Hinokihara and Seiji Miyashita (2020-09-24)}
We investigate various magnetic configurations caused by the dipole-dipole
interaction (DDI) in the thin-film magnet with the perpendicular anisotropy
under the open boundary conditions. Two different approaches are simulated: one
starts from a random magnetic configuration and decreases temperatures
step-wisely; the other starts from the saturated out-of-plane ferromagnetic
state to evaluate its metastability. As typical patterns of magnetic
configuration, five typical configurations are found: an out-of-plane
ferromagnetic, in-plane ferromagnetic, vortex, multi-domain, and canted
multi-domain states. Notably, the canted multi-domain forms a concentric
magnetic-domain-pattern with an in-plane vortex structure, resulting from the
open boundary conditions. Concerning to the coercivity, a comparison of the
magnetic configurations in both processes reveals that the out-of-plane
ferromagnetic state exhibits metastability in the multi-domain state, while not
in the vortex state. We also confirm that the so-called Neel-cap
magnetic-domain-wall structure, which is originally discussed in the in-plane
anisotropy system, appears at the multi-domain state.

\subsection*{\href{http://arxiv.org/abs/2009.11573v1}{Thermoelectric properties of finite two dimensional triangular lattices  coupled to electrodes}}
\subsubsection*{David M T Kuo (2020-09-24)}
Novel intrinsic two-dimensional materials have attracted many researchers'
attention. The unusual transport and optical properties of these materials
originate mainly from triangular lattices (TLs). Therefore, the application of
energy harvesting calls for a study of the thermoelectric properties of 2D TLs
coupled to electrodes. The transmission coefficient of 2D TLs is calculated by
using the Green's function technique to treat ballistic transports. Especially
important among our findings is the electron-hole asymmetric behavior of the
power factor ($PF$). Specifically, the maximum $PF$ of electrons is
significantly larger than that of holes. At room temperature, the maximum $PF$
of electrons is dictated by the position of the chemical potential of
electrodes near the band edge of TLs. The enhancement of $PF$ with increasing
electronic states results from the enhancement of electrical conductance and
constant Seebeck coefficient. When the band gap is ten times larger than the
thermal energy, it is appropriate to make one-band model predictions for
thermoelectric optimization.

\subsection*{\href{http://arxiv.org/abs/2009.11569v1}{IrO2 Surface Complexions Identified Through Machine Learning and Surface  Investigations}}
\subsubsection*{Jakob Timmermann, \dots, and Karsten Reuter (2020-09-24)}
A Gaussian Approximation Potential (GAP) was trained using density-functional
theory data to enable a global geometry optimization of low-index rutile IrO2
facets through simulated annealing. Ab initio thermodynamics identifies (101)
and (111) (1x1)-terminations competitive with (110) in reducing environments.
Experiments on single crystals find that (101) facets dominate, and exhibit the
theoretically predicted (1x1) periodicity and X-ray photoelectron spectroscopy
(XPS) core level shifts. The obtained structures are analogous to the
complexions discussed in the context of ceramic battery materials.

\subsection*{\href{http://arxiv.org/abs/2009.11541v1}{Analytical Model for the Transient Permittivity of Uncured TiO2  Whisker/Liquid Silicone Rubber Composites Under an AC Electric Field}}
\subsubsection*{Zikui Shen, \dots, and Zhidong Jia (2020-09-24)}
The electric field grading of dielectric permittivity gradient devices is an
effective way of enhancing their insulation properties. The in-situ electric
field-driven assembly is an advanced method for the fabrication of insulation
devices with adaptive permittivity gradients, however, there is no theoretical
guidance for design. In this work, an analytical model with a time constant is
developed to determine the transient permittivity of uncured composites under
an applied AC electric field. This model is based on optical image and
dielectric permittivity monitoring, which avoids the direct processing of
complex electrodynamics. For a composite with given components, the increased
filler content and electric field strength can accelerate the transient
process. Compared with the finite element method (FEM) based on differential
equations, this statistical model is simple but efficient, and can be applied
to any low-viscosity uncured composites, which may contain multiple fillers.
More importantly, when a voltage is applied to an uncured composite insulating
device, the proposed model can be used to analyse the spatiotemporal
permittivity characteristics of this device and optimise its permittivity
gradient for electric field grading.

\subsection*{\href{http://arxiv.org/abs/2009.11539v1}{Quantum state rotation}}
\subsubsection*{Yonghae Lee, and Soojoon Lee (2020-09-24)}
Quantum state exchange is a quantum communication task for two users in which
the users exchange their respective quantum information in the asymptotic
scenario. In this work, we generalize the quantum state exchange task to a
quantum communication task for $M$ users in which the users rotate their
respective quantum information. We assume that every two users may share
entanglement resources, and they use local operations and classical
communication in order to perform the task. We call this generalized task the
quantum state rotation. First of all, we formally define the quantum state
rotation task and its optimal entanglement cost, which means the least amount
of total entanglement resources required to carry out the task. We then present
lower and upper bounds on the optimal entanglement cost, and provide conditions
for zero optimal entanglement costs. Based on these results, we find out a
difference between the quantum state rotation task and the quantum state
exchange task.

\subsection*{\href{http://arxiv.org/abs/2009.11529v1}{Quantum phase transition in a quantum Ising chain at nonzero  temperatures}}
\subsubsection*{K. L. Zhang and Z. Song (2020-09-24)}
We study the response of a thermal state of an Ising chain to a nonlocal
non-Hermitian perturbation, which coalesces the topological Kramer-like
degeneracy in the ferromagnetic phase. The dynamic responses for initial
thermal states in different quantum phases are distinct. The final state always
approaches its half component with a fixed parity in the ferromagnetic phase
but remains almost unchanged in the paramagnetic phase. This indicates that the
phase diagram at zero temperature is completely preserved at finite
temperatures. Numerical simulations for Loschmidt echoes demonstrate such
dynamical behaviors in finite-size systems. In addition, it provides a clear
manifestation of the bulk-boundary correspondence at nonzero temperatures. This
work presents an alternative approach to understanding the quantum phase
transitions of quantum spin systems at nonzero temperatures.

\subsection*{\href{http://arxiv.org/abs/2009.11518v1}{Sample optimal Quantum identity testing via Pauli Measurements}}
\subsubsection*{Nengkun Yu (2020-09-24)}
In this paper, we show that
$\Theta(\mathrm{poly}(n)\cdot\frac{4^n}{\epsilon^2})$ is the sample complexity
of testing whether two $n$-qubit quantum states $\rho$ and $\sigma$ are
identical or $\epsilon$-far in trace distance using two-outcome Pauli
measurements.

\subsection*{\href{http://arxiv.org/abs/2009.11517v1}{Threshold conduction in amorphous phase change materials: effects of  temperature}}
\subsubsection*{J C Martinez, \dots, and R E Simpson (2020-09-24)}
We emphasize the role of temperature in explaining the IV ovonic threshold
switching curve of amorphous phase change materials. The Poole-Frankel
conduction model is supplemented by considering effects of temperature on the
conductivity in amorphous materials and we find agreement with a wide variety
of available data. This leads to a simple explanation of the snapback in
threshold switching. We also argue that low frequency current noise in the
amorphous state originates from trains of moving charge carriers derailing and
restarting due to the different local structures within the amorphous material.

\subsection*{\href{http://arxiv.org/abs/2009.11512v1}{Anomalous Hall effect triggered by pressure-induced magnetic phase  transition in $α$-Mn}}
\subsubsection*{Kazuto Akiba, \dots, and Tatsuo C. Kobayashi (2020-09-24)}
Recent interest in topological nature in condensed matter physics has
revealed the essential role of Berry curvature in anomalous Hall effect (AHE).
However, since large Hall response originating from Berry curvature has been
reported in quite limited materials, the detailed mechanism remains unclear at
present. Here, we report the discovery of a large AHE triggered by a
pressure-induced magnetic phase transition in elemental $\alpha$-Mn. The AHE is
absent in the non-collinear antiferromagnetic phase at ambient pressure,
whereas a large AHE is observed in the weak ferromagnetic phase under high
pressure despite the small averaged moment of $\sim 0.02 \mu_B$/Mn. Our results
indicate that the emergence of the AHE in $\alpha$-Mn is governed by the
symmetry of the underlying magnetic structure, providing a direct evidence of a
switch between a zero and non-zero contribution of the Berry curvature across
the phase boundary. $\alpha$-Mn can be an elemental and tunable platform to
reveal the role of Berry curvature in AHE.

\subsection*{\href{http://arxiv.org/abs/2009.11511v1}{Magneto-transport properties of tellurium under extreme conditions}}
\subsubsection*{Kazuto Akiba, \dots, and Masashi Tokunaga (2020-09-24)}
This study investigates the transport properties of a chiral elemental
semiconductor tellurium (Te) under magnetic fields and pressure. Application of
hydrostatic pressure reduces the resistivity of Te, while its temperature
dependence remains semiconducting up to 4 GPa, contrary to recent theoretical
and experimental studies. Application of higher pressure causes structural as
well as semiconductor--metal transitions. The resulting metallic phase above 4
GPa exhibits superconductivity at 2 K along with a noticeable linear
magnetoresistance effect. On the other hand, at ambient pressure, we identified
metallic surface states on the as-cleaved (10$\bar{1}$0) surfaces of Te. The
nature of these metallic surface states has been systematically studied by
analyzing quantum oscillations observed in high magnetic fields. We clarify
that a well-defined metallic surface state exists not only on chemically etched
samples that were previously reported, but also on as-cleaved ones.

\subsection*{\href{http://arxiv.org/abs/2009.11502v1}{Relaxation of the Excited Rydberg States of Surface Electrons on Liquid  Helium}}
\subsubsection*{Erika Kawakami, and Denis Konstantinov (2020-09-24)}
We report the first direct observation of the decay of the excited-state
population in electrons trapped on the surface of liquid helium. The relaxation
dynamics, which are governed by inelastic scattering processes in the system,
are probed by the real-time response of the electrons to a pulsed microwave
excitation. Comparison with theoretical calculations allows us to establish the
dominant mechanisms of inelastic scattering for different temperatures. The
longest measured relaxation time is around 1 us at the lowest temperature of
135 mK, which is determined by the inelastic scattering due to the spontaneous
two-ripplon emission process. Furthermore, the image-charge response shortly
after applying microwave radiation reveals interesting population dynamics due
to the multisubband structure of the system.

\subsection*{\href{http://arxiv.org/abs/2009.11495v1}{Heterostructures of hetero-stack of 2D TMDs (MoS2, WS2 and ReS2) and BN}}
\subsubsection*{Badri Vishal, \dots, and Ranjan Datta (2020-09-24)}
In this manuscript, we describe optical emission of heterostructure of
hetero-stack between 2D TMDs (MoS2, WS2, and ReS2) and BN. Similar to our
previous results on the stack of similar type of TMDs, intense PL emission peak
is observed around 2.13 eV but is split around 2.13eV into two or more peaks
depending on the different stack of TMDs with BN. The transitions from the
valence band of BN to conduction bands of different TMD stacks due to quantum
coupling and specific orientation explain the strong peak in the PL spectra.

\subsection*{\href{http://arxiv.org/abs/2009.11489v2}{Electronic correlations in the semiconducting half-Heusler compound  FeVSb}}
\subsubsection*{Estiaque H. Shourov, \dots, and Jason K. Kawasaki (2020-09-24)}
Electronic correlations are crucial to the low energy physics of metallic
systems with localized $d$ and $f$ states; however, their effect on band
insulators and semiconductors is typically negligible. Here, we measure the
electronic structure of the half-Heusler compound FeVSb, a band insulator with
filled shell configuration of 18 valence electrons per formula unit ($s^2 p^6
d^{10}$). Angle-resolved photoemission spectroscopy (ARPES) reveals a mass
renormalization of $m^{*}/m_{bare}= 1.4$, where $m^{*}$ is the measured
effective mass and $m_{bare}$ is the mass from density functional theory (DFT)
calculations with no added on-site Coulomb repulsion. Our measurements are in
quantitative agreement with dynamical mean field theory (DMFT) calculations,
highlighting the many-body origin of the mass renormalization. This mass
renormalization lies in dramatic contrast to other filled shell intermetallics,
including the thermoelectric materials CoTiSb and NiTiSn; and has a similar
origin to that in FeSi, where Hund's coupling induced fluctuations across the
gap can explain a dynamical self-energy and correlations. Our work calls for a
re-thinking of the role of correlations and Hund's coupling in intermetallic
band insulators.

\subsection*{\href{http://arxiv.org/abs/2009.11488v1}{Pressure-Temperature Phase Diagram of $α$-Mn}}
\subsubsection*{Takaaki Sato, \dots, and Tatsuo C. Kobayashi (2020-09-24)}
Electrical resistivity and ac-susceptibility measurements under high pressure
were carried out in high-quality single crystals of $\alpha$-Mn. The
pressure-temperature phase diagram consists of an antiferromagnetic ordered
phase (0<$P$<1.4 GPa, $T<T_{\rm N}$), a pressure-induced ordered phase
(1.4<$P$<4.2-4.4 GPa, $T<T_{\rm A}$), and a paramagnetic phase. A significant
increase was observed in the temperature dependence of ac-susceptibility at
$T_{\rm A}$, indicating that the pressure-induced ordered phase has a
spontaneous magnetic moment. Ferrimagnetic order and parasitic ferromagnetism
are proposed as candidates for a possible magnetic structure. At the critical
pressure, where the pressure-induced ordered phase disappears, the temperature
dependence of the resistivity below 10 K is proportional to $T^{5/3}$. This
non-Fermi liquid behavior suggests the presence of pronounced magnetic
fluctuation.

\subsection*{\href{http://arxiv.org/abs/2009.11482v1}{Fault-Tolerant Operation of a Quantum Error-Correction Code}}
\subsubsection*{Laird Egan, \dots, and Christopher Monroe (2020-09-24)}
Quantum error correction protects fragile quantum information by encoding it
in a larger quantum system whose extra degrees of freedom enable the detection
and correction of errors. An encoded logical qubit thus carries increased
complexity compared to a bare physical qubit. Fault-tolerant protocols contain
the spread of errors and are essential for realizing error suppression with an
error-corrected logical qubit. Here we experimentally demonstrate
fault-tolerant preparation, rotation, error syndrome extraction, and
measurement on a logical qubit encoded in the 9-qubit Bacon-Shor code. For the
logical qubit, we measure an average fault-tolerant preparation and measurement
error of 0.6\% and a transversal Clifford gate with an error of 0.3\% after error
correction. The result is an encoded logical qubit whose logical fidelity
exceeds the fidelity of the entangling operations used to create it. We compare
these operations with non-fault-tolerant protocols capable of generating
arbitrary logical states, and observe the expected increase in error. We
directly measure the four Bacon-Shor stabilizer generators and are able to
detect single qubit Pauli errors. These results show that fault-tolerant
quantum systems are currently capable of logical primitives with error rates
lower than their constituent parts. With the future addition of intermediate
measurements, the full power of scalable quantum error-correction can be
achieved.

\subsection*{\href{http://arxiv.org/abs/2009.11481v1}{Algorithmic Thresholds in Mean Field Spin Glasses}}
\subsubsection*{Ahmed El Alaoui and Andrea Montanari (2020-09-24)}
Optimizing a high-dimensional non-convex function is, in general,
computationally hard and many problems of this type are hard to solve even
approximately. Complexity theory characterizes the optimal approximation ratios
achievable in polynomial time in the worst case. On the other hand, when the
objective function is random, worst case approximation ratios are overly
pessimistic. Mean field spin glasses are canonical families of random energy
functions over the discrete hypercube $\{-1,+1\}^N$. The near-optima of these
energy landscapes are organized according to an ultrametric tree-like
structure, which enjoys a high degree of universality. Recently, a precise
connection has begun to emerge between this ultrametric structure and the
optimal approximation ratio achievable in polynomial time in the typical case.
A new approximate message passing (AMP) algorithm has been proposed that
leverages this connection. The asymptotic behavior of this algorithm has been
analyzed, conditional on the nature of the solution of a certain variational
problem.
  In this paper we describe the first implementation of this algorithm and the
first numerical solution of the associated variational problem. We test our
approach on two prototypical mean-field spin glasses: the
Sherrington-Kirkpatrick (SK) model, and the $3$-spin Ising spin glass. We
observe that the algorithm works well already at moderate sizes ($N\gtrsim
1000$) and its behavior is consistent with theoretical expectations. For the SK
model it asymptotically achieves arbitrarily good approximations of the global
optimum. For the $3$-spin model, it achieves a constant approximation ratio
that is predicted by the theory, and it appears to beat the `threshold energy'
achieved by Glauber dynamics. Finally, we observe numerically that the
intermediate states generated by the algorithm have the properties of ancestor
states in the ultrametric tree.

\subsection*{\href{http://arxiv.org/abs/2009.11480v1}{Quantum Formation of Topological Defects}}
\subsubsection*{Mainak Mukhopadhyay, and George Zahariade (2020-09-24)}
We consider quantum phase transitions with global symmetry breakings that
result in the formation of topological defects. We evaluate the number
densities of kinks, vortices, and monopoles that are produced in $d=1,2,3$
spatial dimensions respectively and find that they scale as $t^{-d/2}$ and
evolve towards attractor solutions that are independent of the quench
timescale. For $d=1$ our results apply in the region of parameters $\lambda
\tau/m \ll 1$ where $\lambda$ is the quartic self-interaction of the order
parameter, $\tau$ is the quench timescale, and $m$ the mass parameter.

\subsection*{\href{http://arxiv.org/abs/2009.11477v1}{Comparative study of the density matrix embedding theory for the Hubbard  models}}
\subsubsection*{Masataka Kawano and Chisa Hotta (2020-09-24)}
We examine the performance of the density matrix embedding theory (DMET)
recently proposed in [G. Knizia and G. K.-L. Chan, Phys. Rev. Lett. 109, 186404
(2012)]. The core of this method is to find a proper one-body potential that
generates a good trial wave function for projecting a large scale original
Hamiltonian to a local subsystem with a small number of basis. The resultant
ground state of the projected Hamiltonian can locally approximate the true
ground state. However, the lack of the variational principle makes it difficult
to judge the quality of the choice of the potential. Here we focus on the
entanglement spectrum (ES) as a judging criterion; accurate evaluation of the
ES guarantees that the corresponding reduced density matrix well reproduces all
physical quantities on the local subsystem. We apply the DMET to the Hubbard
model on the one-dimensional chain, zigzag chain, and triangular lattice and
test several variants of potentials and cost functions. It turns out that a
symmetric potential reproduces the ES of the phase that continues from a
noninteracting limit. The Mott transition as well as symmetry-breaking
transitions can be detected by the singularities in the ES. However, the
details of the ES in the strongly interacting parameter region depends much on
these choices, meaning that the present algorithm allowing for numerous
variants of the DMET is insufficient to fully fix the quantum many-body
problem.

\subsection*{\href{http://arxiv.org/abs/2009.11472v1}{Strain Tunable Semimetal-Topological-Insulator Transition in Monolayer  1T'-WTe2}}
\subsubsection*{Chenxiao Zhao, \dots, and Jinfeng Jia (2020-09-24)}
A quantum spin hall insulator(QSHI) is manifested by its conducting edge
channels that originate from the nontrivial topology of the insulating bulk
states. Monolayer 1T'-WTe2 exhibits this quantized edge conductance in
transport measurements, but because of its semimetallic nature, the coherence
length is restricted to around 100 nm. To overcome this restriction, we propose
a strain engineering technique to tune the electronic structure, where either a
compressive strain along a axis or a tensile strain along b axis can drive
1T'-WTe2 into an full gap insulating phase. A combined study of molecular beam
epitaxy and in-situ scanning tunneling microscopy/spectroscopy then confirmed
such a phase transition. Meanwhile, the topological edge states were found to
be very robust in the presence of strain.

\subsection*{\href{http://arxiv.org/abs/2009.11470v1}{Tunable layered-magnetism-assisted magneto-Raman effect in a  two-dimensional magnet $\mathrm{CrI_3}$}}
\subsubsection*{Wencan Jin, \dots, and Liuyan Zhao (2020-09-24)}
We use a combination of polarized Raman spectroscopy experiment and model
magnetism-phonon coupling calculations to study the rich magneto-Raman effect
in the two-dimensional (2D) magnet $\mathrm{CrI_3}$. We reveal a novel
layered-magnetism-assisted phonon scattering mechanism below the magnetic onset
temperature, whose Raman excitation breaks time-reversal symmetry, has an
antisymmetric Raman tensor, and follows the magnetic phase transitions across
critical magnetic fields, on top of the presence of the conventional phonon
scattering with symmetric Raman tensors in $N$-layer $\mathrm{CrI_3}$. We
resolve in data and by calculations that the 1st-order $A_g$ phonon of
monolayer splits into a $N$-fold multiplet in $N$-layer $\mathrm{CrI_3}$ due to
the interlayer coupling ($N$>=2) and that the phonons with the multiple show
distinct magnetic field dependence because of their different
layered-magnetism-phonon coupling. We further find that such a
layered-magnetism-phonon coupled Raman scattering mechanism extends beyond
1st-order to higher-order multi-phonon scattering processes. Our results on
magneto-Raman effect of the 1st-order phonons in the multiplet and the
higher-order multi-phonons in $N$-layer $\mathrm{CrI_3}$ demonstrate the rich
and strong behavior of emergent magneto-optical effects in 2D magnets and
underlines the unique opportunities of new spin-phonon physics in van der Waals
layered magnets.

\subsection*{\href{http://arxiv.org/abs/2009.11451v1}{Shear response of granular packings compressed above jamming onset}}
\subsubsection*{Philip Wang, \dots, and Corey S. O'Hern (2020-09-24)}
We investigate the mechanical response of jammed packings of repulsive,
frictionless spherical particles undergoing isotropic compression. Prior
simulations of the soft-particle model, where the repulsive interactions scale
as a power-law in the interparticle overlap with exponent $\alpha$, have found
that the ensemble-averaged shear modulus $\langle G \rangle$ increases with
pressure $P$ as $\sim P^{(\alpha-3/2)/(\alpha-1)}$ at large pressures. However,
a deep theoretical understanding of this scaling behavior is lacking. We show
that the shear modulus of jammed packings of frictionless, spherical particles
has two key contributions: 1) continuous variations as a function of pressure
along geometrical families, for which the interparticle contact network does
not change, and 2) discontinuous jumps during compression that arise from
changes in the contact network. We show that the shear modulus of the first
geometrical family for jammed packings can be collapsed onto a master curve:
$G^{(1)}/G_0 = (P/P_0)^{(\alpha-2)/(\alpha-1)} - P/P_0$, where $P_0 \sim
N^{-2(\alpha-1)}$ is a characteristic pressure that separates the two power-law
scaling regions and $G_0 \sim N^{-2(\alpha-3/2)}$. Deviations from this form
can occur when there is significant non-affine particle motion near changes in
the contact network. We further show that $\langle G (P)\rangle$ is not simply
a sum of two power-laws, but $\langle G \rangle \sim (P/P_c)^a$, where $a
\approx (\alpha -2)/(\alpha-1)$ in the $P \rightarrow 0$ limit and $\langle G
\rangle \sim (P/P_c)^b$, where $b \gtrsim (\alpha -3/2)/(\alpha-1)$ above a
characteristic pressure $P_c$. In addition, the magnitudes of both
contributions to $\langle G\rangle$ from geometrical families and changes in
the contact network remain comparable in the large-system limit for $P >P_c$.

\subsection*{\href{http://arxiv.org/abs/2009.11439v1}{The "isothermal" compressibility of active matter}}
\subsubsection*{Austin R. Dulaney, and John F. Brady (2020-09-24)}
We demonstrate that the mechanically-defined "isothermal" compressibility
behaves as a thermodynamic-like response function for suspensions of active
Brownian particles. The compressibility computed from the active pressure - a
combination of the collision and unique swim pressures - is capable of
predicting the critical point for motility induced phase separation, as
expected from the mechanical stability criterion. We relate this mechanical
definition to the static structure factor via an active form of the
thermodynamic compressibility equation and find the two to be equivalent, as
would be the case for equilibrium systems. This equivalence indicates that
compressibility behaves like a thermodynamic response function, even when
activity is large. Finally, we discuss the importance of the phase interface
when defining an active chemical potential. Previous definitions of the active
chemical potential are shown to be accurate above the critical point but
breakdown in the coexistence region. Inclusion of the swim pressure in the
mechanical compressibility definition suggests that the interface is essential
for determining phase behavior.

\subsection*{\href{http://arxiv.org/abs/2009.11438v1}{Stückelberg interferometry using spin-orbit-coupled cold atoms in an  optical lattice}}
\subsubsection*{Shuang Liang, \dots, and Zhihao Lan (2020-09-24)}
Time evolution of spin-orbit-coupled cold atoms in an optical lattice is
studied, with a two-band energy spectrum having two avoided crossings. A force
is applied such that the atoms experience two consecutive Landau-Zener
tunnelings while transversing the avoided crossings. St\"uckelberg interference
arises from the phase accumulated during the adiabatic evolution between the
two tunnelings. This phase is gauge field-dependent and thus provides new
opportunities to measure the synthetic gauge field, which is verified via
calculation of spin transition probabilities after a double passage process.
Time-dependent and time-averaged spin probabilities are derived, in which
resonances are found. We also demonstrate chiral Bloch oscillation and rich
spin-momentum locking behavior in this system.

\subsection*{\href{http://arxiv.org/abs/2009.11432v1}{Fractional antiferromagnetic skyrmion lattice induced by anisotropic  couplings}}
\subsubsection*{Shang Gao, \dots, and Oksana Zaharko (2020-09-24)}
Magnetic skyrmions are topological solitons with a nanoscale winding spin
texture that hold promise for spintronics applications. Until now, skyrmions
have been observed in a variety of magnets that exhibit nearly parallel
alignment for the neighbouring spins, but theoretically, skyrmions with
anti-parallel neighbouring spins are also possible. The latter,
antiferromagnetic skyrmions, may allow more flexible control compared to the
conventional ferromagnetic skyrmions. Here, by combining neutron scattering and
Monte Carlo simulations, we show that a fractional antiferromagnetic skyrmion
lattice with an incipient meron character is stabilized in MnSc$_2$S$_4$
through anisotropic couplings. Our work demonstrates that the theoretically
proposed antiferromagnetic skyrmions can be stabilized in real materials and
represents an important step towards implementing the
antiferromagnetic-skyrmion based spintronic devices.

\subsection*{\href{http://arxiv.org/abs/2009.11427v1}{Giant strain gradient elasticity in SrTiO3 membranes: bending versus  stretching}}
\subsubsection*{Varun Harbola, \dots, and Harold Y. Hwang (2020-09-24)}
Young's modulus determines the mechanical loads required to elastically
stretch a material, and also, the loads required to bend it, given that bending
stretches one surface while compressing the opposite one. Flexoelectric
materials have the additional property of becoming electrically polarized when
bent. While numerous studies have characterized this flexoelectric coupling,
its impact on the mechanical response, due to the energy cost of polarization
upon bending, is largely unexplored. This intriguing contribution of strain
gradient elasticity is expected to become visible at small length scales where
strain gradients are geometrically enhanced, especially in high permittivity
insulators. Here we present nano-mechanical measurements of freely suspended
SrTiO3 membrane drumheads. We observe a striking non-monotonic thickness
dependence of Young's modulus upon small deflections. Furthermore, the modulus
inferred from a predominantly bending deformation is three times larger than
that of a predominantly stretching deformation for membranes thinner than 20
nm. In this regime we extract a giant strain gradient elastic coupling of
~2.2e-6 N, which could be used in new operational regimes of
nano-electro-mechanics.

\subsection*{\href{http://arxiv.org/abs/2009.11420v1}{Effect of the reaction medium on the progression of NaBH4 hydrolysis  reaction on fcc Co surfaces: A DFT study}}
\subsubsection*{N. Karakaya Akbas and B. Kutlu (2020-09-24)}
In this study, he effect of the reaction medium on the initial hydrogen
decomposition of the NaBH4 hydrolysis reaction in the presence of the Co
catalyst surface has been theoretically investigated using the CASTEP package
program based on density functional theory. The main purpose of the research is
to distinguish the catalytic effects of fcc cobalt surfaces on the
decomposition of hydrogen from NaBH4 and to determine the effects of the
environments formed by the hydrolysis of water at the most efficient cobalt
surface on hydrogen decomposition. Therefore, the common surface types Co(111),
Co(110), and Co(100) surfaces of the fcc cobalt crystal were used to determine
the effect of fcc surface structure on hydrogen decomposition from NaBH4 and
water decomposition. The activation energies required for hydrogen
decomposition from NaBH4 on the catalyst surface of Co(111) with the lowest
activation barrier were determined for each possible reaction medium containing
different concentrations of H* atoms, OH* radicals, and H2O* molecules. The
variation of the activation energies calculated for the reaction paths chosen
for different amounts of H* atoms and OH* radicals are in good agreement with
experimental expectations. According to our results, the value of activation
energy increases as the ratio of OH* radicals increases.

\subsection*{\href{http://arxiv.org/abs/2009.11419v1}{Higher-order Weyl superconductors with anisotropic Weyl-point  connectivity}}
\subsubsection*{W. B. Rui, \dots, and Z. D. Wang (2020-09-23)}
Weyl superconductors feature Weyl points at zero energy in the
three-dimensional (3D) Brillouin zone and arc states that connect the
projections of these Weyl points on the surface. We report that higher-order
Weyl superconductors can be realized in odd-parity topological superconductors
with time-reversal symmetry being broken by periodic driving. Different from
conventional Weyl points, the higher-order Weyl points in the bulk separate 2D
first- and second-order topological phases, while on the surface, their
projections are connected not only by conventional surface Majorana arcs, but
also by hinge Majorana arcs. We show that the Weyl-point connectivity via
Majorana arcs is largely enriched by the underlying higher-order topology and
becomes anisotropic with respect to surface orientations. We identify the
anisotropic Weyl-point connectivity as a characteristic feature of higher-order
Weyl materials. As each 2D subsystem can be singled out by fixing the periodic
driving, we propose how the Majorana zero modes in the 2D higher-order
topological phases can be detected and manipulated in experiments.

\subsection*{\href{http://arxiv.org/abs/2009.11417v1}{A state-averaged orbital-optimized hybrid quantum-classical algorithm  for a democratic description of ground and excited states}}
\subsubsection*{Saad Yalouz, \dots, and Lucas Visscher (2020-09-23)}
In the Noisy Intermediate-Scale Quantum (NISQ) era, solving the electronic
structure problem from chemistry is considered as the "killer application" for
near-term quantum devices. In spite of the success of variational hybrid
quantum/classical algorithms in providing accurate energy profiles for small
molecules, careful considerations are still required for the description of
complicated features of potential energy surfaces. Because the current quantum
resources are very limited, it is common to focus on a restricted part of the
Hilbert space (determined by the set of active orbitals). While physically
motivated, this approximation can severely impact the description of these
complicated features. A perfect example is that of conical intersections (i.e.
a singular point of degeneracy between electronic states), which are of primary
importance to understand many prominent reactions. Designing active spaces so
that the improved accuracy from a quantum computer is not rendered useless is
key to finding useful applications of these promising devices within the field
of chemistry. To answer this issue, we introduce a NISQ-friendly method called
"State-Averaged Orbital-Optimized Variational Quantum Eigensolver" (SA-OO-VQE)
which combines two algorithms: (1) a state-averaged orbital-optimizer, and (2)
a state-averaged VQE. To demonstrate the success of the method, we classically
simulate it on a minimal Schiff base model (namely the formaldimine molecule
CH2NH) relevant also for the photoisomerization in rhodopsin -- a crucial step
in the process of vision mediated by the presence of a conical intersection. We
show that merging both algorithms fulfil the necessary condition to describe
the molecule's conical intersection, i.e. the ability to treat degenerate (or
quasi-degenerate) states on the same footing.

\subsection*{\href{http://arxiv.org/abs/2009.11414v1}{Interplay of rearrangements, strain, and local structure during  avalanche propagation}}
\subsubsection*{Ge Zhang, and Andrea J. Liu (2020-09-23)}
Jammed soft disks exhibit avalanches of particle rearrangements under
quasistatic shear. We follow the avalanches using steepest descent to decompose
them into individual localized rearrangements. We characterize the local
structural environment of each particle by a machine-learned quantity,
softness, designed to be highly correlated with rearrangements, and analyze the
interplay between softness, rearrangements and strain. Our findings form the
foundation of an augmented elastoplastic model that includes local structure.

\subsection*{\href{http://arxiv.org/abs/2009.11411v1}{Dynamics of the sub-ambient gelation and shearing of solutions of P3HT  incorporated with a non-fullerene acceptor o-IDTBR towards active layer  formation in bulk heterojunction organic solar cells}}
\subsubsection*{Li Quan, \dots, and Dilhan M. Kalyon (2020-09-23)}
Organic solar cells (OSCs) containing an active layer consisting of a
nanostructured blend of a conjugated polymer like poly(3-hexylthiophene) (P3HT)
and an electron acceptor molecule have the potential of competing against
silicon-based photovoltaic panels. However, this potential is unfulfilled
primarily due to interrelated production and stability issues. The generally
employed spin coating process for fabricating organic solar cells cannot be
scaled up. Recently, He et al., have reported that the gelation of P3HT with
[6,6]-phenyl-C61-butyric acid methyl ester (PC60BM) under sub-ambient
conditions can provide a continuous extrusion/coating based route to the
processing of organic solar cells and that increases in power conversion
efficiencies (PCEs) of the P3HT/PC60BM active layer are possible under certain
shearing and thermal histories of the P3HT/PC60BM gels. Here oscillatory and
steady torsional flows were used to investigate the gel formation dynamics of
P3HT with a recently proposed non-fullerene o-IDTBR under sub-ambient
conditions. The gel strengths defined on the basis of linear viscoelastic
material functions as determined via small-amplitude oscillatory shear were
observed to be functions of the P3HT and o-IDTBR concentrations, the solvent
used and the shearing conditions. Overall, the gels which formed upon quenching
to sub-zero temperatures were found to be stable during small-amplitude
oscillatory shear (linear viscoelastic range) but broke down even at the
relatively low shear rates associated with steady torsional flows, suggesting
that the shearing conditions used during the processing of gels of P3HT with
small molecule acceptor blends can alter the gel structure and possibly affect
the resulting active layer performance.

\subsection*{\href{http://arxiv.org/abs/2009.11399v1}{'Setting things straight' by twisting and bending?}}
\subsubsection*{Edward T. Samulski, and Demetri J. Photinos (2020-09-23)}
We explain the origins of the controversy about the classification of the
$N_X$ phase observed in cyanobiphenyl dimers and why it is a polar twisted
phase, an entirely new kind of nematic.

\subsection*{\href{http://arxiv.org/abs/2009.11398v1}{Chemical Interaction and Electronic Structure in a Compositionally  Complex Alloy: a Case Study by means of X-ray Absorption and X-ray  Photoelectron Spectroscopy}}
\subsubsection*{S. Kasatikov, \dots, and G. Schumacher (2020-09-23)}
Chemical interaction and changes in local electronic structure of Cr, Fe, Co,
Ni and Cu transition metals (TMs) upon formation of an
$Al_{8}Co_{17}Cr_{17}Cu_{8}Fe_{17}Ni_{33}$ compositionally complex alloy (CCA)
have been studied by X-ray absorption spectroscopy and X-ray photoelectron
spectroscopy. It was found that upon CCA formation, occupancy of the Cr, Co and
Ni 3d states changes and the maximum of the occupied and empty Ni 3d states
density shifts away from Fermi level ($E_f$) by 0.5 and 0.6 eV, respectively,
whereas the Cr 3d empty states maximum shifts towards $E_f$ by 0.3 eV, compared
to the corresponding pure metals. The absence of significant charge transfer
between the elements was established, pointing to the balancing of the 3d
states occupancy change by involvement of delocalized 4s and 4p states into the
charge redistribution. Despite the expected formation of strong Al-TMs covalent
bonds, the Al role in the transformation of the TMs 3d electronic states is
negligible. The work demonstrates a decisive role of Cr in the Ni local
electronic structure transformation and suggests formation of directional Ni-Cr
bonds with covalent character. These findings can be helpful for tuning
deformation properties and phase stability of the CCA.

\subsection*{\href{http://arxiv.org/abs/2009.11396v1}{Azimuthal eigenmodes at strongly non-degenerate parametric  down-conversion}}
\subsubsection*{Lev S. Dvernik and Pavel A. Prudkovskii (2020-09-23)}
Quantum-optical technologies based on the effect of parametric light
down-conversion are not yet applied in the terahertz frequency range. This is
owing to the absence of terahertz single-photon detectors and the strong
entanglement of modes of optical-terahertz biphotons. This study investigates
the angular structure of scattered radiation generated by strongly
non-degenerate parametric down-conversion. It demonstrates that under certain
approximations, it is possible to obtain azimuthal eigenmodes for the
nonlinear-interaction operator. The solution of the evolution equations for the
field operators in these eigenmodes has the form of the Bogolyubov
transformation, which allows a scattering matrix to be obtained for arbitrary
values of the parametric gain. This scattering matrix can describe both the
production of biphoton pairs and the generation of intense fluxes of correlated
optical-terahertz fields that form a macroscopic quantum state of radiation in
two spectral ranges.

\subsection*{\href{http://arxiv.org/abs/2009.11395v1}{Electronic noise of warm electrons in semiconductors from  first-principles}}
\subsubsection*{Alexander Y. Choi, and Austin J. Minnich (2020-09-23)}
The ab-initio theory of low-field electronic transport properties such as
carrier mobility in semiconductors is well-established. However, an equivalent
treatment of electronic fluctuations about a non-equilibrium steady state,
which are readily probed experimentally, remains less explored. Here, we report
a first-principles theory of electronic noise for warm electrons in
semiconductors. In contrast with typical numerical methods used for electronic
noise, no adjustable parameters are required in the present formalism, with the
electronic band structure and scattering rates calculated from
first-principles. We demonstrate the utility of our approach by applying it to
GaAs and show that spectral features in AC transport properties and noise
originate from the disparate time scales of momentum and energy relaxation,
despite the dominance of optical phonon scattering. Our formalism enables a
parameter-free approach to probe the microscopic transport processes that give
rise to electronic noise in semiconductors.

\subsection*{\href{http://arxiv.org/abs/2009.11383v1}{Emergent universality in critical quantum spin chains: entanglement  Virasoro algebra}}
\subsubsection*{Qi Hu, and Guifre Vidal (2020-09-23)}
Entanglement entropy and entanglement spectrum have been widely used to
characterize quantum entanglement in extended many-body systems. Given a pure
state of the system and a division into regions $A$ and $B$, they can be
obtained in terms of the $Schmidt~ values$, or eigenvalues $\lambda_{\alpha}$
of the reduced density matrix $\rho_A$ for region $A$. In this paper we draw
attention instead to the $Schmidt~ vectors$, or eigenvectors
$|v_{\alpha}\rangle$ of $\rho_A$. We consider the ground state of critical
quantum spin chains whose low energy/long distance physics is described by an
emergent conformal field theory (CFT). We show that the Schmidt vectors
$|v_{\alpha}\rangle$ display an emergent universal structure, corresponding to
a realization of the Virasoro algebra of a boundary CFT (a chiral version of
the original CFT). Indeed, we build weighted sums $H_n$ of the lattice
Hamiltonian density $h_{j,j+1}$ over region $A$ and show that the matrix
elements $\langle v_{\alpha}H_n |v_{\alpha'}\rangle$ are universal, up to
finite-size corrections. More concretely, these matrix elements are given by an
analogous expression for $H_n^{\tiny \text{CFT}} = \frac 1 2 (L_n + L_{-n})$ in
the boundary CFT, where $L_n$'s are (one copy of) the Virasoro generators. We
numerically confirm our results using the critical Ising quantum spin chain and
other (free-fermion equivalent) models.

\subsection*{\href{http://arxiv.org/abs/2009.11372v1}{Directional Clogging and Phase Separation for Disk Flow Through Periodic  and Diluted Obstacle Arrays}}
\subsubsection*{C. Reichhardt and C. J. O. Reichhardt (2020-09-23)}
We model collective disk flow though a square array of obstacles as the flow
direction is changed relative to the symmetry directions of the array. At lower
disk densities there is no clogging for any driving direction, but as the disk
density increases, the average disk velocity decreases and develops a drive
angle dependence. For certain driving angles, the flow is reduced or drops to
zero when the system forms a heterogeneous clogged state consisting of high
density clogged regions coexisting with empty regions. The clogged states are
fragile and can be unclogged by changing the driving angle. For large obstacle
sizes, we find a uniform clogged state that is distinct from the collective
clogging regime. Within the clogged phases, depinning transitions can occur as
a function of increasing driving force, with strongly intermittent motion
appearing just above the depinning threshold. The clogging is robust against
the random removal or dilution of the obstacle sites, and the disks are able to
form system-spanning clogged clusters even under increasing dilution. If the
dilution becomes too large, however, the clogging behavior is lost.

\subsection*{\href{http://arxiv.org/abs/2009.11371v1}{Simulating Non Commutative Geometry with Quantum Walks}}
\subsubsection*{Fabrice Debbasch (2020-09-23)}
Non Commutative Geometry (NCG) is considered in the context of a charged
particle moving in a uniform magnetic field. The classical and quantum
mechanical treatments are revisited and a new marker of NCG is introduced. This
marker is then used to investigate NCG in magnetic Quantum Walks. It is proven
that these walks exhibit NCG at and near the continuum limit. For the purely
discrete regime, two illustrative walks of different complexities are studied
in full detail. The most complex walk does exhibit NCG but the simplest, most
degenerate one does not. Thus, NCG can be simulated by QWs, not only in the
continuum limit, but also in the purely discrete regime.

\subsection*{\href{http://arxiv.org/abs/2009.11370v1}{Unravelling the role of coherence in the first law of quantum  thermodynamics}}
\subsubsection*{Bertúlio de Lima Bernardo (2020-09-23)}
One of the fundamental questions in the emerging field of quantum
thermodynamics is the role played by coherence in energetic processes that
occur at the quantum level. Here, we address this issue by investigating two
different quantum versions of the first law of thermodynamics, derived from the
classical definitions of work and heat. By doing so, we find out that there
exists a mathematical inconsistency between both scenarios. We further show
that the energetic contribution of the dynamics of coherence is the key
ingredient to establish the consistency. Some examples involving two-level
atomic systems are discussed in order to illustrate our findings.

\subsection*{\href{http://arxiv.org/abs/2009.11351v1}{Estimating entropy rate from censored symbolic time series: a test for  time-irreversibility}}
\subsubsection*{Raul Salgado-Garcia and Cesar Maldonado (2020-09-23)}
In this work we introduce a method for estimating entropy rate and entropy
production rate from finite symbolic time series. From the point of view of
statistics, estimating entropy from a finite series can be interpreted as a
problem of estimating parameters of a distribution with a censored or truncated
sample. We use this point of view to give estimations of entropy rate and
entropy production rate assuming that this is a parameter of a (limiting)
distribution. The last statement is actually a consequence of the fact that the
distribution of estimators coming from recurrence-time statistics comply with
the central limit theorem. We test our method in a Markov chain model where
these quantities can be exactly computed.

\subsection*{\href{http://arxiv.org/abs/2009.11331v1}{Thermodynamics of Magnetic Monopoles in Square Spin Ice}}
\subsubsection*{Cristiano Nisoli (2020-09-23)}
We describe degenerate square spin ice via its magnetic monopoles coupled to
an emergent entropic field that subsumes the effect of the underlying spin
vacuum. We derive their effective free energy, entropic interaction,
correlations and screening. Unlike in pyrochlore ices, a dimensional mismatch
between real and entropic interactions leads to weak singularities at the pinch
points signaling an algebraic screening, which can be, however, camouflaged by
a pseudo-screening regime.

\subsection*{\href{http://arxiv.org/abs/2009.11328v1}{Entanglement of Two Jaynes-Cummings Atoms In Single Excitation Space}}
\subsubsection*{Ya Yang, \dots, and Lan Zhou (2020-09-23)}
We study the entanglement dynamics of two atoms coupled to their own
Jaynes-Cummings cavities in single-excitation space. Here we use the
concurrence to measure the atomic entanglement. And the partial Bell states as
initial states are considered. Our analysis suggests that there exist collapses
and recovers in the entanglement dynamics. The physical mechanism behind the
entanglement dynamics is the periodical information and energy exchange between
atoms and light fields. For the initial Partial Bell states, only if the ratio
of two atom-cavity coupling strengths is a rational number, the evolutionary
periodicity of the atomic entanglement can be found. And whether there is time
translation between two kinds of initial partial Bell state cases depends on
the odd-even number of the coupling strength ratio.

\subsection*{\href{http://arxiv.org/abs/2009.11324v1}{Local master equations bypass the secular approximation}}
\subsubsection*{Stefano Scali, and Luis A. Correa (2020-09-23)}
Master equations are a vital tool to model heat flow through nanoscale
thermodynamic systems. Most practical devices are made up of interacting
sub-system, and are often modelled using either local master equations (LMEs)
or global master equations (GMEs). While the limiting cases in which either the
LME or the GME breaks down are well understood, there exists a 'grey area' in
which both equations capture steady-state heat currents reliably, but predict
very different transient heat flows. In such cases, which one should we trust?
Here, we show that, when it comes to dynamics, the local approach can be more
reliable than the global one for weakly interacting open quantum systems. This
is due to the fact that the secular approximation, which underpins the GME, can
destroy key dynamical features. To illustrate this, we consider a minimal
transport setup and show that its LME displays exceptional points (EPs). These
singularities have been observed in a superconducting-circuit realisation of
the model [1]. However, in stark contrast to experimental evidence, no EPs
appear within the global approach. We then show that the EPs are a feature
built into the Redfield equation, which is more accurate than the LME and the
GME. Finally, we show that the local approach emerges as the weak-interaction
limit of the Redfield equation, and that it entirely avoids the secular
approximation.

\subsection*{\href{http://arxiv.org/abs/2009.11319v1}{Post-Newtonian Description of Quantum Systems in Gravitational Fields}}
\subsubsection*{Philip K. Schwartz (2020-09-23)}
This thesis deals with the systematic treatment of quantum-mechanical systems
in post-Newtonian gravitational fields. Starting from clearly spelled-out
assumptions, employing a framework of geometric background structures defining
the notion of a post-Newtonian expansion, our systematic approach allows to
properly derive the post-Newtonian coupling of quantum-mechanical systems to
gravity based on first principles. This sets it apart from more heuristic
approaches that are commonly employed, for example, in the description of
quantum-optical experiments under gravity.
  Regarding single particles, we compare simple canonical quantisation of a
free particle in curved spacetime to formal expansions of the minimally coupled
Klein-Gordon equation, which may be motivated from QFT in curved spacetimes.
Specifically, we develop a general WKB-like post-Newtonian expansion of the KG
equation to arbitrary order in $c^{-1}$. Furthermore, for stationary
spacetimes, we show that the Hamiltonians arising from expansions of the KG
equation and from canonical quantisation agree up to linear order in particle
momentum, independent of any expansion in $c^{-1}$.
  Concerning composite systems, we perform a fully detailed systematic
derivation of the first order post-Newtonian quantum Hamiltonian describing the
dynamics of an electromagnetically bound two-particle system situated in
external electromagnetic and gravitational fields, the latter being described
by the Eddington-Robertson PPN metric.
  In the last, independent part of the thesis, we prove two uniqueness results
characterising the Newton--Wigner position observable for Poincar\'e-invariant
classical Hamiltonian systems: one is a direct classical analogue of the
quantum Newton--Wigner theorem, and the other clarifies the geometric
interpretation of the Newton--Wigner position as `centre of spin', as proposed
by Fleming in 1965.

\subsection*{\href{http://arxiv.org/abs/2009.11315v1}{Probing light-driven quantum materials with ultrafast resonant inelastic  X-ray scattering}}
\subsubsection*{Matteo Mitrano and Yao Wang (2020-09-23)}
Ultrafast optical pulses are an increasingly important tool for controlling
quantum materials and triggering novel photo-induced phase transitions.
Understanding these dynamic phenomena requires a probe sensitive to spin,
charge, and orbital degrees of freedom. Time-resolved resonant inelastic X-ray
scattering (trRIXS) is an emerging spectroscopic method, which responds to this
need by providing unprecedented access to the finite-momentum fluctuation
spectrum of photoexcited solids. In this Perspective, we briefly review
state-of-the-art trRIXS experiments on condensed matter systems, as well as
recent theoretical advances. We then describe future research opportunities in
the context of light control of quantum matter.

\subsection*{\href{http://arxiv.org/abs/2009.11313v1}{Taming the infinite: framework for resource quantification in  infinite-dimensional general probabilistic theories}}
\subsubsection*{Ludovico Lami, \dots, and Giovanni Ferrari (2020-09-23)}
Resource theories provide a general framework for the characterization of
properties of physical systems in quantum mechanics and beyond. Here, we
introduce methods for the quantification of resources in general probabilistic
theories (GPTs), focusing in particular on the technical issues associated with
infinite-dimensional state spaces. We define a universal resource quantifier
based on the robustness measure, and show it to admit a direct operational
meaning: in any GPT, it quantifies the advantage that a given resource state
enables in channel discrimination tasks over all resourceless states. We show
that the robustness acts as a faithful and strongly monotonic measure in any
resource theory described by a convex and closed set of free states, and can be
computed through a convex conic optimization problem.
  Specializing to continuous-variable quantum mechanics, we obtain additional
bounds and relations, allowing an efficient computation of the measure and
comparison with other monotones. We demonstrate applications of the robustness
to several resources of physical relevance: optical nonclassicality,
entanglement, genuine non-Gaussianity, and coherence. In particular, we
establish exact expressions for various classes of states, including Fock
states and squeezed states in the resource theory of nonclassicality and
general pure states in the resource theory of entanglement, as well as tight
bounds applicable in general cases.

\subsection*{\href{http://arxiv.org/abs/2009.11312v1}{Open system dynamics and quantum jumps: Divisibility vs. dissipativity}}
\subsubsection*{Dariusz Chruściński, \dots, and Andrea Smirne (2020-09-23)}
Several key properties of quantum evolutions are characterized by
divisibility of the corresponding dynamical maps. In particular, a Markovian
evolution respects CP-divisibility, whereas breaking of P-divisibility provides
a clear sign of non-Markovian effects. We analyze a class of evolutions which
interpolates between CP- and P-divisible classes and is characterized by
dissipativity -- a long known but so far not widely used formal concept to
classify open system dynamics. By making a connection to stochastic jump
unravellings of master equations, we demonstrate that there exists inherent
freedom in how to divide the terms of the underlying master equation into the
deterministic and jump parts for the stochastic description. This leads to a
number of different unravelings, each one with a measurement scheme
interpretation and highlighting different properties of the considered open
system dynamics. Starting from formal mathematical concepts, our results allow
us to get fundamental insights in open system dynamics and to ease their
numerical simulations.

\subsection*{\href{http://arxiv.org/abs/2009.11311v1}{Measurement and entanglement phase transitions in all-to-all quantum  circuits, on quantum trees, and in Landau-Ginsburg theory}}
\subsubsection*{Adam Nahum, \dots, and Jonathan Ruhman (2020-09-23)}
Quantum many-body systems subjected to local measurements at a nonzero rate
can be in distinct dynamical phases, with differing entanglement properties. We
introduce theoretical approaches to measurement-induced phase transitions (MPT)
and also to entanglement transitions in random tensor networks. Many of our
results are for "all-to-all" quantum circuits with unitaries and measurements,
in which any qubit can couple to any other, and related settings where some of
the complications of low-dimensional models are reduced. We also propose field
theory descriptions for spatially local systems of finite dimensionality. To
build intuition, we first solve the simplest "minimal cut" toy model for
entanglement dynamics in all-to-all circuits, finding scaling forms and
exponents within this approximation. We then show that certain all-to-all
measurement circuits allow exact results by exploiting the circuit's local
tree-like structure. For this reason, we make a detour to give universal
results for entanglement phase transitions in a class of random tree tensor
networks, making a connection with the classical theory of directed polymers on
a tree. We then compare these results with numerics in all-to-all circuits,
both for the MPT and for the simpler "Forced Measurement Phase Transition"
(FMPT). We characterize the two different phases in all-to-all circuits using
observables that are sensitive to the amount of information propagated between
the initial and final time. We demonstrate signatures of the two phases that
can be understood from simple models. Finally we propose
Landau-Ginsburg-Wilson-like field theories for the MPT, the FMPT, and for
entanglement transitions in tensor networks. This analysis shows a surprising
difference between the MPT and the other cases. We discuss variants of the
measurement problem with additional structure, and questions for the future.

\subsection*{\href{http://arxiv.org/abs/2009.11301v1}{TBG I: Matrix Elements, Approximations, Perturbation Theory and a  $k\cdot p$ 2-Band Model for Twisted Bilayer Graphene}}
\subsubsection*{B. Andrei Bernevig, \dots, and Biao Lian (2020-09-23)}
We investigate the Twisted Bilayer Graphene (TBG) model to obtain an analytic
understanding of its energetics and wavefunctions needed for many-body
calculations. We provide an approximation scheme which first elucidates why the
BM $K_M$-point centered calculation containing only $4$ plane-waves provides a
good analytical value for the first magic angle. The approximation scheme also
elucidates why most many-body matrix elements in the Coulomb Hamiltonian
projected to the active bands can be neglected. By applying our approximation
scheme at the first magic angle to a $\Gamma_M$-point centered model of 6
plane-waves, we analytically understand the small $\Gamma_M$-point gap between
the active and passive bands in the isotropic limit $w_0=w_1$. Furthermore, we
analytically calculate the group velocities of passive bands in the isotropic
limit, and show that they are \emph{almost} doubly degenerate, while no
symmetry forces them to be. Furthermore, away from $\Gamma_M$ and $K_M$ points,
we provide an explicit analytical perturbative understanding as to why the TBG
bands are flat at the first magic angle, despite it is defined only by
vanishing $K_M$-point Dirac velocity. We derive analytically a connected "magic
manifold" $w_1=2\sqrt{1+w_0^2}-\sqrt{2+3w_0^2}$, on which the bands remain
extremely flat as $w_0$ is tuned between the isotropic ($w_0=w_1$) and chiral
($w_0=0$) limits. We analytically show why going away from the isotropic limit
by making $w_0$ less (but not larger) than $w_1$ increases the $\Gamma_M$-
point gap between active and passive bands. Finally, perturbatively, we provide
an analytic $\Gamma_M$ point $k\cdot p$ $2$-band model that reproduces the TBG
band structure and eigenstates in a certain $w_0,w_1$ parameter range. Further
refinement of this model suggests a possible faithful $2$-band $\Gamma_M$ point
$k\cdot p$ model in the full $w_0, w_1$ parameter range.

\subsection*{\href{http://arxiv.org/abs/2009.11302v1}{Operational quantification of continuous-variable quantum resources}}
\subsubsection*{Bartosz Regula, \dots, and Ryuji Takagi (2020-09-23)}
The diverse range of resources which underlie the utility of quantum states
in practical tasks motivates the development of universally applicable methods
to measure and compare resources of different types. However, many of such
approaches were hitherto limited to the finite-dimensional setting or were not
connected with operational tasks. We overcome this by introducing a general
method of quantifying resources for continuous-variable quantum systems based
on the robustness measure, applicable to a plethora of physically relevant
resources such as nonclassicality, entanglement, genuine non-Gaussianity, and
coherence. We demonstrate in particular that the measure has a direct
operational interpretation as the advantage enabled by a given state in a class
of channel discrimination tasks. We show that the robustness constitutes a
well-behaved, bona fide resource quantifier in any convex resource theory,
contrary to a related negativity-based measure known as the standard
robustness. Furthermore, we show the robustness to be directly observable -- it
can be computed as the expectation value of a single witness operator -- and
establish general methods for evaluating the measure. Explicitly applying our
results to the relevant resources, we demonstrate the exact computability of
the robustness for several classes of states.

\subsection*{\href{http://arxiv.org/abs/2009.11303v2}{Thermodynamics of precision in quantum nano-machines}}
\subsubsection*{Antoine Rignon-Bret, \dots, and Mark T. Mitchison (2020-09-23)}
Fluctuations strongly affect the dynamics and functionality of nanoscale
thermal machines. Recent developments in stochastic thermodynamics have shown
that fluctuations in many far-from-equilibrium systems are constrained by the
rate of entropy production via so-called thermodynamic uncertainty relations.
These relations imply that increasing the reliability or precision of an
engine's power output comes at a greater thermodynamic cost. Here we study the
thermodynamics of precision for small thermal machines in the quantum regime.
In particular, we derive exact relations between the power, power fluctuations,
and entropy production rate for several models of few-qubit engines (both
autonomous and cyclic) that perform work on a quantised load. Depending on the
context, we find that quantum coherence can either help or hinder where power
fluctuations are concerned. We discuss design principles for reducing such
fluctuations in quantum nano-machines, and propose an autonomous three-qubit
engine whose power output for a given entropy production is more reliable than
would be allowed by any classical Markovian model.

\subsection*{\href{http://arxiv.org/abs/2009.11286v1}{Entanglement Properties of Disordered Quantum Spin Chains with  Long-Range Antiferromagnetic Interactions}}
\subsubsection*{Youcef Mohdeb, \dots, and Stephan Haas (2020-09-23)}
We examine the concurrence and entanglement entropy in quantum spin chains
with random long-range couplings, spatially decaying with a power-law exponent
$\alpha$. Using the strong disorder renormalization group (SDRG) technique, we
find by analytical solution of the master equation a strong disorder fixed
point, characterized by a fixed point distribution of the couplings with a
finite dynamical exponent, which describes the system consistently in the
regime $\alpha > 1/2$. A numerical implementation of the SDRG method yields a
power law spatial decay of the average concurrence, which is also confirmed by
exact numerical diagonalization. However, we find that the lowest-order SDRG
approach is not sufficient to obtain the typical value of the concurrence. We
therefore implement a correction scheme which allows us to obtain the leading
order corrections to the random singlet state. This approach yields a power-law
spatial decay of the typical value of the concurrence, which we derive both by
a numerical implementation of the corrections and by analytics. Next, using
numerical SDRG, the entanglement entropy (EE) is found to be logarithmically
enhanced for all $\alpha$, corresponding to a critical behavior with an
effective central charge $c = {\rm ln} 2$, independent of $\alpha$. This is
confirmed by an analytical derivation. Using numerical exact diagonalization
(ED), we confirm the logarithmic enhancement of the EE and a weak dependence on
$\alpha$. For a wide range of distances $l$, the EE fits a critical behavior
with a central charge close to $c=1$, which is the same as for the clean
Haldane-Shastry model with a power-la-decaying interaction with $\alpha =2$.
Consistent with this observation, we find using ED that the concurrence shows
power law decay, albeit with smaller power exponents than obtained by SDRG.

\subsection*{\href{http://arxiv.org/abs/2009.11284v1}{Tilted elastic lines with columnar and point disorder, non-Hermitian  quantum mechanics and spiked random matrices: pinning and localization}}
\subsubsection*{Alexandre Krajenbrink, and Neil O'Connell (2020-09-23)}
We revisit the problem of an elastic line (e.g. a vortex line in a
superconductor) subject to both columnar disorder and point disorder in
dimension $d=1+1$. Upon applying a transverse field, a delocalization
transition is expected, beyond which the line is tilted macroscopically. We
investigate this transition in the fixed tilt angle ensemble and within a
one-way model where backward jumps are neglected. From recent results about
directed polymers and their connections to random matrix theory, we find that
for a single line and a single strong defect this transition in presence of
point disorder coincides with the Baik-Ben Arous-Peche (BBP) transition for the
appearance of outliers in the spectrum of a perturbed random matrix in the GUE.
This transition is conveniently described in the polymer picture by a
variational calculation. In the delocalized phase, the ground state energy
exhibits Tracy-Widom fluctuations. In the localized phase we show, using the
variational calculation, that the fluctuations of the occupation length along
the columnar defect are described by $f_{KPZ}$, a distribution which appears
ubiquitously in the Kardar-Parisi-Zhang universality class. We then consider a
smooth density of columnar defect energies. Depending on how this density
vanishes at its lower edge we find either (i) a delocalized phase only (ii) a
localized phase with a delocalization transition. We analyze this transition
which is an infinite-rank extension of the BBP transition. The fluctuations of
the ground state energy of a single elastic line in the localized phase (for
fixed columnar defect energies) are described by a Fredholm determinant based
on a new kernel. The case of many columns and many non-intersecting lines,
relevant for the study of the Bose glass phase, is also analyzed. The ground
state energy is obtained using free probability and the Burgers equation.

\subsection*{\href{http://arxiv.org/abs/2009.11283v1}{Robust metastable skyrmions with tunable size in the chiral magnet  FePtMo$_3$N}}
\subsubsection*{A. S. Sukhanov, \dots, and D. S. Inosov (2020-09-23)}
Synthesis of new materials that can host magnetic skyrmions and their
thorough experimental and theoretical characterization are essential for future
technological applications. The $\beta$-Mn-type compound FePtMo$_3$N is one
such novel material that belongs to the chiral space group $P4_132$, where the
antisymmetric Dzyaloshinkii-Moriya interaction is allowed due to the absence of
inversion symmetry. We report the results of small-angle neutron scattering
(SANS) measurements of FePtMo$_3$N and demonstrate that its magnetic ground
state is a long-period spin helix with a Curie temperature of 222~K. The
magnetic field-induced redistribution of the SANS intensity showed that the
helical structure transforms to a lattice of skyrmions at $\sim$13~mT at
temperatures just below $T_{\text C}$. Our key observation is that the skyrmion
state in FePtMo$_3$N is robust against field cooling down to the lowest
temperatures. Moreover, once the metastable state is prepared by field cooling,
the skyrmion lattice exists even in zero field. Furthermore, we show that the
skyrmion size in FePtMo$_3$N exhibits high sensitivity to the sample
temperature and can be continuously tuned between 120 and 210~nm. This offers
new prospects in the control of topological properties of chiral magnets.

\subsection*{\href{http://arxiv.org/abs/2009.11280v1}{Specific Heat of a Quantum Critical Metal}}
\subsubsection*{Ori Grossman, \dots, and Erez Berg (2020-09-23)}
We investigate the specific heat, $c$, near an Ising nematic quantum critical
point (QCP), using sign problem-free quantum Monte Carlo simulations. Cooling
towards the QCP, we find a broad regime of temperature where $c/T$ is close to
the value expected from the non-interacting band structure, even for a
moderately large coupling strength. At lower temperature, we observe a rapid
rise of $c/T$, followed by a drop to zero as the system becomes
superconducting. The spin susceptibility begins to drop at roughly the same
temperature where the enhancement of $c/T$ onsets, most likely due to the
opening of a gap associated with superconducting fluctuations. These findings
suggest that superconductivity and non-Fermi liquid behavior (manifested in an
enhancement of the effective mass) onset at comparable energy scales. We
support these conclusions with an analytical perturbative calculation.

\subsection*{\href{http://arxiv.org/abs/2009.11271v2}{Towards the Heisenberg limit in microwave photon detection by a qubit  array}}
\subsubsection*{P. Navez, \dots, and A. M. Zagoskin (2020-09-23)}
Using an analytically solvable model, we show that a qubit array-based
detector allows to achieve the fundamental Heisenberg limit in detecting single
photons. In case of superconducting qubits, this opens new opportunities for
quantum sensing and communications in the important microwave range.

\subsection*{\href{http://arxiv.org/abs/2009.11270v1}{Simpler (classical) and faster (quantum) algorithms for Gibbs partition  functions}}
\subsubsection*{Srinivasan Arunachalam, \dots, and Pawel Wocjan (2020-09-23)}
We consider the problem of approximating the partition function of a
classical Hamiltonian using simulated annealing. This requires the computation
of a cooling schedule, and the ability to estimate the mean of the Gibbs
distributions at the corresponding inverse temperatures. We propose classical
and quantum algorithms for these two tasks, achieving two goals: (i) we
simplify the seminal work of \v{S}tefankovi\v{c}, Vempala and Vigoda
(\emph{J.~ACM}, 56(3), 2009), improving their running time and almost matching
that of the current classical state of the art; (ii) we quantize our new simple
algorithm, improving upon the best known algorithm for computing partition
functions of many problems, due to Harrow and Wei (SODA 2020). A key ingredient
of our method is the paired-product estimator of Huber (\emph{Ann.\ Appl.\
Probab.}, 25(2),~2015). The proposed quantum algorithm has two advantages over
the classical algorithm: it has quadratically faster dependence on the spectral
gap of the Markov chains as well as the precision, and it computes a shorter
cooling schedule, which matches the length conjectured to be optimal by
\v{S}tefankovi\v{c}, Vempala and Vigoda.

\subsection*{\href{http://arxiv.org/abs/2009.11265v1}{Ergotropy from indefinite causal orders}}
\subsubsection*{Kyrylo Simonov, \dots, and Mauro Paternostro (2020-09-23)}
In this work we characterize the impact that the application of two
consecutive quantum channels or their quantum superposition (thus, without a
definite causal order) has on ergotropy, i.e. the maximum work that can be
extracted from a system through a cyclic unitary transformation. First of all,
we show that commutative channels always lead to a non-negative gain;
non-commutative channels, on the other hand, can entail both an increase and a
decrease in ergotropy. We then perform a thorough analysis for qubit channels
and provide general conditions for achieving a positive gain on the incoherent
part of ergotropy. Finally, we extend our results to d-dimensional quantum
systems undergoing a pair of completely depolarizing channels.

\subsection*{\href{http://arxiv.org/abs/2009.11255v1}{Thermoelectric performance of P-N-P abrupt heterostructures vertical to  temperature gradient}}
\subsubsection*{Bohang Nan, \dots, and Tao Guo (2020-09-23)}
We present a model for P-N-P abrupt heterostructures vertical to temperature
gradient to improve the thermoelectric performance. The P-N-P heterostructure
is considered as an abrupt bipolar junction transistor due to an externally
applied temperature gradient paralleled to depletion layers. Taking Bi2Te3 and
Bi0.5Sb1.5Te3 as N-type and P-type thermoelectric materials respectively for
example, we achieve the purpose of controlling the Seebeck coefficient and the
electrical conductivity independently while amplifying operation power. The
calculated results show that the Seebeck coefficient can reach 3312V/K, and the
ZTmax values of this model are 45 or 425, which are tens or even hundreds of
times greater than those of bulk materials and films.

\subsection*{\href{http://arxiv.org/abs/2009.11254v1}{Chiral states and nonreciprocal phases in a Josephson junction ring}}
\subsubsection*{R. Asensio-Perea, \dots, and E. Rico (2020-09-23)}
In this work, we propose how to load and manipulate chiral states in a
Josephson junction ring in the so called transmon regimen. We characterise
these states by their symmetry properties under time reversal and parity
transformations. We describe an explicit protocol to load and detect the states
within a realistic set of circuit parameters and show simulations that reveal
the chiral nature. Finally, we explore the utility of these states in quantum
technological nonreciprocal devices.

\subsection*{\href{http://arxiv.org/abs/2009.11246v1}{Anisotropic superconductivity and Fermi surface reconstruction in the  spin-vortex antiferromagnetic superconductor  CaK(Fe$_{0.95}$Ni$_{0.05}$)$_4$As$_4$}}
\subsubsection*{José Benito Llorens, \dots, and Hermann Suderow (2020-09-23)}
High critical temperature superconductivity often occurs in systems where an
antiferromagnetic order is brought near $T=0K$ by slightly modifying pressure
or doping. CaKFe$_4$As$_4$ is a superconducting, stoichiometric iron pnictide
compound showing optimal superconducting critical temperature with $T_c$ as
large as $38$ K. Doping with Ni induces a decrease in $T_c$ and the onset of
spin-vortex antiferromagnetic order, which consists of spins pointing inwards
to or outwards from alternating As sites on the diagonals of the in-plane
square Fe lattice. Here we study the band structure of
CaK(Fe$_{0.95}$Ni$_{0.05}$)$_4$As$_4$ (T$_c$ = 10 K, T$_N$ = 50 K) using
quasiparticle interference with a Scanning Tunneling Microscope (STM) and show
that the spin-vortex order induces a Fermi surface reconstruction and a
fourfold superconducting gap anisotropy.

\subsection*{\href{http://arxiv.org/abs/2009.11238v1}{Capillary condensation under atomic-scale confinement}}
\subsubsection*{Qian Yang, \dots, and A. K. Geim (2020-09-23)}
Capillary condensation of water is ubiquitous in nature and technology. It
routinely occurs in granular and porous media, can strongly alter such
properties as adhesion, lubrication, friction and corrosion, and is important
in many processes employed by microelectronics, pharmaceutical, food and other
industries. The century-old Kelvin equation is commonly used to describe
condensation phenomena and shown to hold well for liquid menisci with diameters
as small as several nm. For even smaller capillaries that are involved in
condensation under ambient humidity and, hence, of particular practical
interest, the Kelvin equation is expected to break down, because the required
confinement becomes comparable to the size of water molecules. Here we take
advantage of van der Waals assembly of two-dimensional crystals to create
atomic-scale capillaries and study condensation inside. Our smallest
capillaries are less than 4 angstroms in height and can accommodate just a
monolayer of water. Surprisingly, even at this scale, the macroscopic Kelvin
equation using the characteristics of bulk water is found to describe
accurately the condensation transition in strongly hydrophilic (mica)
capillaries and remains qualitatively valid for weakly hydrophilic (graphene)
ones. We show that this agreement is somewhat fortuitous and can be attributed
to elastic deformation of capillary walls, which suppresses giant oscillatory
behavior expected due to commensurability between atomic-scale confinement and
water molecules. Our work provides a much-needed basis for understanding of
capillary effects at the smallest possible scale important in many realistic
situations.

\subsection*{\href{http://arxiv.org/abs/2009.11237v1}{Experimental observation of topological exciton-polaritons in transition  metal dichalcogenide monolayers}}
\subsubsection*{Mengyao Li, \dots, and Alexander B. Khanikaev (2020-09-23)}
The rise of quantum science and technologies motivates photonics research to
seek new platforms with strong light-matter interactions to facilitate quantum
behaviors at moderate light intensities. One promising platform to reach such
strong light-matter interacting regimes is offered by polaritonic metasurfaces,
which represent ultrathin artificial media structured on nano-scale and
designed to support polaritons - half-light half-matter quasiparticles.
Topological polaritons, or 'topolaritons', offer an ideal platform in this
context, with unique properties stemming from topological phases of light
strongly coupled with matter. Here we explore polaritonic metasurfaces based on
2D transition metal dichalcogenides (TMDs) supporting in-plane polarized
exciton resonances as a promising platform for topological polaritonics. We
enable a spin-Hall topolaritonic phase by strongly coupling valley polarized
in-plane excitons in a TMD monolayer with a suitably engineered all-dielectric
topological photonic metasurface. We first show that the strong coupling
between topological photonic bands supported by the metasurface and excitonic
bands in MoSe2 yields an effective phase winding and transition to a
topolaritonic spin-Hall state. We then experimentally realize this phenomenon
and confirm the presence of one-way spin-polarized edge topolaritons. Combined
with the valley polarization in a MoSe2 monolayer, the proposed system enables
a new approach to engage the photonic angular momentum and valley degree of
freedom in TMDs, offering a promising platform for photonic/solid-state
interfaces for valleytronics and spintronics.

\subsection*{\href{http://arxiv.org/abs/2009.11223v1}{Crystal structure and phase transitions at high pressures in the  superconductor FeSe0.89S0.11}}
\subsubsection*{Yulia A. Nikiforova, \dots, and Mahmoud Abdel-Hafiez (2020-09-23)}
We report on the structural phase transitions in the S doped FeSe
superconductor by powder synchrotron X ray diffraction at high pressures up to
18.5 GPa under compression and decompression modes. In order to create high
quasi hydrostatic pressures, diamond anvil cells filled with helium as a
pressure transmitting medium were used. It was found that at ambient pressure
and room temperature, S doped FeSe has a tetragonal structure. Under
compression, in the region of 10 GPa, a phase transition from the tetragonal
into the orthorhombic structure is observed, which persists up to 18.5 GPa. Our
results strongly suggest that, at decompression, as the applied pressure
decreases to 6 GPa and then is completely removed, most of the sample
recrystallizes into the hexagonal phase of the structural type NiAs. However,
the other part of the sample remains in the high pressure orthorhombic phase,
while the tetragonal phase is not restored. These observations illustrate a
strong hysteresis of the structural properties of S doped FeSe during a phase
transition under pressure.

\subsection*{\href{http://arxiv.org/abs/2009.11220v1}{Floquet generation of Second Order Topological Superconductor}}
\subsubsection*{Arnob Kumar Ghosh, and Arijit Saha (2020-09-23)}
We theoretically investigate the Floquet generation of second-order
topological superconducting (SOTSC) phase, hosting Majorana corner modes
(MCMs), considering a quantum spin Hall insulator (QSHI) with proximity induced
superconducting $s$-wave pairing in it. Our dynamical prescription consists of
the periodic kick in time-reversal symmetry breaking in-plane magnetic field
and four-fold rotational symmetry breaking mass term while these Floquet MCMs
are preserved by anti-unitary particle-hole symmetry. The first driving
protocol always leads to four zero energy MCMs (i.e. one Majorana per corner)
as a sign of a {\it{strong}} SOTSC phase. Interestingly, the second protocol
can result in a {\it{weak}} SOTSC phase, harbouring eight zero energy MCMs (two
Majorana states per corner), in addition to the {\it{strong}} SOTSC phase. We
characterize the topological nature of these phases by Floquet quadrupolar
moment and Floquet Wannier spectrum. We believe that relying on the recent
experimental advancement in the driven systems and proximity induced
superconductivity, our schemes may be possible to test in the future.

\subsection*{\href{http://arxiv.org/abs/2009.11215v1}{Structure, Magnetism and First Principles Modeling of the Na0.5La0.5RuO3  Perovskite}}
\subsubsection*{Loi T. Nguyen, and Robert J. Cava (2020-09-23)}
High purity polycrystalline Na0.5La0.5RuO3 was synthesized by a solid state
method, and its properties were studied by magnetic susceptibility, heat
capacity and resistivity measurements. We find it to be a tetragonal
perovskite, in contrast to an earlier report, with random La/Na mixing. With a
Curie-Weiss temperature of -231 K and effective moment of 2.74 uB/mol-Ru, there
is no magnetic ordering down to 1.8 K. A broad hump at 1.4 K in the heat
capacity, however, indicates the presence of a glassy magnetic transition,
which we attribute to the influence of the random distribution of Na and La on
the perovskite A sites. Comparison to CaRuO3, a structurally ordered ruthenate
perovskite with similar properties, is presented. First-principle calculations
indicate that the Na-La distribution determines the local magnetic exchange
inter-actions between Ru ions, favoring either antiferromagnetic or
ferromagnetic coupling when the local environment is Na or La rich. Thus our
data and analysis suggest that mixing cations with different charges and sizes
on the A site in this perovskite results in magnetic frustration through a
balance of local magnetic exchange interactions.

\subsection*{\href{http://arxiv.org/abs/2009.11211v1}{Dynamical systems on large networks with predator-prey interactions are  stable and exhibit oscillations}}
\subsubsection*{Andrea Marcello Mambuca, and Izaak Neri (2020-09-23)}
We analyse the linear stability of fixed points in large dynamical systems
defined on sparse, random graphs with predator-prey, competitive, and
mutualistic interactions. These systems are aimed at modelling, among others,
ecosystems consisting of a large number of species that interact through a
food-web. We develop an exact theory for the spectral distribution and the
leading eigenvalue of the corresponding sparse Jacobian matrices. This theory
reveals that the nature of local interactions have a strong influence on
system's stability. In particular, we show that fixed points of dynamical
systems defined on random graphs are always unstable if they are large enough,
except if all interactions are of the predator-prey type. This qualitatively
new feature for antagonistic systems is accompanied by a peculiar oscillatory
behaviour of the dynamical response of the system after a perturbation, when
the mean degree of the graph is small enough. Moreover we find that there exist
a dynamical phase transition and critical mean degree above which the response
becomes non-oscillatory also for antagonistic systems.

\subsection*{\href{http://arxiv.org/abs/2009.11209v1}{Structure of the Lennard-Jones liquid estimated from a single simulation}}
\subsubsection*{Shibu Saw and Jeppe C. Dyre (2020-09-23)}
Combining the recent Piskulich-Thompson approach [Z. A. Piskulich and W. H.
Thompson, {\it J. Chem. Phys.} {\bf 152}, 011102 (2020)] with isomorph theory,
the structure of a single-component Lennard-Jones system (LJ) is obtained at an
arbitrary state point in almost the whole liquid region of the
temperature-density phase diagram from a single simulation. The LJ exhibits two
temperature range where the van't Hoff's assumption that energetic and entropic
forces are temperature independent is valid. A method to evaluate the structure
at an arbitrary state point along an isochore from the knowledge of only
structures at two temperatures on the isochore is also discussed. We argue that
the structure of any R-simple system obeying van't Hoff's assumption in the
whole range of temperatures can be determined in the whole liquid region of
phase-diagram from only a single simulation.

\subsection*{\href{http://arxiv.org/abs/2009.11202v1}{Kinetics of Domain Growth and Aging in a Two-Dimensional Off-lattice  System}}
\subsubsection*{Jiarul Midya and Subir K. Das (2020-09-23)}
We have used molecular dynamics simulations for a comprehensive study of
phase separation in a two-dimensional single component off-lattice model where
particles interact through the Lennard-Jones potential. Via state-of-the-art
methods we have analyzed simulation data on structure, growth and aging for
nonequilibrium evolutions in the model. These data were obtained following
quenches of well-equilibrated homogeneous configurations, with density close to
the critical value, to various temperatures inside the miscibility gap, having
vapor-"liquid" as well as vapor-"solid" coexistence. For the vapor-liquid phase
separation we observe that $\ell$, the average domain length, grows with time
($t$) as $t^{1/2}$, a behavior that has connection with hydrodynamics. At low
enough temperature, a sharp crossover of this time dependence to a much slower,
temperature dependent, growth is identified within the time scale of our
simulations, implying "solid"-like final state of the high density phase. This
crossover is, interestingly, accompanied by strong differences in domain
morphology and other structural aspects between the two situations. For aging,
we have presented results for the order-parameter autocorrelation function.
This quantity exhibits data-collapse with respect to $\ell/\ell_w$, $\ell$ and
$\ell_w$ being the average domain lengths at times $t$ and $t_w$ ($\leq t$),
respectively, the latter being the age of a system. Corresponding scaling
function follows a power-law decay: $~\sim (\ell/\ell_w)^{-\lambda}$, for $t\gg
t_w$. The decay exponent $\lambda$, for the vapor-liquid case, is accurately
estimated via the application of an advanced finite-size scaling method. The
obtained value is observed to satisfy a bound.

\subsection*{\href{http://arxiv.org/abs/2009.12246v1}{Message passing for probabilistic models on networks with loops}}
\subsubsection*{Alec Kirkley, and M. E. J. Newman (2020-09-23)}
In this paper, we extend a recently proposed framework for message passing on
"loopy" networks to the solution of probabilistic models. We derive a
self-consistent set of message passing equations that allow for fast
computation of probability distributions in systems that contain short loops,
potentially with high density, as well as expressions for the entropy and
partition function of such systems, which are notoriously difficult quantities
to compute. Using the Ising model as an example, we show that our solutions are
asymptotically exact on certain classes of networks with short loops and offer
a good approximation on more general networks, improving significantly on
results derived from standard belief propagation. We also discuss potential
applications of our method to a variety of other problems.

\subsection*{\href{http://arxiv.org/abs/2009.11194v1}{Comment on "Kosterlitz-Thouless-type caging-uncaging transition in a  quasi-one-dimensional hard disk system" [Phys. Rev. Research 2, 033351  (2020)]}}
\subsubsection*{Yi Hu and Patrick Charbonneau (2020-09-23)}
Huerta et al. [Phys. Rev. Research 2, 033351 (2020)] report a power-law decay
of positional order in numerical simulations of hard disks confined within hard
parallel walls, which they interpret as a Kosterlitz-Thouless-type
caging-uncaging transition. The proposed existence of such a transition in a
quasi-one-dimensional (q1D) system, however, contradicts long-held physical
expectations. To clarify if the proposed ordering persists in the thermodynamic
limit, we introduce an exact transfer matrix approach to expeditiously generate
equilibrium configurations for systems of arbitrary size. The power-law decay
of positional order is found to extend only over finite distances. We conclude
that the numerical simulation results reported are associated with a crossover,
and not a proper thermodynamic phase transition.

\subsection*{\href{http://arxiv.org/abs/2009.11191v1}{Morris-Shore transformation for non-degenerate systems}}
\subsubsection*{K. N. Zlatanov, and N. V. Vitanov (2020-09-23)}
The Morris-Shore (MS) transformation is a powerful tool for decomposition of
the dynamics of multistate quantum systems to a set of two-state systems and
uncoupled single states. It assumes two sets of states wherein any state in the
first set can be coupled to any state in the second set but the states within
each set are not coupled between themselves. Another important condition is the
degeneracy of the states in each set, although all couplings between the states
from different sets can be detuned from resonance by the same detuning. The
degeneracy condition limits the application of the MS transformation in various
physically interesting situations, e.g. in the presence of electric and/or
magnetic fields or light shifts, which lift the degeneracy in each set of
states, e.g. when these sets comprise the magnetic sublevels of levels with
nonzero angular momentum. This paper extends the MS transformation to such
situations, in which the states in each of the two sets are nondegenerate. To
this end, we develop an alternative way for the derivation of Morris-Shore
transformation, which can be applied to non-degenerate sets of states. We
present a generalized eigenvalue approach, by which, in the limit of small
detunings from degeneracy, we are able to generate an effective Hamiltonian
that is dynamically equivalent to the non-degenerate Hamiltonian. The effective
Hamiltonian can be mapped to the Morris-Shore basis with a two-step similarity
transformation. After the derivation of the general framework, we demonstrate
the application of this technique to the popular Lambda three-state system, and
the four-state tripod, double-Lambda and diamond systems. In all of these
systems, our formalism allows us to reduce their quantum dynamics to simpler
two-state systems even in the presence of various detunings, e.g. generated by
external fields of frequency drifts.

\subsection*{\href{http://arxiv.org/abs/2009.11183v1}{Biorthogonal quantum criticality in non-Hermitian many-body systems}}
\subsubsection*{Gaoyong Sun and Su-Peng Kou (2020-09-23)}
We develop the perturbation theory of the fidelity susceptibility in
biorthogonal bases for arbitrary interacting non-Hermitian many-body systems
with real eigenvalues. The quantum criticality in the non-Hermitian transverse
field Ising chain is investigated by the second derivative of ground-state
energy and the ground-state fidelity susceptibility. We show that the system
undergoes a second-order phase transition with the Ising universal class by
numerically computing the critical points and the critical exponents from the
finite-size scaling theory. Interestingly, our results indicate that the
biorthogonal quantum phase transitions are described by the biorthogonal
fidelity susceptibility instead of the conventional fidelity susceptibility.

\subsection*{\href{http://arxiv.org/abs/2009.11181v1}{Detailed Electron Energy Loss Spectroscopy (EELS) Microanalysis of Data  Collected Under Semi-Angle Less Than Both Plasmon Cutoff Angle and Incident  Beam Convergence Semi-Angle}}
\subsubsection*{Noureddine Hadji (2020-09-23)}
In previous work a different and powerful, analytical, technique was used to
get data, such as the absolute atom concentration (AAC), specimen thickness
etc., from public domain boron nitride EELS spectrum collected under a
collection semi-angle, $\beta$, less than the plasmon cutoff angle,
$\theta_{c}$, but large relative to incident beam convergence, $\alpha$. Here,
seeking for some completeness, another, numerical, technique usable together
with, also, $\beta < \theta_{c}$ is described in minute detail and applied to
data obtained with $\beta/\alpha < 2$, so necessitating incident beam
convergence-related corrections. A lot of experimental physical parameters all
fully relevant to one another are produced from a single EELS spectrum. Of
public domain silicon nitride, Si$_{3}$N$_{4}$, EELS spectrum used. Comparison
between results producible by the two $\beta<\theta_{c}$-related techniques
made. Results range from parameters such as AAC, density, plasmon critical
vector, plasmon dispersion coefficient, Fermi energy to specimen thickness.
Results were obtained using version 5 of eelsMicr program and compared with
existing results obtained using non EELS techniques.

\subsection*{\href{http://arxiv.org/abs/2009.11165v1}{Impact of domains on the orthorhombic-tetragonal transition of  BaTiO$_3$: an ab initio study}}
\subsubsection*{Anna Grünebohm and Madhura Marathe (2020-09-23)}
We investigate the multi-domain structures in the tetragonal and orthorhombic
phases of BaTiO$_3$ and the impact of the presence of domain walls on the
intermediary phase transition. We focus on the change in the transition
temperatures resulting from various types of domain walls and their coupling
with an external electric field. We employ molecular dynamics simulations of an
ab initio effective Hamiltonian in this study. After confirming that this model
is applicable to multi-domain configurations, we show that the phase transition
temperatures strongly depend on the presence of domains walls. Notably we show
that elastic 90$^{\circ}$ walls can strongly reduce thermal hysteresis. Further
analysis shows that the change in transition temperatures can be attributed to
two main factors - long-range monoclinic distortions induced by walls within
domains and domain wall widths. We also show that the coupling with the field
further facilitates the reduction of thermal hysteresis for orthorhombic
90$^{\circ}$ walls making this configuration attractive for future
applications.

\subsection*{\href{http://arxiv.org/abs/2009.11163v2}{Confine Electrons in Effective Plane Fractals}}
\subsubsection*{Xiaotian Yang, \dots, and Shengjun Yuan (2020-09-23)}
As an emerging complex two-dimensional structure, plane fractal has attracted
much attention due to its novel dimension-related physical properties. In this
paper, we check the feasibility to create an effective Sierpinski carpet (SC),
a plane fractal with Hausdorff dimension intermediate between one and two, by
applying an external electric field to a square or a honeycomb lattice. The
electric field forms a fractal geometry but the atomic structure of the
underlying lattice remains the same. By calculating and comparing various
electronic properties, we find parts of the electrons can be confined
effectively in a fractional dimension with a relatively small field, and
representing properties very close to these in a real fractal. In particular,
compared to the square lattice, the external field required to effectively
confine the electron is smaller in the hexagonal lattice, suggesting that a
graphene-like system will be an ideal platform to construct an effective SC
experimentally. Our work paves a new way to build fractals from a top-down
perspective, and can motivate more studies of fractional dimensions in real
systems.

\subsection*{\href{http://arxiv.org/abs/2009.11161v1}{Indium gallium nitride quantum dots: Consequence of random alloy  fluctuations for polarization entangled photon emission}}
\subsubsection*{Saroj Kanta Patra and Stefan Schulz (2020-09-23)}
We analyze the potential of the $c$-plane InGaN/GaN quantum dots for
polarization entangled photon emission by means of an atomistic many-body
framework. Special attention is paid to the impact of random alloy fluctuations
on the excitonic fine structure and the excitonic binding energy. Our
calculations show that $c$-plane InGaN/GaN quantum dots are ideal candidates
for high temperature entangled photon emission as long as the underlying
$C_{3v}$-symmetry is preserved. However, when assuming random alloy
fluctuations in the dot, our atomistic calculations reveal that while the large
excitonic binding energies are only slightly affected, the $C_{3v}$ symmetry is
basically lost due to the alloy fluctuations. We find that this loss in
symmetry significantly impacts the excitonic fine structure. The observed
changes in fine structure and the accompanied light polarization
characteristics have a detrimental effect for polarization entangled photon
pair emission via the biexciton-exciton cascade. Here, we also discuss possible
alternative schemes that benefit from the large excitonic binding energies, to
enable non-classical light emission from $c$-plane InGaN/GaN quantum dots at
elevated temperatures.

\subsection*{\href{http://arxiv.org/abs/2009.11158v1}{Dissipative dynamics at first-order quantum transitions}}
\subsubsection*{Giovanni Di Meglio, and Ettore Vicari (2020-09-23)}
We investigate the effects of dissipation on the quantum dynamics of
many-body systems at quantum transitions, especially considering those of the
first order. This issue is studied within the paradigmatic one-dimensional
quantum Ising model. We analyze the out-of-equilibrium dynamics arising from
quenches of the Hamiltonian parameters and dissipative mechanisms modeled by a
Lindblad master equation, with either local or global spin operators acting as
dissipative operators. Analogously to what happens at continuous quantum
transitions, we observe a regime where the system develops a nontrivial dynamic
scaling behavior, which is realized when the dissipation parameter $u$
(globally controlling the decay rate of the dissipation within the Lindblad
framework) scales as the energy difference $\Delta$ of the lowest levels of the
Hamiltonian, i.e., $u\sim \Delta$. However, unlike continuous quantum
transitions where $\Delta$ is power-law suppressed, at first-order quantum
transitions $\Delta$ is exponentially suppressed with increasing the system
size (provided the boundary conditions do not favor any particular phase).

\subsection*{\href{http://arxiv.org/abs/2009.11157v1}{Effect of reduced local lattice disorder on the magnetic properties of  B-site substituted La0.8Sr0.2MnO3}}
\subsubsection*{Sagar Ghorai, \dots, and Peter Svedlindh (2020-09-23)}
Disorder induced by chemical inhomogeneity and Jahn-Teller (JT) distortions
is often observed in mixed valence perovskite manganites. The main reasons for
the evolution of this disorder are connected with the cationic size differences
and the ratio between JT active and non-JT active ions. The quenched disorder
leads to a spin-cluster state above the magnetic transition temperature. The
effect of Cu, a B-site substitution in the La0.8Sr0.2MnO3 compound, on the
disordered phase has been addressed here. X-ray powder diffraction reveals
rhombohedral (R-3c) structures for the two compounds with negligible change of
lattice volume. The chemical compositions of the two compounds were verified by
ion beam analysis technique. With the change of electronic bandwidth, the
magnetic phase transition temperature has been tuned towards room temperature
(318 K), an important requirement for room temperature magnetic refrigeration.
However, a small decrease of the isothermal entropy was observed with
Cu-substitution, related to the decrease of the saturation magnetization.

\subsection*{\href{http://arxiv.org/abs/2009.11153v1}{High-pressure structural study of a-Mn: solving a three decades-old  mystery}}
\subsubsection*{Logan K. Magad-Weiss, \dots, and Elissaios Stavrou (2020-09-23)}
Manganese, in the a-Mn structure, has been studied using synchrotron powder
x-ray diffraction in a diamond anvil cell up to 220 GPa at room temperature
combined with density functional calculations (DFT). The experiment reveals an
extended pressure stability of the a-Mn phase up to the highest pressure of
this study, in contrast with previous experimental and theoretical studies. On
the other hand, calculations reveal that the previously predicted hcp-Mn phase
becomes lower in enthalpy than the a-Mn phase above 160 GPa. The apparent
discrepancy is explained due to a substantial electron transfer between Mn
ions, which stabilizes the a-Mn phase through the formation of ionic bonding
between monatomic ions under pressure.

\subsection*{\href{http://arxiv.org/abs/2009.11151v1}{Threshold theorem in isolated quantum dynamics with local stochastic  control errors}}
\subsubsection*{Manaka Okuyama, and Masayuki Ohzeki (2020-09-23)}
We investigate the effect of local stochastic control errors in the
time-dependent Hamiltonian on isolated quantum dynamics. The control errors are
formulated as time-dependent stochastic noise in the Schr\"odinger equation.
For any local stochastic control errors, we establish a threshold theorem that
provides a sufficient condition to obtain the target state, which should be
determined in noiseless isolated quantum dynamics, as a relation between the
number of measurements required and noise strength. The theorem guarantees that
if the sum of the noise strengths is less than the inverse of computational
time, the target state can be obtained through a constant-order number of
measurements. If the opposite is true, the required number of measurements
increases exponentially with computational time. Our threshold theorem can be
applied to any isolated quantum dynamics such as quantum annealing, adiabatic
quantum computation, the quantum approximate optimization algorithm, and the
quantum circuit model.

\subsection*{\href{http://arxiv.org/abs/2009.11145v1}{Emergence of effective temperatures in an out-of-equilibrium model of  RNA folding}}
\subsubsection*{Marco Ancona, \dots, and Alessandro Pelizzola (2020-09-23)}
We investigate the possibility of extending the notion of temperature in a
stochastic model for the RNA/protein folding driven out of equilibrium. We
simulate the dynamics of a small RNA hairpin subject to an external pulling
force, which is time-dependent. First, we consider a fluctuation-dissipation
relation (FDR), that extends the fluctuation-dissipation theorem in
nonequilibrium contexts. We verify that various effective temperatures values
can be obtained in a range of parameters and for different observables, only
when the slowest intrinsic relaxation timescale of the system regulates the
dynamics of the system. Then, we introduce a different nonequilibrium
temperature, which is defined from the rate of heat exchanged with a
weakly-interacting thermal bath. Notably, this `kinetic' temperature can be
defined for any frequency of the external switching force. We also discuss and
compare the behavior of these two emerging parameters, by discriminating the
time-delayed nature of the FDR temperature from the instantaneous character of
the kinetic temperature.

\subsection*{\href{http://arxiv.org/abs/2009.11137v1}{Partial Order-Disorder Transition Driving Closure of Band Gap: Example  of Thermoelectric Clathrates}}
\subsubsection*{Maria Troppenz, \dots, and Claudia Draxl (2020-09-23)}
On the quest for efficient thermoelectrics, semiconducting behavior is a
targeted property. Yet, this is often difficult to achieve due to the complex
interplay between electronic structure, temperature, and disorder. We find this
to be the case for the thermoelectric clathrate Ba$_8$Al$_{16}$Si$_{30}$:
Although this material exhibits a band gap in its groundstate, a
temperature-driven partial order-disorder transition leads to its effective
closing. This finding is enabled by a novel approach to calculate the
temperature-dependent effective band structure of alloys. Our method fully
accounts for the effects of short-range order and can be applied to complex
alloys with many atoms in the primitive cell, without relying on effective
medium approximations.

\subsection*{\href{http://arxiv.org/abs/2009.11132v1}{Solute-point defect interactions, coupled diffusion, and radiation  induced segregation in fcc nickel}}
\subsubsection*{E. Toijer, \dots, and P. Olsson (2020-09-23)}
Radiation-induced segregation (RIS) of solutes in materials exposed to
irradiation is a well-known problem. It affects the life-time of nuclear
reactor core components by favouring radiation-induced degradation phenomena
such as hardening and embrittlement. In this work, RIS tendencies in
face-centered cubic (fcc) Ni-X (X = Cr, Fe, Ti, Mn, Si, P) dilute binary alloys
are examined. The goal is to investigate the driving forces and kinetic
mechanisms behind the experimentally observed segregation. By means of ab
initio calculations, point-defect stabilities and interactions with solutes are
determined, together with migration energies and attempt frequencies. Transport
and diffusion coefficients are then calculated in a mean-field framework, to
get a full picture of solute-defect kinetic coupling in the alloys. Results
show that all solutes considered, with the exception of Cr, prefer
vacancy-mediated over interstitial-mediated diffusion during both thermal and
radiation-induced migration. Cr, on the other hand, preferentially migrates in
a mixed-dumbbell configuration. P and Si are here shown to be enriched, and Fe
and Mn to be depleted at sinks during irradiation of the material. Ti and Cr,
on the other hand, display a crossover between enrichment at lower
temperatures, and depletion in the higher temperature range. Results in this
work are compared with previous studies in body-centered cubic (bcc) Fe, and
discussed in the context of RIS in austenitic alloys.

\subsection*{\href{http://arxiv.org/abs/2009.11125v1}{The quantum canonical ensemble in phase space}}
\subsubsection*{Alfredo M. Ozorio de Almeida, and Olivier Brodier (2020-09-23)}
The density operator for a quantum system in thermal equilibrium with its
environment depends on Planck's constant, as well as the temperature. At high
temperatures, the Weyl representation, that is, the thermal Wigner function,
becomes indistinguishable from the corresponding classical distribution in
phase space, whereas the low temperature limit singles out the quantum ground
state of the system's Hamiltonian. In all regimes, thermal averages of
arbitrary observables are evaluated by integrals, as if the thermal Wigner
function were a classical distribution.
  The extension of the semiclassical approximation for quantum propagators to
an imaginary thermal time, bridges the complex intervening region between the
high and the low temperature limit. This leads to a simple quantum correction
to the classical high temperature regime, irrespective of whether the motion is
regular or chaotic. A variant of the full semiclassical approximation with a
real thermal time, though in a doubled phase space, avoids any search for
particular trajectories in the evaluation of thermal averages. The double
Hamiltonian substitutes the stable minimum of the original system's Hamiltonian
by a saddle, which eliminates local periodic orbits from the stationary phase
evaluation of the integrals for the partition function and thermal averages.

\subsection*{\href{http://arxiv.org/abs/2009.11664v1}{High reflectance and optical dynamics of a metasurface comprizing  quantum $Λ$-emitters}}
\subsubsection*{Igor V. Ryzhov, and Victor A. Malyshev (2020-09-23)}
In this Letter, we study theoretically reflectance of a monolayer comprizing
regularly spaced quantum $\Lambda$-emitters. Due to high density of the latter,
the monolayer almost totally reflects the incident field in the vicinity of the
system's collective (excitonic) resonance. The emitter self-action through the
secondary field provides a positive feedback, interplay of which with the
inherent nonlinearity of an emitter itself, results in an exotic behavior of
the system reflectance, including bistability, self-oscillations, and chaotic
dynamics. All these features might be of interest for nanophotonic applications

\subsection*{\href{http://arxiv.org/abs/2009.11115v1}{Thermodynamic bound on speed limit in systems with unidirectional  transitions and a tighter bound}}
\subsubsection*{Deepak Gupta and Daniel M. Busiello (2020-09-23)}
We consider a general discrete state-space system with both unidirectional
and bidirectional links. In contrast to bidirectional links, there is no
reverse transition along the unidirectional links. Herein, we first compute the
statistical length and the thermodynamic cost function for transitions in the
probability space, highlighting contributions from total, environmental, and
resetting (unidirectional) entropy production. Then, we derive the
thermodynamic bound on the speed limit to connect two distributions separated
by a finite time, showing the effect of the presence of unidirectional
transitions. Novel uncertainty relationships can be found for the
\textit{temporal} first and second moments of the average resetting entropy
production. We derived simple expressions in the limit of slow unidirectional
transition rates. Finally, we present a refinement of the thermodynamic bound,
by means of an optimization procedure. We numerically investigate these results
on systems that stochastically reset with constant and periodic resetting rate.

\subsection*{\href{http://arxiv.org/abs/2009.11091v1}{Generalized Langevin equations and fluctuation-dissipation theorem for  particle-bath systems in electric and magnetic fields}}
\subsubsection*{Vladimir Lisy and Jana Tothova (2020-09-23)}
The Brownian motion of a particle immersed in a medium of charged particles
is considered when the system is placed in magnetic or electric fields. Coming
from the Zwanzig-Caldeira-Legget particle-bath model, we modify it so that not
only the charged Brownian particle (BP) but also the bath particles respond to
the external fields. For stationary systems the generalized Langevin equations
are derived. Arbitrarily time-dependent electric fields do not affect the
memory functions, the thermal noise force, and the BP velocity correlation
functions. In the case of a constant magnetic field two equations with
different memory functions are obtained for the BP motion in the plane
perpendicular to the field. As distinct from the previous theories, the random
thermal force depends on the field magnitude. Its time correlation function is
connected with one of the found memory functions through the familiar second
fluctuation-dissipation theorem.

\subsection*{\href{http://arxiv.org/abs/2009.11088v1}{Gauge Fixing for Strongly Correlated Electrons coupled to Quantum Light}}
\subsubsection*{Olesia Dmytruk and Marco Schiró (2020-09-23)}
We discuss the problem of gauge fixing for strongly correlated electrons
coupled to quantum light, described by projected low-energy models such as
those obtained within tight-binding methods. Drawing from recent results in the
field of quantum optics, we present a general approach to write down quantum
light-matter Hamiltonian in either dipole or Coulomb gauge which are explicitly
connected by a unitary transformation, thus ensuring gauge equivalence even
after projection. The projected dipole gauge Hamiltonian features a linear
light-matter coupling and an instantaneous self-interaction for the electrons,
similar to the structure in the full continuum theory. On the other hand, in
the Coulomb gauge the photon field enters in a highly non-linear way, through
phase factors that dress the electronic degrees of freedom. We show that our
approach generalises the well-known Peierls approximation, to which it reduces
when local, on-site orbital contributions to light-matter coupling are
disregarded. As an application, we study a two-orbital model of interacting
electrons coupled to a uniform cavity mode, recently studied in the context of
excitonic superradiance and associated no-go theorems. Using both gauges we
recover the absence of superradiant phase in the ground state and show that
excitations on top of it, described by polariton modes, contain instead
non-trivial light-matter entanglement. Our results highlight the importance of
treating the non-linear light-matter interaction of the Coulomb gauge
non-perturbatively, to obtain a well-defined ultrastrong coupling limit and to
not spoil gauge equivalence.

\subsection*{\href{http://arxiv.org/abs/2009.11057v1}{Theoretical analysis of quantum turbulence using the Onsager "ideal  turbulence" theory}}
\subsubsection*{Tomohiro Tanogami (2020-09-23)}
To understand the effect of quantum stress on energy transfer across scales,
we study three-dimensional quantum turbulence as described by the
Gross-Pitaevskii equation by using the analytical method exploited in the
Onsager "ideal turbulence" theory. It is shown that two types of scale-to-scale
energy flux exist that contribute to the energy cascade: the energy flux that
is the same as in classical turbulence, and that induced by quantum stress. We
then propose a definition of the inertial range for quantum turbulence
$\ell_{\mathrm{small}}\ll\ell\ll\ell_{\mathrm{large}}$, where
$\ell_{\mathrm{large}}$ is determined through pressure-dilatation and
$\ell_{\mathrm{small}}$ through quantum-stress-strain. Under assumptions on the
regularity of the velocity and density fields, we show that, in the steady
state in which the total mean kinetic energy is constant, the classical-type
energy flux becomes dominant in $\ell_i\ll\ell\ll\ell_{\mathrm{large}}$, while
the quantum-type flux becomes dominant in
$\ell_{\mathrm{small}}\ll\ell\ll\ell_i$, where $\ell_i$ is the mean intervortex
distance. Correspondingly, the velocity power spectrum exhibits a power-law
behavior: $k^{-5/3}$ in $\ell^{-1}_{\mathrm{large}}\ll k\ll\ell^{-1}_i$ and
$k^{-3}$ in $\ell^{-1}_i\ll k\ll\ell^{-1}_{\mathrm{small}}$.

\subsection*{\href{http://arxiv.org/abs/2009.11052v1}{Kondo effects in small bandgap carbon nanotube quantum dots}}
\subsubsection*{P. Florków, and S. Lipiński (2020-09-23)}
We study magnetoconductance of the small bandgap carbon nanotube quantum dots
in the presence of spin-orbit coupling in the strong correlations regime. The
finite-U mean field slave boson approach is used to study many-body effects.
Different degeneracies are restored in magnetic field and Kondo effects of
different symmetries arise including SU(3) effects of different types. Full
spin-orbital degeneracy might be recovered for zero field and correspondingly
SU(4) Kondo effect sets in. We point out on the possibility of the occurrence
of electron-hole Kondo effects in slanting magnetic fields, which we predict
will occur in the available magnetic fields for orientation of fields close to
perpendicular. When the field approaches transverse orientation a crossover
from SU(2) or SU(3) symmetry into SU(4) is observed.

\subsection*{\href{http://arxiv.org/abs/2009.11047v1}{Universal dynamics of superradiant phase transition in the anisotropic  quantum Rabi model}}
\subsubsection*{Xunda Jiang, \dots, and Chaohong Lee (2020-09-23)}
We investigate the universally non-equilibrium dynamics of superradiant phase
transition in the anisotropic quantum Rabi model. By introducing position and
momentum operators, we obtain the ground states and their excitation gaps for
both normal and superradiant phases via perturbation theory. We analytically
extract the critical exponents from the excitation gap and the diverging length
scale near the critical point, and find that the critical exponents are
independent upon the anisotropy ratio. Moreover, by simulating the real-time
dynamics across the critical point, we numerically extract the critical
exponents from the phase transition delay and the diverging length scale, which
are well consistent with the analytical ones. Our study provides a dynamic way
to explore universal critical behaviors in the quantum Rabi model.

\subsection*{\href{http://arxiv.org/abs/2009.11046v1}{A search for neutron to mirror-neutron oscillations}}
\subsubsection*{C. Abel, \dots, and G. Zsigmond (2020-09-23)}
It has been proposed that there could be a mirror copy of the standard model
particles, restoring the parity symmetry in the weak interaction on the global
level. Oscillations between a neutral standard model particle, such as the
neutron, and its mirror counterpart could potentially answer various standing
issues in physics today. Astrophysical studies and terrestrial experiments led
by ultracold neutron storage measurements have investigated neutron to
mirror-neutron oscillations and imposed stringent constraints. Recently,
further analysis of these ultracold neutron storage experiments has yielded
statistically significant anomalous signals that may be interpreted as neutron
to mirror-neutron oscillations, assuming nonzero mirror magnetic fields. The
neutron electric dipole moment collaboration performed a dedicated search at
the Paul Scherrer Institute and found no evidence of neutron to mirror-neutron
oscillations. Thereby, the following new lower limits on the oscillation time
were obtained: $\tau_{nn'} > 352~$s at $B'=0$ (95\% C.L.), $\tau_{nn'} >
6~\text{s}$ for all $0.4~\mu\text{T}9~\text{s}$ for all
$5.0~\mu\text{T}<B'<25.4~\mu\text{T}$ (95\% C.L.), where $\beta$ is the fixed
angle between the applied magnetic field and the ambient mirror magnetic field
which is assumed to be bound to a reference frame rotating with the Earth.
These new constraints are the best measured so far around $B'\sim10~\mu$T, and
$B'\sim20~\mu$T.

\subsection*{\href{http://arxiv.org/abs/2009.11036v1}{Attenuating the fermion sign problem in path integral Monte Carlo  simulations using the Bogoliubov inequality and thermodynamic integration}}
\subsubsection*{Tobias Dornheim, \dots, and Barak Hirshberg (2020-09-23)}
Accurate thermodynamic simulations of correlated fermions using path integral
Monte Carlo (PIMC) methods are of paramount importance for many applications
such as the description of ultracold atoms, electrons in quantum dots, and
warm-dense matter. The main obstacle is the fermion sign problem (FSP), which
leads to an exponential increase in computation time both with increasing the
system-size and with decreasing temperature. Very recently, Hirshberg et al.
[J. Chem. Phys. 152, 171102 (2020)] have proposed to alleviate the FSP based on
the Bogoliubov inequality. In the present work, we extend this approach by
adding a parameter that controls the perturbation, allowing for an
extrapolation to the exact result. In this way, we can also use thermodynamic
integration to obtain an improved estimate of the fermionic energy. As a test
system, we choose electrons in 2D and 3D quantum dots and find in some cases a
speed-up exceeding 10\^6 , as compared to standard PIMC, while retaining a
relative accuracy of $\sim0.1\%$. Our approach is quite general and can readily
be adapted to other simulation methods.

\subsection*{\href{http://arxiv.org/abs/2009.11031v1}{Phonon-assisted Exciton Dissociation in Transition Metal Dichalcogenides}}
\subsubsection*{Raul Perea-Causin, and Ermin Malic (2020-09-23)}
Monolayers of transition metal dichalcogenides (TMDs) have been established
in the last years as promising materials for novel optoelectronic devices.
However, the performance of such devices is often limited by the dissociation
of tightly bound excitons into free electrons and holes. While previous studies
have investigated tunneling at large electric fields, we focus in this work on
phonon-assisted exciton dissociation that is expected to be the dominant
mechanism at small fields. We present a microscopic model based on the density
matrix formalism providing access to time- and momentum-resolved exciton
dynamics including phonon-assisted dissociation. We track the pathway of
excitons from optical excitation via thermalization to dissociation,
identifying the main transitions and dissociation channels. Furthermore, we
find intrinsic limits for the quantum efficiency and response time of a
TMD-based photodetector and investigate their tunability with externally
accessible knobs, such as excitation energy, substrate screening, temperature
and strain. Our work provides microscopic insights in fundamental mechanisms
behind exciton dissociation and can serve as a guide for the optimization of
TMD-based optoelectronic devices.

\subsection*{\href{http://arxiv.org/abs/2009.11028v2}{Particle-hole symmetry breaking in a spin-dimer system TlCuCl$_3$  observed at 100 T}}
\subsubsection*{X. -G. Zhou, \dots, and H. Tanaka (2020-09-23)}
The entire magnetization process of TlCuCl$_3$ has been experimentally
investigated up to 100 T employing the single turn technique. The upper
critical field $H_{c2}$ is observed to be 86.1 T at 2 K. A convex slope of the
$M$-$H$ curve between the lower and upper critical fields ($H_{c1}$ and
$H_{c2}$) is clearly observed, which indicates that a particle-hole symmetry is
broken in TlCuCl$_3$. By quantum Monte Carlo simulation and the bond-operator
theory method, we find that the particle-hole symmetry breaking results from
strong inter-dimer interactions.

\subsection*{\href{http://arxiv.org/abs/2009.11656v1}{Skyrmion lattice creep at ultra-low current densities}}
\subsubsection*{Yongkang Luo, \dots, and Boris Maiorov (2020-09-23)}
Magnetic skyrmions are well-suited for encoding information because they are
nano-sized, topologically stable, and only require ultra-low critical current
densities $j_c$ to depin from the underlying atomic lattice. Above $j_c$
skyrmions exhibit well-controlled motion, making them prime candidates for
race-track memories. In thin films thermally-activated creep motion of isolated
skyrmions was observed below $j_c$ as predicted by theory. Uncontrolled
skyrmion motion is detrimental for race-track memories and is not fully
understood. Notably, the creep of skyrmion lattices in bulk materials remains
to be explored. Here we show using resonant ultrasound spectroscopy--a probe
highly sensitive to the coupling between skyrmion and atomic lattices--that in
the prototypical skyrmion lattice material MnSi depinning occurs at $j_c^*$
that is only 4 percent of $j_c$. Our experiments are in excellent agreement
with Anderson-Kim theory for creep and allow us to reveal a new dynamic regime
at ultra-low current densities characterized by thermally-activated
skyrmion-lattice-creep with important consequences for applications.

\subsection*{\href{http://arxiv.org/abs/2009.11026v1}{Analytical solution to the surface states of antiferromagnetic  topological insulator MnBi$_2$Te$_4$}}
\subsubsection*{Hai-Peng Sun, \dots, and X. C. Xie (2020-09-23)}
Recently, the intrinsic magnetic topological insulator MnBi$_2$Te$_4$ has
attracted great attention. It has an out-of-plane antiferromagnetic order,
which is believed to open a sizable energy gap in the surface states. This gap,
however, was not always observable in the latest ARPES experiments. To address
this issue, we analytically derive an effective model for the 2D surface states
by starting from a 3D Hamiltonian for bulk MnBi$_2$Te$_4$ and taking into
account the spatial profile of the bulk magnetization. We suggest that the Bi
antisite defects in the Mn atomic layers in the bulk may be one of the reasons
for the varied experimental results, since the Bi antisite defects may result
in a much smaller and more localized intralayer ferromagnetic order, leading to
the diminished surface gap. In addition, we calculate the spatial distribution
and penetration depth of the surface states, which are mainly embedded in the
first two septuple layers from the terminating surface. From our analytical
results, the influence of the bulk parameters on the surface states can be
found explicitly. Furthermore, we derive a $\bf{k}\cdot \bf{p}$ model for
MnBi$_2$Te$_4$ thin films and show the oscillation of the Chern number between
odd and even septuple layers. Our results will be helpful for the ongoing
explorations of the MnBi$_x$Te$_y$ family.

\subsection*{\href{http://arxiv.org/abs/2009.11024v1}{Multiscale network renormalization: scale-invariance without geometry}}
\subsubsection*{Elena Garuccio, and Diego Garlaschelli (2020-09-23)}
Systems with lattice geometry can be renormalized exploiting their embedding
in metric space, which naturally defines the coarse-grained nodes. By contrast,
complex networks defy the usual techniques because of their small-world
character and lack of explicit metric embedding. Current network
renormalization approaches require strong assumptions (e.g. community
structure, hyperbolicity, scale-free topology), thus remaining incompatible
with generic graphs and ordinary lattices. Here we introduce a graph
renormalization scheme valid for any hierarchy of coarse-grainings, thereby
allowing for the definition of `block nodes' across multiple scales. This
approach reveals a necessary and specific dependence of network topology on an
additive hidden variable attached to nodes, plus optional dyadic factors.
Renormalizable networks turn out to be consistent with a unique specification
of the fitness model, while they are incompatible with preferential attachment,
the configuration model or the stochastic blockmodel. These results highlight a
deep conceptual distinction between scale-free and scale-invariant networks,
and provide a geometry-free route to renormalization. If the hidden variables
are annealed, the model spontaneously leads to realistic scale-free networks
with cut-off. If they are quenched, the model can be used to renormalize
real-world networks with node attributes and distance-dependence or
communities. As an example we derive an accurate multiscale model of the
International Trade Network applicable across hierarchical geographic
partitions.

\subsection*{\href{http://arxiv.org/abs/2009.11021v1}{Quantum frequency conversion based on resonant four-wave mixing}}
\subsubsection*{Chin-Yao Cheng, \dots, and Yong-Fan Chen (2020-09-23)}
Quantum frequency conversion (QFC), a critical technology in photonic quantum
information science, requires that the quantum characteristics of the
frequency-converted photon must be the same as the input photon except for the
color. In nonlinear optics, the wave mixing effect far away from the resonance
condition is often used to realize QFC because it can prevent the vacuum field
reservoir from destroying the quantum state of the converted photon
effectively. Under conditions far away from resonance, experiments typically
require strong pump light to generate large nonlinear interactions to achieve
high-efficiency QFC. However, strong pump light often generates additional
noise photons through spontaneous Raman or parameter conversion processes.
Herein, we theoretically study another efficient QFC scheme based on a resonant
four-wave mixing system. Due to the effect of electromagnetically induced
transparency (EIT), this resonant QFC scheme can greatly suppress vacuum field
noise at low light levels; consequently, the converted photons can inherit the
quantum state of the input photon with high fidelity. Our research demonstrates
that if the conversion efficiency of the EIT-based QFC is close to 100\%, the
wave function and quadrature variance of the converted photons are almost the
same as the input probe photons.

\subsection*{\href{http://arxiv.org/abs/2009.11012v1}{Non-local cooperative atomic motions that govern dissipation in  amorphous tantala unveiled by dynamical mechanical spectroscopy}}
\subsubsection*{Francesco Puosi, \dots, and Dino Leporini (2020-09-23)}
The mechanisms governing mechanical dissipation in amorphous tantala are
studied at microscopic scale via Molecular Dynamics simulations, namely by
mechanical spectroscopy in a wide range of temperature and frequency. We find
that dissipation is associated with irreversible atomic rearrangements with a
sharp cooperative character, involving tens to hundreds of atoms arranged in
spatially extended clusters of polyhedra. Remarkably, at low temperature we
observe an excess of plastically rearranging oxygen atoms which correlates with
the experimental peak in the macroscopic mechanical losses. A detailed
structural analysis reveals preferential connections of the irreversibly
rearranging polyhedra, corresponding to edge and face sharing. These results
might lead to microscopically informed design rules for reducing mechanical
losses in relevant materials for structural, optical, and sensing applications.

\subsection*{\href{http://arxiv.org/abs/2009.11006v1}{Enhancement of anomalous Nernst effect in Ni/Pt superlattice}}
\subsubsection*{T. Seki, \dots, and K. Takanashi (2020-09-23)}
We report an enhancement of the anomalous Nernst effect (ANE) in Ni/Pt (001)
epitaxial superlattices. The transport and magneto-thermoelectric properties
were investigated for the Ni/Pt superlattices with various Ni layer thicknesses
(${\it t}$). The anomalous Nernst coefficient was increased up to more than 1
${\mu}$V K$^{-1}$ for 2.0 nm ${\leq}$ ${\it t}$ ${\leq}$ 4.0 nm, which was the
remarkable enhancement compared to the bulk Ni. It has been found that the
large transverse Peltier coefficient (${\alpha}$$_{xy}$), reaching
${\alpha}$$_{xy}$ = 4.8 A K$^{-1}$ m$^{-1}$ for ${\it t}$ = 4.0 nm, plays a
prime role for the enhanced ANE of the Ni/Pt (001) superlattices.

\subsection*{\href{http://arxiv.org/abs/2009.11003v1}{Non-trivial retardation effects in dispersion forces: From anomalous  distance dependence to novel traps}}
\subsubsection*{Johannes Fiedler, \dots, and Mathias Boström (2020-09-23)}
In the study of dispersion forces, nonretarded, retarded and thermal
asymptotes with their distinct scaling laws are regarded as cornerstone results
governing interactions at different separations. Here, we show that when
particles interact in a medium, the influence of retardation is qualitatively
different, making it necessary to consider the non-monotonous potential in
full. We discuss different regimes for several cases and find an anomalous
behaviour of the retarded asymptote. It can change sign, and lead to a trapping
potential.

\subsection*{\href{http://arxiv.org/abs/2009.11001v1}{Quantum Assisted Eigensolver}}
\subsubsection*{Kishor Bharti (2020-09-23)}
We propose a hybrid quantum-classical algorithm for approximating the ground
state and ground state energy of a Hamiltonian. Once the Ansatz has been
decided, the quantum part of the algorithm involves the calculation of two
overlap matrices. The output from the quantum part of the algorithm is utilized
as input for the classical computer. The classical part of the algorithm is a
quadratically constrained quadratic program with a single quadratic equality
constraint. Unlike the variational quantum eigensolver algorithm, our algorithm
does not have any classical-quantum feedback loop. Using convex relaxation
techniques, we provide an efficiently computable lower bound to the classical
optimization program. Furthermore, using results from Bar-On et al. (Journal of
Optimization Theory and Applications, 82(2):379--386, 1994), we provide a
sufficient condition for a local minimum to be a global minimum. A solver can
use such a condition as a stopping criterion. We expect our algorithm to
deliver the first practical application of existing quantum computers.

\subsection*{\href{http://arxiv.org/abs/2009.10996v1}{Long-Term Stabilization of Two-Dimensional Perovskites by Encapsulation  with Hexagonal Boron Nitride}}
\subsubsection*{Michael Seitz, \dots, and Ferry Prins (2020-09-23)}
Metal halide perovskites are known to suffer from rapid degradation, limiting
their direct applicability. Here, the degradation of phenethylammonium lead
iodide (PEA2PbI4) two-dimensional perovskites under ambient conditions is
studied using fluorescence, absorbance and fluorescence lifetime measurements.
It is demonstrated that a long-term stability of two-dimensional perovskites
can be achieved through the encapsulation with hexagonal boron nitride. While
un-encapsulated perovskite flakes degrade within hours, the encapsulated
perovskites are stable for at least three months. In addition, encapsulation
considerably improves the stability under laser irradiation. The environmental
stability, combined with the improved durability under illumination, is a
critical ingredient for thorough spectroscopic studies of the intrinsic
opto-electronic properties of this material platform.

\subsection*{\href{http://arxiv.org/abs/2009.10986v1}{Technological implementation of a photonic Bier-Glass cavity}}
\subsubsection*{Jonathan Jurkat, \dots, and Christian Schneider (2020-09-23)}
In this paper, we introduce a novel quantum photonic device, which we term
photonic Bier-Glass cavity. We discuss its fabrication and functionality, which
is based on the coupling of integrated In(Ga)As quantum dots to a broadband
photonic cavity resonance. By design, the device architecture uniquely combines
the Purcell enhancement of a photonic micropillar structure with broadband
photonic mode shaping of a vertical, tapered waveguide, making it an
interesting candidate to enable the efficient extraction of entangled photon
pairs. We detail the epitaxial growth of the heterostructure and the necessary
lithography steps to approach a GaAs-based photonic device with a height
exceeding 15 $\mu$m, supported on a pedestal that can be as thin as 20 nm. We
present an optical characterization, which confirms the presence of broadband
optical resonances, in conjunction with amplified spontaneous emission of
single photons.

\subsection*{\href{http://arxiv.org/abs/2009.10985v1}{Transport experiments in semiconductor-superconductor hybrid Majorana  devices}}
\subsubsection*{Bin Li, \dots, and Ming-Tang Deng (2020-09-23)}
As the condensed matter analog of Majorana fermion, Majorana zero-mode is
well known as a building block of fault-tolerant topological quantum computing.
In this review, we focus on the recent progress of Majorana experiments,
especially experiments about semiconductor-superconductor hybrid devices. We
first sketch Majorana zero-mode formation from a bottom-up view, which is more
suitable for beginners and experimentalist. Then, we survey the status of
zero-energy state signatures reported recently, from zero-energy conductance
peaks, the oscillations, the quantization, and the interactions with
extra-degrees of freedom. This paper also gives prospects on future experiments
for advancing one-dimensional semiconductor nanowire-superconductor hybrid
materials and devices.

\subsection*{\href{http://arxiv.org/abs/2009.10977v1}{Structural diversity in electrohydrodynamically driven active and  organized liquids}}
\subsubsection*{Geet Raju, and Jaakko V. I. Timonen (2020-09-23)}
Spontaneous emergence of organized states of matter driven by non-equilibrium
conditions is of significant fundamental and technological interest. In many
cases, the organized states are complex, hence, with some well-studied
exceptions, their emergence is challenging to predict. In this letter, we show
that an unexpectedly diverse collection of dissipative organized states can
emerge in a simple biphasic system consisting of two liquids under planar
confinement. We drive the liquid-liquid interface, which is held together by
capillary forces, out of thermodynamic equilibrium using DC electrohydrodynamic
shearing. As a result, the interface goes through multiple spontaneous symmetry
breakings, leading to various organized non-equilibrium states. First, at low
shearing, the shear-deformed interface becomes unstable and a 1D quasi-static
corrugation pattern emerges. At slightly higher shearing, we observe
topological changes that lead to emergence of active self-propulsive fluidic
filaments and filament networks, as well as ordered bicontinuous fluidic
lattices, including a Kagome lattice. Finally, the system transitions into
active self-propulsive droplets, quasi-stationary dissipating polygonal and
toroidal droplets, and ultimately to a chaotic active emulsion of
non-coalescing droplets with complex interactions. Interestingly, this single
system captures many features from continuum non-equilibrium pattern formation
and discrete active particles, which are often considered separate fields of
study. The diversity of observed dissipative organized states is exceptional
and points towards many new avenues in the study of electrohydrodynamics,
capillary phenomena, non-equilibrium pattern formation, and active materials.

\subsection*{\href{http://arxiv.org/abs/2009.10969v1}{Magnetoelectric coupling and multi-blocking effect in Ising-chain magnet  Sr2Ca2CoMn2O9}}
\subsubsection*{Tathamay Basu, \dots, and Vincent Hardy (2020-09-23)}
We have demonstrated magnetoelectric (ME) coupling in an Ising-chain magnet
Sr2Ca2CoMn2O9, via detailed investigation of ac susceptibility and dielectric
constant as a function of temperature, magnetic field and frequency.
Sr2Ca2CoMn2O9 consists of spin-chains of one CoO6 trigonal prismatic and two
alternating MnO6 octahedra polyhedron. The (Co2+ Mn4+ Mn4+) distribution
stabilizes a (up-down-up) spin-state along the chains which are distributed on
a triangular lattice. This compound undergoes a partially disordered
antiferromagnetic transition at TN~28 K. The dielectric constant exhibits a
clear peak at TN only in presence of an external magnetic field (H>5 kOe),
evidencing the presence of ME coupling, which is further confirmed by
H-dependence dielectric measurements. The mechanism of this ME coupling is
discussed as a result of exchange-striction in an Ising-chain magnet. In
addition to this strong spin-lattice coupling, we report a dipolar relaxation
phenomenon similar to spin-relaxation arising from the single-ion magnetism
(spin-blocking effect). We term such phenomenon a multi-blocking effect.

\subsection*{\href{http://arxiv.org/abs/2009.10964v1}{Thickness dependence of antiferromagnetic phase transition in  Heisenberg-type MnPS3}}
\subsubsection*{Soo Yeon Lim, \dots, and Hyeonsik Cheong (2020-09-23)}
The behavior of 2-dimensional (2D) van der Waals (vdW) layered magnetic
materials in the 2D limit of the few-layer thickness is an important
fundamental issue for the understanding of the magnetic ordering in lower
dimensions. The antiferromagnetic transition temperature TN of the
Heisenberg-type 2D magnetic vdW material MnPS3 was estimated as a function of
the number of layers. The antiferromagnetic transition was identified by
temperature-dependent Raman spectroscopy, from the broadening of a phonon peak
at 155 cm-1, accompanied by an abrupt redshift and an increase of its spectral
weight. TN is found to decrease only slightly from ~78 K for bulk to ~ 66 K for
3L. The small reduction of TN in thin MnPS3 approaching the 2D limit implies
that the interlayer vdW interaction is playing an important role in stabilizing
magnetic ordering in layered magnetic materials.

\subsection*{\href{http://arxiv.org/abs/2009.10960v1}{Intrinsic and extrinsic effects on intraband optical conductivity of hot  carriers in photoexcited graphene}}
\subsubsection*{Masatsugu Yamashita and Chiko Otani (2020-09-23)}
We present a numerical study on the intraband optical conductivity of hot
carriers at quasiequilibria in photoexcited graphene based on the semiclassical
Boltzmann transport equations (BTE) with the aim of understanding the effects
of intrinsic optical phonon and extrinsic coulomb scattering caused by charged
impurities at the graphene{substrate interface. We employ iterative solutions
of the BTE and the comprehensive model for the temporal evolutions of
hot-carrier temperature and hot-optical-phonon occupations to reduce
computational costs, instead of using full-BTE solutions. Undoped graphene
exhibited large positive photoconductivity owing to the increase in thermally
excited carriers and the reduction in charged impurity scattering. The
frequency dependencies of the photoconductivity in undoped graphene with high
concentrations of charged impurities significantly deviated from that observed
in the simple Drude model, which is attributed to temporally varying charged
impurity scattering during terahertz (THz) probing in the hot-carrier cooling
process. Heavily doped graphene exhibited small negative photoconductivity
similar to that of the Drude model. In this case, charged impurity scattering
is substantially suppressed by the carrier-screening effect, and the
temperature dependencies of the Drude weight and optical phonon scattering
govern negative photoconductivity. In lightly doped graphene, the
photoconductivity changes its sign temporally after the photoexcitation,
depending on the carrier and optical phonon temperatures and the pump uence.
Moreover, the photoconductivity spectra depend not only on the material
property of graphene sample but also on the waveform of the THz-probe pulse.
Our approach provides a quantitative understanding of non-Drude behaviors and
the temporal evolution of photoconductivity in graphene.

\subsection*{\href{http://arxiv.org/abs/2009.10954v1}{Polytypism in Few-Layer Gallium Selenide}}
\subsubsection*{Soo Yeon Lim, \dots, and Hyeonsik Cheong (2020-09-23)}
Gallium selenide (GaSe) is one of layered group-III metal monochalcogenides,
which has an indirect bandgap in monolayer and direct bandgap in bulk unlike
other conventional transition metal dichalcogenides (TMDs) such as MoX2 and WX2
(X=S and Se). Four polytypes of bulk GaSe, designated as beta-, epsilon-,
gamma-, and delta-GaSe, have been reported. Since different polytypes result in
different optical and electrical properties even for the same thickness,
identifying the polytype is essential in utilizing this material for various
optoelectronic applications. We performed polarized Raman measurement on GaSe
and found different ultra-low-frequency Raman spectra of inter-layer
vibrational modes even for the same thickness due to different stacking
sequences of the polytypes. By comparing the ultra-low-frequency Raman spectra
with theoretical calculations and high-resolution electron microscopy
measurements, we established the correlation between the ultra-low-frequency
Raman spectra and the stacking sequences for trilayer GaSe. We further found
that the AB-type stacking is more stable than the AA'-type stacking in GaSe.

\subsection*{\href{http://arxiv.org/abs/2009.10952v1}{Growth and Characterization of Low Sheet Resistance Metalorganic  Vapor-Phase Epitaxy-Grown \b{eta}-(AlxGa1-x)2O3/\b{eta}-Ga2O3 Heterostructure  Channels}}
\subsubsection*{Praneeth Ranga, \dots, and Sriram Krishnamoorthy (2020-09-23)}
We report on growth and characterization of metalorganic vapor-phase
epitaxy-grown \b{eta}-(AlxGa1-x)2O3/ \b{eta}-Ga2O3 modulation-doped
heterostructure. Electron channel is realized in the heterostructure by
utilizing a delta-doped \b{eta}-(AlxGa1-x)2O3 barrier. Electron channel
characteristics are studied using transfer length method, capacitance-voltage
and Hall characterization. Hall sheet charge density of 1.1 x 10\^13 cm\^-2 and
mobility of 111 cm\^2/Vs is measured at room temperature. Fabricated transistor
showed peak current of 22 mA/mm and on-off ratio of 8 x 106. Sheet resistance
of 5.3 k{\Omega}/Square is measured at room temperature, which is the lowest
reported value for a single \b{eta}-(AlxGa1-x)2O3/\b{eta}-Ga2O3
heterostructure.

\subsection*{\href{http://arxiv.org/abs/2009.10949v1}{Rotating edge-field driven processing of chiral spin textures in  racetrack devices}}
\subsubsection*{Alexander F. Schäffer, \dots, and Elena Y. Vedmedenko (2020-09-23)}
Topologically distinct magnetic structures like skyrmions, domain walls, and
the uniformly magnetized state have multiple applications in logic devices,
sensors, and as bits of information. One of the most promising concepts for
applying these bits is the racetrack architecture controlled by electric
currents or magnetic driving fields. In state-of-the-art racetracks, these
fields or currents are applied to the whole circuit. Here, we employ
micromagnetic and atomistic simulations to establish a concept for racetrack
memories free of global driving forces. Surprisingly, we realize that mixed
sequences of topologically distinct objects can be created and propagated over
far distances exclusively by local rotation of magnetization at the sample
boundaries. We reveal the dependence between the chirality of the rotation and
the direction of propagation and define the phase space where the proposed
procedure can be realized. The advantages of this approach are the exclusion of
high current and field densities as well as its compatibility with an
energy-efficient three-dimensional design.

\subsection*{\href{http://arxiv.org/abs/2009.10946v1}{An endoreversible quantum heat engine driven by atomic collisions}}
\subsubsection*{Quentin Bouton, \dots, and Artur Widera (2020-09-23)}
Quantum heat engines are subjected to quantum fluctuations related to their
discrete energy spectra. Such fluctuations question the reliable operation of
quantum engines in the microscopic realm. We here realize an endoreversible
quantum Otto cycle in the large quasi-spin states of Cesium impurities immersed
in an ultracold Rubidium bath. Endoreversible machines are internally
reversible and irreversible losses only occur via thermal contact. We employ
quantum control over both machine and bath to suppress internal dissipation and
regulate the direction of heat transfer that occurs via inelastic spin-exchange
collisions. We additionally use full-counting statistics of individual atoms to
monitor heat exchange between engine and bath at the level of single quanta,
and evaluate average and variance of the power output. We optimize the
performance as well as the stability of the quantum engine, achieving high
efficiency, large power output and small power output fluctuations.

\subsection*{\href{http://arxiv.org/abs/2009.10944v1}{Local trade-off between information and disturbance in quantum  measurements}}
\subsubsection*{Hiroaki Terashima (2020-09-23)}
This study confirms a local trade-off between information and disturbance in
quantum measurements, and establishes the correlation between the changes in
these two quantities when the measurement is slightly modified. The correlation
indicates that when the measurement is modified to increase the obtained
information, the disturbance also increases. However, the information can be
increased while decreasing the disturbance because the correlation is not
necessarily perfect. For measurements with imperfect correlation between the
information and disturbance, this paper discusses a general scheme that raises
the amount of information in the measurement while diminishing the disturbance.

\subsection*{\href{http://arxiv.org/abs/2009.10943v1}{Asymptotic property of current for a conduction model of Fermi particles  on finite lattice}}
\subsubsection*{Kazuki Yamaga (2020-09-23)}
In this paper, we introduce a conduction model of Fermi particles on a finite
sample, and investigate the asymptotic behavior of stationary current for large
sample size. In our model a sample is described by a one-dimensional finite
lattice on which Fermi particles injected at both ends move under various
potentials and noise from the environment. We obtain a simple current formula.
The formula has broad applicability and is used to study various potentials.
When the noise is absent, it provides the asymptotic behavior of the current in
terms of a transfer matrix. In particular, for dynamically defined potential
cases, a relation between exponential decay of the current and the Lyapunov
exponent of a relevant transfer matrix is obtained. For example, it is shown
that the current decays exponentially for the Anderson model. On the other
hand, when the noise exists but the potential does not, an explicit form of the
current is obtained, which scales as 1/N for large sample size N. Moreover, we
provide an extension to higher dimensional systems. For a three-dimensional
case, it is shown that the current increases in proportion to cross section and
decreases in inverse proportion to the length of the sample.

\subsection*{\href{http://arxiv.org/abs/2009.10932v1}{Fluctuating superconductivity in overdoped cuprate with a flat antinodal  dispersion}}
\subsubsection*{Yu He, \dots, and Zhi-Xun Shen (2020-09-23)}
A major unsolved puzzle in cuprate superconductivity is that, despite
accumulated evidence of more conventional normal state properties over the last
30 years, the superconducting $T_c$ of the overdoped cuprates seems to be still
controlled by phase coherence rather than the Cooper pair formation. So far, a
microscopic understanding of this unexpected behavior is lacking. Here we
report angle-resolved photoemission, magnetic and thermodynamic evidence that
Cooper pairs form at temperatures more than 30\% above $T_c$ in overdoped
metallic Bi$_2$Sr$_2$CaCu$_2$O$_{8+\delta}$ (Bi-2212). More importantly, our
data lead to a microscopic understanding where the phase fluctuation is
enhanced by the flat dispersion near the Brillouin zone boundary. This proposal
is tested by a sign-problem free quantum Monte Carlo simulation. Such a
microscopic mechanism is likely to find applications in other flat band
superconductors, such as twisted bilayer-bilayer graphene and NdNiO$_2$

\subsection*{\href{http://arxiv.org/abs/2009.10928v1}{Gamow vectors formalism applied to the Loschmidt echo}}
\subsubsection*{Sebastian Fortin, \dots, and Marcelo Losada (2020-09-23)}
Gamow vectors have been developed in order to give a mathematical description
for quantum decay phenomena. Mainly, they have been applied to radioactive
phenomena, scattering and to some decoherence models. They play a crucial role
in the description of quantum irreversible processes, and in the formulation of
time asymmetry in quantum mechanics. In this paper, we use this formalism to
describe a well-known phenomenon of irreversibility: the Loschmidt echo. The
standard approach considers that the irreversibility of this phenomenon is the
result of an additional term in the backward Hamiltonian. Here, we use the
non-Hermitian formalism, where the time evolution is non-unitary. Additionally,
we compare the characteristic decay times of this phenomenon with the
decoherence ones. We conclude that the Loschmidt echo and the decoherence can
be considered as two aspects of the same phenomenon, and that there is a
mathematical relationship between their corresponding characteristic times.

\subsection*{\href{http://arxiv.org/abs/2009.10921v2}{On Quantum Simulation Of Cosmic Inflation}}
\subsubsection*{Yue-Zhou Li and Junyu Liu (2020-09-23)}
In this paper, we generalize Jordan-Lee-Preskill, an algorithm for simulating
flat-space quantum field theories, to 3+1 dimensional inflationary spacetime.
The generalized algorithm contains the encoding treatment, the initial state
preparation, the inflation process, and the quantum measurement of cosmological
observables at late time. The algorithm is helpful for obtaining predictions of
cosmic non-Gaussianities, serving as useful benchmark problems for quantum
devices, and checking assumptions made about interacting vacuum in the
inflationary perturbation theory.
  Components of our work also include a detailed discussion about the lattice
regularization of the cosmic perturbation theory, a detailed discussion about
the in-in formalism, a discussion about encoding using the HKLL-type formula
that might apply for both dS and AdS spacetimes, a discussion about bounding
curvature perturbations, a description of the three-party Trotter simulation
algorithm for time-dependent Hamiltonians, a ground state projection algorithm
for simulating gapless theories, a discussion about the quantum-extended
Church-Turing Thesis, and a discussion about simulating cosmic reheating in
quantum devices.

\subsection*{\href{http://arxiv.org/abs/2009.10919v1}{Finding high-order Hadamard matrices by using quantum computers}}
\subsubsection*{Andriyan Bayu Suksmono and Yuichiro Minato (2020-09-23)}
Solving hard problems is one of the most important issues in computing to be
addressed by a quantum computer. Previously, we have shown that the H-SEARCH;
which is the problem of finding a Hadamard matrix (H-matrix) among all possible
binary matrices of corresponding order, is a hard problem that can be solved by
a quantum computer. However, due to the limitation on the number of qubits and
connections in present day quantum processors, only low orders H-SEARCH are
implementable. In this paper, we show that by adopting classical
construction/search techniques of the H-matrix, we can develop new quantum
computing methods to find higher order H-matrices. Especially, the Turyn-based
quantum computing method can be further developed to find an arbitrarily high
order H-matrix by balancing the classical and quantum resources. This method is
potentially capable to find some unknown H-matrices of practical and scientific
interests, where a classical computer alone cannot do because of the
exponential grow of the complexity. We present some results of finding H-matrix
of order more than one hundred and a prototypical experiment to find even
higher order matrix by using the classical-quantum resource balancing method.
Although heuristic optimizations generally only achieve approximate solutions,
whereas the exact one should be determined by exhaustive listing; which is
difficult to perform, in the H-SEARCH we can assure such exactness in
polynomial time by checking the orthogonality of the solution. Since quantum
advantage over the classical computing should have been measured by comparing
the performance in solving a problem up to a definitive solution, the proposed
method may lead to an alternate route for demonstrating practical quantum
supremacy in the near future.

\subsection*{\href{http://arxiv.org/abs/2009.10911v2}{Interplay between superconductivity and non-Fermi liquid at a  quantum-critical point in a metal. IV: The $γ$ model and its phase  diagram at $1<γ<2$}}
\subsubsection*{Yi-Ming Wu, and Andrey V. Chubukov (2020-09-23)}
In this paper, we continue our analysis of the interplay between pairing and
non-Fermi liquid behavior for a set of itinerant quantum-critical systems with
an effective dynamical electron-electron interaction $V(\Omega_m) \propto
1/|\Omega_m|^\gamma$ (the $\gamma$-model). In previous papers, we considered
the cases $0 < \gamma <1$ and $\gamma \approx 1$. We argued that (i) at $T=0$,
there exists an infinite, discrete set of solutions of the gap equation
$\Delta_n (\omega_m)$, all with the same spatial symmetry, but with different
number ($n$) of sign changes, (ii) each $\Delta_n$ develops at a particular
onset temperature $T_{p,n}$, and (iii) $\Delta_n (\omega_m)$ and $T_{p,n}$
change smoothly through $\gamma =1$, despite that the self-energy and the
pairing vertex diverge at $\gamma \geq 1$. The condensation energy, $E_{c,n}$
is the largest for $n=0$, but the presence of other solutions gives rise to
additional "longitudinal" gap fluctuations. Here, we analyze how the system
behavior evolves at $1<\gamma <2$, both on the Matsubara axis and in the upper
half-plane of frequency. We report two key new features, which develop at
$\gamma \geq 1$ and become crucial at $\gamma \to 2$. First, the spectrum of
condensation energies become progressively more dense. Second, an array of
vortices emerges in the upper frequency half-plane, and the number of such
vortices increases with $\gamma$. We argue that this behavior is a non-trivial
system reaction to the fact that on a real axis, the real part of the dynamical
electron-electron interaction $V' (\Omega) \propto \cos(\pi
\gamma/2)/|\Omega|^\gamma$ becomes repulsive for $\gamma >1$, and the imaginary
part $V^{''} (\Omega) \propto \sin(\pi \gamma/2)/|\Omega|^\gamma$ decreases as
$\gamma \to 2$.

\subsection*{\href{http://arxiv.org/abs/2009.10904v1}{Optimal energy conversion through anti-adiabatic driving breaking  time-reversal symmetry}}
\subsubsection*{Loris Maria Cangemi, \dots, and Giuliano Benenti (2020-09-23)}
Starting with Carnot engine, the ideal efficiency of a heat engine has been
associated with quasi-static transformations and vanishingly small output
power. Here, we exactly calculate the thermodynamic properties of a isothermal
heat engine, in which the working medium is a periodically driven underdamped
harmonic oscillator, focusing instead on the opposite, anti-adiabatic limit,
where the period of a cycle is the fastest time scale in the problem. We show
that in that limit it is possible to approach the ideal energy conversion
efficiency $\eta=1$, with finite output power and vanishingly small relative
power fluctuations. The simultaneous realization of all the three desiderata of
a heat engine is possible thanks to the breaking of time-reversal symmetry. We
also show that non-Markovian dynamics can further improve the power-efficiency
trade-off.

\subsection*{\href{http://arxiv.org/abs/2009.10896v1}{Sequential pairwise reactions dictate phase evolution in the solid-state  synthesis of multicomponent ceramics}}
\subsubsection*{Akira Miura, \dots, and Wenhao Sun (2020-09-23)}
Solid-state synthesis from powder precursors is the primary processing route
to advanced multicomponent ceramic materials. Optimizing ceramic synthesis
routes is usually a laborious, trial-and-error process, as heterogeneous
mixtures of powder precursors often evolve through a complicated series of
reaction intermediates. Here, we show that phase evolution from multiple
precursors can be modeled as a sequence of interfacial reactions which initiate
between two phases at a time. By using ab initio thermodynamics to calculate
which pairwise interface is most reactive, we can understand and anticipate
which non-equilibrium phases will form during solid-state synthesis. Using the
classic high-temperature superconductor YBa$_2$Cu$_3$O$_{6+x}$ (YBCO) as a
model system, we directly observe these sequential pairwise reactions with in
situ X-ray diffraction and transmission electron microscopy. Our model
rationalizes a remarkable observation--that YBCO can be synthesized in 30
minutes when starting with a BaO$_2$ precursor, as opposed to 12+ hours with
the traditional BaCO$_3$ precursor.

\subsection*{\href{http://arxiv.org/abs/2009.10895v1}{On the control of interference and diffraction of a 3-level atom in a  double-slit scheme with cavity fields}}
\subsubsection*{Mario Miranda Rojas and Miguel Orszag Posa (2020-09-23)}
A double cavity with a quantum mechanical and a classical field is located
immediately behind of a double-slit in order to analyse the wave-particle
duality. Both fields have common nodes and antinodes through which a
three-level atom passes after crossing the double-slit. The atom-field
interaction is maximum when the atom crosses a common antinode and
path-information can be recorded on the phase of the quantum field. On other
hand, if the atom crosses a common node, the interaction is null and no
path-information is stored. A quadrature measurement on the quantum field can
reveal the path followed by the atom, depending on its initial amplitude
$\alpha$ and the classical amplitude $\varepsilon$. In this report we show that
the classical radiation acts like a focusing element of the interference and
diffraction patterns and how it alters the visibility and distinguishabilily.
Furthermore, in our double-slit scheme the two possible paths are correlated
with the internal atomic states, which allows us to study the relationship
between concurrence and wave-particle duality considering different cases.

\subsection*{\href{http://arxiv.org/abs/2009.13264v1}{A Derivative-free Method for Quantum Perceptron Training in  Multi-layered Neural Networks}}
\subsubsection*{Tariq M. Khan and Antonio Robles-Kelly (2020-09-23)}
In this paper, we present a gradient-free approach for training multi-layered
neural networks based upon quantum perceptrons. Here, we depart from the
classical perceptron and the elemental operations on quantum bits, i.e. qubits,
so as to formulate the problem in terms of quantum perceptrons. We then make
use of measurable operators to define the states of the network in a manner
consistent with a Markov process. This yields a Dirac-Von Neumann formulation
consistent with quantum mechanics. Moreover, the formulation presented here has
the advantage of having a computational efficiency devoid of the number of
layers in the network. This, paired with the natural efficiency of quantum
computing, can imply a significant improvement in efficiency, particularly for
deep networks. Finally, but not least, the developments here are quite general
in nature since the approach presented here can also be used for
quantum-inspired neural networks implemented on conventional computers.

\subsection*{\href{http://arxiv.org/abs/2009.10888v1}{Distributing Graph States Across Quantum Networks}}
\subsubsection*{Alex Fischer and Don Towsley (2020-09-23)}
Graph states are an important class of multipartite entangled quantum states.
We propose a new approach for distributing graph states across a quantum
network. We consider a quantum network consisting of nodes-quantum computers
within which local operations are free-and EPR pairs shared between nodes that
can continually be generated. We prove upper bounds for our approach on the
number of EPR pairs consumed, number of timesteps taken, and amount of
classical communication required, all of which are equal to or better than that
of prior work. We also reduce the problem of minimizing the number of timesteps
taken to distribute a graph state using our approach to a network flow problem
having polynomial time complexity.

\subsection*{\href{http://arxiv.org/abs/2009.10870v1}{Third order Møller-Plesset theory made more useful? The role of  density functional theory orbitals}}
\subsubsection*{Adam Rettig, \dots, and Martin Head-Gordon (2020-09-23)}
The practical utility of M{\o}ller-Plesset (MP) perturbation theory is
severely constrained by the use of Hartree-Fock (HF) orbitals. It has recently
been shown that use of regularized orbital-optimized MP2 orbitals and scaling
of MP3 energy could lead to a significant reduction in MP3 error (J. Phys.
Chem. Lett. 10, 4170, 2019). In this work we examine whether density functional
theory (DFT) optimized orbitals can be similarly employed to improve the
performance of MP theory at both the MP2 and MP3 levels. We find that use of
DFT orbitals leads to significantly improved performance for prediction of
thermochemistry, barrier heights, non-covalent interactions, and dipole moments
relative to standard HF based MP theory. Indeed MP3 (with or without scaling)
with DFT orbitals is found to surpass the accuracy of coupled cluster singles
and doubles (CCSD) for several datasets. We also found that the results are not
particularly functional sensitive in most cases, (although range-separated
hybrid functionals with low delocalization error perform the best). MP3 based
on DFT orbitals thus appears to be an efficient, non-iterative $O(N^6)$ scaling
wave function approach for single-reference electronic structure computations.
Scaled MP2 with DFT orbitals is also found to be quite accurate in many cases,
although modern double hybrid functionals are likely to be considerably more
accurate.

\subsection*{\href{http://arxiv.org/abs/2009.10869v1}{Stabilization of NbTe3, VTe3, and TiTe3 via Nanotube Encapsulation}}
\subsubsection*{Scott Stonemeyer, \dots, and Alex Zettl (2020-09-23)}
The structure of MX3 transition metal trichalcogenides (TMTs, with M a
transition metal and X a chalcogen) is typified by one-dimensional (1D) chains
weakly bound together via van der Waals interactions. This structural motif is
common across a range of M and X atoms (e.g. NbSe3, HfTe3, TaS3), but not all M
and X combinations are stable. We report here that three new members of the MX3
family which are not stable in bulk, specifically NbTe3, VTe3, and TiTe3, can
be synthesized in the few- to single-chain limit via nano-confined growth
within the stabilizing cavity of multi-walled carbon nanotubes. Transmission
electron microscopy (TEM) and atomic-resolution scanning transmission electron
microscopy (STEM) reveal the chain-like nature and the detailed atomic
structure. The synthesized materials exhibit behavior unique to few-chain
quasi-1D structures, such as multi-chain spiraling and a trigonal
anti-prismatic rocking distortion in the single-chain limit. Density functional
theory (DFT) calculations provide insight into the crystal structure and
stability of the materials, as well as their electronic structure.

\subsection*{\href{http://arxiv.org/abs/2009.10856v1}{Effects of Non-locality in Gravity and Quantum Theory}}
\subsubsection*{Jens Boos (2020-09-22)}
Spacetime---the union of space and time---is both the actor and the stage
during physical processes in our fascinating Universe. In Lorentz invariant
local theories, the existence of a maximum signalling speed (the "speed of
light") determines a notion of causality in spacetime, distinguishing the past
from the future, and the cause from the effect. This thesis is dedicated to the
study of \emph{deviations} from locality. Focussing on a particular class of
\emph{non-local} theories that is both Lorentz invariant and free of ghosts, we
aim to understand the effects of such non-local physics in both gravity and
quantum theory. Non-local ghost-free theories are accompanied by a parameter
$\ell$ of dimension length that parametrizes the scale of non-locality, and for
that reason we strive to express all effects of non-locality in terms of this
symbol. In the limiting case of $\ell=0$ one recovers the local theory, and the
effects of non-locality vanish. In order to address these questions we develop
the notion of non-local Green functions [...]. The results presented in this
thesis establish several effects of a Lorentz invariant, ghost-free
non-locality in the areas of both gravitational and quantum physics. (Full
abstract in document.)

\subsection*{\href{http://arxiv.org/abs/2009.10840v1}{On the net displacement of contractile elastic bodies subjected to  surface friction}}
\subsubsection*{Jose J. Munoz, and David Doste (2020-09-22)}
We show that for an elastic body undergoing a set of self-equilibrated
contractile forces and negligible inertial forces, and subjected to a linear
frictional boundary condition, the mean displacement averaged on the contact
surface is zero. This fact demonstrates the inability of slender organisms to
move under this friction condition if the contact surface remains constant,
regardless of the contractility strategy employed. We extend our results to
non-homogeneous and anisotropic friction, illustrate our conclusions with two
examples, and comment on strategies for net propulsion.

\subsection*{\href{http://arxiv.org/abs/2009.10836v1}{Spacetime as a Tightly Bound Quantum Crystal}}
\subsubsection*{Vlatko Vedral (2020-09-22)}
We review how reparametrization of space and time, namely the procedure where
both are made to depend on yet another parameter, can be used to formulate
quantum physics in a way that is naturally conducive to relativity. This leads
us to a second quantised formulation of quantum dynamics in which different
points of spacetime represent different modes. We speculate on the fact that
our formulation can be used to model dynamics in spacetime the same way that
one models propagation of an electron through a crystal lattice in solid state
physics. We comment on the implications of this for the notion of mode
entanglement as well as for the fully relativistic Page-Wootters formulation of
the wavefunction of the Universe.

\subsection*{\href{http://arxiv.org/abs/2009.10821v1}{Optical response of noble metal nanostructures: Quantum surface effects  in crystallographic facets}}
\subsubsection*{A. Rodríguez Echarri, \dots, and Joel D. Cox (2020-09-22)}
Noble metal nanostructures are ubiquitous elements in nano-optics, supporting
plasmon modes that can focus light down to length scales commensurate with
nonlocal effects associated with quantum confinement and spatial dispersion in
the underlying electron gas. Nonlocal effects are naturally more prominent for
crystalline noble metals, which potentially offer lower intrinsic loss than
their amorphous counterparts, and with particular crystal facets giving rise to
distinct electronic surface states. Here, we employ a quantum-mechanical model
to describe nonclassical effects impacting the optical response of crystalline
noble-metal films and demonstrate that these can be well-captured using a set
of surface-response functions known as Feibelman $d$-parameters. In particular,
we characterize the $d$-parameters associated with the (111) and (100) crystal
facets of gold, silver, and copper, emphasizing the importance of surface
effects arising due to electron wave function spill-out and the
surface-projected band gap emerging from atomic-layer corrugation. We then show
that the extracted $d$-parameters can be straightforwardly applied to describe
the optical response of various nanoscale metal morphologies of interest,
including metallic ultra-thin films, graphene-metal heterostructures hosting
extremely confined acoustic graphene plasmons, and crystallographic faceted
metallic nanoparticles supporting localized surface plasmons. The tabulated
$d$-parameters reported here can circumvent computationally expensive
first-principles atomistic simulations to describe microscopic nonlocal effects
in the optical response of mesoscopic crystalline metal surfaces, which are
becoming widely available with increasing control over morphology down to
atomic length scales for state-of-the-art experiments in nano-optics.

\subsection*{\href{http://arxiv.org/abs/2009.10809v1}{Quantitative prediction of the fracture toughness of amorphous carbon  from atomic-scale simulations}}
\subsubsection*{S. Mostafa Khosrownejad, and Lars Pastewka (2020-09-22)}
Fracture is the ultimate source of failure of amorphous carbon (a-C) films,
however it is challenging to measure fracture properties of a-C from
nano-indentation tests and results of reported experiments are not consistent.
Here, we use atomic-scale simulations to make quantitative and mechanistic
predictions on fracture of a-C. Systematic large-scale K-field controlled
atomic-scale simulations of crack propagation are performed for a-C samples
with densities of $\rho=2.5, \, 3.0 \, \text{ and } 3.5~\text{g/cm}^{3}$
created by liquid quenches for a range of quench rates $\dot{T}_q = 10 -
1000~\text{K/ps}$. The simulations show that the crack propagates by
nucleation, growth, and coalescence of voids. Distances of $ \approx 1\,
\text{nm}$ between nucleated voids result in a brittle-like fracture toughness.
We use a crack growth criterion proposed by Drugan, Rice \& Sham to estimate
steady-state fracture toughness based on our short crack-length fracture
simulations. Fracture toughness values of $2.4-6.0\,\text{MPa}\sqrt{\text{m}}$
for initiation and $3-10\,\text{MPa}\sqrt{\text{m}}$ for the steady-state crack
growth are within the experimentally reported range. These findings demonstrate
that atomic-scale simulations can provide quantitatively predictive results
even for fracture of materials with a ductile crack propagation mechanism.

\subsection*{\href{http://arxiv.org/abs/2009.10806v1}{Effects of Electron-Vibrational Interaction in Exciton-Polariton  Luminescence and Relaxation. Time-dependent Polariton Luminescence Spectrum}}
\subsubsection*{B. D. Fainberg and V. Al. Osipov (2020-09-22)}
The non-equilibrium polariton luminescence spectrum is calculated within the
framework of the non-Markovian molecular relaxation model. The model accounts
for both the high-frequency (HF) and the low frequency (LF) optically active
(OA) molecular vibrations. For the calculation of the composite
vibration-polariton operators, we employ the Lang-Firsov transformation for the
HFOA vibration mode and account for the classical LFOA vibrations
stochastically. We also propose a polariton fluorescence mechanism in which the
spreading of the two-particle polariton expectation value outside the
nano-sample is considered as the decay of the composite polariton particle.
This opens a way for observation of the hot exciton-polaritons luminescence in
organic-based nano-devices in analogy with the hot luminescence of molecules
and crystals. The theory provides a clear simple physical picture of the
polariton luminescence line-shape relaxation and, as it is demonstrated, agrees
with the experiment.

\subsection*{\href{http://arxiv.org/abs/2009.10779v1}{Using Quantum Annealers to Calculate Ground State Properties of  Molecules}}
\subsubsection*{Justin Copenhaver, and Birgit Wehefritz-Kaufmann (2020-09-22)}
Quantum annealers, which make use of the adiabatic theorem to efficiently
find the ground state of an Ising model Hamiltonian, provide an alternative
approach to quantum computing. Such devices are currently commercially
available and have been successfully applied to several combinatorial and
discrete optimization problems. However, the application of quantum annealers
to problems in chemistry remains a relatively sparse area of research due to
the difficulty in mapping molecular systems to the Ising model Hamiltonian. In
this paper we review two different methods for finding the ground state of
molecular Hamiltonians using quantum annealers. In addition, we compare the
relative effectiveness of each method by calculating some of the ground state
properties of the H3+ and H2O molecules. We find that while each of these
methods is capable of accurately predicting the ground state properties of
small molecules, it is unlikely that either could accurately simulate larger
classes of molecules or outperform modern classical algorithms.

\subsection*{\href{http://arxiv.org/abs/2009.10759v1}{Chaos exponents of SYK traversable wormholes}}
\subsubsection*{Tomoki Nosaka and Tokiro Numasawa (2020-09-22)}
In this paper we study the chaos exponent, the exponential growth rate of the
out-of-time-ordered four point functions, in a two coupled SYK models which
exhibits a first order phase transition between the high temperature black hole
phase and the low temperature gapped phase interpreted as a traversable
wormhole. We see that as the temperature decreases the chaos exponent exhibits
a discontinuous fall-off from the value of order the universal bound
$2\pi/\beta$ at the critical temperature of the phase transition, which is
consistent with the expected relation between black holes and strong chaos.
Interestingly, the chaos exponent is small but non-zero even in the wormhole
phase. This is surprising but consistent with the observation on the decay rate
of the two point function [arXiv:2003.03916], and we found the chaos exponent
and the decay rate indeed obey the same temperature dependence in this regime.
We also studied the chaos exponent of a closely related model with single SYK
term, and found that the chaos exponent of this model is always greater than
that of the two coupled model in the entire parameter space.

\subsection*{\href{http://arxiv.org/abs/2009.10758v1}{Probing atomic-scale symmetry breaking by rotationally invariant machine  learning of multidimensional electron scattering}}
\subsubsection*{Mark P. Oxley, \dots, and Sergei V. Kalinin (2020-09-22)}
The 4D scanning transmission electron microscopy (STEM) method has enabled
mapping of the structure and functionality of solids on the atomic scale,
yielding information-rich data sets containing information on the interatomic
electric and magnetic fields, structural and electronic order parameters, and
other symmetry breaking distortions. A critical bottleneck on the pathway
toward harnessing 4D-STEM for materials exploration is the dearth of analytical
tools that can reduce complex 4D-STEM data sets to physically relevant
descriptors. Classical machine learning (ML) methods such as principal
component analysis and other linear unmixing techniques are limited by the
presence of multiple point-group symmetric variants, where diffractograms from
each rotationally equivalent position will form its own component. This
limitation even holds for more complex ML methods, such as convolutional neural
networks. Here, we propose and implement an approach for the systematic
exploration of symmetry breaking phenomena from 4D-STEM data sets using
rotationally invariant variational autoencoders (rrVAE), which is designed to
disentangle the general rotation of the object from other latent
representations. The implementation of purely rotational rrVAE is discussed as
are applications to simulated data for graphene and zincblende structures that
illustrate the effect of site symmetry breaking. Finally, the rrVAE analysis of
4D-STEM data of vacancies in graphene is illustrated and compared to the
classical center-of-mass (COM) analysis. This approach is universal for probing
of symmetry breaking phenomena in complex systems and can be implemented for a
broad range of diffraction methods exploring the 2D diffraction space of the
system, including X-ray ptychography, electron backscatter diffraction (EBSD),
and more complex methods.

\subsection*{\href{http://arxiv.org/abs/2009.10756v1}{Error Rates and Resource Overheads of Repetition Cat Qubits}}
\subsubsection*{Jérémie Guillaud and Mazyar Mirrahimi (2020-09-22)}
We estimate and analyze the error rates and the resource overheads of the
repetition cat qubit approach to universal and fault-tolerant quantum
computation. The cat qubits stabilized by two-photon dissipation exhibit an
extremely biased noise where the bit-flip error rate is exponentially
suppressed with the mean number of photons. In a recent work, we suggested that
the remaining phase-flip error channel could be suppressed using a 1D
repetition code. Indeed, using only bias-preserving gates on the cat-qubits, it
is possible to build a universal set of fault-tolerant logical gates at the
level of the repetition cat qubit. In this paper, we perform Monte-Carlo
simulations of all the circuits implementing the protected logical gates, using
a circuit-level error model. Furthermore, we analyze two different approaches
to implement a fault-tolerant Toffoli gate on repetition cat qubits. These
numerical simulations indicate that very low logical error rates could be
achieved with a reasonable resource overhead, and with parameters that are
within the reach of near-term circuit QED experiments.

\subsection*{\href{http://arxiv.org/abs/2009.10751v1}{Measuring the Mermin-Peres magic square using an online quantum computer}}
\subsubsection*{A. Dikme, \dots, and G. Björk (2020-09-22)}
We have implemented the six series of three commuting measurement of the
Mermin-Peres magic square on an online, five qubit, quantum computer. The magic
square tests if the measurements of the system can be described by physical
realism (in the EPR sense) and simultaneously are non-contextual. We find that
our measurement results violate any realistic and non-contextual model by
almost 28 standard deviations. We also find that although the quantum computer
we used for the measurements leaves much to be desired in producing accurate
and reproducible results, the simplicity, the ease of re-running the
measurement programs, and the user friendliness compensates for this fact.

\subsection*{\href{http://arxiv.org/abs/2009.10742v1}{Evolution of Non-Gaussian Hydrodynamic Fluctuations}}
\subsubsection*{Xin An, \dots, and Ho-Ung Yee (2020-09-22)}
In the context of the search for the QCD critical point using non-Gaussian
fluctuations, we obtain the evolution equations for non-Gaussian cumulants to
leading order of the systematic expansion in the magnitude of thermal
fluctuations. We develop diagrammatic technique in which the leading order
contributions are given by tree diagrams. We introduce Wigner transform for
multipoint correlators and derive the evolution equations for three- and
four-point Wigner functions for the problem of nonlinear stochastic diffusion
with multiplicative noise.

\subsection*{\href{http://arxiv.org/abs/2009.10736v1}{Stripes, Antiferromagnetism, and the Pseudogap in the Doped Hubbard  Model at Finite Temperature}}
\subsubsection*{Alexander Wietek, \dots, and E. Miles Stoudenmire (2020-09-22)}
The interplay between thermal and quantum fluctuations controls the
competition between phases of matter in strongly correlated electron systems.
We study finite-temperature properties of the strongly coupled two-dimensional
doped Hubbard model using the minimally-entangled typical thermal states
(METTS) method on width $4$ cylinders. We discover that a novel phase
characterized by commensurate short-range antiferromagnetic correlations and no
charge ordering occurs at temperatures above the half-filled stripe phase
extending to zero temperature. The transition from the antiferromagnetic phase
to the stripe phase takes place at temperature $T/t \approx 0.05$ and is
accompanied by a step-like feature of the specific heat. We find the
single-particle gap to be smallest close to the nodal point at
$\mathbf{k}=(\pi/2, \pi/2)$ and detect a maximum in the magnetic
susceptibility. These features bear a strong resemblance to the pseudogap phase
of high-temperature cuprate superconductors. The simulations are verified using
a variety of different unbiased numerical methods in the three limiting cases
of zero temperature, small lattice sizes, and half-filling.

\subsection*{\href{http://arxiv.org/abs/2009.10730v1}{Josephson effect in graphene bilayers with adjustable relative  displacement}}
\subsubsection*{Mohammad Alidoust, and Jaakko Akola (2020-09-22)}
The Josephson current is investigated in a superconducting graphene bilayer
where the pristine graphene sheets can make in-plane or out-of-plane
displacements with respect to each other. The superconductivity can be of
intrinsic nature, or due to a proximity effect. The results demonstrate that
the supercurrent responds qualitatively differently to relative displacement if
the superconductivity is due to either intralayer or interlayer spin-singlet
electron-electron pairing, thus providing a tool to distinguish between the two
mechanisms. Specifically, both the AA and AB stacking orders are studied with
antiferromagnetic spin alignment. For the AA stacking order with intralayer and
on-site pairing no current reversal is found. In contrast, the supercurrent may
switch its direction as a function of the in-plane displacement and
out-of-plane interlayer coupling for the cases of AA ordering with interlayer
pairing and AB ordering with either intralayer or interlayer pairing. In
addition to sign reversal, the Josephson signal displays many characteristic
fingerprints which derive directly from the pairing mechanism. Thus,
measurements of the Josephson current as a function of the graphene bilayer
displacement open up means for achieving a deeper insight of the
superconducting pairing mechanism.

\subsection*{\href{http://arxiv.org/abs/2009.10723v1}{Defect Dynamics in Active Polar Fluids vs. Active Nematics}}
\subsubsection*{Farzan Vafa (2020-09-22)}
Topological defects play a key role in two-dimensional active nematics, and a
transient role in two-dimensional active polar fluids. In this paper, we study
both the transient and long-time behavior of defects in two-dimensional active
polar fluids in the limit of strong order and overdamped, compressible flow,
and compare the defect dynamics with the corresponding active nematics model
studied recently. One result is non-central interactions between defect pairs
for active polar fluids, and by extending our analysis to allow orientation
dynamics of defects, we find that the orientation of $+1$ defects, unlike that
of $\pm 1/2$ defects in active nematics, is not locked to defect positions and
relaxes to asters. Moreover, using a scaling argument, we explain the transient
feature of active polar defects and show that in the steady state, active polar
fluids are either devoid of defects or consist of a single aster. We argue that
for contractile (extensile) active nematic systems, $+1$ vortices (asters)
should emerge as bound states of a pair of $+1/2$ defects, which has been
recently observed. Moreover, unlike the polar case, we show that for active
nematics, a linear chain of equally spaced bound states of pairs of $+1/2$
defects can screen the activity term. A common feature in both models is the
appearance of $+1$ defects (elementary in polar and composite in nematic) in
the steady state.

\subsection*{\href{http://arxiv.org/abs/2009.10720v1}{Magnetic mixed valent semimetal EuZnSb$_2$ with Dirac states in the band  structure}}
\subsubsection*{Aifeng Wang, \dots, and C. Petrovic (2020-09-22)}
We report discovery of new antiferromagnetic semimetal EuZnSb$_2$, obtained
and studied in the form of single crystals. Electric resistivity, magnetic
susceptibility and heat capacity indicate antiferromagnetic order of Eu with
$T_N$ = 20 K. The effective moment of Eu$^{2+}$ inferred from the magnetization
and specific heat measurement is 3.5 $\mu_B$, smaller than the theoretical
value of Eu$^{2+}$ due to presence of both Eu$^{3+}$ and Eu$^{2+}$. Magnetic
field-dependent resistivity measurements suggest dominant quasi two dimensional
Fermi surfaces whereas the first-principle calculations point to the presence
of Dirac fermions. Therefore, EuZnSb$_2$ could represent the first platform to
study the interplay of dynamical charge fluctuations, localized magnetic 4$f$
moments and Dirac states with Sb orbital character.

\subsection*{\href{http://arxiv.org/abs/2009.10719v1}{Hierarchy of higher-order Floquet topological phases in three dimensions}}
\subsubsection*{Tanay Nag, and Bitan Roy (2020-09-22)}
Following a general protocol of periodically driving static first-order
topological phases (supporting surface states) with suitable discrete symmetry
breaking Wilson-Dirac masses, here we construct a hierarchy of higher-order
Floquet topological phases in three dimensions. In particular, we demonstrate
realizations of both second-order and third-order Floquet topological states,
respectively supporting dynamic hinge and corner modes at zero quasienergy, by
periodically driving their static first-order parent states with one and two
discrete symmetry breaking Wilson-Dirac mass(es). While the static surface
states are characterized by codimension $d_c=1$, the resulting dynamic hinge
(corner) modes, protected by \emph{antiunitary} spectral or particle-hole
symmetries, live on the boundaries with $d_c=2$ $(3)$. We exemplify these
outcomes for three-dimensional topological insulators and Dirac semimetals,
with the latter ones following an arbitrary spin-$j$ representation.

\subsection*{\href{http://arxiv.org/abs/2009.10718v1}{Thermodynamic Casimir forces in strongly anisotropic systems within the  $N\to \infty$ class}}
\subsubsection*{Maciej Łebek and Paweł Jakubczyk (2020-09-22)}
We analyze the thermodynamic Casimir effect in strongly anizotropic systems
from the vectorial $N\to\infty$ class in a slab geometry. Employing the
imperfect (mean-field) Bose gas as a representative example, we demonstrate the
key role of spatial dimensionality $d$ in determining the character of the
effective fluctuation-mediated interaction between the confining walls. For a
particular, physically conceivable choice of anisotropic dispersion and
periodic boundary conditions, we show that the Casimir force at criticality as
well as within the low-temperature phase is repulsive for dimensionality $d\in
(\frac{5}{2},4)\cup (6,8)\cup (10,12)\cup\dots$ and attractive for $d\in
(4,6)\cup (8,10)\cup \dots$. We argue, that for $d\in\{4,6,8\dots\}$ the
Casimir interaction entirely vanishes in the scaling limit. We discuss
implications of our results for systems characterized by $1/N>0$ and possible
realizations in the context of quantum phase transitions.

\subsection*{\href{http://arxiv.org/abs/2009.10709v1}{Fast Black-Box Quantum State Preparation}}
\subsubsection*{Johannes Bausch (2020-09-22)}
Quantum state preparation is an important ingredient for other higher-level
quantum algorithms, such as Hamiltonian simulation, or for loading
distributions into a quantum device to be used e.g. in the context of
optimization tasks such as machine learning. Starting with a generic "black
box" method devised by Grover in 2000, which employs amplitude amplification to
load coefficients calculated by an oracle, there has been a long series of
results and improvements with various additional conditions on the amplitudes
to be loaded, culminating in Sanders et al.'s work which avoids almost all
arithmetic during the preparation stage. In this work, we improve upon this
routine in two aspects: we reduce the required qubit overhead from $g$ to
$\log_2(g)$ in the bit precision $g$ (at a cost of slightly increasing the
count of non-Clifford operations), and show how various sets of $N$
coefficients can be loaded significantly faster than in $O(\sqrt N)$ rounds of
amplitude amplification - up to only $O(1)$ many - by bootstrapping the
procedure with an optimised initial state.

\subsection*{\href{http://arxiv.org/abs/2009.10705v1}{Faster Exact Exchange in Periodic Systems using Single-precision  Arithmetic}}
\subsubsection*{John Vinson (2020-09-22)}
Density-functional theory simplifies many-electron calculations by
approximating the exchange and correlation interactions with a one-electron
operator that is a functional of the density. Hybrid functionals incorporate
some amount of exact exchange, improving agreement with measured electronic and
structural properties. However, calculations with hybrid functionals require
substantial computational resources, limiting their use. By calculating the
exchange interaction of periodic systems with single-precision arithmetic, the
computation time is cut nearly in half with a negligible loss in accuracy. This
improvement makes exact exchange calculations quicker and more feasible,
especially for high-throughput calculations. Example hybrid density-functional
theory calculations of band energies, forces, and x-ray absorption spectra show
that this single-precision implementation maintains accuracy with significantly
reduced runtime and memory requirements.

\subsection*{\href{http://arxiv.org/abs/2009.10704v1}{Wigner-Weyl calculus in Keldysh technique}}
\subsubsection*{C. Banerjee, \dots, and M. A. Zubkov (2020-09-22)}
We discuss the non-equilibrium dynamics of condensed matter/quantum field
systems in the framework of Keldysh technique. In order to deal with the
inhomogeneous systems we use the Wigner-Weyl formalism. Unification of the
mentioned two approaches is demonstrated on the example of Hall conductivity.
We express Hall conductivity through the Wigner transformed two-point Green's
functions. We demonstrate how this expression is reduced to the topological
number in thermal equilibrium at zero temperature. At the same time both at
finite temperature and out of equilibrium the topological invariance is lost.
Moreover, Hall conductivity becomes sensitive to interaction corrections.

\subsection*{\href{http://arxiv.org/abs/2009.10703v1}{Magnon Spin Edelstein Effect Detected by Inverse Spin Hall Effect}}
\subsubsection*{Hantao Zhang and Ran Cheng (2020-09-22)}
In an easy-plane antiferromagnet with the Dzyaloshinskii-Moriya interaction
(DMI), magnons are subject to an effective spin-momentum locking. A temperature
gradient can generate interfacial accumulation of magnons with a specified
polarization, realizing the magnon spin Edelstein effect. We investigate the
injection and detection of this thermally-driven spin polarization in an
adjacent heavy metal with strong spin Hall effect. We find that the inverse
spin Hall voltage depends monotonically on both temperature and the DMI but
non-monotonically on the hard-axis anisotropy. Counterintuitively, the magnon
spin Edelstein effect is even in a magnetic field applied along the N\'eel
vector.

\subsection*{\href{http://arxiv.org/abs/2009.10699v1}{Double-Fano resonance in a two-level quantum system coupled to zigzag  Phosphorene nanoribbon}}
\subsubsection*{Mohsen Amini, \dots, and Mohsen Rezaei (2020-09-22)}
Double-level quantum systems are good candidates for revealing coherent
quantum transport properties. Here, we consider quantum interference effects
due to the formation of a two-level system (TLS) coupled to the edge channel of
a zigzag Phosphorene nanoribbon (ZPNR). Using the tight-binding approach, we
first demonstrate the formation of a TLS in bulk Phosphorene sheet due to the
existence of two nearby vacancy impurities. Then, we show that such a TLS can
couple to the quasi-one-dimensional continuum of the edge states in a ZPNR
which results in the the appearance of two-dip Fano-type line shapes. To this
end, we generalize the Lippmann-Schwinger approach to study the scattering of
edge electrons in a ZPNR by two coupled impurity defects. We obtain an
analytical expression of the transmission coefficient which shows that the
positions and widths of the anti-resonances can be controlled by changing the
intervacancy distance as well as their distance from the edge of the ribbon.
This work constitutes a clear example of the multiple Fano resonances in
mesoscopic transport.

\subsection*{\href{http://arxiv.org/abs/2009.10695v1}{Accurate simulation of q-state clock model}}
\subsubsection*{Guanrong Li, and Zheng-Cheng Gu (2020-09-22)}
We accurately simulate the phase diagram and critical behavior of the
$q$-state clock model on the square lattice by using the state-of-the-art loop
optimization for tensor network renormalzation(loop-TNR) algorithm. The two
phase transition points for $q \geq 5$ are determined with very high accuracy.
Furthermore, by computing the conformal scaling dimensions, we are able to
accurately determine the compactification radius $R$ of the compactified boson
theories at both phase transition points. In particular, the compactification
radius $R$ at high-temperature critical point is precisely the same as the
predicted $R$ for Berezinskii-Kosterlitz-Thouless (BKT) transition. Moreover,
we find that the fixed point tensors at high-temperature critical point also
converge(up to numerical errors) to the same one for large enough $q$ and the
corresponding operator product expansion(OPE) coefficient of the compactified
boson theory can also be read out directly from the fixed point tensor.

\subsection*{\href{http://arxiv.org/abs/2009.10678v2}{Quantum Polar Duality and the Symplectic Camel: a Geometric Approach to  Quantization}}
\subsubsection*{Maurice de Gosson (2020-09-22)}
We define and study the notion of quantum polarity, which is a kind of
geometric Fourier transform between sets of positions and sets of momenta. This
notion exhibits a strong interplay between the uncertainty principle and convex
geometry, suggesting that there should be alternative ways to measure quantum
uncertainty. Extending previous work of ours, we show that the orthogonal
projections of the covariance ellipsoid of a quantum state on the configuration
and momentum spaces form what we call a dual quantum pair. Our approach could
pave the way for a geometric and topological version of quantum indeterminacy.
We relate this result to the Blaschke-Santal\'o inequality and to the Mahler
conjecture. We also discuss the Hardy uncertainty principle and the less-known
Donoho--Stark principle from the point of view of polar duality.

\subsection*{\href{http://arxiv.org/abs/2009.10662v1}{Implications of gauge-freedom for non-relativistic quantum  electrodynamics}}
\subsubsection*{Adam Stokes and Ahsan Nazir (2020-09-22)}
We review gauge-freedom in quantum electrodynamics (QED) outside of textbook
regimes. We emphasise that QED subsystems are defined {\em relative} to a
choice of gauge. Each definition uses different gauge-invariant observables. We
show that this relativity is only eliminated if a sufficient number of
Markovian and weak-coupling approximations are employed. All physical
predictions are gauge-invariant, including subsystem properties such as photon
number and entanglement. However, subsystem properties naturally differ for
different physical subsystems. Gauge-ambiguities arise not because it is
unclear how to obtain gauge-invariant predictions, but because it is not always
clear which physical observables are the most operationally relevant. The
gauge-invariance of a prediction is necessary but not sufficient to ensure its
operational relevance. We show that in controlling which gauge-invariant
observables are used to define a material system, the choice of gauge affects
the balance between the material system's localisation and its electromagnetic
dressing. We review various implications of subsystem gauge-relativity for
deriving effective models, for describing time-dependent interactions, for
photodetection theory, and for describing matter within a cavity.

\subsection*{\href{http://arxiv.org/abs/2009.10659v1}{Effect of Self-Lubricating Carbon Materials on the Tribological  Performance of Ultra-High-Molecular-Weight Polyethylene}}
\subsubsection*{N. Camacho, \dots, and G. C. Mondragón-Rodríguez (2020-09-22)}
Ultra-high-molecular-weight polyethylene (UHMWPE) has been the gold standard
for total knee replacements (TKR). Due to the knee's natural movements, the
wear debris of the UHMWPE is inevitable. The debris results in the joint's
mechanical instability, reduced mobility, increased pain, and implant
loosening. TKRs have increased their survival rate, but improvements to
increase the UHMWPE component's longevity are needed. Here, Ti-doped
diamond-like carbon coatings (Ti-DLC) and multiwalled carbon nanotubes (MWCNTs)
were used to decrease the UHMWPE wear. After 400,000 cycles, the UHMWPE-MWCNT
diminishes the mass loss compared to UHMWPE, and the combination with Ti-DLC
decreased the material loss by ~ 43.7 \% compared to the reference pair.
Cold-flow and burnishing were the predominant wear modes.

\subsection*{\href{http://arxiv.org/abs/2009.10650v1}{Phonon Renormalization in Reconstructed MoS$_2$ Moiré Superlattices}}
\subsubsection*{Jiamin Quan, \dots, and Xiaoqin Li (2020-09-22)}
In moir\'e crystals formed by stacking van der Waals (vdW) materials,
surprisingly diverse correlated electronic phases and optical properties can be
realized by a subtle change in the twist angle. Here, we discover that phonon
spectra are also renormalized in MoS$_2$ twisted bilayers, adding a new
perspective to moir\'e physics. Over a range of small twist angles, the phonon
spectra evolve rapidly due to ultra-strong coupling between different phonon
modes and atomic reconstructions of the moir\'e pattern. We develop a new
low-energy continuum model for phonons that overcomes the outstanding challenge
of calculating properties of large moir\'e supercells and successfully captures
essential experimental observations. Remarkably, simple optical spectroscopy
experiments can provide information on strain and lattice distortions in
moir\'e crystals with nanometer-size supercells. The newly developed theory
promotes a comprehensive and unified understanding of structural, optical, and
electronic properties of moir\'e superlattices.

\subsection*{\href{http://arxiv.org/abs/2009.10628v1}{From Chaotic Spin Dynamics to Non-collinear Spin Textures in YIG  Nano-films by Spin Current Injection}}
\subsubsection*{Henning Ulrichs (2020-09-22)}
In this article I report about a numerical investigation of nonlinear spin
dynamics in a magnetic thin-film, made of Yttrium-Iron-Garnet (YIG). This film
is exposed to a small in-plane oriented magnetic field, and strong spin
currents. The rich variety of findings encompass dynamic regimes hosting
localized, non-propagating solitons, a turbulent chaotic regime, which
condenses into a quasi-static phase featuring a non-collinear spin texture.
Eventually, at largest spin current, a homogeneously switched state is
established.

\subsection*{\href{http://arxiv.org/abs/2009.10626v1}{Strain Engineering 2D MoS$_{2}$ with Thin Film Stress Capping Layers}}
\subsubsection*{Tara Peña, \dots, and Stephen M. Wu (2020-09-22)}
In this work, we induce on-chip static strain into the transition metal
dichalcogenide (TMDC) MoS$_{2}$ with e-beam evaporated stressed thin film
multilayers. These thin film stressors are analogous to SiN$_{x}$ based
stressors utilized in CMOS technology. We choose optically transparent thin
film stressors to allow us to probe the strain transferred into the MoS$_{2}$
with Raman spectroscopy. We combine thickness dependent analyses from Raman
peak shifts in MoS$_{2}$ and atomistic simulations to understand the strain
transferred throughout each layer. This collaboration between experimental and
theoretical efforts allows us to conclude that strain is transferred from the
stressor into the top few layers of MoS$_{2}$ and the bottom layer is always
partially fixed to the substrate. This proof of concept suggests that commonly
used industrial strain engineering techniques may be easily implemented with 2D
materials, as long as the c-axis strain transfer is considered.

\subsection*{\href{http://arxiv.org/abs/2009.10624v1}{Spherical Density Functional Theory}}
\subsubsection*{Ágnes Nagy, \dots, and Levente Vitos (2020-09-22)}
Recently, Theophilou (J. Chem.Phys {\bf 149} 074104 (2018)) showed that a set
of spherically symmetric densities determines uniquely the external potential
in molecules and solids. Here, spherically symmetric Kohn-Sham-like equations
are derived. The spherical densities can be expressed with radial wave
functions. Expression for the total energy is also presented.

\subsection*{\href{http://arxiv.org/abs/2009.10621v1}{The fitness landscapes of translation}}
\subsubsection*{Mario Josupeit and Joachim Krug (2020-09-22)}
Motivated by recent experiments on an antibiotic resistance gene, we
investigate genetic interactions between synonymous mutations in the framework
of exclusion models of translation. We show that the range of possible
interactions is markedly different depending on whether translation efficiency
is assumed to be proportional to ribosome current or ribosome speed. In the
first case every mutational effect has a definite sign that is independent of
genetic background, whereas in the second case the effect-sign can vary
depending on the presence of other mutations. The latter result is demonstrated
using configurations of multiple translational bottlenecks induced by slow
codons.

\subsection*{\href{http://arxiv.org/abs/2009.10611v1}{Electronic and magnetic properties of transition-metal doped ScN for  spintronics applications}}
\subsubsection*{Fares Benissad and Abdesalem Houari (2020-09-22)}
Motivated by the ongoing interest in nitrides as materials for spintronics
applications we have studied effects of doping with magnetic transition-metal
elements (T=Cr,Mn,Fe,Co and Ni) on the electronic properties of semiconducting
scandium nitride. Using density functional together with the generalized
gradient approximation (GGA) as well as PBE0r hybrid functional (with different
mixing of the exact exchange), two different doping amounts 25\% ($\rm
Sc_{0.75}T_{0.25}N$) and 10\% ($\rm Sc_{0.9}T_{0.1}N$) have been investigated.
This is done in comparison to the reference compound ScN with a strong focus on
identifying candidates for half-metallic or semiconducting ferromagnetic ground
states. Within GGA, only $\rm Sc_{0.75}Cr_{0.25}N$ and $\rm
Sc_{0.75}Mn_{0.25}N$ are found to be semiconducting and half-metallic,
respectively. The use of hybrid functional changes drastically these finding,
where $\rm Sc_{0.75}Fe(Co,Ni)_{0.25}N$ become half-metals and $\rm
Sc_{0.75}Cr(Mn)_{0.25}N$ are found both semiconductors. However, additional
calculations assuming antiferromagnetic ordering revealed that $\rm
Sc_{0.75}Cr_{0.25}N$ is the only compound of this series, which prefers an
antiferromagnetic (and semiconducting) ground state. For the lower
concentration, $\rm Sc_{0.9}T_{0.1}N$, similar results have been predicted, and
all the doped nitrides are found to prefer ferromagnetic ground state over an
antiferromagnetic one.

\subsection*{\href{http://arxiv.org/abs/2009.10605v1}{Hidden non-Markovianity in open quantum systems}}
\subsubsection*{Daniel Burgarth, \dots, and Davide Lonigro (2020-09-22)}
We show that non-Markovian open quantum systems can exhibit exact Markovian
dynamics up to an arbitrarily long time; the non-Markovianity of such systems
is thus perfectly "hidden", i.e. not experimentally detectable by looking at
the reduced dynamics alone. This shows that non-Markovianity is physically
undecidable and extremely counterintuitive, since its features can change at
any time, without precursors. Some interesting examples are discussed.

\subsection*{\href{http://arxiv.org/abs/2009.10596v1}{Stochastic motion of finite-size immiscible impurities in a dilute  quantum fluid at finite temperature}}
\subsubsection*{Umberto Giuriato and Giorgio Krstulovic (2020-09-22)}
The dynamics of an active, finite-size and immiscible impurity in a dilute
quantum fluid at finite temperature is characterized by means of numerical
simulations of the projected Gross--Pitaevskii equation. The impurity is
modeled as a localized repulsive potential and described with classical degrees
of freedom. It is shown that impurities of different sizes thermalize with the
fluid and undergo a stochastic dynamics compatible with an Ornstein--Uhlenbeck
process at sufficiently large time-lags. The velocity correlation function and
the displacement of the impurity are measured and an increment of the friction
with temperature is observed. Such behavior is phenomenologically explained in
a scenario where the impurity exchanges momentum with a dilute gas of thermal
excitations, experiencing an Epstein drag.

\subsection*{\href{http://arxiv.org/abs/2009.10584v1}{Quantum Detection of Inertial Frame Dragging}}
\subsubsection*{Wan Cong, \dots, and Robert B. Mann (2020-09-22)}
We study the response function of Unruh De-Witt detectors placed in a slowly
rotating shell. We show that the response function picks up the presence of
rotation even though the spacetime inside the shell is flat and the detector is
locally inertial. Moreover, it can do so when the detector is switched on for a
finite time interval within which a light signal cannot travel to the shell and
back to convey the presence of rotation.

\subsection*{\href{http://arxiv.org/abs/2009.10577v1}{On the Spatial Locality of Electronic Correlations in LiFeAs}}
\subsubsection*{Minjae Kim, \dots, and Gabriel Kotliar (2020-09-22)}
We address the question of the degree of spatial non-locality of the self
energy in the iron-based superconductors, a subject which is receiving
considerable attention. Using LiFeAs as a prototypical example, we extract the
self energy from angular-resolved photoemission spectroscopy (ARPES) data. We
use two distinct electronic structure references: density functional theory in
the local density approximation and linearized quasiparticle self consistent GW
(LQSGW). We find that in the LQSGW reference, spatially local dynamical
correlations provide a consistent description of the experimental data, and
account for some surprising aspects of the data such as the substantial out of
plane dispersion of the electron Fermi surface having dominant xz/yz character.
Hall effect and resistivity data are shown to be consistent with the proposed
description.

\subsection*{\href{http://arxiv.org/abs/2009.10567v1}{A cracking oxygen story: a new view of stress corrosion cracking in  titanium alloys}}
\subsubsection*{Sudha Josepha, \dots, and David Dye (2020-09-22)}
Titanium alloys can suffer from halide-associated stress corrosion cracking
at elevated temperatures e.g. in jet engines, where chlorides and Ti-oxide
promote the cracking of water vapour in the gas stream, depositing embrittling
species at the crack tip. Here we report, using isotopically-labelled
experiments, that crack tips in an industrial Ti-6Al-2Sn-4Zr-6Mo alloy are
strongly enriched (>5 at.\%) in oxygen from the water vapour, far greater than
the amounts (0.25 at.\%) required to embrittle the material. Surprisingly,
relatively little hydrogen is measured, despite careful preparation. Therefore,
we suggest that a synergistic effect of O and H leads to cracking, with O
playing a vital role, since O is well-known to cause embrittlement of the
alloy. In contrast it appears that in alpha-beta Ti alloys, it may be that H
may drain away into the bulk owing to its high solubility in beta-Ti, rather
than being retained in the stress field of the crack tip. Therefore, whilst
hydrides may form on the fracture surface, hydrogen ingress might not result in
embrittlement of the underlying matrix. This possibility challenges decades of
understanding of stress-corrosion cracking as being related only to the
hydrogen enhanced localised plasticity (HELP) mechanism, which explains why
H-doped Ti alloys are embrittled. This would change the perspective on stress
corrosion embrittlement away from a focus on hydrogen towards the ingress of O
originating from the water vapour, insights critical for designing corrosion
resistant materials.

\subsection*{\href{http://arxiv.org/abs/2009.10565v1}{Properties of equilibria and glassy phases of the random Lotka-Volterra  model with demographic noise}}
\subsubsection*{Ada Altieri, \dots, and Giulio Biroli (2020-09-22)}
In this letter we study a reference model in theoretical ecology, the
disordered Lotka-Volterra model for ecological communities, in the presence of
finite demographic noise. Our theoretical analysis, which takes advantage of a
mapping to an equilibrium disordered system, proves that for sufficiently
heterogeneous interactions and low demographic noise the system displays a
multiple equilibria phase, which we fully characterize. In particular, we show
that in this phase the number of stable equilibria is exponential in the number
of species. Upon further decreasing the demographic noise, we unveil a
"Gardner" transition to a marginally stable phase, similar to that observed in
jamming of amorphous materials. We confirm and complement our analytical
results by numerical simulations. Furthermore, we extend their relevance by
showing that they hold for others interacting random dynamical systems, such as
the Random Replicant Model. Finally, we discuss their extension to the case of
asymmetric couplings.

\subsection*{\href{http://arxiv.org/abs/2009.10561v1}{Comment on: "Some quantum aspects of a particle with electric quadrupole  moment interacting with an electric field subject to confining potentials".  Int. J. Mod. Phys. A 29 (2014) 1450117}}
\subsubsection*{Francisco M. Fernández (2020-09-22)}
We analyze the results obtained from a model consisting of the interaction
etween the electric quadrupole moment of a moving particle and an electric
field. We argue that the system does not support bound states because the
motion along the $z$ axis is unbounded. It is shown that the author obtains a
wrong bound-state spectrum for the motion in the $x-y$ plane and that the
existence of allowed cyclotron frequencies is an artifact of the approach.

\subsection*{\href{http://arxiv.org/abs/2009.10560v1}{Outgassing rate comparison of seven geometrically similar vacuum  chambers of different materials and heat treatments}}
\subsubsection*{James A. Fedchak, \dots, and Perry Henderson (2020-09-22)}
We have measured the water and hydrogen outgassing rates of seven vacuum
chambers of identical geometry but constructed of different materials and heat
treatments. Chambers of five different materials were tested: 304L, 316L, and
316LN stainless steels; titanium; and aluminum. In addition, chambers
constructed of 316L and 316LN stainless steel were subjected to a vacuum-fire
process, where they were heated to approximately 950 $^{\circ}$C for 24 hours while
under vacuum. These latter two chambers are designated as 316L-XHV and
316LN-XHV. Because all the chambers were of identical geometry and made by the
same manufacturer, a relative comparison of the outgassing rates among these
chambers can be made. Water outgassing rates were measured as a function of
time using the throughput technique. The water outgassing results for the 316L,
316LN, 316L-XHV, 316LN-XHV were all similar, but lower than those of 304L by a
factor of 3 to 5 lower at 10,000 s. The water outgassing results for Ti and Al
chambers were closer to that of 304L, Ti being slightly lower. Hydrogen
outgassing rates were measured using the rate-of-rise method and performed
after a low-temperature bake of 125 $^{\circ}$C to 150 $^{\circ}$C for a minimum of 72
hours. The Ti, Al, 316L-XHV, and 316LN-XHV chambers all have ultra-low specific
outgassing rates below 1.0E-11 Pa L s-1 cm-2 and are a factor of 100 or better
than the 304L chamber. The 304L, 316L, and 316LN chambers with no vacuum-fire
heat treatment have larger hydrogen outgassing rates than the other chambers,
with determined specific outgassing rates ranging between 4.0E-11 Pa L s-1 cm-2
and 8.0E-11 Pa L s-1 cm-2. We conclude that Ti, Al, 316L-XHV, and 316LN-XHV
have hydrogen outgassing rates that make them excellent choices for ultra-high
vacuum (UHV) and extreme-high vacuum (XHV) applications, the choice depending
on cost and other material properties.

\subsection*{\href{http://arxiv.org/abs/2009.10559v1}{Quarkonium in Quark-Gluon Plasma: Open Quantum System Approaches  Re-examined}}
\subsubsection*{Yukinao Akamatsu (2020-09-22)}
Dissociation of quarkonium in quark-gluon plasma (QGP) is a long standing
topic in relativistic heavy-ion collisions because it signals one of the
fundamental natures of the QGP -- Debye screening due to the liberation of
color degrees of freedom. Among recent new theoretical developments is the
application of open quantum system framework to quarkonium in the QGP. Open
system approach enables us to describe how dynamical as well as static
properties of QGP influences the time evolution of quarkonium.
  Currently, there are several master equations for quarkonium corresponding to
various scale assumptions, each derived in different theoretical frameworks. In
this review, all of the existing master equations are systematically rederived
as Lindblad equations in a uniform framework. Also, as one of the most relevant
descriptions in relativistic heavy-ion collisions, quantum Brownian motion of
heavy quark pair in the QGP is studied in detail. The quantum Brownian motion
is parametrized by a few fundamental quantities of QGP such as real and
imaginary parts of heavy quark potential (complex potential), heavy quark
momentum diffusion constant, and thermal dipole self-energy constant. This
indicates that the yields of quarkonia such as $J/\psi$ and $\Upsilon$ in the
relativistic heavy-ion collisions have the potential to determine these
fundamental quantities.

\subsection*{\href{http://arxiv.org/abs/2009.10552v1}{Negative probabilities: What they are and what they are for}}
\subsubsection*{Andreas Blass and Yuri Gurevich (2020-09-22)}
In quantum mechanics, the probability distributions of position and momentum
of a particle are normally not the marginals of a joint distribution, that is
unless -- as shown by Wigner in 1932 -- negative probabilities are allowed.
Since then, much theoretical work has been done to study negative
probabilities; most of this work is about what those probabilities are. We
suggest shifting the emphasis to what negative probabilities are for. In this
connection, we introduce the framework of observation spaces. An observation
space is a family $\mathcal S = \big\langle\mathcal P_i: i\in I\big\rangle$ of
probability distributions sharing a common sample space in a consistent way; a
grounding for $\mathcal S$ is a signed probability distribution $\mathcal P$
such that every $\mathcal P_i$ is a restriction of $\mathcal P$; and the
grounding problem for $\mathcal S$ is the problem of describing the groundings
for $\mathcal S$. We show that a wide variety of quantum scenarios can be
formalized as observation spaces, and we solve the grounding problem for a
number of quantum observation spaces. Our main technical result is a rigorous
proof that Wigner's distribution is the unique signed probability distribution
yielding the correct marginal distributions for position and momentum and all
their linear combinations.

\subsection*{\href{http://arxiv.org/abs/2009.10545v1}{Terahertz Magneto-optical investigation of quadrupolar spin-lattice  effects in magnetically frustrated Tb2Ti2O7}}
\subsubsection*{K. Amelin, \dots, and S. de Brion (2020-09-22)}
The significance of spin-lattice coupling in the phase diagram of the quantum
spin-icepyrochlore Tb2Ti2O7 has been a topic of debate for some time. Here, we
focus on the aspect of vibronic coupling, which occurs between the Tb3+
electronic levels and transverse acoustic phonons, by studying their dependence
on a magnetic field applied along the cubic h111i direction. Our experimental
THz spectroscopy measurements, combined with quantitative theoretical quantum
calculations, show that indeed vibronic effects are observed at 3 K. An
analysis incorporating quadrupolar spin-lattice effects in the Hamiltonian is
therefore relevant in this compound, which is no longer optically isotropic but
magnetically birefringent.

\subsection*{\href{http://arxiv.org/abs/2009.10539v1}{Emergence of spin-active channels at a quantum Hall interface}}
\subsubsection*{Amartya Saha, \dots, and Ganpathy Murthy (2020-09-22)}
We study the ground state of a system with an interface between $\nu=4$ and
$\nu=3$ in the quantum Hall regime. Far from the interface, for a range of
interaction strengths, the $\nu=3$ region is fully polarized but $\nu=4$ region
is locally a singlet. Upon varying the strength of the interactions and the
width of the interface, the system chooses one of two distinct edge/interface
phases. In phase A, stabilized for wide interfaces, spin is a good quantum
number, and there are no gapless long-wavelength spin fluctuations. In phase B,
stabilized for narrow interfaces, spin symmetry is spontaneously broken at the
Hartree-Fock level. Going beyond Hartree-Fock, we argue that phase B is
distinguished by the emergence of gapless long-wavelength spin excitations
bound to the interface, which can, in principle, be detected by a measurement
of the relaxation time $T_2$ in nuclear magnetic resonance.

\subsection*{\href{http://arxiv.org/abs/2009.10534v1}{Dynamics of a vortex lattice in a non-equilibrium polariton superfluid}}
\subsubsection*{Riccardo Panico, \dots, and Dario Ballarini (2020-09-22)}
If a quantum fluid is put in motion with enough angular momentum, at
equilibrium the ground state of the system is given by an array of quantised
vortices. In a driven-dissipative polariton fluid, we demonstrate that the
reverse process is also possible. Upon initially imprinting a static and
regular vortex array, the quantum fluid starts rotating. By tracking on
picosecond time scales many quantized vortices, we present the first measure of
rigid-body rotation in a polariton condensate. Such many-body motion agrees
with the Feynman quantization of superfluid velocity, which we show to be valid
even if our system is expanding and equilibrium is never attained.

\subsection*{\href{http://arxiv.org/abs/2009.10519v1}{The Impact of Anionic Vacancies on the Mechanical Properties of NbC and  NbN: An ab initio Study}}
\subsubsection*{P. W. Muchiri, \dots, and G. O. Amolo (2020-09-22)}
The development of super-hard materials has recently focused on systems
containing a heavy transition metal and light main group elements. Niobium
carbides and nitrides have previously been identified as potential candidates,
however, the volatility of carbon and nitrogen during synthesis makes them
prone to the formation of anionic vacancies, which have the ability to change
the electronic structure, dynamical stability and adversely affecting the
mechanical properties. Here, we present ab initio Density Functional Theory
calculations that probe the occurrence of anionic vacancies as a function of
concentration, thereafter, pertinent mechanical properties are investigated.
Our results showed that the presence of anionic vacancies in NbC and NbN tends
to deteriorate the mechanical properties and ultimately the mechanical hardness
due to vacancy softening that can be attributed to defect induced covalent to
metallic bond transition. Further, it was observed that anionic vacancies in
NbC tend to modify its toughness, in particular, NbC in ZB becomes brittle
while NbC in WZ becomes ductile in the presences of C vacancies of up to 6\%. On
the other hand, the toughness of NbN was found to be insensitive to defect
concentration of even up to 8\%. Consequently, stringent control of anionic
defects during the synthesis of NbC and NbN is critical for the realization of
the desired mechanical response that can make these materials ideal for
super-hard and related applications.

\subsection*{\href{http://arxiv.org/abs/2009.10516v1}{Dynamic $^{14}\rm N$ nuclear spin polarization in nitrogen-vacancy  centers in diamond}}
\subsubsection*{Laima Busaite, \dots, and Florian Gahbauer (2020-09-22)}
We studied the dynamic nuclear spin polarization of nitrogen in negatively
charged nitrogen-vacancy (NV) centers in diamond both experimentally and
theoretically over a wide range of magnetic fields from 0 to 1100 G covering
both the excited-state level anti-crossing and the ground-state level
anti-crossing magnetic field regions. Special attention was paid to the less
studied ground-state level anti-crossing region. The nuclear spin polarization
was inferred from measurements of the optically detected magnetic resonance
signal. These measurements show that a very large (up to $96 \pm 2\%$) nuclear
spin polarization of nitrogen can be achieved over a very broad range of
magnetic field starting from around 400 G up to magnetic field values
substantially exceeding the ground-state level anti-crossing at 1024 G. We
measured the influence of angular deviations of the magnetic field from the NV
axis on the nuclear spin polarization efficiency and found that, in the
vicinity of the ground-state level anti-crossing, the nuclear spin polarization
is more sensitive to this angle than in the vicinity of the excited-state level
anti-crossing. Indeed, an angle as small as a tenth of a degree of arc can
destroy almost completely the spin polarization of a nitrogen nucleus. In
addition, we investigated theoretically the influence of strain and optical
excitation power on the nuclear spin polarization.

\subsection*{\href{http://arxiv.org/abs/2009.10511v1}{Spatiotemporal entanglement in a noncollinear optical parametric  amplifier}}
\subsubsection*{L. La Volpe, \dots, and D. B. Horoshko (2020-09-22)}
We theoretically investigate the generation of two entangled beams of light
in the process of single-pass type-I noncollinear frequency degenerate
parametric downconversion with an ultrashort pulsed pump. We find the
spatio-temporal squeezing eigenmodes and the corresponding squeezing
eigenvalues of the generated field both numerically and analytically. The
analytical solution is obtained by modeling the joint spectral amplitude of the
field by a Gaussian function in curvilinear coordinates. We show that this
method is highly efficient and is in a good agreement with the numerical
solution. We also reveal that when the total bandwidth of the generated beams
is sufficiently high, the modal functions cannot be factored into a spatial and
a temporal parts, but exhibit a spatio-temporal coupling, whose strength can be
increased by shortening the pump.

\subsection*{\href{http://arxiv.org/abs/2009.10508v1}{Anomalous topological edge states in non-Hermitian piezophononic media}}
\subsubsection*{Penglin Gao, and Johan Christensen (2020-09-22)}
The bulk-boundary or bulk-edge correspondence is a principle relating surface
confined states to the topological classification of the bulk. By combining
non-Hermitian ingredients in terms of gain or loss with media that violate
reciprocity, an unconventional non-Bloch bulk-boundary correspondence leads to
unusual localization of bulk states at boundaries$-$a phenomenon coined
non-Hermitian skin effect. Here we \textcolor{black}{numerically} employ the
acoustoelectric effect in electrically biased and layered piezophononic media
as a solid framework for non-Hermitian and nonreciprocal topological mechanics
in the MHz regime. Thanks to a non-Hermitian skin effect for mechanical
vibrations, we find that the bulk bands of finite systems are highly sensitive
to the type of crystal termination, which indicates a failure of using
traditional Bloch bands to predict the wave characteristics. More surprisingly,
when reversing the electrical bias, we unveil how topological edge and bulk
vibrations can be harnessed either at the same or opposite interface. Yet,
while bulk states are found to display this unconventional skin effect, we
further discuss how in-gap edge states in the same instant, counterintuitively
are able to delocalize along the entire layered medium. We foresee that our
predictions will stimulate new avenues in echo-less ultrasonics based on exotic
wave physics.

\subsection*{\href{http://arxiv.org/abs/2009.10507v1}{Transfer matrix in scattering theory: A survey of basic properties and  recent developments}}
\subsubsection*{Ali Mostafazadeh (2020-09-22)}
We give a pedagogical introduction to time-independent scattering theory in
one dimension focusing on the basic properties and recent applications of
transfer matrices. In particular, we begin surveying some basic notions of
potential scattering such as transfer matrix and its analyticity,
multi-delta-function and locally periodic potentials, Jost solutions, spectral
singularities and their time-reversal, and unidirectional reflectionlessness
and invisibility. We then offer a simple derivation of the Lippmann-Schwinger
equation and Born series, and discuss the Born approximation. Next, we outline
a recently developed dynamical formulation of time-independent scattering
theory in one dimension. This formulation relates the transfer matrix and
therefore the solution of the scattering problem for a given potential to the
solution of the time-dependent Schr\"odinger equation for an effective
non-unitary two-level quantum system. We provide a self-contained treatment of
this formulation and some of its most important applications. Specifically, we
use it to devise a powerful alternative to the Born series and Born
approximation, derive dynamical equations for the reflection and transmission
amplitudes, discuss their application in constructing exact tunable
unidirectionally invisible potentials, and use them to provide an exact
solution for single-mode inverse scattering problems. The latter, which has
important applications in designing optical devices with a variety of
functionalities, amounts to providing an explicit construction for a
finite-range complex potential whose reflection and transmission amplitudes
take arbitrary prescribed values at any given wavenumber.

\subsection*{\href{http://arxiv.org/abs/2009.10506v1}{Superlattices based on van der Waals 2D materials}}
\subsubsection*{Yu Kyoung Ryu, and Andres Castellanos-Gomez (2020-09-22)}
Two-dimensional (2D) materials exhibit a number of improved mechanical,
optical, electronic properties compared to their bulk counterparts. The absence
of dangling bonds in the cleaved surfaces of these materials allows combining
different 2D materials into van der Waals heterostructures to fabricate p-n
junctions, photodetectors, 2D-2D ohmic contacts that show unexpected
performances. These intriguing results are regularly summarized in
comprehensive reviews. A strategy to tailor their properties even further and
to observe novel quantum phenomena consists in the fabrication of superlattices
whose unit cell is formed either by two dissimilar 2D materials or by a 2D
material subjected to a periodical perturbation, each component contributing
with different characteristics. Furthermore, in a 2D materials-based
superlattice, the interlayer interaction between the layers mediated by van der
Waals forces constitutes a key parameter to tune the global properties of the
superlattice. The above-mentioned factors reflect the potential to devise
countless combinations of van der Waals 2D materials based superlattices. In
the present feature article, we explain in detail the state-of-the-art of 2D
materials-based superlattices and we describe the different methods to
fabricate them, classified as vertical stacking, intercalation with atoms or
molecules, moir\'e patterning, strain engineering and lithographic design. We
also aim to highlight some of the specific applications for each type of
superlattices.

\subsection*{\href{http://arxiv.org/abs/2009.10500v1}{Single-electron emission from degenerate quantum levels}}
\subsubsection*{Michael Moskalets (2020-09-22)}
Single-electron sources on-demand are requisite for a promising fully
electronic platform for solid-state quantum information processing. Most of the
experimentally realized sources use the fact that only one electron can be
taken from a singly occupied quantum level. Here I take the next step and
discuss emission from orbitally degenerate quantum levels that arise, for
example, in quantum dots with a nontrivial ring topology. I show that
degeneracy provides additional flexibility for single- and two-electron
emission. Indeed, the small Aharonov-Bohm flux, which slightly lifts the
degeneracy, is a powerful tool for changing the relative width of the emitted
wave packets over a wide range. In a ring with one lead, even electrons emitted
from completely degenerate levels can be separated in time if the driving
potential changes at the appropriate rate. In a ring with two leads, electrons
can be emitted to different leads if one of the leads has a bound state with
Fermi energy at its end.

\subsection*{\href{http://arxiv.org/abs/2009.10499v1}{Bethe strings in the dynamical structure factor of the spin-1/2  Heisenberg XXX chain}}
\subsubsection*{J. M. P. Carmelo, and P. D. Sacramento (2020-09-22)}
Recently there has been a renewed interest in the spectra and role in
dynamical properties of excited states of the spin-1/2 Heisenberg
antiferromagnetic chain in longitudinal magnetic fields associated with Bethe
strings. The latter are bound states of elementary magnetic excitations
described by Bethe-ansatz complex non-real rapidities. Previous studies on this
problem referred to finite-size systems. Here we consider the thermodynamic
limit and study it for the isotropic spin-1/2 Heisenberg XXX chain in a
longitudinal magnetic field. We confirm that also in that limit the most
significant spectral weight contribution from Bethe strings leads to gapped
continua in the spectra of the spin +- and xx dynamical structure factors. The
contribution of Bethe strings to the zz dynamical structure factor is found to
be small at low spin densities and to become negligible upon increasing that
density above 0.317. For the -+ dynamical structure factor, that contribution
is found to be negligible at finite magnetic field. We derive analytical
expressions for the line shapes of the +-, xx, and zz dynamical structure
factors valid in the vicinity of singularities located at and just above the
gapped lower thresholds of the Bethe-string states's spectra. As a side result
and in order to provide an overall physical picture that includes the relative
location of all spectra with a significant amount of spectral weight, we
revisit the general problem of the line-shape of the transverse and
longitudinal spin dynamical structure factors at finite magnetic field and
excitation energies in the vicinity of other singularities. This includes those
located at and just above the lower thresholds of the spectra that stem from
excited states described by only real Bethe-ansatz rapidities.

\subsection*{\href{http://arxiv.org/abs/2009.10495v1}{Optical-based thickness measurement of MoO3 nanosheets}}
\subsubsection*{Sergio Puebla, \dots, and Andres Castellanos-Gomez (2020-09-22)}
Considering that two-dimensional (2D) molybdenum trioxide has acquired more
attention in the last few years, it is relevant to speed up thickness
identification of this material. We provide two fast and non-destructive
methods to evaluate the thickness of MoO3 flakes on SiO2/Si substrates. First,
by means of quantitative analysis of the apparent color of the flakes in
optical microscopy images, one can make a first approximation of the thickness
with an uncertainty of +-3 nm. The second method is based on the fit of optical
contrast spectra, acquired with micro-reflectance measurements, to a Fresnel
law-based model that provides an accurate measurement of the flake thickness
with +-2 nm of uncertainty.

\subsection*{\href{http://arxiv.org/abs/2009.10489v1}{Thickness identification of thin InSe by optical microscopy methods}}
\subsubsection*{Qinghua Zhao, \dots, and Andres Castellanos-Gomez (2020-09-22)}
Indium selenide (InSe), as a novel van der Waals layered semiconductor, has
attracted a large research interest thanks to its excellent optical and
electrical properties in the ultra-thin limit. Here, we discuss four different
optical methods to quantitatively identify the thickness of thin InSe flakes on
various substrates, such as SiO2/Si or transparent polymeric substrates. In the
case of thin InSe deposited on a transparent substrate, the transmittance of
the flake in the blue region of the visible spectrum can be used to estimate
the thickness. For InSe supported by SiO2/Si, the thickness of the flakes can
be estimated either by assessing their apparent colors or accurately analyzed
using a Fresnel-law based fitting model of the optical contrast spectra.
Finally, we also studied the thickness dependency of the InSe photoluminescence
emission energy, which provides an additional tool to estimate the InSe
thickness and it works both for InSe deposited on SiO2/Si and on a transparent
polymeric substrate.

\subsection*{\href{http://arxiv.org/abs/2009.10479v1}{Homogenisation Theory of Space-Time Metamaterials}}
\subsubsection*{Paloma Arroyo Huidobro, \dots, and J. B. Pendry (2020-09-22)}
We present a general framework for the homogenisation theory of space-time
metamaterials. By mapping to a frame co-moving with the space-time modulation,
we derive analytical formulae for the effective material parameters for
travelling wave modulations in the low frequency limit: electric permittivity,
magnetic permeability and magnetoelectric coupling. Remarkably, we show that
the theory is exact at all frequencies in the absence of back-reflections, and
exact at low frequencies when that condition is relaxed. This allows us to
derive exact formulae for the Fresnel drag experienced by light travelling
through travelling-wave modulations of electromagnetic media.

\subsection*{\href{http://arxiv.org/abs/2009.10477v1}{Non-equilibrium evolution of Bose-Einstein condensate deformation in  temporally controlled weak disorder}}
\subsubsection*{Milan Radonjić and Axel Pelster (2020-09-22)}
We consider a time-dependent extension of a perturbative mean-field approach
to the dirty boson problem by considering how switching on and off a weak
disorder potential affects the stationary state of an initially homogeneous
Bose-Einstein condensate by the emergence of a disorder-induced condensate
deformation. We find that in the switch on scenario the stationary condensate
deformation turns out to be a sum of an equilibrium part and a
dynamically-induced part, where the latter depends on the particular driving
protocol. If the disorder is switched off afterwards, the resulting condensate
deformation acquires an additional dynamically-induced part in the long-time
limit, while the equilibrium part vanishes. Our results demonstrate that the
condensate deformation represents an indicator of the generically
non-equilibrium nature of steady states of a Bose gas in a temporally
controlled weak disorder.

\subsection*{\href{http://arxiv.org/abs/2009.10472v1}{Interpolated Collision Model Formalism}}
\subsubsection*{Daniel Grimmer (2020-09-22)}
The dynamics of open quantum systems (i.e., of quantum systems interacting
with an uncontrolled environment) forms the basis of numerous active areas of
research from quantum thermodynamics to quantum computing. One approach to
modeling open quantum systems is via a Collision Model. For instance, one could
model the environment as being composed of many small quantum systems
(ancillas) which interact with the target system sequentially, in a series of
"collisions". In this thesis I will discuss a novel method for constructing a
continuous-time master equation from the discrete-time dynamics given by any
such collision model. This new approach works for any interaction duration,
$\delta t$, by interpolating the dynamics between the time-points $t =
n\,\delta t$. I will contrast this with previous methods which only work in the
continuum limit (as $\delta t\to 0$). Moreover, I will show that any
continuum-limit-based approach will always yield unitary dynamics unless it is
fine-tuned in some way. For instance, it is common to find non-unitary dynamics
in the continuum limit by taking an (I will argue unphysical) divergence in the
interaction strengths, $g$, such that $g^2 \delta t$ is constant as $\delta t
\to 0$.

\subsection*{\href{http://arxiv.org/abs/2009.10469v1}{Tuning the structure of Skyrmion lattice system Cu2OSeO3 under pressure}}
\subsubsection*{Srishti Pal, \dots, and A. K. Sood (2020-09-22)}
The insulating ferrimagnet Cu2OSeO3 shows a rich variety of phases such as
skyrmion lattice and helical magnetism controlled by interplay of different
exchange interactions which can be tuned by external pressure. In this work we
have investigated pressure-induced phase transitions at room temperature using
synchrotron based x- ray diffraction and Raman scattering measurements. The
ambient cubic phase transforms to a monoclinic phase above 7 GPa and then to
the triclinic phase above 11 GPa. Emergence of new phonon modes in the Raman
spectra confirms these structural phase transitions. Notably, upon
decompression, the crystal undergoes transition to a new monoclinic structure.
Atomic coordinates have been refined in the low pressure cubic phase to capture
the Cu-tetrahedra evolution responsible for the earlier reported magnetic
behavior under pressure. Our experiments will motivate further studies of its
emergent magnetic behavior under pressure.

\subsection*{\href{http://arxiv.org/abs/2009.10463v1}{Half-Metal Spin-Gapless Semiconductor Junctions as a Route to the Ideal  Diode}}
\subsubsection*{E. Şaşıoğlu, \dots, and I. Mertig (2020-09-22)}
The ideal diode is a theoretical concept that completely conducts the
electric current under forward bias without any loss and that behaves like a
perfect insulator under reverse bias. However, real diodes have a junction
barrier that electrons have to overcome and thus they have a threshold voltage
$V_T$, which must be supplied to the diode to turn it on. This threshold
voltage gives rise to power dissipation in the form of heat and hence is an
undesirable feature. In this work, based on half-metallic magnets and
spin-gapless semiconductors we propose a diode concept that does not have a
junction barrier and the operation principle of which relies on the
spin-dependent transport properties of the HMM and SGS materials. We show that
the HMM and SGS materials form an Ohmic contact under any finite forward bias,
while for a reverse bias the current is blocked due to spin-dependent filtering
of the electrons. Thus, the HMM-SGS junctions act as a diode with zero
threshold voltage $V_T$, and linear $I-V$ characteristics as well as an
infinite on:off ratio at zero temperature. However, at finite temperatures,
non-spin-flip thermally excited high-energy electrons as well as low-energy
spin-flip excitations can give rise to a leakage current and thus reduce the
on:off ratio under a reverse bias. Furthermore, a zero threshold voltage allows
one to detect extremely weak signals and due to the Ohmic HMM-SGS contact, the
proposed diode has a much higher current drive capability and low resistance,
which is advantageous compared to conventional semiconductor diodes. We employ
the NEGF method combined with DFT to demonstrate the linear $I-V$
characteristics of the proposed diode based on two-dimensional half-metallic
Fe/MoS$_2$ and spin-gapless semiconducting VS$_2$ planar heterojunctions.

\subsection*{\href{http://arxiv.org/abs/2009.10455v1}{Universal Hall conductance scaling in non-Hermitian Chern insulators}}
\subsubsection*{Solofo Groenendijk, and Tobias Meng (2020-09-22)}
We investigate the Hall conductance of a two-dimensional Chern insulator
coupled to an environment causing gain and loss. Introducing a biorthogonal
linear response theory, we show that sufficiently strong gain and loss lead to
a characteristic non-analytical contribution to the Hall conductance. Near its
onset, this contribution exhibits a universal power-law with a power 3/2 as a
function of Dirac mass, chemical potential and gain strength. Our results pave
the way for the study of non-Hermitian topology in electronic transport
experiments.

\subsection*{\href{http://arxiv.org/abs/2009.10448v1}{Validating N-body code Chrono for granular DEM simulations in  reduced-gravity environments}}
\subsubsection*{Cecily Sunday, \dots, and Patrick Michel (2020-09-22)}
The Discrete Element Method (DEM) is frequently used to model complex
granular systems and to augment the knowledge that we obtain through theory,
experimentation, and real-world observations. Numerical simulations are a
particularly powerful tool for studying the regolith-covered surfaces of
asteroids, comets, and small moons, where reduced-gravity environments produce
ill-defined flow behaviors. In this work, we present a method for validating
soft-sphere DEM codes for both terrestrial and small-body granular
environments. The open-source code Chrono is modified and evaluated first with
a series of simple two-body-collision tests, and then, with a set of piling and
tumbler tests. In the piling tests, we vary the coefficient of rolling friction
to calibrate the simulations against experiments with 1 mm glass beads. Then,
we use the friction coefficient to model the flow of 1 mm glass beads in a
rotating drum, using a drum configuration from a previous experimental study.
We measure the dynamic angle of repose, the flowing layer thickness, and the
flowing layer velocity for tests with different particle sizes, contact force
models, coefficients of rolling friction, cohesion levels, drum rotation speeds
and gravity levels. The tests show that the same flow patterns can be observed
at Earth and reduced-gravity levels if the drum rotation speed and the
gravity-level are set according to the dimensionless parameter known as the
Froude number. Chrono is successfully validated against known flow behaviors at
different gravity and cohesion levels, and will be used to study small-body
regolith dynamics in future works.

\subsection*{\href{http://arxiv.org/abs/2009.10429v2}{Classical-noise-free sensing based on quantum correlation measurement}}
\subsubsection*{Ping Wang, and Renbao Liu (2020-09-22)}
Quantum sensing, using quantum properties of sensors, can enhance resolution,
precision, and sensitivity of imaging, spectroscopy, and detection. An
intriguing question is: Can the quantum nature (quantumness) of sensors and
targets be exploited to enable schemes that are not possible for classical
probes or classical targets? Here we show that measurement of the quantum
correlations of a quantum target indeed allows for sensing schemes that have no
classical counterparts. As a concrete example, in case where the second-order
classical correlation of a quantum target could be totally concealed by
non-stationary classical noise, the higher-order quantum correlations can
single out a quantum target from the classical noise background, regardless of
the spectrum, statistics, or intensity of the noise. Hence a
classical-noise-free sensing scheme is proposed. This finding suggests that the
quantumness of sensors and targets is still to be explored to realize the full
potential of quantum sensing. New opportunities include sensitivity beyond
classical approaches, non-classical correlations as a new approach to quantum
many-body physics, loophole-free tests of the quantum foundation, et cetera.

\subsection*{\href{http://arxiv.org/abs/2009.10416v1}{Thermalization of isolated quantum many-body system and entanglement}}
\subsubsection*{Prasenjit Deb, \dots, and Abhishek Banerjee (2020-09-22)}
Thermalization of an isolated quantum system has been a non-trivial problem
since the early days of quantum mechanics. In generic isolated systems,
non-equilibrium dynamics is expected to result in thermalization, indicating
the emergence of statistical mechanics from quantum dynamics. However, what
feature of many-body quantum system facilitates quantum thermalization is still
not well understood. Here we revisit this problem and show that introduction of
entanglement in the system gives rise to thermalization, and it takes place at
the level of individual eigenstate. We also show that the expectation value in
the energy eigenstate of each subsystem is close to the canonical average.

\subsection*{\href{http://arxiv.org/abs/2009.10415v1}{Diagrammatic technique for simulation of large-scale quantum repeater  networks with dissipating quantum memories}}
\subsubsection*{Viacheslav V. Kuzmin and Denis V. Vasilyev (2020-09-22)}
We present a detailed description of the diagrammatic technique, recently
devised in [V. V. Kuzmin et.al., npj Quantum Information 5, 115 (2019)], for
semi-analytical description of large-scale quantum-repeater networks. The
technique takes into account all essential experimental imperfections,
including dissipative Liouville dynamics of the network quantum memories and
the classical communication delays. The results obtained with the semi-analytic
method match the exact Monte Carlo simulations while the required computational
resources scale only linearly with the network size. The presented approach
opens new possibilities for the development and efficient optimization of
future quantum networks.

\subsection*{\href{http://arxiv.org/abs/2009.10413v1}{Record spintronic harvesting of thermal fluctuations using paramagnetic  molecular centers}}
\subsubsection*{Bhavishya Chowrira, \dots, and Martin Bowen (2020-09-22)}
Several experiments have suggested that building a quantum engine using the
electron spin enables the harvesting of thermal fluctuations on paramagnetic
centers even though the device is at thermal equilibrium. We illustrate this
thermodynamics conundrum through measurements on several devices of large
output power, which endures beyond room temperature. We've inserted the Co
paramagnetic center in Co phthalocyanine molecules between electron
spin-selecting Fe/C60 interfaces within vertical molecular nanojunctions. We
observe output power as high as 450nW(24nW) at 40K(360K), which leapfrogs
previous results, as well as classical spintronic energy harvesting strategies
involving a thermal gradient. Our data links magnetic correlations between the
fluctuating paramagnetic centers and output power. This device class also
behaves as a spintronically controlled switch of current flow, and of its
direction. We discuss the conceptual challenges raised by these measurements.
Better understanding the phenomenon and further developing this technology
could help accelerate the transition to clean energy.

\subsection*{\href{http://arxiv.org/abs/2009.10412v1}{Surface-induced linear magnetoresistance in antiferromagnetic  topological insulator MnBi2Te4}}
\subsubsection*{X. Lei, \dots, and H. T. He (2020-09-22)}
Through a thorough magneto-transport study of antiferromagnetic topological
insulator MnBi2Te4 (MBT) thick films, a positive linear magnetoresistance (LMR)
with a two-dimensional (2D) character is found in high perpendicular magnetic
fields and temperatures up to at least 260 K. The nonlinear Hall effect further
reveals the existence of high-mobility surface states in addition to the bulk
states in MBT. We ascribe the 2D LMR to the high-mobility surface states of
MBT, thus unveiling a transport signature of surface states in thick MBT films.
A suppression of LMR near the Neel temperature of MBT is also noticed, which
might suggest the gap opening of surface states due to the
paramagnetic-antiferromagnetic phase transition of MBT. Besides these, the
failure of the disorder and quantum LMR model in explaining the observed LMR
indicates new physics must be invoked to understand this phenomenon.

\subsection*{\href{http://arxiv.org/abs/2009.10408v1}{Control dynamics using quantum memory}}
\subsubsection*{Mathieu Roget, and Giuseppe Di Molfetta (2020-09-22)}
We propose a new quantum numerical scheme to control the dynamics of a
quantum walker in a two dimensional space-time grid. More specifically, we show
how, introducing a quantum memory for each of the spatial grid, this result can
be achieved simply by acting on the initial state of the whole system, and
therefore can be exactly controlled once for all. As example we prove
analytically how to encode in the initial state any arbitrary walker's mean
trajectory and variance. This brings significantly closer the possibility of
implementing dynamically interesting physics models on medium term quantum
devices, and introduces a new direction in simulating aspects of quantum field
theories (QFTs), notably on curved manifold.

\subsection*{\href{http://arxiv.org/abs/2009.10402v1}{Giant piezoresistive effect and strong band gap tunability in ultrathin  InSe upon biaxial strain}}
\subsubsection*{Qinghua Zhao, \dots, and Andres Castellanos-Gomez (2020-09-22)}
The ultrathin nature and dangling bonds free surface of two-dimensional (2D)
semiconductors allow for significant modifications of their band gap through
strain engineering. Here, thin InSe photodetector devices are biaxially
stretched, finding, a strong band gap tunability upon strain. The applied
biaxial strain is controlled through the substrate expansion upon temperature
increase and the effective strain transfer from the substrate to the thin InSe
is confirmed by Raman spectroscopy. The band gap change upon biaxial strain is
determined through photoluminescence measurements, finding a gauge factor of up
to ~200 meV/\%. We further characterize the effect of biaxial strain on the
electrical properties of the InSe devices. In the dark state, a large increase
of the current is observed upon applied strain which gives a piezoresistive
gauge factor value of ~450-1000, ~5-12 times larger than that of other 2D
materials and of state-of-the-art silicon strain gauges. Moreover, the biaxial
strain tuning of the InSe band gap also translates in a strain-induced redshift
of the spectral response of our InSe photodetectors with {\Delta}Ecut-off ~173
meV at a rate of ~360 meV/\% of strain, indicating a strong strain tunability of
the spectral bandwidth of the photodetectors.

\subsection*{\href{http://arxiv.org/abs/2009.10398v1}{Optical Hall response of bilayer graphene: the manifestation of chiral  hybridised states in broken mirror symmetry lattices}}
\subsubsection*{V. Nam Do, \dots, and D. Bercioux (2020-09-22)}
Understanding mechanisms governing the optical activity of layered-stacked
materials is crucial for designing devices aimed to manipulate light at the
nanoscale. Here, we show that both the twisted and slid bilayer graphene are
chiral systems able to deflect the polarization of the linear polarized light.
However, only the twisted bilayer graphene supports the circular dichroism. Our
calculation scheme, based on the time-dependent Schr\"odinger equation, is
specifically efficient for calculating the optical-conductivity tensor. In
particular, it allows us showing the chirality of hybridized states as the
handedness-dependent bending of the trajectory of kicked Gaussian wave packets
in the bilayer lattices. We show that the nonzero Hall conductivity is the
result of the non-cancelling manifestation of hybridized states in chiral
lattices. We also demonstrate the continuous dependence of the conductivity
tensor on the twist angle and the sliding vector.

\subsection*{\href{http://arxiv.org/abs/2009.10382v1}{All-dielectric silicon metalens for two-dimensional particle  manipulation in optical tweezers}}
\subsubsection*{Teanchai Chantakit, \dots, and Thomas Zentgraf (2020-09-22)}
Dynamic control of compact chip-scale contactless manipulation of particles
for bioscience applications remains a challenging endeavor, which is restrained
by the balance between trapping efficiency and scalable apparatus. Metasurfaces
offer the implementation of feasible optical tweezers on a planar platform for
shaping the exerted optical force by a microscale-integrated device. Here, we
design and experimentally demonstrate a highly efficient silicon-based metalens
for two-dimensional optical trapping in the near-infrared. Our metalens concept
is based on the Pancharatnam-Berry phase, which enables the device for
polarization-sensitive particle manipulation. Our optical trapping setup is
capable of adjusting the position of both the metasurface lens and the particle
chamber freely in three directions, which offers great freedom for optical trap
adjustment and alignment. Two-dimensional (2D) particle manipulation is done
with a relatively low numerical aperture metalens ($NA_{ML}=0.6$). We
experimentally demonstrate both 2D polarization sensitive drag and drop
manipulation of polystyrene particles suspended in water and transfer of
angular orbital momentum to these particles with a single tailored beam. Our
work may open new possibilities for lab-on-a-chip optical trapping for
bioscience applications and micro to nanoscale optical tweezers.

\subsection*{\href{http://arxiv.org/abs/2009.10368v1}{Shor-Movassagh chain leads to unusual integrable model}}
\subsubsection*{Bin Tong, \dots, and Vladimir Korepin (2020-09-22)}
The ground state of Shor-Movassagh chain can be analytically described by the
Motzkin paths. There is no analytical description of the excited states, the
model is not solvable. We prove the integrability of the model without
interacting part in this paper [free Shor-Movassagh]. The Lax pair for the free
Shor-Movassagh open chain is explicitly constructed. We further obtain the
boundary $K$-matrices compatible with the integrability of the model on the
open interval. Our construction provides a direct demonstration for the quantum
integrability of the model, described by Yang-Baxter algebra. Due to the lack
of crossing unitarity, the integrable open chain can not be constructed by the
reflection equation (boundary Yang-Baxter equation).

\subsection*{\href{http://arxiv.org/abs/2009.10357v1}{Rainbow Nambu-Goldstone modes under a nonequilibrium steady flow}}
\subsubsection*{Yuki Minami, and Yoshimasa Hidaka (2020-09-22)}
We study an $O(N)$ scalar model under shear flow and its Nambu-Goldstone
modes associated with spontaneous symmetry breaking $O(N) \to O(N-1)$. We find
that the Nambu-Goldstone mode splits into an infinite number of gapless modes,
which we call the rainbow Nambu-Goldstone modes. They have different group
velocities and the fractional dispersion relation $\omega \sim k_1^{2/3}$,
where $k_1$ is the wavenumber along the flow. Such behaviors do not have
counterparts in an equilibrium state.

\subsection*{\href{http://arxiv.org/abs/2009.10344v1}{Surrogate representation of sink strengths and the long-term role of  crystalline interfaces in the development of irradiation-induced bubbles}}
\subsubsection*{Jing Luo, \dots, and Xu Guo (2020-09-22)}
The present article addresses an early-stage attempt on replacing the
analyticity-based sink strength terms in rate equations by surrogate models of
machine learning representation. Here we emphasise, in the context of
multiscale modelling, a combinative use of machine learning with scale
analysis, through which a set of fine-resolution problems of partial
differential equations describing the (quasi-steady) short-range individual
sink behaviour can be asymptotically sorted out from the mean-field kinetics.
Hence the training of machine learning is restrictively oriented, that is, to
express the local and already identified, but analytically unavailable
nonlinear functional relationships between the sink strengths and other local
continuum field quantities. With the trained models, one is enabled to
quantitatively investigate the biased effect shown by a void/bubble being a
point defect sink, and the results are compared with existing ones over
well-studied scenarios. Moreover, the faster diffusive mechanisms on
crystalline interfaces are distinguishingly modelled by locally planar rate
equations, and their linkages with rate equations for bulk diffusion are
formulated through derivative jumps of point defect concentrations across the
interfaces. Thus the distinctive role of crystalline interfaces as partial
sinks and quick diffusive channels can be investigated. Methodologicalwise, the
present treatment is also applicable for studying more complicated situation of
long-term sink behaviour observed in irradiated materials.

\subsection*{\href{http://arxiv.org/abs/2009.10339v1}{Universal scaling behaviour near vortex-solid/glass to vortex-fluid  transition in type-II superconductors in two- and three-dimensions}}
\subsubsection*{Hemanta Kumar Kundu, \dots, and Aveek Bid (2020-09-22)}
In this article, we present evidence for the existence of vortex-solid/glass
(VG) to vortex-fluid (VF) transition in a type-II superconductor (SC), NbN. We
probed the VG to VF transition in both 2D and 3D films of NbN through studies
of magnetoresistance and current-voltage characteristics. The dynamical
exponents corresponding to this phase transition were extracted independently
from the two sets of measurements. The $H$-$T$ phase diagram for the 2D and 3D
SC are found to be significantly different near the critical point. In the case
of 3D SC, the exponent values obtained from the two independent measurements
show excellent match. On the other hand, for the 2D SC, the exponents obtained
from the two experiments were significantly different. We attribute this to the
fact that the characteristic length scale diverges near the critical point in a
2D SC in a distinctly different way from its 3D counterpart form scaling
behaviour.

\subsection*{\href{http://arxiv.org/abs/2009.10329v1}{Tensor-network codes}}
\subsubsection*{Terry Farrelly, \dots, and Thomas M. Stace (2020-09-22)}
Inspired by holographic codes and tensor-network decoders, we introduce
tensor-network stabilizer codes which come with a natural tensor-network
decoder. These codes can correspond to any geometry, but, as a special case, we
generalize holographic codes beyond those constructed from perfect or
block-perfect isometries, and we give an example that corresponds to neither.
Using the tensor-network decoder, we find a threshold of 18.8\% for this code
under depolarizing noise. We also show that for holographic codes the exact
tensor-network decoder (with no bond-dimension truncation) is efficient with a
complexity that is polynomial in the number of physical qubits, even for
locally correlated noise.

\subsection*{\href{http://arxiv.org/abs/2009.10328v1}{Noise correlation and success probability in coherent Ising machines}}
\subsubsection*{Yoshitaka Inui and Yoshihisa Yamamoto (2020-09-22)}
We compared the noise correlation and the success probability of coherent
Ising machines (CIMs) with optical delay-line, measurement feedback, and
mean-field couplings. We theoretically studied three metrics for the noise
correlations in these CIMs: quantum entanglement, quantum discord, and
normalized correlation of canonical coordinates. The success probability was
obtained through numerical simulations of truncated stochastic differential
equations based on the Wigner distribution function. The results indicate that
the success probability is more directly related to the normalized correlation
function rather than entanglement or quantum discord.

\subsection*{\href{http://arxiv.org/abs/2009.10323v1}{Ferromagnetic phase of spinel compound MgV$_2$O$_4$ and its spintronics  properties}}
\subsubsection*{Javad G. Azadani, \dots, and Tony Low (2020-09-22)}
Spinel compound, MgV$_2$O$_4$, known as a highly frustrated magnet has been
extensively studied both experimentally and theoretically for its exotic
quantum magnetic states. However, due to its intrinsic insulating nature in its
antiferromagnetic (AFM) ground state, its realistic applications in spintronics
are quite limited. Here, based on first-principles calculations, we examine the
ferromagnetic (FM) phase of MgV$_2$O$_4$, which was found to host
three-dimensional flat band (FB) right near the Fermi level, consequently
yielding a large anomalous Hall effect (AHE, $\sigma \approx
670\,\Omega^{-1}\cdot cm^{-1}$). Our calculations suggest that the
half-metallicity feature of MgV$_2$O$_4$ is preserved even after interfacing
with MgO due to the excellent lattice matching, which could be a promising spin
filtering material for spintronics applications. Lastly, we explore
experimental feasibility of stabilizing this FM phase through strain and doping
engineering. Our study suggests that experimentally accessible amount of hole
doping might induce a AFM-FM phase transition.

\subsection*{\href{http://arxiv.org/abs/2009.10314v1}{Detector self-tomography}}
\subsubsection*{Raúl Cónsul and Alfredo Luis (2020-09-22)}
We present an intuitive model of detector self-tomography. Two identical
realisations of the detector are illuminated by an entangled state that
connects the joint statistics in a way in which each detector sees the other as
a kind of mirror reflection. A suitable analysis of the statistics reveals the
possibility of fully characterizing the detector. We apply this idea to
Bell-type experiments revealing their nonclassical nature.

\subsection*{\href{http://arxiv.org/abs/2009.10299v1}{Magnon-mediated spin currents in Tm3Fe5O12/Pt with perpendicular  magnetic anisotropy}}
\subsubsection*{G. L. S. Vilela, \dots, and J. S. Moodera (2020-09-22)}
The control of pure spin currents carried by magnons in magnetic insulator
(MI) garnet films with a robust perpendicular magnetic anisotropy (PMA) is of
great interest to spintronic technology as they can be used to carry, transport
and process information. Garnet films with PMA present labyrinth domain
magnetic structures that enrich the magnetization dynamics, and could be
employed in more efficient wave-based logic and memory computing devices. In
MI/NM bilayers, where NM being a normal metal providing a strong spin-orbit
coupling, the PMA benefits the spin-orbit torque (SOT) driven magnetization's
switching by lowering the needed current and rendering the process faster,
crucial for developing magnetic random-access memories (SOT-MRAM). In this
work, we investigated the magnetic anisotropies in thulium iron garnet (TIG)
films with PMA via ferromagnetic resonance measurements, followed by the
excitation and detection of magnon-mediated pure spin currents in TIG/Pt driven
by microwaves and heat currents. TIG films presented a Gilbert damping constant
{\alpha}~0.01, with resonance fields above 3.5 kOe and half linewidths broader
than 60 Oe, at 300 K and 9.5 GHz. The spin-to-charge current conversion through
TIG/Pt was observed as a micro-voltage generated at the edges of the Pt film.
The obtained spin Seebeck coefficient was 0.54 {\mu}V/K, confirming also the
high interfacial spin transparency.

\subsection*{\href{http://arxiv.org/abs/2009.10296v1}{Influence of particle size on the thermoresponsive and rheological  properties of aqueous poly(N-isopropylacrylamide) colloidal suspensions}}
\subsubsection*{Chandeshwar Misra, and Ranjini Bandyopadhyay (2020-09-22)}
Thermoresponsive poly(N-isopropylacrylamide) (PNIPAM) particles of different
sizes are synthesized by varying the concentration of sodium dodecyl sulphate
(SDS) in a one-pot method. The sizes, size polydispersities and the
thermoresponsivity of the PNIPAM particles are characterized by using dynamic
light scattering and scanning electron microscopy. It is observed that the
sizes of these particles decrease with increase in SDS concentration. Swelling
ratios of PNIPAM particles measured from the thermoresponsive curves are
observed to increase with decrease in particle size. This observation is
understood by minimizing the Helmholtz free energy of the system with respect
to the swelling ratio of the particles. Finally, the dynamics of these
particles in jammed aqueous suspensions are investigated by performing
rheological measurements.

\subsection*{\href{http://arxiv.org/abs/2009.10287v1}{Spin-orbital coupled real topological phases}}
\subsubsection*{Y. X. Zhao, \dots, and Shengyuan A. Yang (2020-09-22)}
Real topological phases stemmed from spacetime-inversion ($PT$) symmetry have
caught considerable interest recently, because of their extraordinary
properties, such as real Dirac semimetals, nontrivial nodal-line linking
structures, non-Abelian topological charges, higher-order topological phases,
and boundary topological phase transitions with unchanged bulk topological
invariants. Such phases rely on the algebraic identity $(PT)^2=1$. Since the
identity holds only for spinless fermions, it is a common wisdom that these
phases will be destroyed by spin-orbital coupling or magnetic orders. Here, we
show that in the presence of $\mathbb{Z}_2$ gauge fields, the real and
symplectic symmetry classes with $(PT)^2=\pm 1$, respectively, can be exchanged
due to the projective representation of the symmetry algebra. In other words,
we can effectively turn spinful fermions into spinless ones, and hence achieve
real topological phases also in spin-orbit coupled systems. This is explicitly
demonstrated by a $3$D generalized Kane-Mele model, with the gauge flux
configuration that minimizes the ground-state energy due to the Lieb theorem.
In the presence of spin-orbital coupling and magnetic ordering, the model
realizes novel real topological semimetal phases characterized by the
Stiefel-Whitney classes with the aforementioned boundary phase transition. Our
work broadens the scope of real topological phases, and more importantly, it
reveals a new avenue, namely the projective representation of symmetries, to
switch the fundamental categories of system topologies.

\subsection*{\href{http://arxiv.org/abs/2009.10275v1}{Quantum optimal control using phase-modulated driving fields}}
\subsubsection*{Jiazhao Tian, \dots, and Jianming Cai (2020-09-22)}
Quantum optimal control represents a powerful technique to enhance the
performance of quantum experiments by engineering the controllable parameters
of the Hamiltonian. However, the computational overhead for the necessary
optimization of these control parameters drastically increases as their number
grows. We devise a novel variant of a gradient-free optimal-control method by
introducing the idea of phase-modulated driving fields, which allows us to find
optimal control fields efficiently. We numerically evaluate its performance and
demonstrate the advantages over standard Fourier-basis methods in controlling
an ensemble of two-level systems showing an inhomogeneous broadening. The
control fields optimized with the phase-modulated method provide an increased
robustness against such ensemble inhomogeneities as well as control-field
fluctuations and environmental noise, with one order of magnitude less of
average search time. Robustness enhancement of single quantum gates is also
achieved by the phase-modulated method. Under environmental noise, an XY-8
sequence constituted by optimized gates prolongs the coherence time by $50\%$
compared with standard rectangular pulses in our numerical simulations, showing
the application potential of our phase-modulated method in improving the
precision of signal detection in the field of quantum sensing.

\subsection*{\href{http://arxiv.org/abs/2009.10234v1}{Standard and inverse site percolation of straight rigid rods on  triangular lattices: Isotropic and nematic deposition/removal}}
\subsubsection*{L. S. Ramirez, \dots, and A. J. Ramirez-Pastor (2020-09-22)}
Numerical simulations and finite-size scaling analysis have been carried out
to study standard and inverse percolation of straight rigid rods on triangular
lattices. In the case of standard (inverse) percolation, the lattice is
initially empty(occupied) and linear $k$-mers ($k$ linear consecutive sites)
are randomly and sequentially deposited on(removed from) the lattice,
considering an isotropic and nematic scheme. The study is conducted by
following the behavior of four critical concentrations with the size $k$,
determined for a wide range of $k$ : $(i)$[$(ii)$] standard isotropic[nematic]
percolation threshold $\theta_{c,k}$[$\vartheta_{c,k}$], and $(iii)$[$(iv)$]
inverse isotropic[nematic] percolation threshold
$\theta^i_{c,k}$[$\vartheta^i_{c,k}$]. The obtained results indicate that:
$(1)$ $\theta_{c,k}$[$\theta^i_{c,k}$] exhibits a non-monotonous dependence
with $k$. It decreases[increases], goes through a minimum[maximum] around $k =
11$, then increases and asymptotically converges towards a definite value for
large $k$ $\theta_{c,k \rightarrow \infty}=0.500(2)$[$\theta^i_{c,k \rightarrow
\infty}=0.500(1)$]; $(2)$ $\vartheta_{c,k}$[$\vartheta^i_{c,k}$] rapidly
increases[decreases] and asymptotically converges towards a definite value for
infinitely long $k$-mers $\vartheta_{c,k \rightarrow
\infty}=0.5334(6)$[$\vartheta^i_{c,k \rightarrow \infty}=0.4666(6)$]; $(3)$ for
both models, the curves of standard and inverse percolation thresholds are
symmetric with respect to $\theta = 0.5$. Thus, a complementary property is
found $\theta_{c,k} + \theta^i_{c,k} = 1$ (and $\vartheta_{c,k} +
\vartheta^i_{c,k} = 1$), which has not been observed in other regular lattices.
This condition is analytically validated by using exact enumeration of
configurations for small systems; and $(4)$ in all cases, the model presents
percolation transition for the whole range of $k$.

\subsection*{\href{http://arxiv.org/abs/2009.10227v1}{Electronic properties of Bi$_{2}$Se$_{3}$ dopped by $3d$ transition  metal (Mn, Fe, Co, or Ni) ions}}
\subsubsection*{Andrzej Ptok, and Anna Ciechan (2020-09-22)}
Topological insulators are characterized by the existence of band inversion
and the possibility of the realization of surface states. Doping with a
magnetic atom, which is a source of the time-reversal symmetry breaking, can
lead to realization of novel magneto-electronic properties of the system. In
this paper, we study effects of substitution by the transition metal ions (Mn,
Fe, Co and Ni) into Bi$_{2}$Se$_{3}$ on its electric properties. Using the ab
inito supercell technique, we investigate the density of states and the
projected band structure. Under such substitution the shift of the Fermi level
is observed. We find the existence of nearly dispersionless bands around the
Fermi level associated with substituted atoms, especially, in the case of the
Co and Ni. Additionally, we discuss the modification of the electron
localization function as well as charge and spin redistribution in the system.
Our study shows a strong influence of the transition metal--Se bond on local
modifications of the physical properties. The results are also discussed in the
context of the interplay between energy levels of the magnetic impurities and
topological surface states.

